%% Dokumentenklasse (Koma Script) -----------------------------------------
\documentclass[%
  %draft,     % Entwurfsstadium
  final, % fertiges Dokument
  % --- Paper Settings ---
  paper=a4,paper=portrait, pagesize=auto, % [Todo: add alternatives]% landscape% driver
  % --- Base Font Size ---
  fontsize=11pt,%
  % --- Koma Script Version ---
  version=last, ]{scrreprt} %% Classes: scrartcl, scrreprt, scrbook

\usepackage{scrhack}

% Encoding der Dateien (sonst funktionieren Umlaute nicht)
% Fuer Linux -> utf8
% Fuer Windows, alte Linux Distributionen -> latin1

% Empfohlen latin1, da einige Pakete mit utf8 Zeichen nicht
% funktionieren, z.B: listings, soul.
%\usepackage[latin1]{inputenc}
%\usepackage[ansinew]{inputenc}
\usepackage[utf8]{inputenc}
\usepackage{xurl}
%\usepackage{ucs}
%\usepackage[utf8x]{inputenc}

%%%%%%%%%%%%%%%%%%%%%%%%%
%    SET LANGUAGE       %
%%%%%%%%%%%%%%%%%%%%%%%%%
\def\lang{ngerman} % Options: english or ngerman

% The used tool and the reason; remove command if it does not apply
\def\declareUseOfGenerativeAITool{ChatGPT~4}

%%% Preambel
\input{preambel/settings}
\input{preambel/preambel}

\author{Jan Ole Schmidt}
\thesis{Bachelor Thesis}
%\thesis{Master Thesis}
\title{Die Zukunft der Software-Entwicklung unter dem Einfluss künstlicher Intelligenz: Chancen, Herausforderungen und praxisorientierte Anwendungen}
\academicTitle{Bachelor of Science}
%\academicTitle{Master of Science}
\firstReferee{Prof. Dr. Dennis Priefer}
\secondReferee{Kevin Linne}

%
%%%% Neue Befehle
\input{macros/newcommands}
\input{macros/TableCommands}

%%% Silbentrennung
\input{preambel/Hyphenation}

%% Dokument Beginn %%%%%%%%%%%%%%%%%%%%%%%%%%%%%%%%%%%%%%%%%%%%%%%%%%%%%%%%

\begin{document}
% Deckblatt
\input{content/00_Titel}

% Nicht vergessen zu Entfernen wenn alles Fertig:
%\listoftodos

% Abstract - Kurzfassung der Arbeit ohne Wertung
% % Wie schreibe ich das Abstract?
% Bei der Vorbereitung des Abstracts sollten Sie sich an folgenden grundlegenden Punkten orientieren:

%% Motivation des Textes:
%% worin liegt die Bedeutung der entsprechenden Forschung, warum sollte der längere Text gelesen werden?

%% Fragestellung:
%% welche Fragestellung(en) versucht der Text zu beantworten, was ist der Umfang der Forschung, was sind die zentralen Argumente und Behauptungen?

%% Methodologie:
%% welche Methoden/Zugänge nutzt der Autor/die Autorin, auf welche empirische Basis stützt sich der Text?

%% Ergebnisse:
%% zu welchen Ergebnissen kam die Forschung, was sind die zentralen Schlussfolgerungen des Textes?

%% Implikationen:
%% welche Schlussfolgerungen ergeben sich aus dem Text für die Forschung, was fügt der Text unserem Wissen über das Thema hinzu?
 
\begin{abstract}
    Künstliche Intelligenz (KI) hat sich in den letzten Jahren von einem theoretischen Konzept zu einer zentralen Technologie in vielen Bereichen der Softwareentwicklung entwickelt. Besonders generative KI-Modelle wie GitHub Copilot und Cursor AI revolutionieren den Entwicklungsprozess, indem sie automatisierte Code-Vorschläge, Fehlerkorrekturen und sogar vollständige Funktionen generieren. Diese Fortschritte werfen jedoch grundlegende Fragen auf: Welche langfristigen Auswirkungen hat der verstärkte Einsatz von KI auf Softwareentwickler? Inwieweit verändert KI etablierte Prinzipien und Methoden der Softwareentwicklung? Diese Arbeit untersucht die Chancen, Herausforderungen und praxisorientierten Anwendungen von KI in der Zukunft der Softwareentwicklung.
\end{abstract}
\cleardoublepage
\frontmatter
% Inhaltsverzeichnis in den PDF-Links eintragen
\pdfbookmark[1]{Inhaltsverzeichnis}{toc}
\tableofcontents

% Hauptteil
\mainmatter
\chapter{Einführung}
Künstliche Intelligenz (KI) verändert zunehmend die Softwareentwicklung, indem sie Automatisierungsmöglichkeiten bietet und Entwicklungsprozesse effizienter gestaltet. In den letzten Jahren haben generative KI-Modelle erhebliche Fortschritte erzielt, insbesondere in der automatisierten Codegenerierung und Qualitätssicherung. Erste Studien zeigen, dass KI-gestützte Tools Produktivitätssteigerungen für Entwickler ermöglichen, indem sie automatisierte Codevorschläge liefern und repetitive Aufgaben reduzieren.

Neben den Vorteilen dieser Technologien gibt es jedoch auch Herausforderungen. Sicherheitsaspekte, ethische Fragen und die langfristige Veränderung der Arbeitsweise von Softwareentwicklern sind zentrale Diskussionspunkte in der aktuellen Forschung. Die Auswirkungen von KI auf die Softwareentwicklung sind nicht nur technischer Natur, sondern haben auch weitreichende Konsequenzen für Unternehmensstrukturen und die Ausbildung zukünftiger Entwickler.

Diese Arbeit setzt sich mit den Chancen, Herausforderungen und praxisorientierten Anwendungen von KI in der Softwareentwicklung auseinander. Dabei wird untersucht, wie KI-gestützte Softwareentwicklung sowohl die Produktivität als auch die Qualität von Code beeinflusst und welche ethischen und sicherheitstechnischen Fragen sich daraus ergeben.
\newpage
\section{Kontext und Motivation}
\subsection{Relevanz des Themas}
Diese Arbeit adressiert diese Problematik, indem sie die Chancen und Herausforderungen von KI in der Softwareentwicklung analysiert. Ziel ist es, sowohl wissenschaftliche als auch praktische Erkenntnisse zu gewinnen, die Unternehmen und Entwicklern helfen, informierte Entscheidungen über den Einsatz von KI-gestützten Technologien zu treffen.

Künstliche Intelligenz (KI) hat sich in den letzten Jahren als transformative Technologie in der Softwareentwicklung etabliert. Insbesondere generative KI-Modelle wie Large Language Models (LLMs) haben das Potenzial, Entwicklungsprozesse signifikant zu verändern. Die Automatisierung von Codegenerierung, Fehleranalyse und Softwarewartung führt zu einer gesteigerten Effizienz und ermöglicht es Entwicklern, sich auf konzeptionell anspruchsvollere Aufgaben zu konzentrieren.

Die zunehmende Integration von KI in den Softwareentwicklungsprozess eröffnet neue Möglichkeiten, birgt jedoch auch Herausforderungen. Während einige Studien eine gesteigerte Produktivität und Codequalität durch KI-gestützte Tools belegen, gibt es gleichzeitig Bedenken hinsichtlich Sicherheitsrisiken, algorithmischer Verzerrung und langfristigen Veränderungen in der Arbeitsweise von Entwicklern. Diese Gegensätze verdeutlichen die Notwendigkeit einer fundierten wissenschaftlichen Auseinandersetzung mit den Auswirkungen von KI auf die Softwareentwicklung.

\subsection{Motivation}
Die zunehmende Verbreitung künstlicher Intelligenz (KI) in der Softwareentwicklung stellt sowohl Wissenschaft als auch Praxis vor bedeutende Herausforderungen und Chancen. Unternehmen integrieren KI-gestützte Tools in ihre Entwicklungsprozesse, um Produktivität und Codequalität zu steigern, doch der langfristige Einfluss dieser Technologie auf die Arbeitsweise von Softwareentwicklern ist noch nicht vollständig erforscht.

Besonders relevant ist die Frage, wie sich KI-gestützte Entwicklungsumgebungen auf traditionelle Softwareentwicklungspraktiken auswirken. Während einige Forschungen darauf hindeuten, dass KI-Tools repetitive Aufgaben reduzieren und Entwicklern mehr Raum für kreative Problemlösungen geben, bestehen weiterhin Unsicherheiten hinsichtlich der Verlässlichkeit der generierten Codevorschläge und möglicher ethischer Bedenken.

\section{Zielsetzung und Fragestellungen}

Ziel dieser Arbeit ist es, die Auswirkungen generativer KI auf die
Softwareentwicklung umfassend zu analysieren und daraus praxisnahe
Handlungsempfehlungen für Unternehmen und Entwickler:innen abzuleiten. Im Fokus
steht insbesondere, wie KI-gestützte Tools bestehende Entwicklungspraktiken
verändern, welche Herausforderungen damit verbunden sind und wie sich Chancen
und Risiken in der Praxis ausbalancieren lassen.

Daraus ergeben sich folgende zentrale Forschungsfragen:
\begin{itemize}
      \item Wie verändert generative KI traditionelle Entwicklungspraktiken in der
            Softwareentwicklung?
      \item Welche spezifischen Herausforderungen entstehen durch KI-gestützte
            Softwareentwicklung hinsichtlich Sicherheit, Ethik und Code-Qualität?
      \item Wie kann generative KI Softwareentwickler:innen in einem agilen
            Entwicklungsprozess unterstützen?
      \item Wie lassen sich bestehende generative KI-Tools, wie Cursor, GitHub Copilot,
            bolt.new (Bolt), in den Entwicklungsprozess einer React-Native-App integrieren
            und welchen Einfluss hat dies auf Entwicklungszeit und Code-Qualität?
\end{itemize}

Um diese Fragen nicht nur theoretisch, sondern auch praxisnah zu beleuchten,
wird im praktischen Teil der Arbeit eine React-Native-App als Fallstudie
herangezogen. Ziel ist es, exemplarisch zu untersuchen, wie generative KI-Tools
in reale Entwicklungsprozesse integriert werden können und welchen konkreten
Mehrwert sie im Entwicklungsalltag bieten.


\section{Methodik und Vorgehensweise}
Diese Arbeit verfolgt eine theoretische und literaturbasierte Herangehensweise. Es wird eine systematische Analyse bestehender wissenschaftlicher Literatur durchgeführt, um ein umfassendes Verständnis der aktuellen Forschungslage zu generativer KI in der Softwareentwicklung zu erhalten. Die Methodik umfasst folgende Schritte:
\begin{enumerate}
    \item \textbf{Literaturrecherche:} Auswahl relevanter wissenschaftlicher Publikationen aus anerkannten Datenbanken wie IEEE Xplore, arXiv und SpringerLink. Dabei wird ein Fokus auf aktuelle Studien gelegt, die die Auswirkungen von KI auf die Softwareentwicklung untersuchen.
    \item \textbf{Kategorisierung der Forschungsthemen:} Identifikation und Gruppierung zentraler Themenfelder wie Produktivitätssteigerung, Automatisierung, Sicherheitsrisiken und ethische Fragestellungen.
    \item \textbf{Vergleichende Analyse:} Gegenüberstellung der identifizierten Chancen und Herausforderungen durch KI in der Softwareentwicklung.
    \item \textbf{Synthese und Ableitung von Schlussfolgerungen:} Basierend auf der Literaturauswertung werden praxisorientierte Empfehlungen für den Einsatz von KI in der Softwareentwicklung erarbeitet.
\end{enumerate}

\section{Aufbau der Arbeit}
Die Arbeit gliedert sich in folgende Kapitel:

\begin{itemize}
    \item \textbf{Kapitel 1: Einleitung} 
        \begin{itemize}
            \item Darstellung von Hintergrund, Motivation, Zielsetzung, Forschungsfragen, methodischer Vorgehensweise und Abgrenzung.
        \end{itemize}
    \item \textbf{Kapitel 2: Theoretische Grundlagen}   
        \begin{itemize}
            \item Definition und Funktionsweise von generativer KI in der Softwareentwicklung
            \item Übersicht relevanter KI-Modelle, Algorithmen und Beispiele für KI-gestützte Entwicklungswerkzeuge
        \end{itemize}
    \item \textbf{Kapitel 3: Praktische Demonstration}
        \begin{itemize}
            \item Vorstellung des Projekts "Locals" und dessen Architektur
            \item Implementierung einer interaktiven Kartenansicht in der React Native-Anwendung mithilfe generativer KI-Technologien
            \item Darstellung der Implementierungsschritte, Code-Beispiele und erste Evaluationsergebnisse 
        \end{itemize}
    \item \textbf{Kapitel 4: Chancen durch KI} 
        \begin{itemize}
            \item Effizienzsteigerung und Automatisierung
            \item Neue Werkzeuge und Methoden
            \item Verbesserte Code-Qualität und Fehlerminimierung
            \item Einfluss von KI auf agile Entwicklungsmethoden
            \item Bezugnahme auf die Erkenntnisse aus Kapitel 3
        \end{itemize}
    \item \textbf{Kapitel 5: Herausforderungen durch KI} 
        \begin{itemize}
            \item Sicherheits- und Datenschutzaspekte
            \item Ethische Implikationen und Bias in KI-Modellen
            \item Langfristige Auswirkungen auf Entwickler:innen-Rollen
            \item Organisatorische und technologische Hürden
            \item Risiken durch Abhängigkeit von KI-generiertem Code
            \item Analyse der in Kapitel 3 möglicherweise aufgetretenen Herausforderungen und Problematiken
        \end{itemize}
    \item \textbf{Kapitel 6: Wirtschaftliche und gesellschaftliche Auswirkungen} 
        \begin{itemize}
            \item Veränderungen in Softwareunternehmen
            \item Auswirkungen auf den Arbeitsmarkt und Entwickler:innen-Rollen
            \item Zukunftsperspektiven und strategische Empfehlungen
            \item Kosten-Nutzen-Analyse von KI-gestützter Softwareentwicklung
        \end{itemize}
    \item \textbf{Kapitel 7: Fazit und Ausblick} 
        \begin{itemize}
            \item Zusammenfassung der theoretischen und praktischen Erkenntnisse
            \item Diskussion offener Forschungsfragen
            \item Ableitung von Handlungsempfehlungen und Ausblick auf zukünftige Entwicklungen
        \end{itemize}
\end{itemize}

\section{Abgrenzung}

Die Arbeit konzentriert sich auf die theoretische Analyse der Chancen und
Herausforderungen von KI in der Softwareentwicklung. Folgende Aspekte werden
bewusst ausgeklammert:

\begin{itemize}
    \item \textbf{Technische Implementierungen:} Es werden keine eigenen KI-Modelle oder neuen Algorithmen entwickelt.
    \item \textbf{Empirische Studien:} Die Arbeit basiert auf einer literaturgestützten Analyse; eigene Befragungen, Experimente oder quantitative Erhebungen werden nicht durchgeführt.
    \item \textbf{Rechtliche Rahmenbedingungen:} Eine detaillierte Untersuchung rechtlicher oder regulatorischer Aspekte findet nicht statt.
\end{itemize}

Obwohl ein begrenzter praktischer Teil in Form einer exemplarischen
Funktionsimplementierung integriert wird, dient dieser ausschließlich als Proof
of Concept zur Veranschaulichung des KI-Einsatzes. Eine umfassende empirische
oder technische Evaluation erfolgt nicht.

Diese Abgrenzungen und methodischen Einschränkungen sind bei der Interpretation
der Ergebnisse sowie im abschließenden Fazit und Ausblick dieser Arbeit stets
zu berücksichtigen.



\chapter{Theoretische Grundlagen}

\section{Künstliche Intelligenz: Definitionen und Technologien}

Künstliche Intelligenz (KI) leitet einen grundlegenden Wandel in der
Softwareentwicklung ein. Die aktuelle, rechtlich verbindliche Definition der
Europäischen Union beschreibt KI als maschinengestützte Systeme, die mit
unterschiedlichem Grad an Autonomie Vorhersagen, Empfehlungen oder
Entscheidungen generieren, welche physische oder virtuelle Umgebungen
beeinflussen können \cite{noauthor_verordnung_nodate}. Diese Definition
etabliert sich zunehmend als Referenzrahmen in Forschung und Praxis.

Moderne KI-Systeme unterscheiden sich deutlich von traditionellen
softwarebasierten Ansätzen. Während frühe KI-Tools vor allem Aufgaben wie
Syntaxprüfung unterstützten, übernehmen generative KI-Systeme (GenAI) heute
komplexe Aufgaben im Design, der Entwicklung und Wartung von Software, etwa
durch die automatische Generierung von Code-Snippets, Unterstützung bei
Unit-Tests oder die Automatisierung von Deployment-Prozessen
\cite{donvir_role_2024}.

Zu den wichtigsten Architekturen zählen Transformer-Modelle (insbesondere Large
Language Models wie GPT), Generative Adversarial Networks (GANs), Variational
Autoencoders (VAEs) und Diffusion Models. Während LLMs primär für Text- und
Codegenerierung eingesetzt werden, kommen GANs und Diffusion Models vor allem
in der Bild- und Medienerzeugung zum Einsatz \cite{donvir_role_2024}. Viele
aktuelle Tools wie ChatGPT, GitHub Copilot oder Stable Diffusion basieren auf
diesen Architekturen und treiben Innovationen in der Softwareentwicklung
maßgeblich voran.

Der Einsatz generativer KI verändert Methoden und Arbeitsweisen grundlegend.
Besonders Large Language Models prägen sämtliche Phasen des
Softwareentwicklungszyklus, von der Anforderungsanalyse bis zur Umsetzung und
Wartung. Entscheidungsunterstützung, die Rekonstruktion von
Softwarearchitekturen sowie Methoden wie Few-Shot-Prompting und
Retrieval-Augmented Generation (RAG) sind dabei zentrale Elemente
\cite{esposito_generative_2025}.

Trotz fortschreitender Automatisierung bleibt die sorgfältige Validierung der
von KI generierten Ergebnisse durch Entwickler:innen unerlässlich. Die größten
Herausforderungen betreffen Präzision und Verlässlichkeit der Modelle, den
Umgang mit sogenannten „Halluzinationen“, ethische Aspekte sowie das Fehlen
domänenspezifischer Benchmarks und Standards \cite{esposito_generative_2025}.

Die Entwicklung der Softwaretechnik unter dem Einfluss von KI lässt sich grob
in drei Phasen gliedern. Während Software Engineering~1.0 klassische,
code-zentrierte Entwicklungspraktiken beschreibt, kennzeichnet SE~2.0 die
Integration KI-gestützter Assistenzsysteme (z.\,B. Copilots) in etablierte
Prozesse. Die Vision von SE~3.0 sieht einen Paradigmenwechsel zu einer
intent-basierten, dialogorientierten Entwicklung vor, in der KI-Systeme als
intelligente Partner agieren~\cite{hassan_towards_2024}.

\begin{table}[H]
    \centering
    \vspace{1em}
    \footnotesize
    \begin{tabular}{|p{3cm}|p{3.3cm}|p{3.3cm}|p{3.3cm}|}
        \hline
                                                 & \textbf{SE 1.0 (Vergangenheit)} & \textbf{SE 2.0 (Gegenwart)} & \textbf{SE 3.0 (Zukunft)} \\
        \hline
        Leitmotiv                                &
        Code-first                               &
        Code-first + KI-Unterstützung            &
        Intent-first, KI als Partner                                                                                                         \\
        \hline
        Technologie                              &
        Klassische Tools, Programmanalyse        &
        KI-gestützte Copilots, Foundation Models &
        Konversationsorientiert, Reasoning-Modelle                                                                                           \\
        \hline
        Rolle von Mensch/KI                      &
        Mensch zentral, alles manuell            &
        Mensch + KI, KI assistiert               &
        Symbiose: Mensch \& KI kollaborieren                                                                                                 \\
        \hline
        Besonderheiten                           &
        Fokus auf Code, klare Abläufe            &
        Effizienzsteigerung, hohe kognitive Last &
        Dialog, Automatisierung, wissensgetrieben                                                                                            \\
        \hline
    \end{tabular}
    \caption{Entwicklung der Softwaretechnik unter KI-Einfluss (SE~1.0 bis SE~3.0). Quelle: Eigene Darstellung in Anlehnung an Hassan et al.~\cite{hassan_towards_2024}.}
    \label{tab:se-evolution}
\end{table}

% \vspace{1em}
% \noindent
% \textbf{Quellen:}
% \begin{itemize}
%     \item Verordnung (EU) 2024/1689 des Europäischen Parlaments und des Rates
%           \cite{noauthor_verordnung_nodate}
%     \item Donvir, A. et al. (2024): \textit{The Role of Generative AI Tools in
%               Application Development: A Comprehensive Review of Current Technologies and
%               Practices} \cite{donvir_role_2024}
%     \item Esposito, M. et al. (2025): \textit{Generative AI for Software Architecture.
%               Applications, Trends, Challenges, and Future Directions}
%           \cite{esposito_generative_2025}
% \end{itemize}


\subsection{Beispiele für generative KI-Tools in der Praxis}

Generative KI-Tools sind mittlerweile aus der Softwareentwicklung nicht mehr
wegzudenken und decken ein breites Spektrum an Aufgaben ab, von der
Code-Vervollständigung über automatische Testgenerierung bis hin zur
Unterstützung ganzer Entwicklungsprojekte \cite{donvir_role_2024}. Zu den
praxisrelevanten Vertretern zählen unter anderem GitHub Copilot, TabNine,
Cursor AI und Devin AI.

\textit{GitHub Copilot}
ist ein KI-basierter Codeassistent, der Entwickelnden direkt im Kontext der
Entwicklungsumgebung Vorschläge für Code-Snippets, komplette Funktionen und
sogar Tests unterbreitet. Die Integration erfolgt nahtlos in gängige IDEs wie
Visual Studio Code, IntelliJ oder Eclipse. Das zugrunde liegende Modell –
OpenAI Codex – wird mit Kommentaren oder natürlicher Sprache angesteuert.
Typische Anwendungsfelder sind schnelles Prototyping, die Generierung von
Boilerplate-Code und die Unterstützung beim Onboarding neuer Teammitglieder
\cite{donvir_role_2024}. Praxiserfahrungen zeigen, dass Copilot die Entwicklung
– etwa von React-Anwendungen – erheblich beschleunigen kann, indem es gezielte
Codevorschläge für Authentifizierung, Routing oder Formularvalidierung liefert
und bei der Fehlerbehebung unterstützt. Dennoch bleibt eine kritische
Überprüfung der KI-Vorschläge unerlässlich \cite{kerr_github_nodate}.

\textit{TabNine}
ist ein weiteres KI-gestütztes Tool zur Code-Vervollständigung, das
ursprünglich auf GPT-2 basierte und mittlerweile ein eigenes Modell verwendet.
Es generiert Codevorschläge in Echtzeit für zahlreiche Programmiersprachen und
passt sich sukzessive dem Stil der jeweiligen Entwicklerperson an. TabNine
unterstützt alle gängigen Entwicklungsumgebungen und bietet zusätzlich eine
Chat-Funktion für gezielte Code-Fragen. Die Flexibilität durch wahlweise lokale
oder cloudbasierte Modelle wird besonders im Hinblick auf unterschiedliche
Datenschutzanforderungen geschätzt \cite{donvir_role_2024}.

\textit{Cursor AI}
steht für die nächste Generation KI-basierter Entwicklungstools. Es kann auf
Grundlage natürlicher Sprache vollständige Applikationen generieren und nutzt
dabei fortschrittliche Ansätze wie Retrieval-Augmented Generation (RAG) und
Agentic AI. Die Stärke von Cursor AI liegt in der End-to-End-Generierung
kompletter Projekte, was insbesondere für schnelles Prototyping oder den Aufbau
komplexer Softwarelösungen von Vorteil ist \cite{donvir_role_2024}.

\textit{Devin AI}
geht noch einen Schritt weiter und versteht sich als \enquote{AI Software
    Engineer}. Das Tool setzt komplette Softwareprojekte auf Basis natürlicher
Sprache um, bricht Anforderungen in Aufgaben herunter, automatisiert
Testprozesse und erstellt Deployment-Skripte. Besonders hervorzuheben ist die
Fähigkeit von Devin, langfristige Planungen umzusetzen und kontinuierliche
Anpassungen an neue Anforderungen vorzunehmen \cite{donvir_role_2024}.

\vspace{1em}
\noindent
\textbf{Typische Einsatzszenarien}

In der Praxis kommen nach \cite{donvir_role_2024} diese Tools in verschiedenen
Bereichen zum Einsatz:
\begin{itemize}
    \item \textbf{Code-Generierung und Vervollständigung:} Automatisiertes Schreiben von Code, Vorschläge für Funktionen, Klassen oder API-Integrationen.
    \item \textbf{Test- und Debugging-Unterstützung:} Generierung von Unit- und Integrationstests, Erkennung von Fehlern und Vorschläge für Bugfixes.
    \item \textbf{Projekt-Scaffolding und Boilerplate:} Automatisches Erstellen von Grundstrukturen für neue Projekte.
    \item \textbf{End-to-End-Entwicklung:} Vollständige Umsetzung von Projektanforderungen inklusive Deployment-Skripten und CI/CD-Konfiguration.
\end{itemize}

\vspace{1em}
\noindent
\textbf{Vorteile und Grenzen in der Praxis}

Die Integration generativer KI-Tools führt nachweislich zu erheblichen
Zeitersparnissen, konsistenterem Code und einer schnelleren Einarbeitung neuer
Teammitglieder. Gleichzeitig bleiben strukturierte Review-Prozesse und ein
kritischer Umgang mit KI-generierten Vorschlägen unverzichtbar, um Qualitäts-
und Sicherheitsrisiken zu minimieren. Fortgeschrittene Werkzeuge wie Cursor AI
oder Devin AI bieten ein hohes Maß an Automatisierung, sind jedoch häufig
kostenintensiv und nicht in jedem Anwendungsfall ausgereift
\cite{donvir_role_2024}.

% \vspace{1em}
% \noindent
% \textbf{Quellen:}
% \begin{itemize}
%     \item Donvir, A. et al. (2024): \textit{The Role of Generative AI Tools in
%               Application Development: A Comprehensive Review of Current Technologies and
%               Practices} \cite{donvir_role_2024}
%     \item Kerr, K. (2025): \textit{GitHub for Beginners: Building a React App with GitHub
%               Copilot - The GitHub Blog} \cite{kerr_github_nodate}
% \end{itemize}


\subsection{Wichtige Algorithmen und Modelle in der Softwareentwicklung}
% TODO : Quellenangaben für absatz?
% Die Integration generativer Künstlicher Intelligenz (KI) in die
% Softwareentwicklung beruht maßgeblich auf dem Einsatz fortschrittlicher Modelle
% und Algorithmen. Insbesondere Large Language Models (LLMs) und
% Transformer-Architekturen bilden die technologische Grundlage moderner
% Coding-Tools wie GitHub Copilot, Cursor oder v0. Im Folgenden werden zentrale
% Modelle, deren Funktionsweise und Bedeutung für typische Entwicklungsaufgaben
% dargestellt.

\label{sec:wichtige-algorithmen-modelle}

Aktuelle Forschung betont, dass ein zukunftsorientiertes Software-Ökosystem
nicht nur technologische Innovation, sondern auch eine optimierte
Zusammenarbeit von Mensch und KI erfordert. Ein systematischer Umgang mit
technischer Verschuldung sowie die gezielte Nutzung externer Wissensquellen
gelten als Schlüsselfaktoren moderner Entwicklungsframeworks
\cite{matsumoto_conceptual_2021}. Darüber hinaus zeigen empirische Studien,
dass KI-Assistenzsysteme die Codequalität und Wartbarkeit nachweislich
verbessern können \cite{martinovic_impact_2024}. Der Ansatz des sogenannten
\glqq AI-Native Software Engineering\grqq{} (SE 3.0) fokussiert eine enge
Verzahnung von KI, Entwicklerkompetenz und Geschäftsprozessen und steht für
einen Wandel hin zu einer kooperativen, dialogorientierten Softwareentwicklung
\cite{hassan_towards_2024}.

Im Kern basieren moderne KI-Tools wie GitHub Copilot, Cursor oder Bolt auf
sogenannten Large Language Models (LLMs) und verwandten Architekturen wie
Transformers. Diese tiefenlernenden neuronalen Netze werden auf großen Mengen
an Quellcode und natürlicher Sprache trainiert und nutzen Mechanismen wie
Self-Attention, um den Kontext über längere Sequenzen hinweg zu erfassen. So
gelingt es den Modellen, sowohl syntaktisch als auch semantisch komplexe
Strukturen zu erkennen und zu generieren, etwa im Bereich der Code-Logik oder
bei der Anwendung typischer Designmuster \cite{nguyen-duc_generative_2023,
    esposito_generative_2025}. Während Diffusionsmodelle in der Bild- und
Mediengenerierung dominieren, bleiben für text- und codebasierte
Softwareentwicklung weiterhin LLMs und Transformer-Architekturen zentral
\cite{weisz_design_2024}.

In der praktischen Anwendung nutzen moderne Assistenzsysteme typischerweise
spezialisierte Sprachmodelle wie OpenAI Codex, Code Llama oder StarCoder, die
gezielt auf Programmcode vortrainiert wurden. Diese Modelle sind in der Lage,
ausgehend vom Kontext, etwa bestehender Code, Kommentare oder Projekthistorie,
automatisiert neue Code-Abschnitte zu generieren, Fehlerkorrekturen
vorzuschlagen oder passende Testfälle zu erstellen \cite{coutinho_role_2024,
    esposito_generative_2025}. Ihr Funktionsspektrum reicht von der automatischen
Erstellung kompletter Funktionen und Module auf Basis kurzer Beschreibungen in
natürlicher Sprache über die Generierung von Unit-Tests und die Unterstützung
beim Review bis hin zur Entwicklung und Auswahl geeigneter
Softwarearchitekturen oder Design Patterns, insbesondere im projektspezifischen
Kontext \cite{coutinho_role_2024, esposito_generative_2025, donvir_role_2024}.

LLMs übernehmen damit zunehmend Aufgaben, die von der reinen Code-Generierung
bis hin zu komplexeren Anforderungen reichen. Sie erstellen automatisiert
Tests, variieren Testdaten, erkennen typische Fehlerbilder und unterstützen
Entwickler:innen durch Vorschläge zur Architektur oder durch das Übersetzen von
Requirements in konkrete Design-Entwürfe oder Diagramme
\cite{esposito_generative_2025, nguyen-duc_generative_2023}.

Trotz des enormen Potenzials solcher Modelle sind wesentliche Herausforderungen
zu beachten. So können LLMs zwar häufig syntaktisch korrekten Code generieren,
dieser ist jedoch nicht immer inhaltlich passend oder sicher, insbesondere bei
vagen Prompts oder fehlendem Kontext (\textit{Halluzinationen}). Die
Erklärbarkeit und Transparenz der generierten Vorschläge bleibt oftmals
eingeschränkt, da die Entscheidungswege der Modelle schwer nachvollziehbar
sind. Ohne gezieltes Fine-Tuning auf spezifische Projekte oder Domänen bleibt
das Wissen zudem meist allgemein und kann individuellen Anforderungen nicht
immer gerecht werden \cite{esposito_generative_2025,
    nguyen-duc_generative_2023, donvir_role_2024}.

Im Praxisteil dieser Arbeit werden die beschriebenen Modelle durch Tools wie
GitHub Copilot, Cursor und Bolt eingesetzt, um Entwickler:innen in sämtlichen
Phasen der Softwareentwicklung, von Architekturentwurf bis Testing, aktiv zu
unterstützen. Diese Werkzeuge etablieren sich damit zunehmend als kollaborative
Partner und verändern klassische Entwicklungsprozesse nachhaltig
\cite{esposito_generative_2025, nguyen-duc_generative_2023}.



\section{Generative KI-Tools: Funktion und Anwendung}
\label{sec:generative-ki-tools}

Generative KI-Tools wie GitHub Copilot, Cursor oder v0 prägen den modernen
Softwareentwicklungsprozess entscheidend. Ihre Hauptfunktion besteht darin,
natürliche Sprache, sogenannte Prompts, in ausführbaren Code, Testfälle oder
Dokumentationen umzusetzen. Damit verändern sie sowohl technische Workflows als
auch die Zusammenarbeit in Entwicklungsteams und stellen neue Anforderungen an
die Kompetenzen der Beteiligten \cite{weisz_design_2024}.

Die zugrundeliegenden Large Language Models (LLMs) ermöglichen ein
Prompt-basiertes Entwicklungsparadigma: Entwickler:innen beschreiben Aufgaben
in natürlicher Sprache und erhalten daraufhin passende Vorschläge. Diese
erscheinen entweder als \textit{Code Completion} direkt beim Tippen oder als
vollständige Funktionsblöcke \cite{kerr_github_nodate, weisz_design_2024}.
Neuere Ansätze wie SENAI integrieren generative KI von Beginn an in den
Software-Engineering-Prozess und erlauben so eine hochautomatisierte,
KI-zentrierte Entwicklung \cite{saad_senai_2025}.

Systematische Literaturübersichten zeigen, dass die Integration generativer KI
in Entwicklungsumgebungen das Nutzererlebnis, die Akzeptanz und den praktischen
Nutzen maßgeblich beeinflusst. Besonders Faktoren wie Transparenz, Usability
und intelligente Feedbackmechanismen sind entscheidend für den Erfolg im Alltag
\cite{sergeyuk_human-ai_2025}. Die praktische Nutzung generativer KI erfolgt
heute meist direkt über Plugins für etablierte IDEs wie Visual Studio Code oder
JetBrains, oftmals auch über API-Schnittstellen. Dadurch lassen sich Aufgaben
wie Refactoring, Testing oder Dokumentation unmittelbar und nahtlos in die
gewohnte Arbeitsumgebung integrieren \cite{kerr_github_nodate,
    shi_ai-assisted_2023, weisz_design_2024}. Besonders hervorzuheben ist zudem,
dass KI-gestützte Assistenzsysteme neue Teilhabemöglichkeiten für
sehbeeinträchtigte Entwickler:innen eröffnen können, vorausgesetzt, die Tools
sind barrierefrei gestaltet \cite{flores-saviaga_impact_2025}.

Ein typischer Workflow zeigt sich etwa beim Pair Programming mit Copilot:
Aufgaben werden in Form von Prompts gestellt, das Tool generiert passende
Codevorschläge, die geprüft, angepasst oder verworfen werden können. Studien
belegen, dass diese Arbeitsweise Routineaufgaben wie Testautomatisierung,
Refactoring und Dokumentation signifikant beschleunigt
\cite{kerr_github_nodate, weisz_design_2024, shi_ai-assisted_2023}. In agilen
Teams kann generative KI zudem einen Beitrag zur Qualitätsbewertung und zur
Dokumentation von Anforderungen leisten \cite{geyer_case_2025}.

Der Einsatz generativer KI bietet zahlreiche Vorteile, die in aktuellen Studien
hervorgehoben werden: So wird insbesondere die Automatisierung von
Routineaufgaben und die damit verbundene Zeitersparnis hervorgehoben, genauso
wie die Verbesserung der Codequalität durch die Erkennung häufiger Fehler und
gezielte Empfehlungen für Best Practices. Außerdem werden niedrigere
Einstiegshürden für weniger erfahrene Entwickler:innen genannt, denen durch
kontextbasierte Vorschläge der Zugang zur Entwicklung erleichtert wird
\cite{donvir_role_2024, sergeyuk_human-ai_2025}. Darüber hinaus fördern
KI-gestützte Code Reviews die Kollaboration zwischen Mensch und Maschine und
erhöhen die Transparenz im Entwicklungsprozess \cite{alami_human_2025}. Durch
kontinuierliche Modellverbesserung und flexible Integration in unterschiedliche
Projekte kann das Optimierungspotenzial weiter gesteigert werden
\cite{kerr_github_nodate, weisz_design_2024}.

Trotz dieser Vorteile bestehen zentrale Herausforderungen. Ein häufig
diskutiertes Problem sind sogenannte \textit{Halluzinationen}: Modelle
generieren zwar syntaktisch korrekten, inhaltlich aber fehlerhaften oder
unsicheren Code, besonders bei unscharfen oder vagen Prompts
\cite{shi_ai-assisted_2023, weisz_design_2024}. Hinzu kommen Bias und
Kontextdefizite, also die Übernahme von Vorurteilen aus den Trainingsdaten oder
das Nichtbeachten projektspezifischer Regeln. Ein weiteres Risiko besteht in
der Generierung unsicheren Codes, etwa durch Vorschläge mit hardcodierten
Zugangsdaten, die explizit überprüft werden müssen. Zudem beobachten Studien
ein übermäßiges Vertrauen vieler Entwickler:innen in die Vorschläge der KI,
eine unkritische Übernahme ohne manuelle Überprüfung kann zu schwerwiegenden
Fehlern führen \cite{shi_ai-assisted_2023, weisz_design_2024}.

Um einen produktiven und sicheren Einsatz zu gewährleisten, empfiehlt die
Literatur spezifische Designprinzipien für generative KI-Tools
\cite{weisz_design_2024}: Erstens sollte die Gestaltung der Systeme so
erfolgen, dass Nutzer:innen die Funktionsweise und Grenzen nachvollziehen
können (\textit{Design for Mental Models}). Zweitens müssen Feedbackmechanismen
sowohl Vertrauen fördern als auch zur kritischen Prüfung anregen
(\textit{Design for Appropriate Trust \& Reliance}). Drittens ist die
Berücksichtigung von Fehlern essenziell: Tools sollten aktiv auf mögliche
Fehler hinweisen und die Nutzer:innen zur Korrektur befähigen (\textit{Design
    for Imperfection}). Die Gestaltung generativer GUIs und Entwicklungsprozesse
muss diese Prinzipien berücksichtigen, um eine nachhaltige und
verantwortungsvolle Integration zu ermöglichen \cite{lee_towards_2025,
    chen_genui_2025, gill_agile_2025}. Flexibilität, Feedback und kontinuierliche
Anpassung sind hier Schlüsselfaktoren.

Insgesamt zeigt sich, dass der Mehrwert generativer KI-Tools vor allem dann zum
Tragen kommt, wenn sie sinnvoll in bestehende Entwicklungsumgebungen
integriert, kritisch überprüft und auf die jeweiligen Team- und
Projektanforderungen angepasst werden.

% Optional: Kurzbezug auf Kapitel 3 (Praxisbeispiel), falls erwünscht
% Wie in Kapitel 3 praktisch demonstriert, lassen sich diese Prinzipien in der Entwicklung von React Native-Anwendungen mit KI-Tools exemplarisch umsetzen.




\chapter{Praktische Demonstration}

\section{Zielsetzung und Vorgehen}

Die Zielsetzung dieses Kapitels besteht darin, den Einsatz generativer KI-Tools
in der Softwareentwicklung nicht nur theoretisch zu betrachten, sondern anhand
eines konkreten, praxisnahen Beispiels zu evaluieren. Im Zentrum steht die
Implementierung eines interaktiven Map-Screens für die mobile App „Locals“. Das
Vorgehen umfasst die iterative Umsetzung dieses Features mithilfe mehrerer
moderner KI-gestützter Entwicklungstools, um deren Stärken, Schwächen und das
Entwicklererlebnis unter realen Bedingungen vergleichend zu analysieren.

Die Auswahl der eingesetzten Tools orientiert sich an aktuellen Empfehlungen
aus Wissenschaft und Praxis. Donvir et~al.~\cite{donvir_role_2024} zeigen in
ihrem Überblick, dass GitHub Copilot, Cursor und Bolt zu den
fortschrittlichsten und am weitesten verbreiteten generativen
KI-Entwicklungsumgebungen zählen und sich besonders für vergleichende Studien
in der Softwareentwicklung eignen. Auch Rogachev~\cite{rogachev_my_nodate}
beschreibt praxisnah die Vorteile und Grenzen von Copilot-Agenten bei der
Entwicklung von React Native-Anwendungen.

\subsubsection{Motivation für das praktische Beispiel}

Die Entscheidung, als praktisches Beispiel die Entwicklung einer interaktiven
Kartenansicht zu wählen, beruht auf mehreren Überlegungen:
\begin{itemize}
      \item Die Map-View ist ein zentrales und anspruchsvolles Feature moderner Event- und
            Social-Apps und erfordert die Integration unterschiedlicher Technologien (u. a.
            Geolocation, Datenmanagement, UI/UX-Design, Filter- und Suchfunktionen).
      \item Die Aufgabe vereint sowohl klassische Herausforderungen der
            Frontend-Entwicklung (State-Management, UI-Logik) als auch typische
            Stolpersteine in der Zusammenarbeit mit externen Libraries (z. B.
            \texttt{react-native-maps}, Google Maps API).
      \item Durch die Komplexität und Vielschichtigkeit eignet sich das Beispiel besonders
            gut, um die Leistungsfähigkeit generativer KI-Tools im realen
            Entwicklungsprozess kritisch zu beleuchten.
      \item Das Feature ist in modernen Event-Apps allgegenwärtig und der Nutzen für
            Nutzer:innen unmittelbar erlebbar.
\end{itemize}
Die Auswahl des Map-Screens als Demonstrationsobjekt ermöglicht somit einen fundierten und praxisnahen Einblick in die Potenziale und Grenzen generativer KI im täglichen Entwickleralltag.

\subsubsection{Eingesetzte KI-Tools}

Für die Implementierung des Map-Screens wurden folgende KI-gestützte
Entwicklungstools eingesetzt:

\begin{itemize}
      \item \textbf{GitHub Copilot:} KI-basierter Code-Assistent mit kontextsensitiver Echtzeit-Code-Vervollständigung, automatischen Codevorschlägen für Funktionen, Tests und Dokumentation. Bietet einen integrierten Chat für Refactoring, Hilfestellungen und Testgenerierung, Pull-Request-Zusammenfassungen sowie projektübergreifende Kontextfunktionen (z.B. Copilot Spaces und Wissensdatenbanken). Unterstützt die gängigen IDEs und arbeitet kontinuierlich im Hintergrund, um den gesamten Entwicklungsprozess zu begleiten~\cite{github_copilot_2025}.
      \item \textbf{Cursor:} Spezialisierter KI-Code-Editor mit Funktionen wie Multi-Line Edits, Smart Rewrites, Tab-Kommandos zur schnellen Navigation durch Änderungsvorschläge sowie vollständige Indexierung und kontextabhängige Analyse des gesamten Repos. Der Agent Mode erlaubt Prompt Chaining, Terminalbefehle, Fehlerdiagnose und komplette Aufgabenabwicklung, unterstützt durch ein dialogorientiertes Chat-Interface. Cursor ist plattformübergreifend (inkl. Web/Mobile) einsetzbar~\cite{cursor_welcome_2025}.
      \item \textbf{Bolt.new:} Cloudbasierte, browserbasierte Entwicklungsumgebung ohne lokalen Setup-Bedarf. Ermöglicht die Erstellung und das Deployment von Web- und Mobile-Anwendungen direkt aus Prompts heraus. Unterstützt Multi-Plattform-Development mit automatischer Paket-Installation, Live-Code-Bearbeitung und direkter Anbindung an GitHub. Bolt bietet ein tokenbasiertes Preismodell sowie Funktionen für Kollaboration und Teamarbeit~\cite{bolt_support_2025}.
\end{itemize}

Die Auswahl dieser Tools ermöglicht eine umfassende Betrachtung verschiedener
Ansätze generativer KI in der Softwareentwicklung – vom klassischen Pair
Programming bis zur cloudbasierten Komplettlösung.



\section{Vorstellung der App \glqq Locals\grqq}

\subsection{Architektur und technischer Stack}

Die App \textit{Locals} ist eine mobile Plattform, die Nutzer:innen ermöglicht,
lokale Events zu entdecken, zu erstellen und zu verwalten. Sie richtet sich an
ein junges, urbanes Publikum und unterstützt soziale Interaktionen rund um
Veranstaltungen.

Die modulare Architektur setzt auf moderne, plattformübergreifende
Technologien:

% TODO: einheitlich schreiben mit punkt oder ohne generell überall?
\begin{itemize}
    \item \textbf{Frontend:} React Native, TypeScript, Expo (für schnelle, konsistente Entwicklung auf iOS/Android).
    \item \textbf{Backend:} Firebase (Authentifizierung, Datenhaltung, Synchronisation).
    \item \textbf{Navigation:} \texttt{@react-navigation/native}, \texttt{expo-router}.
    \item \textbf{State-Management:} Eigene Context-Provider (\texttt{AuthProvider}, \texttt{EventsProvider}).
    \item \textbf{UI:} \texttt{@expo/vector-icons}, \texttt{lucide-react-native}.
    \item \textbf{Karten/Location:} \texttt{react-native-maps}, \texttt{expo-location}.
\end{itemize}

Die Anwendung ist in drei Hauptbereiche gegliedert:
\begin{itemize}
    \item \textbf{Explore-Screen:} Event-Feed nach Standort/Interesse.
    \item \textbf{Map-Screen:} Interaktive Kartenansicht mit Event-Markern und Filtern.
    \item \textbf{Profil-Screen:} Übersicht und Verwaltung eigener Events/Profile.
\end{itemize}

Der Map-Screen wird in dieser Arbeit als prototypisches, KI-gestütztes Feature
exemplarisch entwickelt und dient als Grundlage für die weitere Analyse.

\subsubsection{Einstiegskomponente und Navigation}

Das zentrale \texttt{RootLayout} lädt Fonts, Authentifizierungs- und
Event-Kontexte und steuert die Nutzerführung abhängig vom Login-Status. Die
Navigation erfolgt strikt zustandsbasiert (\texttt{useAuth},
\texttt{useSegments}, \texttt{useRouter}). Dieses Architekturprinzip
gewährleistet eine saubere Trennung von Zustandsverwaltung und Routing, was
spätere Erweiterungen und Integrationen (z.B. von KI-gestützten Funktionen)
erleichtert.

\subsection{Bestehende Funktionalitäten}

Bereits vor der KI-Integration implementiert sind:
\begin{itemize}
    \item \textbf{Benutzerauthentifizierung:} Firebase Authentication.
    \item \textbf{Profilverwaltung:} Eigene und besuchte Events.
    \item \textbf{Eventverwaltung:} Anlegen, Bearbeiten, Löschen.
    \item \textbf{Tab-Navigation:} Zwischen Explore, Map, Profil.
    \item \textbf{Responsives Design:} Einheitliche Darstellung auf allen Endgeräten.
\end{itemize}

Die modulare Struktur und klare Abgrenzung der App-Komponenten bieten die
ideale Basis, um generative KI-Tools gezielt einzusetzen und deren Auswirkungen
im Entwicklungsprozess zu evaluieren.

\section{Implementierung der interaktiven Kartenansicht mit KI-Unterstützung}
% \subsection{Integration der KI-Tools}

\subsection{Demonstration mit GitHub Copilot}

\subsubsection{Setup und Vorgehen}
Die Entwicklung des Map-Screens wurde exemplarisch mit \textbf{GitHub Copilot}
in Visual Studio Code durchgeführt. Für größtmögliche Vergleichbarkeit kamen
ausschließlich die Copilot-Funktionen zum Einsatz, keine weiteren KI-Plugins.
Ziel war die Umsetzung eines Map-Screens mit Event-Markern, Filterfunktion
sowie Detailansicht – sämtliche Eventdaten stammen dabei aus dem zentralen
EventsProvider.

\begin{figure}[htbp]
      \centering
      \includegraphics[width=1\textwidth]{images/copilot_screenshots/Screenshots Ist-Zustand-copilot.png}
      \caption{Ausgangszustand der Anwendung vor dem Einsatz von Copilot. \textit{Copilot-Demo}}
      \label{fig:copilot-istzustand}
\end{figure}

Vor Beginn der Implementierung wurde Copilot aktiviert und die
Entwicklungsumgebung vorbereitet (z.~B.\ Installation von
\texttt{react-native-maps} und \texttt{expo-location}). Die Aufgabenstellung
für Copilot wurde jeweils als präziser Kommentar oder Docstring formuliert.
Beispielsweise:

\begin{lstlisting}[language=HTML]
// Create a React Native component for a map with event markers and filter functionality
\end{lstlisting}

Oder als umfangreicher Prompt:
\begin{lstlisting}[language=HTML]
// Create a Map Screen with Event Markers and Filter in React Native. Requirements:
//
// - Use event data from the EventsProvider context
// - Show markers, enable filtering, display event details in modal/callout
// - Clean, modular, and maintainable code, with modern UI/UX
\end{lstlisting}

% \begin{figure}[htbp]
%       \centering
%       \includegraphics[width=0.55\textwidth]{images/copilot_screenshots/rückmeldung nach 1. prompt eingabe-copilot.png}
%       \caption{Reaktion von GitHub Copilot auf den ersten Prompt zur Implementierung des MapScreens. \textit{Copilot-Demo}}
%       \label{fig:copilot-erster-prompt}
% \end{figure}

\begin{figure}[htbp]
      \centering
      \begin{minipage}{0.48\textwidth}
            \centering
            \includegraphics[width=0.98\textwidth]{images/copilot_screenshots/implementation-rückmeldung-copilot-1.png}
      \end{minipage}
      \hfill
      \begin{minipage}{0.48\textwidth}
            \centering
            \includegraphics[width=0.98\textwidth]{images/copilot_screenshots/implementation-rückmeldung-copilot-2.png}
      \end{minipage}
      \caption{Rückmeldung und Hinweise von Copilot während der Implementierung. \textit{Copilot-Demo}}
      \label{fig:copilot-impl-pair}
\end{figure}

\subsubsection{Schrittweise Umsetzung und Reflexion}

Die Entwicklung erfolgte nach folgendem Muster:
\begin{enumerate}
      \item \textbf{Prompt definieren:} Pro Feature (z.\,B. Marker, Filter, Event-Details) wurde ein spezifischer Kommentar als Arbeitsanweisung eingefügt.
      \item \textbf{Vorschläge von Copilot akzeptieren oder anpassen:} Vorschläge wurden Schritt für Schritt übernommen, angepasst oder verworfen.
      \item \textbf{Test und Dokumentation:} Nach jeder Änderung wurde der Code getestet und die Funktionsweise reflektiert.
      \item \textbf{Fehlersuche und Nacharbeit:} Fehlerhafte Vorschläge oder Bugs wurden nach Rücksprache mit Copilot, durch Google-Suche oder manuelle Nacharbeit behoben.
\end{enumerate}

\textbf{Beispiel:}
Zur Erstellung des MapScreens wurde folgender Prompt verwendet (gekürzt):

\begin{lstlisting}[language=HTML]
// Create a React Native component called `MapScreen` that displays event markers on a map using event data from the `EventsProvider` context. Requirements:
//
// - Show all events as markers on a map
// - When a marker is tapped, show details
// - Add filter options above the map
// - Handle loading/error states appropriately
\end{lstlisting}

Die technische Umsetzung umfasste:
\begin{itemize}
      \item Anzeige aller Events als Marker (mit Kategorie und Titel) auf der Karte
            (\texttt{react-native-maps}).
      \item Umsetzung einer Filterleiste zur Auswahl nach Kategorie und Datum.
      \item Darstellung von Event-Details beim Tippen auf einen Marker (Callout oder
            Modal).
      \item Responsives Layout und modernes UI-Design nach Vorgabe.
      \item Fehler- und Ladezustände wurden entsprechend behandelt.
\end{itemize}

\begin{figure}[htbp]
      \centering
      \begin{minipage}{0.48\textwidth}
            \centering
            \includegraphics[width=\textwidth]{images/copilot_screenshots/6. 1. Version das MapScreens - Screenshot-copilot.png}
            \caption{Erste lauffähige Version des MapScreens nach KI-gestützter Entwicklung. \textit{Copilot-Demo}}
      \end{minipage}
      \hfill
      \begin{minipage}{0.48\textwidth}
            \centering
            \includegraphics[width=\textwidth]{images/copilot_screenshots/Callouts+modal-copilot.png}
            \caption{Event-Details und Callouts auf dem MapScreen. \textit{Copilot-Demo}}
            \label{fig:copilot-callouts}
      \end{minipage}
\end{figure}

Die Code-Generierung erfolgte modular und meist nachvollziehbar. Beispiel:
Copilot erstellte automatisch das Grundgerüst einer Map-Komponente, ergänzte
dann Schritt für Schritt die Logik für Marker, Filter und Event-Details.

Die in dieser Arbeit beobachteten Stärken und Schwächen von GitHub Copilot
decken sich mit den Ergebnissen anderer Studien. So berichtet
Kerr~\cite{kerr_github_nodate}, dass Copilot die Entwicklung von
React-Anwendungen deutlich beschleunigen kann, jedoch eine kritische
Überprüfung der Vorschläge unerlässlich bleibt. Weisz
et~al.~\cite{weisz_design_2024} betonen zudem, dass die Gestaltung der
Interaktion und Feedbackmechanismen einen maßgeblichen Einfluss auf die
Akzeptanz der KI-gestützten Codegenerierung hat.

\subsubsection{Herausforderungen und Beobachtungen}
Im Verlauf der Entwicklung zeigte Copilot verschiedene Stärken und Schwächen:

\textbf{Stärken:}
\begin{itemize}
      \item Effiziente Generierung von Boilerplate-Code und wiederkehrenden Patterns
            (z.\,B. Hook-Nutzung, State-Management).
      \item Schnelle Vorschläge für UI-Komponenten und einfache Interaktionslogik.
      \item Erkennung von einfachen Fehlern und automatische Typanpassung nach Änderungen
            (z.\,B. bei Änderung der Datenstruktur für Kategorien).
\end{itemize}

\textbf{Schwächen und typische Fehlerquellen:}
\begin{itemize}
      \item Teilweise fehlerhafte oder veraltete Import-Pfade (z.\,B. bei Kategorien).
      \item Missverständnisse bei nicht exakt spezifizierten Datenstrukturen (z.\,B.
            Umwandlung von Kategorien-Array von String zu Objekt wurde erst nach manuellem
            Eingriff korrekt erkannt).
      \item Filter-Logik für den ``All''-Filter funktionierte zunächst nicht wie erwartet;
            Copilot schlug Anpassungen vor, die jedoch neue Fehler erzeugten (zwei
            ``All''-Filter, Duplicate-Key-Warning).
      \item In mehreren Iterationen blieb die Korrektur von Spezialfällen (wie
            Filter-Probleme oder Typkonflikte) hinter den Erwartungen zurück, sodass
            letztlich manuelle Nachbesserung notwendig wurde.
\end{itemize}

\textbf{Reflexion:}
\begin{itemize}
      \item Die Vorschläge waren bei Standardaufgaben meist brauchbar (\textit{Subjektive
                  Zufriedenheit: 4/5}), bei komplexeren State- oder Typ-Logiken aber oft
            unvollständig.
      \item Die Interaktion mit Copilot war intuitiv, erfordert jedoch genaue Prompts und
            ein grundsätzliches Verständnis für die Implementierung, um Fehlerquellen zu
            erkennen und zu korrigieren.
      \item Bei UI-Details oder individuelleren Anforderungen blieb Nacharbeit
            unerlässlich.
\end{itemize}

\textbf{Fazit:}
Copilot ist ein sehr leistungsfähiges Assistenz-Tool, das Routinearbeiten und Standardaufgaben erheblich beschleunigt. Bei komplexeren oder individuelleren Anforderungen stößt es jedoch an Grenzen, sodass eine kritische Prüfung und manuelle Nacharbeit weiterhin unverzichtbar bleibt. Die Demonstration belegt, dass Copilot einen relevanten Effizienzgewinn für erfahrene Entwickler:innen bieten kann, den Anspruch auf vollständige Automatisierung jedoch (noch) nicht erfüllt.

\subsection{Demonstration mit Cursor}

\subsubsection{Setup und Vorgehen}
Für die Entwicklung des Map-Screens wurde \textbf{Cursor} als spezialisierte
KI-basierte Entwicklungsumgebung genutzt (Branch: \texttt{cursor},
Sprachmodell: Claude 3.7 Sonnet, Agent mode). Cursor ermöglicht Prompt
Chaining, die Verarbeitung von Screenshots als Referenz und einen
dialogbasierten Entwicklungsprozess.

\begin{figure}[htbp]
      \centering
      \includegraphics[width=1\textwidth]{images/cursor_screenshots/Screenshots Ist-Zustand-cursor.png}
      \caption{Ausgangszustand der Anwendung vor Einsatz von Cursor. \textit{Cursor Demo}}
      \label{fig:cursor-istzustand}
\end{figure}

Zu Beginn wurden Screenshots des aktuellen App-Zustands sowie relevanter
Komponenten (u.\,a. \texttt{\_layout.tsx}, Event Provider) als Kontext
bereitgestellt. Der Hauptprompt für den Einstieg lautete:

\begin{lstlisting}[language=HTML]
// Create a Map Screen with Event Markers and Filter in React Native. Requirements:
//
// - Use event data from the EventsProvider context
// - Show markers, enable filtering, display event details in modal/callout
// - Clean, modular, and maintainable code, with modern UI/UX 
// - Use provided screenshots/layouts as visual reference
\end{lstlisting}

\subsubsection{Schrittweise Umsetzung und Reflexion}

Der Entwicklungsprozess war durch mehrere Besonderheiten gekennzeichnet:

\begin{enumerate}
      \item \textbf{Prompt Chaining und Screenshot-Kontext:} Zu jedem Entwicklungsschritt wurden gezielt neue Prompts mit aktualisierten Anforderungen und Referenz-Screenshots gestellt.
      \item \textbf{Terminal-Steuerung:} Cursor führte notwendige Terminalbefehle (z.\,B. Paketinstallationen) eigenständig aus und dokumentierte Fehlermeldungen sowie Lösungsvorschläge direkt im Chat.
      \item \textbf{Debugging und Package-Kompatibilität:} Cursor identifizierte eigenständig Kompatibilitätsprobleme, z.\,B. bei der \texttt{react-native-maps}-Version (\^1.24.3 statt \^1.18.0), und schlug proaktiv eine Anpassung auf die funktionierende Version vor.
\end{enumerate}

\begin{figure}[htbp]
      \centering
      \begin{minipage}{0.48\textwidth}
            \centering
            \includegraphics[width=0.98\textwidth]{images/cursor_screenshots/(NOBRIDGE) ERROR-cursor.png}
      \end{minipage}
      \hfill
      \begin{minipage}{0.48\textwidth}
            \centering
            \includegraphics[width=0.98\textwidth]{images/cursor_screenshots/Cursor führt terminal befehle eigenständig aus.png}
      \end{minipage}
      \caption{Typische Fehlermeldung und autonomes Ausführen von Terminalbefehlen durch Cursor beim Einrichten des MapScreens. \textit{Cursor-Demo}}
      \label{fig:cursor-error-terminal}
\end{figure}

\begin{enumerate}[resume]
      \item \textbf{Iterative Korrekturen und UX-Verbesserungen:} Bei UI-Problemen (z.\,B. überlagernde Filter/Buttons) wurden nach Rückmeldung gezielt Layout-Vorschläge unterbreitet.
      \item \textbf{Feature-Integration:} Funktionen wie Filter, Refresh-Button und Navigation zu Event-Standorten wurden auf Nachfrage oder eigenständig ergänzt.
\end{enumerate}

\begin{figure}[htbp]
      \centering
      \includegraphics[width=0.48\textwidth]{images/cursor_screenshots/erster durchgang-cursor.png}
      \caption{Erste Umsetzungsschritte nach Bereitstellung des Kontexts und initialem Prompt. \textit{Cursor-Demo}}
      \label{fig:cursor-erster-durchgang}
\end{figure}

\textbf{Besonders positiv fiel auf:}
\begin{itemize}
      \item Cursor war bei der Behebung von Package-Fehlern und bei der automatischen
            Adaption von Code an neue Datenstrukturen (z.\,B. Kategorien als Objekte statt
            Strings) sehr präzise.
      \item Im Vergleich zu Copilot wurde die grundlegende Kartenfunktion schneller
            funktionsfähig, auch wenn die erste Map-Anzeige erst nach mehreren Prompts
            erschien.
      \item Cursor dokumentierte seine Debugging-Schritte transparent und schlug auch
            Lösungen für übersehene Fehlerquellen vor.
\end{itemize}

\begin{figure}[htbp]
      \centering
      \begin{minipage}{0.48\textwidth}
            \centering
            \includegraphics[width=0.98\textwidth]{images/cursor_screenshots/final-mapscreen-cursor-3.png}
      \end{minipage}
      \hfill
      \begin{minipage}{0.48\textwidth}
            \centering
            \includegraphics[width=0.98\textwidth]{images/cursor_screenshots/final-mapscreen-cursor-2.png}
      \end{minipage}
      \caption{MapScreen in der finalen Implementierung mit Cursor – zwei verschiedene Zustände/Ansichten. \textit{Cursor-Demo}}
      \label{fig:cursor-finalpair}
\end{figure}

\subsubsection{Herausforderungen und Learnings}
\begin{itemize}
      \item \textbf{Versionierung und Abhängigkeiten:} Kompatibilitätsprobleme zwischen \texttt{react-native-maps} und Expo führten zu Fehlern, die erst nach mehreren Iterationen und Prompts gelöst wurden.
      \item \textbf{Automatische Code-Generierung:} Cursor wechselte in einem Schritt das Map-Framework (von \texttt{react-native-maps} auf \texttt{react-native-webview}), was unerwünscht war und manuell rückgängig gemacht wurde.
      \item \textbf{Filterfunktion:} Die Behandlung von Kategorie-Filtern (\texttt{All} vs. \texttt{all}) führte zu denselben Herausforderungen wie bei Copilot. Allerdings erkannte Cursor den Fehler nach gezieltem Prompt korrekt und schlug eine funktionierende Lösung vor.
      \item \textbf{Erkennung von Typfehlern:} Cursor reagierte auf TypeErrors (Objekte statt Strings in den Kategorien) konsistent und ergänzte die notwendigen Anpassungen selbstständig.
\end{itemize}

\textbf{Reflexion:}
\begin{itemize}
      \item Die Entwicklung mit Cursor verlief insgesamt sehr zügig, da
            Kontextinformationen (Screenshots, Codeausschnitte) effektiv genutzt wurden.
      \item Die Vorschläge für komplexe UI- und Layout-Probleme waren oft präziser und
            anpassungsfähiger als bei Copilot.
      \item Der dialogische Ablauf mit Feedback-Loops und Prompt Chaining war besonders für
            iteratives Refactoring und das Lösen komplexerer Zusammenhänge hilfreich.
      \item Bei seltenen „KI-Fehlinterpretationen“ (z.B. Austausch ganzer Packages) war
            weiterhin manuelle Kontrolle nötig.
\end{itemize}

\textbf{Fazit:}
Cursor bewährt sich vor allem durch die Fähigkeit, Kontext (Screenshots, Code, Fehlermeldungen) aktiv in die Entwicklung einzubinden. Im Vergleich zu Copilot zeigte Cursor bei Debugging, Package-Fehlern und iterativen Verbesserungen eine hohe Präzision und Transparenz.

\subsection{Demonstration mit Bolt}

\subsubsection{Setup und Vorgehen}
Für die Entwicklung des Map-Screens wurde das KI-Assistenztool
\textbf{Bolt.new} eingesetzt. Bolt ermöglichte dabei den direkten Zugriff auf
das bestehende \texttt{Locals}-GitHub-Repository und bot eine integrierte
Umgebung für Prompt Chaining und Live-Code-Editing. Ein dedizierter Branch
(\texttt{bolt}) wurde verwendet. Das zugrundeliegende Sprachmodell war Claude
3.7 Sonnet (nicht direkt auswählbar).

\begin{figure}[htbp]
      \centering
      \begin{minipage}{0.48\textwidth}
            \centering
            \includegraphics[width=0.98\textwidth]{images/bolt_screenshots/startseite-what-do-you-wanna-build-mit-github-branch.png}
      \end{minipage}
      \hfill
      \begin{minipage}{0.48\textwidth}
            \centering
            \includegraphics[width=0.98\textwidth]{images/bolt_screenshots/ der server funktioniert und bolt installiert die restlichen dependencies. scheinbar wird react-native-web benötigt um die laufende app zu sehen.png}
      \end{minipage}
      \caption{Links: Start mit Bolt.new und Auswahl des Locals-Repos. Rechts: Bolt erkennt fehlende Dependencies und installiert diese selbstständig. \textit{bolt-Demo}}
      \label{fig:bolt-setup}
\end{figure}

Zunächst wurden Screenshots des aktuellen App-Zustands sowie zentrale
Komponenten (u.\,a. \texttt{\_layout.tsx}, Event Provider) als Kontext
bereitgestellt. Die Aufgabenstellung für Bolt wurde als umfangreicher Prompt
formuliert:

\begin{lstlisting}[language=HTML]
// Create a Map Screen with Event Markers and Filter in React Native.
// Create a React Native component called MapScreen that displays event markers on a map using event data from the EventsProvider context.
// Requirements:
//
// - Use the list of events from the EventsProvider context
// - Display all events as markers on a map
// - Add filter options above the map (by category, date, distance)
// - When a marker is tapped, show a callout or modal with event details
// - Modern mobile UI, modular and maintainable code
// - Use provided screenshots as visual reference
\end{lstlisting}

\subsubsection{Schrittweise Umsetzung und Reflexion}
Die Besonderheit bei Bolt lag im engen Zusammenspiel mit GitHub, den
automatisch ausführbaren Terminalbefehlen sowie der Möglichkeit, nativ Pakete
zu installieren und Fehler im laufenden Betrieb zu beheben.

\begin{figure}[htbp]
      \centering
      \begin{minipage}{0.48\textwidth}
            \centering
            \includegraphics[width=0.98\textwidth]{images/bolt_screenshots/eingabe des prompts-rueckmeldung von bolt-1.png}
      \end{minipage}
      \hfill
      \begin{minipage}{0.48\textwidth}
            \centering
            \includegraphics[width=0.98\textwidth]{images/bolt_screenshots/eingabe des prompts-rueckmeldung von bolt-2.png.png}
      \end{minipage}
      \caption{Bolt reagiert interaktiv auf Prompts, führt Terminalbefehle aus und gibt strukturiertes Feedback im Interface. \textit{bolt-Demo}}
      \label{fig:bolt-prompts}
\end{figure}

\textbf{Ablauf:}
\begin{enumerate}
      \item \textbf{Repository-Anbindung und Initialisierung:} Über die GitHub-Integration wurde direkt auf das Locals-Repo zugegriffen, ein neuer Branch erstellt und Bolt konnte sämtliche Projektdaten einsehen.
      \item \textbf{Prompt Chaining und Kontextgabe:} Für jede Aufgabe wurden Prompts mit Screenshots und Codeausschnitten ergänzt, etwa zur Installation fehlender Abhängigkeiten wie \texttt{react-native-web}.
      \item \textbf{Automatisiertes Debugging:} Terminalbefehle wie \texttt{npm run dev} wurden selbständig ausgeführt, Fehler wie inkompatible Packages oder fehlende Dependencies eigenständig erkannt und (teilweise) gelöst.
      \item \textbf{Feature-Integration:} Bolt erstellte zentrale Komponenten (\texttt{EventsProvider}, \texttt{EventMarker}, \texttt{MapFilters}) und aktualisierte \texttt{map.tsx} und \texttt{map.web.tsx} für mobile und Web.
      \item \textbf{Multi-Plattform-Support:} Bei Problemen mit \texttt{react-native-maps} auf Web wurde automatisch auf \texttt{react-google-maps} gewechselt und eine alternative Map-Implementierung für Web ergänzt.
      \item \textbf{Fehler-Handling und Limits:} Bei aufwendigen Operationen wurde das Tageslimit des kostenlosen Bolt-Plans schnell erreicht, was ein Upgrade auf Pro erforderte.
\end{enumerate}

\begin{figure}[htbp]
      \centering
      \begin{minipage}{0.48\textwidth}
            \centering
            \includegraphics[width=0.98\textwidth]{images/bolt_screenshots/funktionierender map screen.png}
      \end{minipage}
      \hfill
      \begin{minipage}{0.48\textwidth}
            \centering
            \includegraphics[width=0.98\textwidth]{images/bolt_screenshots/funktionierender map screen + callout & filter angewendet.png}
      \end{minipage}
      \caption{Finale Web-Umsetzung: Interaktiver MapScreen mit Event-Details, Filteroptionen und Callouts. \textit{bolt-Demo}}
      \label{fig:bolt-final}
\end{figure}

\textbf{Herausforderungen und Learnings:}
\begin{itemize}
      \item Bolt war in der Lage, das Setup und viele Standard-Probleme selbständig zu
            lösen und bot zudem intuitive Web-Deployments.
      \item Die Filterlogik, insbesondere der Kategorie-Filter, wurde als einziges KI-Tool
            auf Anhieb korrekt umgesetzt.
      \item Komplexere Fehler (z.\,B. inkompatible native Packages bei mobilen Builds,
            Firebase-Probleme auf iOS) konnten von Bolt erkannt, aber nicht nachhaltig
            gelöst werden.
      \item Die parallele Unterstützung für mobile und Web führte zu umfangreichen
            Anpassungen und wiederkehrenden Fehlern, insbesondere bei der Koordination von
            Dependencies.
      \item Das Token- und Tageslimit der Free-Version wurde durch wiederholte Fehlersuche
            und Build-Versuche schnell ausgereizt.
\end{itemize}

\textbf{Reflexion:}
\begin{itemize}
      \item Bolt eignet sich besonders gut für grüne Wiese-Projekte, kleine neue Repos oder
            als Unterstützung bei der Initialisierung und Standardisierung. Die direkte
            GitHub-, Stripe- und Supabase-Integration sowie das Web-Deployment sind hier
            besonders hilfreich.
      \item Bei größeren, bereits bestehenden Projekten stößt Bolt aktuell jedoch an seine
            Grenzen: Zwar konnte ein funktionierender Map-Screen für das Web erstellt
            werden, für mobile Builds blieben die Fehler jedoch bestehen.
      \item Die Fehlererkennung und automatische Problemlösung war überzeugend, für
            nicht-triviale Projekte bleibt jedoch manuelle Nacharbeit und kritisches Review
            unerlässlich.
      \item Die Web-App ist insgesamt modern und intuitiv gestaltet.
\end{itemize}

\textbf{Fazit:}
Bolt kann den Entwicklungsprozess – besonders in neuen Projekten – signifikant beschleunigen und standardisieren. Bei komplexeren Setups oder plattformübergreifenden Anforderungen treten jedoch noch Limitierungen auf, die nicht ohne weiteres automatisch gelöst werden können.

\subsection{Vergleich und Bewertung der eingesetzten KI-Tools}

Nach der Durchführung der Demonstrationen mit Copilot, Cursor und Bolt lassen
sich deutliche Unterschiede und Gemeinsamkeiten in Bezug auf Funktionalität,
Effizienz, Entwicklererlebnis und Ergebnisqualität feststellen.

\subsubsection{Direkter Vergleich}

\begin{itemize}
      \item \textbf{Copilot} überzeugte besonders bei Standardaufgaben und bewährten Patterns in der Codegenerierung. Die Effizienzsteigerung bei Routinearbeiten ist erheblich, bei komplexeren Aufgaben oder individuellen Anforderungen blieb jedoch manuelle Nacharbeit nötig.
      \item \textbf{Cursor} ermöglichte durch den Kontextbezug (Screenshots, Code, Fehlermeldungen) und das dialogische Prompt Chaining eine besonders zielgerichtete und schnelle Entwicklung. Die Fehlerdiagnose und Lösungsvorschläge waren präziser als bei Copilot, allerdings erforderte auch Cursor für Spezialfälle aktive Begleitung durch die Entwickler:in.
      \item \textbf{Bolt} überzeugte mit seiner umfassenden Plattformintegration (z.B. GitHub, Web, App Store) und der Möglichkeit, Projekte direkt aus der Cloud-IDE zu initialisieren und zu deployen. Besonders im Web-Kontext zeigte Bolt große Stärken, stieß jedoch bei der mobilen Entwicklung und beim refactoring bestehender, größerer Projekte an Grenzen.
\end{itemize}

Die im Praxisteil gewonnenen Erkenntnisse stehen im Einklang mit aktuellen
Studien, die Effizienzgewinne und Qualitätsverbesserungen durch generative
KI-Tools nachweisen. Coutinho et~al.~\cite{coutinho_role_2024} berichten in
einer Pilotstudie von einer deutlichen Zeitersparnis bei Standardaufgaben,
während Braun~\cite{braun_ki_2024} und Sulabh~\cite{s_future_2024} auf die
weiterhin bestehende Notwendigkeit manueller Kontrolle und Überprüfung
hinweisen. Schmitt et~al.~\cite{schmitt_generative_2024} machen zudem darauf
aufmerksam, dass die Nutzung generativer KI-Tools auch soziale und
identitätsbezogene Veränderungen im Entwickleralltag nach sich zieht.

\subsubsection{Lessons Learned und Best Practices}

Die praktische Evaluation zeigt: Generative KI-Tools sind eine wertvolle
Unterstützung im Entwicklungsprozess und führen – bei richtiger Anwendung – zu
Effizienzgewinnen, höherer Codequalität und einer Steigerung des
Entwicklererlebnisses. Wesentliche Empfehlungen lassen sich ableiten:

\begin{itemize}
      \item \textbf{Präzise Prompts sind essenziell:} Je konkreter und kontextreicher die Anweisungen, desto hochwertiger und passgenauer ist das Ergebnis. Unscharfe Prompts führen häufiger zu Fehlinterpretationen oder unbrauchbaren Vorschlägen.
      \item \textbf{Manuelle Kontrolle bleibt notwendig:} Auch bei hoher Automatisierung ist die Überprüfung aller KI-generierten Änderungen unerlässlich – insbesondere bei komplexen State- oder Typ-Logiken und individuellen Anforderungen.
      \item \textbf{Kontextnutzung und Feedback-Loops:} Tools wie Cursor, die Kontextinformationen (z. B. Screenshots, Fehlermeldungen) aktiv verarbeiten, bieten klare Vorteile bei komplexeren Aufgaben. Iteratives Prompt Chaining und direkte Rückmeldung sind Best Practice.
      \item \textbf{Tool-Auswahl nach Projekttyp:} Für „grüne Wiese“-Projekte, schnelle Prototypen und Web-Deployments sind Tools wie Bolt ideal. Für laufende oder größere Projekte empfiehlt sich der gezielte Einsatz von Copilot (bei Routineaufgaben) oder Cursor (bei komplexeren, dialogischen Arbeitsweisen).
      \item \textbf{Plattformkompatibilität beachten:} Unterschiedliche Plattformen (Web, Mobile) erfordern ggf. verschiedene Libraries oder Anpassungen – dies wird von den Tools unterschiedlich gut unterstützt.
\end{itemize}

\subsubsection{Qualitative Bewertung: Zeiteffizienz, Codequalität und Wartbarkeit}

Die qualitative Analyse der Entwicklung mit den drei generativen KI-Tools zeigt
differenzierte Ergebnisse hinsichtlich Effizienz, Codequalität und Wartbarkeit:

\begin{itemize}
      \item \textbf{Copilot} ermöglichte insbesondere bei Standardaufgaben eine spürbare Zeitersparnis. Routinemäßige Komponenten und einfache Logik wurden effizient und weitgehend fehlerfrei generiert. Die Codequalität war bei wiederkehrenden Patterns solide, bei komplexeren Strukturen und individueller Logik jedoch schwankend – hier war zusätzliche Nacharbeit erforderlich, um Wartbarkeit und Konsistenz zu gewährleisten.
      \item \textbf{Cursor} überzeugte durch schnelles Debugging und zielgerichtete Entwicklung, insbesondere bei der Integration von Kontextinformationen (Screenshots, Fehlermeldungen). Die Zeiteffizienz war bei komplexeren Aufgaben und bei Fehlerbehebung deutlich höher als bei Copilot. Die generierte Codequalität profitierte von der iterativen, dialogischen Zusammenarbeit, wodurch auch die Wartbarkeit des Endprodukts verbessert werden konnte.
      \item \textbf{Bolt} zeigte große Stärken in der Initialisierung neuer Projekte und bei Multi-Plattform-Support (Web, Mobile). Die Zeitersparnis war insbesondere im Setup und bei Standardfunktionalitäten signifikant. Im laufenden Betrieb und bei bestehenden Projekten traten jedoch häufiger Kompatibilitätsprobleme auf, die die Wartbarkeit und langfristige Codequalität beeinträchtigten und zusätzliche manuelle Eingriffe erforderten.
\end{itemize}

\subsection{Zwischenfazit}

Die praktische Demonstration hat zentrale Erkenntnisse für die weitere Analyse
geliefert: Die Arbeit mit generativen KI-Tools ermöglicht eine signifikante
Effizienzsteigerung bei der Umsetzung wiederkehrender Entwicklungsaufgaben und
zeigt deutliche Potenziale im Bereich der Codequalität und Wartbarkeit.
Gleichzeitig wurden spezifische Herausforderungen im Bereich der
Fehlerbehebung, Tool-Integration und plattformübergreifenden Entwicklung
sichtbar. Die im praktischen Teil gewonnenen Erfahrungen bilden die Grundlage
für die weiterführende Bewertung der Chancen und Risiken generativer KI in der
Softwareentwicklung, wie sie in den folgenden Kapiteln systematisch analysiert
werden.

Die festgestellten Effizienzgewinne und Herausforderungen im Umgang mit
KI-Tools spiegeln sich auch in der aktuellen empirischen Forschung wider.
Flores-Saviaga et~al.~\cite{flores-saviaga_impact_2025} betonen, dass
insbesondere Aspekte wie Barrierefreiheit und Usability von
Coding-Assistenzsystemen künftig verstärkt zu berücksichtigen sind. Geyer
et~al.~\cite{geyer_case_2025} zeigen, dass die Integration generativer KI-Tools
in agile Entwicklungsprozesse auch positive Effekte auf die Zusammenarbeit und
die Qualitätssicherung haben kann.

% Für die Entwicklung des Map-Screens kamen verschiedene generative KI-Tools zum
% Einsatz: \textbf{GitHub Copilot}, \textbf{Cursor} sowie \textbf{bold.ai}. Die
% Integration erfolgte direkt in der jeweiligen Entwicklungsumgebung: Copilot
% wurde in Visual Studio Code verwendet, Cursor als eigenständige KI-basierte
% Entwicklungsumgebung, bold.ai als weiteres KI-Assistenzsystem mit eigenem
% Workflow.

% Vor Beginn der eigentlichen Implementierung wurden die jeweiligen
% Entwicklungsumgebungen eingerichtet, erforderliche Pakete wie
% \texttt{react-native-maps} und \texttt{expo-location} installiert sowie das
% bestehende App-Setup um die Kartenfunktionalität erweitert. Anschließend wurden
% gezielte Prompts an die KI-Tools formuliert, um die Implementierung effizient
% zu unterstützen.

% \begin{itemize}
%     \item Beispiel-Prompt für Copilot:
%           \begin{quote}
%               ``Erstelle einen Map-Screen in React Native, der Event-Marker basierend auf Eventdaten aus dem Context anzeigt.''
%           \end{quote}
%     \item Beispiel-Prompt für Cursor:
%           \begin{quote}
%               ``Integriere eine Suchleiste mit Places Autocomplete über der Kartenansicht und einen Filter-Button, der zusätzliche Filteroptionen öffnet.''
%           \end{quote}
%     \item Beispiel-Prompt für bold.ai:
%           \begin{quote}
%               ``Erstelle einen übersichtlichen Map-Screen mit Such- und Filterfunktionen in einer bestehenden React-Native-App, nutze dabei die Events aus dem Provider-Kontext.''
%           \end{quote}
% \end{itemize}

% Die jeweiligen KI-Tools generierten daraufhin Codevorschläge für Komponenten,
% Logik zur Datenintegration und Interaktionen mit der Karte. Besonders hilfreich
% waren die Vorschläge zu komplexeren UI-Elementen (wie das Overlay der
% Suchleiste oder die Filter-Logik), aber auch zu typischen Herausforderungen wie
% State-Management und Performance-Optimierung bei vielen Markern.

% \subsection{Entwicklungsprozess mit KI-Unterstützung}
% Der Entwicklungsprozess wurde für jedes Tool vergleichbar durchgeführt, wobei
% die generativen KI-Tools als ``Pair Programming Partner'' dienten. Typischer
% Ablauf:

% \begin{enumerate}
%     \item \textbf{Initiales Setup:} Basierend auf den Vorschlägen des jeweiligen KI-Tools wurde die Kartenkomponente implementiert und mit den Eventdaten aus dem \texttt{EventsProvider} verbunden.
%     \item \textbf{Integration zusätzlicher Features:} Durch gezielte Prompts wurden u.\,a. folgende Features ergänzt:
%           \begin{itemize}
%               \item Suchleiste mit Places Autocomplete für die Standortsuche
%               \item Filter-Button zum Einblenden weiterer Filteroptionen
%               \item Dynamische Anzeige von Event-Markern auf Basis der aktuellen Filter
%           \end{itemize}
%     \item \textbf{Refinement und Debugging:} Die KI-Tools unterstützten bei der Analyse von Fehlern, dem Vorschlagen alternativer Lösungsansätze sowie bei der Verbesserung von Codequalität und Lesbarkeit.
%     \item \textbf{Dokumentation und Code-Reviews:} Automatisch generierte Kommentare und Hilfstexte der KI wurden für die Dokumentation genutzt und anschließend kritisch überprüft.
% \end{enumerate}

% Die Nutzung der unterschiedlichen KI-Tools ermöglichte eine deutliche
% Beschleunigung der Implementierung, insbesondere bei Routinetätigkeiten und der
% Recherche nach Best Practices. Gleichzeitig wurde die Codequalität durch
% automatisierte Vorschläge und Unterstützung bei der Fehleranalyse verbessert.
% Herausfordernd war gelegentlich die präzise Formulierung der Prompts, um
% optimale und kontextsensitive Antworten zu erhalten.

% Die einzelnen Entwicklungsschritte sowie der Vergleich der generierten und
% manuell überarbeiteten Codeabschnitte werden in den folgenden Abschnitten
% exemplarisch dargestellt.



\section{Erste Evaluierung}
\subsection{Erfolge und Herausforderungen}
Die praktische Implementierung hat gezeigt...

\subsection{Qualitative Bewertung}
Die Analyse der Entwicklung fokussiert sich auf:
\begin{itemize}
    \item Zeiteffizienz
    \item Code-Qualität
    \item Wartbarkeit
\end{itemize} 

\section{Zwischenfazit}
Die praktische Demonstration hat wichtige Erkenntnisse für die weitere Analyse geliefert:
\begin{itemize}
    \item [Kernerkenntnisse...]
    \item [Überleitung zu folgenden Kapiteln...]
\end{itemize}  
\chapter{Chancen}
Die Integration generativer KI-Technologien eröffnet der modernen
Softwareentwicklung zahlreiche neue Möglichkeiten. Neben der Automatisierung
repetitiver Aufgaben bieten KI-Tools erhebliche Potenziale zur Steigerung der
Effizienz, zur Verbesserung der Codequalität und zur Einführung innovativer
Entwicklungspraktiken. Wie die praktische Demonstration in Kapitel~3 gezeigt
hat, führt der gezielte Einsatz generativer KI-Tools nicht nur zu einer
spürbaren Zeitersparnis, sondern kann auch die Wartbarkeit und Zuverlässigkeit
von Software nachhaltig erhöhen.

\section{Effizienzsteigerung und Automatisierung}
Zahlreiche Studien und Fallanalysen bescheinigen generativen KI-Tools das
Potenzial, die Effizienz im Entwicklungsprozess maßgeblich zu
steigern~\cite{donvir_role_2024,coutinho_role_2024,s_future_2024,esposito_generative_2025,braun_ki_2024,siebert_generative_2024}.
Auch in der eigenen praktischen Demonstration (vgl. Kapitel~3) zeigte sich,
dass Werkzeuge wie GitHub Copilot oder Cursor repetitive Aufgaben wie das
Erstellen von Boilerplate-Code, Standardkomponenten oder einfachen UI-Logiken
erheblich beschleunigen können. So konnte das Grundgerüst des Map-Screens in
der Locals-App mit Unterstützung von Copilot innerhalb weniger Minuten
generiert werden, während vergleichbare Aufgaben ohne KI deutlich
zeitaufwändiger wären.

Aktuelle Literatur und Praxisberichte belegen, dass der gezielte Einsatz
generativer KI-Tools zu signifikanten Effizienzsteigerungen in der
Softwareentwicklung führt. Donvir et~al.~\cite{donvir_role_2024} betonen, dass
moderne Coding-Assistenzsysteme wie Copilot oder Cursor insbesondere bei
repetitiven Aufgaben für eine starke Beschleunigung sorgen. Auch die Fallstudie
von Coutinho et~al.~\cite{coutinho_role_2024} weist nach, dass sich die
Entwicklungszeit bei Routineaufgaben durch KI-gestützte Werkzeuge deutlich
verringert. Sulabh~\cite{s_future_2024} und das Fraunhofer
IESE~\cite{siebert_generative_2024} berichten von Effizienzgewinnen von bis zu
50\,\%. Esposito et~al.~\cite{esposito_generative_2025} unterstreichen zudem,
dass der Einsatz von Large Language Models neue Automatisierungs- und
Optimierungsmöglichkeiten eröffnet.

\begin{quote}
    \enquote{GitHub Copilot can assist in quick prototyping of code by generating foundational code structure based on natural language description of the feature. It can assist in boilerplate code generation by providing the class and interface definition generation, API and Database Schema creation. Both of these features combined improve the developer efficiency and enhanced code quality.}
    \cite[S.~4]{donvir_role_2024}
\end{quote}

Generative KI-Tools wirken sich auf sämtliche Phasen des
Softwareentwicklungsprozesses aus – von der Planung über die Implementierung
bis hin zu Test und Deployment – und eröffnen dadurch neue Potenziale für die
Effizienzsteigerung~\cite{minikiewicz_impact_nodate}. Feldexperimente mit
Softwareentwickler:innen bestätigen, dass sich der Einsatz solcher Werkzeuge
unmittelbar positiv auf Produktivität und Arbeitsweise
auswirkt~\cite{cui_effects_2024}.

Auch komplexere Aufgaben wie Debugging oder die automatische Anpassung von
Datenstrukturen profitieren von KI-Unterstützung, wie insbesondere der
Vergleich zwischen Copilot und Cursor verdeutlicht. Die Literatur verweist
dabei auf Effizienzsteigerungen von bis zu 50\,\% bei
Routinetätigkeiten~\cite{s_future_2024}, was sich mit den im Praxisteil
beobachteten Zeitersparnissen und Produktivitätsgewinnen deckt.

Die Qualität der Automatisierung bleibt jedoch stark abhängig von der Präzision
der Prompts und der Kontextintegration der eingesetzten Tools. Wie die Arbeit
mit Cursor gezeigt hat, ist gerade bei komplexeren Aufgaben ein dialogischer
Ansatz mit Feedback-Loops und manueller Kontrolle weiterhin unverzichtbar.
Dennoch legen sowohl Forschung als auch Praxis nahe, dass generative KI einen
spürbaren Effizienzgewinn im Entwicklungsalltag ermöglicht.
Wangoo~\cite{wangoo_artificial_2018} hebt hervor, dass KI-Technologien nicht
nur den Entwicklungsprozess beschleunigen, sondern auch die Wiederverwendung
bestehender Komponenten und das Design von Software nachhaltig vereinfachen
können.


\section{Neue Werkzeuge und Methoden}
Der verstärkte Einsatz generativer KI hat in den letzten Jahren eine Vielzahl
neuer Werkzeuge und Methoden in der Softwareentwicklung etabliert. Besonders
die Integration von Large Language Models (LLMs) in Entwicklungsumgebungen hat
die Art, wie Entwickler*innen arbeiten, maßgeblich verändert.

Zu den wichtigsten Werkzeugen zählen unter anderem \textbf{GitHub Copilot},
\textbf{Cursor AI}, \textbf{Amazon CodeWhisperer} und \textbf{Devin AI}. Diese
Tools werden in der Literatur umfassend dargestellt und spielen laut Esposito
et al.~\cite{esposito_generative_2025} sowie Nguyen-Duc et
al.~\cite{duc_generative_2023} eine zentrale Rolle in der aktuellen
Entwicklungspraxis.

GitHub Copilot wird besonders häufig eingesetzt und unterstützt
Entwickler*innen bei der automatischen Codegenerierung und Vervollständigung
direkt in der IDE. Esposito et al.~\cite[S.~2]{esposito_generative_2025}
beschreiben, dass solche Werkzeuge zunehmend in frühen Phasen des
Entwicklungsprozesses verwendet werden, etwa beim Übergang von Anforderungen zu
Architektur oder bei der Erstellung von Code aus natürlichsprachigen
Beschreibungen.

Cursor AI und ähnliche Tools ermöglichen einen dialogorientierten Workflow, bei
dem nicht nur einzelne Codezeilen, sondern ganze Features, Module oder sogar
Projekte automatisch erstellt und verfeinert werden können. Dabei kommen
Methoden wie Prompt Engineering, Retrieval-Augmented Generation (RAG) und
agentenbasierte Ansätze zum Einsatz (vgl. Esposito et
al.,~\cite[S.~3--4]{esposito_generative_2025}).

Im praktischen Teil dieser Arbeit (vgl. Kapitel~3) zeigte sich, dass die
Kombination dieser Werkzeuge erhebliche Produktivitätsgewinne ermöglicht, vor
allem beim schnellen Prototyping, bei Standardaufgaben (Boilerplate) und bei
der automatischen Generierung von Tests. Cursor AI konnte darüber hinaus durch
die Möglichkeit, Kontext wie Screenshots oder Fehlermeldungen einzubinden, bei
der Fehlersuche und dem Debugging zusätzliche Mehrwerte bieten.

Neben den Werkzeugen haben sich auch neue Methoden etabliert:
\begin{itemize}
    \item \textbf{Prompt Engineering:} Entwickler*innen formulieren Anforderungen in natürlicher Sprache, die direkt von der KI interpretiert werden (vgl. Esposito et al.,~\cite[S.~2--3]{esposito_generative_2025}).
    \item \textbf{Retrieval-Augmented Generation (RAG):} KI-Tools kombinieren projektspezifische Kontextdaten (z.\,B. Dokumentation, vorhandener Code) mit aktuellen Benutzeranfragen, um passgenaue Lösungen zu generieren (vgl. Esposito et al.,~\cite[S.~4]{esposito_generative_2025}).
    \item \textbf{Human-in-the-Loop und Pair Programming:} Laut Nguyen-Duc et al.~\cite[S.~8]{nguyen-duc_generative_2023} und Fraunhofer IESE~\cite{siebert_generative_2024} wird die Zusammenarbeit von Mensch und KI (z.\,B. durch Feedback-Loops) immer wichtiger, um Qualität und Anpassungsfähigkeit der Entwicklung zu sichern.
\end{itemize}

Im Blog von Fraunhofer IESE~\cite{siebert_generative_2024} wird betont, dass
diese neuen Tools nicht nur als Autovervollständigung dienen, sondern immer
mehr Aufgaben im gesamten Entwicklungsprozess übernehmen – bis hin zur
automatischen Erstellung von Tests und zum Refactoring.

% \begin{itemize}
%     \item Innovative Ansätze für die Softwareentwicklung
% \end{itemize}
% ... Hier kommt der Text für die Subsektion Optimierung der Kollaboration durch KI ... 

% \chapter{Chancen}
% Der Einsatz von KI in der Softwareentwicklung bietet eine Vielzahl an Vorteilen. Besonders hervorzuheben sind Effizienzsteigerungen durch Automatisierung, die Entwickler:innen von repetitiven Aufgaben entlasten und ihnen mehr Zeit für kreative und konzeptionelle Arbeit geben. Dieses Kapitel untersucht die wichtigsten Potenziale, die KI-Technologien für den Softwareentwicklungsprozess mit sich bringen.

% Im praktischen Teil dieser Arbeit wird zudem untersucht, wie sich durch KI-Tools Entwicklungsaufgaben für eine interaktive Kartenansicht automatisieren lassen und welche konkreten Effizienzsteigerungen hier auftreten können.
% \section{Effizienzsteigerung und Automatisierung}
% Zahlreiche Studien und Fallanalysen bescheinigen generativen KI-Tools das
Potenzial, die Effizienz im Entwicklungsprozess maßgeblich zu
steigern~\cite{donvir_role_2024,coutinho_role_2024,s_future_2024,esposito_generative_2025,braun_ki_2024,siebert_generative_2024}.
Auch in der eigenen praktischen Demonstration (vgl. Kapitel~3) zeigte sich,
dass Werkzeuge wie GitHub Copilot oder Cursor repetitive Aufgaben wie das
Erstellen von Boilerplate-Code, Standardkomponenten oder einfachen UI-Logiken
erheblich beschleunigen können. So konnte das Grundgerüst des Map-Screens in
der Locals-App mit Unterstützung von Copilot innerhalb weniger Minuten
generiert werden, während vergleichbare Aufgaben ohne KI deutlich
zeitaufwändiger wären.

Aktuelle Literatur und Praxisberichte belegen, dass der gezielte Einsatz
generativer KI-Tools zu signifikanten Effizienzsteigerungen in der
Softwareentwicklung führt. Donvir et~al.~\cite{donvir_role_2024} betonen, dass
moderne Coding-Assistenzsysteme wie Copilot oder Cursor insbesondere bei
repetitiven Aufgaben für eine starke Beschleunigung sorgen. Auch die Fallstudie
von Coutinho et~al.~\cite{coutinho_role_2024} weist nach, dass sich die
Entwicklungszeit bei Routineaufgaben durch KI-gestützte Werkzeuge deutlich
verringert. Sulabh~\cite{s_future_2024} und das Fraunhofer
IESE~\cite{siebert_generative_2024} berichten von Effizienzgewinnen von bis zu
50\,\%. Esposito et~al.~\cite{esposito_generative_2025} unterstreichen zudem,
dass der Einsatz von Large Language Models neue Automatisierungs- und
Optimierungsmöglichkeiten eröffnet.

\begin{quote}
    \enquote{GitHub Copilot can assist in quick prototyping of code by generating foundational code structure based on natural language description of the feature. It can assist in boilerplate code generation by providing the class and interface definition generation, API and Database Schema creation. Both of these features combined improve the developer efficiency and enhanced code quality.}
    \cite[S.~4]{donvir_role_2024}
\end{quote}

Generative KI-Tools wirken sich auf sämtliche Phasen des
Softwareentwicklungsprozesses aus – von der Planung über die Implementierung
bis hin zu Test und Deployment – und eröffnen dadurch neue Potenziale für die
Effizienzsteigerung~\cite{minikiewicz_impact_nodate}. Feldexperimente mit
Softwareentwickler:innen bestätigen, dass sich der Einsatz solcher Werkzeuge
unmittelbar positiv auf Produktivität und Arbeitsweise
auswirkt~\cite{cui_effects_2024}.

Auch komplexere Aufgaben wie Debugging oder die automatische Anpassung von
Datenstrukturen profitieren von KI-Unterstützung, wie insbesondere der
Vergleich zwischen Copilot und Cursor verdeutlicht. Die Literatur verweist
dabei auf Effizienzsteigerungen von bis zu 50\,\% bei
Routinetätigkeiten~\cite{s_future_2024}, was sich mit den im Praxisteil
beobachteten Zeitersparnissen und Produktivitätsgewinnen deckt.

Die Qualität der Automatisierung bleibt jedoch stark abhängig von der Präzision
der Prompts und der Kontextintegration der eingesetzten Tools. Wie die Arbeit
mit Cursor gezeigt hat, ist gerade bei komplexeren Aufgaben ein dialogischer
Ansatz mit Feedback-Loops und manueller Kontrolle weiterhin unverzichtbar.
Dennoch legen sowohl Forschung als auch Praxis nahe, dass generative KI einen
spürbaren Effizienzgewinn im Entwicklungsalltag ermöglicht.
Wangoo~\cite{wangoo_artificial_2018} hebt hervor, dass KI-Technologien nicht
nur den Entwicklungsprozess beschleunigen, sondern auch die Wiederverwendung
bestehender Komponenten und das Design von Software nachhaltig vereinfachen
können.


% \section{Neue Werkzeuge und Methoden}
% Der verstärkte Einsatz generativer KI hat in den letzten Jahren eine Vielzahl
neuer Werkzeuge und Methoden in der Softwareentwicklung etabliert. Besonders
die Integration von Large Language Models (LLMs) in Entwicklungsumgebungen hat
die Art, wie Entwickler*innen arbeiten, maßgeblich verändert.

Zu den wichtigsten Werkzeugen zählen unter anderem \textbf{GitHub Copilot},
\textbf{Cursor AI}, \textbf{Amazon CodeWhisperer} und \textbf{Devin AI}. Diese
Tools werden in der Literatur umfassend dargestellt und spielen laut Esposito
et al.~\cite{esposito_generative_2025} sowie Nguyen-Duc et
al.~\cite{duc_generative_2023} eine zentrale Rolle in der aktuellen
Entwicklungspraxis.

GitHub Copilot wird besonders häufig eingesetzt und unterstützt
Entwickler*innen bei der automatischen Codegenerierung und Vervollständigung
direkt in der IDE. Esposito et al.~\cite[S.~2]{esposito_generative_2025}
beschreiben, dass solche Werkzeuge zunehmend in frühen Phasen des
Entwicklungsprozesses verwendet werden, etwa beim Übergang von Anforderungen zu
Architektur oder bei der Erstellung von Code aus natürlichsprachigen
Beschreibungen.

Cursor AI und ähnliche Tools ermöglichen einen dialogorientierten Workflow, bei
dem nicht nur einzelne Codezeilen, sondern ganze Features, Module oder sogar
Projekte automatisch erstellt und verfeinert werden können. Dabei kommen
Methoden wie Prompt Engineering, Retrieval-Augmented Generation (RAG) und
agentenbasierte Ansätze zum Einsatz (vgl. Esposito et
al.,~\cite[S.~3--4]{esposito_generative_2025}).

Im praktischen Teil dieser Arbeit (vgl. Kapitel~3) zeigte sich, dass die
Kombination dieser Werkzeuge erhebliche Produktivitätsgewinne ermöglicht, vor
allem beim schnellen Prototyping, bei Standardaufgaben (Boilerplate) und bei
der automatischen Generierung von Tests. Cursor AI konnte darüber hinaus durch
die Möglichkeit, Kontext wie Screenshots oder Fehlermeldungen einzubinden, bei
der Fehlersuche und dem Debugging zusätzliche Mehrwerte bieten.

Neben den Werkzeugen haben sich auch neue Methoden etabliert:
\begin{itemize}
    \item \textbf{Prompt Engineering:} Entwickler*innen formulieren Anforderungen in natürlicher Sprache, die direkt von der KI interpretiert werden (vgl. Esposito et al.,~\cite[S.~2--3]{esposito_generative_2025}).
    \item \textbf{Retrieval-Augmented Generation (RAG):} KI-Tools kombinieren projektspezifische Kontextdaten (z.\,B. Dokumentation, vorhandener Code) mit aktuellen Benutzeranfragen, um passgenaue Lösungen zu generieren (vgl. Esposito et al.,~\cite[S.~4]{esposito_generative_2025}).
    \item \textbf{Human-in-the-Loop und Pair Programming:} Laut Nguyen-Duc et al.~\cite[S.~8]{nguyen-duc_generative_2023} und Fraunhofer IESE~\cite{siebert_generative_2024} wird die Zusammenarbeit von Mensch und KI (z.\,B. durch Feedback-Loops) immer wichtiger, um Qualität und Anpassungsfähigkeit der Entwicklung zu sichern.
\end{itemize}

Im Blog von Fraunhofer IESE~\cite{siebert_generative_2024} wird betont, dass
diese neuen Tools nicht nur als Autovervollständigung dienen, sondern immer
mehr Aufgaben im gesamten Entwicklungsprozess übernehmen – bis hin zur
automatischen Erstellung von Tests und zum Refactoring.

% \begin{itemize}
%     \item Innovative Ansätze für die Softwareentwicklung
% \end{itemize}
% ... Hier kommt der Text für die Subsektion Optimierung der Kollaboration durch KI ... 


\chapter{Herausforderungen durch KI in der Softwareentwicklung}
Trotz der erheblichen Chancen, die der Einsatz generativer KI in der
Softwareentwicklung bietet, sind mit ihrer Integration zahlreiche
Herausforderungen verbunden. Neben technischen Fragen stehen insbesondere
Aspekte der Sicherheit, des Datenschutzes, der ethischen Verantwortung sowie
soziale und organisatorische Veränderungen im Mittelpunkt der aktuellen
Diskussion. Die folgenden Abschnitte beleuchten diese Herausforderungen anhand
aktueller Literatur und reflektieren sie unter Einbezug der im praktischen Teil
gewonnenen Erfahrungen.

\section{Sicherheits- und Datenschutzaspekte}

Die Integration generativer KI-Tools in die Softwareentwicklung eröffnet nicht
nur neue Chancen, sondern bringt auch Risiken für IT-Sicherheit und Datenschutz
mit sich. Aktuelle Studien zeigen, dass generative KI einerseits dazu beitragen
kann, Sicherheitslücken zu erkennen und Best Practices wie Security-by-Design
umzusetzen, andererseits aber auch neue Angriffsflächen schafft, wenn
KI-generierte Vorschläge ungeprüft übernommen werden
\cite{shi_ai-assisted_2023, alwageed_role_nodate}.

Insbesondere wird darauf hingewiesen, dass der Einsatz generativer KI zu
spezifischen Risiken wie \enquote{Prompt Injection} und \enquote{adversarial
    attacks} führen kann. Dabei werden durch gezielte oder manipulierte Eingaben
der KI unsichere oder schadhafte Codefragmente entlockt
\cite{shi_ai-assisted_2023}. Ein weiteres Risiko ist das sogenannte
\enquote{Model Poisoning}, bei dem durch fehlerhafte oder bösartige
Trainingsdaten gezielt Schwachstellen in das Modell eingeschleust werden
\cite{alwageed_role_nodate}.

Regelmäßige Security-Reviews, Audit-Trails und eine konsequente Einbindung
menschlicher Expertise (\enquote{Human-in-the-Loop}) werden in der Literatur
als zentrale Maßnahmen zur Absicherung empfohlen. Die Gefahr besteht
insbesondere darin, dass KI-Tools potenziell sicherheitskritische Muster zwar
erkennen und kennzeichnen können, ihre Vorschläge jedoch stets von
Entwickler:innen geprüft und angepasst werden müssen
\cite{shi_ai-assisted_2023, alwageed_role_nodate, siebert_generative_2024}.

Auch Datenschutzfragen treten verstärkt in den Vordergrund, insbesondere beim
Einsatz externer KI-Modelle in Cloud-Umgebungen. Hier besteht die Gefahr, dass
sensible Daten unbeabsichtigt an Dritte weitergegeben werden oder aus den
generierten Vorschlägen rekonstruiert werden können
\cite{siebert_generative_2024}.

Im praktischen Teil dieser Arbeit zeigte sich, dass KI-Tools wie Copilot und
Cursor zwar potenziell sicherheitskritische Muster (z.\,B. hardcodierte
Passwörter) erkennen können, ihre Vorschläge aber stets kritisch geprüft und
gegebenenfalls angepasst werden müssen.


\subsection{Sicherheitsrisiken durch generative Modelle}

Generative KI-Modelle bringen spezifische neue Bedrohungen mit sich. Zu den
wichtigsten Risiken zählen sogenannte \enquote{Prompt Injections}, bei denen
durch manipulierte Eingaben unsicherer oder schädlicher Code erzeugt werden
kann, sowie \enquote{adversarial attacks}, bei denen minimale Änderungen an den
Eingabedaten zu sicherheitskritischen Verhaltensweisen führen können
\cite{shi_ai-assisted_2023}.

Ein weiteres Risiko besteht im sogenannten \enquote{Model Poisoning}, bei dem
während des Trainings gezielt fehlerhafte oder bösartige Daten eingespeist
werden, um Schwachstellen in der KI zu platzieren. Besonders große generative
Modelle wie LLMs sind hierfür anfällig, da sie oft auf umfangreiche und
öffentlich verfügbare Datenquellen zurückgreifen \cite{alwageed_role_nodate}.

Auch im Bereich der Software-Supply-Chain entstehen neue Risiken. Unzureichend
geprüfter oder von Dritten generierter Code kann unbemerkt Schwachstellen in
den Entwicklungsprozess einschleusen. Entlang der gesamten Kette, von der
Entwicklung bis zur Bereitstellung, muss daher auf Sicherheit und regelmäßige
Überprüfung geachtet werden \cite{siebert_generative_2024}.

Die Literatur empfiehlt, Sicherheitsmechanismen wie regelmäßige
Security-Reviews, Human-in-the-Loop-Prozesse und gezielte Trainingsmaßnahmen
gegen adversarielle Angriffe und Model Poisoning zu etablieren, um die
Robustheit generativer KI-Systeme zu erhöhen \cite{shi_ai-assisted_2023,
    alwageed_role_nodate, siebert_generative_2024}.


\section{Ethische und soziale Implikationen}

Die zunehmende Integration generativer KI in die Softwareentwicklung wirft
weitreichende ethische und soziale Fragen auf. Die Automatisierung von
Entwicklungsaufgaben macht es notwendig, Verantwortung und
Entscheidungsprozesse klar zu definieren. Insbesondere Nachvollziehbarkeit und
Transparenz von KI-generierten Lösungen sind zentrale Herausforderungen für
ethische Standards und die Überprüfbarkeit von Ergebnissen
\cite{weisz_design_2024}.

Ein wesentliches ethisches Problem besteht im sogenannten Bias. Generative
KI-Modelle übernehmen häufig bestehende Vorurteile oder Stereotypen aus den
Trainingsdaten und können diese unreflektiert reproduzieren. Um
Diskriminierung, Fehlinformationen und unfaire Vorschläge zu vermeiden, sind
technische und organisatorische Kontrollmechanismen erforderlich, wie etwa
Guardrails, kontrollierte Testdatensätze und Diversity-Checks
\cite{weisz_design_2024, schmitt_generative_2024}.

Ethische und soziale Fragestellungen gewinnen immer mehr an Bedeutung, da
Barrierefreiheit und Teilhabe bei der Entwicklung von KI-Systemen noch oft
unzureichend berücksichtigt werden, obwohl KI-Technologien neue Möglichkeiten
der Inklusion bieten könnten \cite{flores-saviaga_impact_2025}.

Gleichzeitig zeigt die Literatur, dass der Einsatz generativer KI die
berufliche Identität von Entwickler:innen beeinflusst. Es entstehen
Unsicherheiten hinsichtlich der eigenen Rolle und Wertschätzung, aber auch neue
Möglichkeiten zur Kompetenzentwicklung und Zusammenarbeit, wenn der Fokus auf
Mensch-KI-Kollaboration gelegt wird \cite{schmitt_generative_2024}.

Auch auf organisatorischer Ebene sind Unternehmen gefordert, klare Leitlinien
für die Nutzung von GenAI-Tools zu formulieren und Verantwortlichkeiten,
Qualitätsstandards sowie ethische Prinzipien verbindlich zu verankern
\cite{nguyen-duc_generative_2023}.


\subsection{Ethische Konflikte und Bias in KI-Systemen}
Eines der zentralen ethischen Probleme beim Einsatz generativer KI in der
Softwareentwicklung ist die Gefahr von Bias und Diskriminierung. Wie Weisz et
al.~\cite{weisz_design_2024} herausstellen, können große Sprachmodelle und
generative Systeme bestehende Vorurteile, Diskriminierungen oder Stereotypen
aus den Trainingsdaten übernehmen und diese im erzeugten Code oder in den
Vorschlägen reproduzieren. Dies kann zu unfairen, potenziell diskriminierenden
Ergebnissen führen und damit ethische Grundsätze sowie
Gleichbehandlungsprinzipien verletzen.

Um solchen Risiken zu begegnen, empfehlen Weisz et
al.~\cite{weisz_design_2024}, dass Entwickler:innen und Organisationen
technische und organisatorische Maßnahmen (\enquote{Guardrails})
implementieren, die sicherstellen, dass generative KI-Lösungen regelmäßig auf
Fairness, Transparenz und mögliche Verzerrungen geprüft werden. Dazu zählen
etwa kontrollierte Testdatensätze, Diversity-Checks oder der Einsatz
spezialisierter Überwachungsmechanismen.

Schmitt et al.~\cite{schmitt_generative_2024} weisen darauf hin, dass die
Gefahr von Bias nicht nur technischer Natur ist, sondern auch soziale und
organisationale Auswirkungen haben kann. Insbesondere im Kontext beruflicher
Identität und Teamdynamik kann eine unkritische Nutzung von KI-Systemen zu
Unsicherheiten, Vertrauensverlust und Spannungen führen – etwa wenn Vorschläge
der KI als neutral oder objektiv wahrgenommen werden, obwohl sie verzerrt oder
unvollständig sind.

Insgesamt machen beide Quellen deutlich, dass ethische Konflikte und der Umgang
mit Bias zentrale Herausforderungen für den erfolgreichen und
verantwortungsvollen Einsatz generativer KI in der Softwareentwicklung
darstellen.


\subsection{Langfristige Auswirkungen auf Entwickler:innen-Rollen}

Der verstärkte Einsatz generativer KI-Tools in der Softwareentwicklung
verändert die Rolle von Entwickler:innen grundlegend. Während
Routinetätigkeiten und repetitive Aufgaben zunehmend automatisiert werden,
gewinnen Kompetenzen wie Prompt-Engineering, Systembewertung und die kritische
Reflexion von KI-Ergebnissen deutlich an Bedeutung
\cite{schmitt_generative_2024}.

Viele Entwickler:innen sehen in der Integration von GenAI neue Möglichkeiten
zur Kompetenzentwicklung, etwa in der Mensch-KI-Kollaboration oder im Aufbau
von Schnittstellenwissen zwischen Entwicklung, Domänenkenntnis und KI-Nutzung.
Gleichzeitig entstehen durch die Neuverteilung von Aufgaben und die wachsende
Abhängigkeit von KI-Systemen Unsicherheiten und Identitätskonflikte
\cite{schmitt_generative_2024}.

Auf organisatorischer Ebene sind Anpassungen der Rollenprofile, neue
Schulungskonzepte und die Überarbeitung von Verantwortlichkeiten notwendig, um
den Wandel aktiv zu gestalten und kontinuierliche Weiterbildung zu ermöglichen.
Entscheidend ist, dass Unternehmen und Entwickler:innen sich auf die
Veränderungen einlassen und die Transformation aktiv begleiten
\cite{nguyen-duc_generative_2023}.

Auch die Erfahrungen aus dem Praxisteil dieser Arbeit zeigen, dass der Fokus
bei der Entwicklung zunehmend auf der Formulierung präziser Prompts, der
kritischen Prüfung von KI-Vorschlägen und der aktiven Gestaltung der
Mensch-KI-Zusammenarbeit liegt. Die Rolle von Entwickler:innen wandelt sich
damit immer stärker hin zu einer Schnittstellenfunktion zwischen Mensch und
Maschine.


\subsection{Technische und organisatorische Hürden bei der Einführung von KI}
Die Einführung generativer KI ist mit zahlreichen technischen und
organisatorischen Hürden verbunden. Nguyen-Duc
et~al.~\cite{nguyen-duc_generative_2023} betonen den Mangel an Standards,
geeigneten Benchmarks und validen Testdaten als zentrale Herausforderungen.
Gill~\cite{Gil} verweist auf die Notwendigkeit, agile Prozesse speziell für
KI-Projekte weiterzuentwickeln, während das Fraunhofer IESE~\cite{Sie} auf
Unsicherheiten im Team und fehlende Akzeptanz hinweist. Sergeyuk
et~al.~\cite{Ser} unterstreichen in ihrer Übersicht, dass auch Usability und
UX-Design bei der Integration von KI-Tools stärker berücksichtigt werden
müssen. Sifi~\cite{sifi_how_2025} hebt hervor, dass die subjektive
Nutzererfahrung bei der Einführung neuer KI-Tools oft unterschätzt wird und die
Akzeptanz entscheidend von transparenter Kommunikation und kontinuierlichem
Feedback abhängt.

Weitere technische Herausforderungen bestehen in der Qualitätssicherung der
generierten Ergebnisse. Laut Nguyen-Duc et
al.~\cite{nguyen-duc_generative_2023} ist es bislang schwierig, die
Zuverlässigkeit, Sicherheit und Wartbarkeit von KI-generiertem Code
systematisch zu überprüfen. Auch der Mangel an geeigneten Benchmarks, Testdaten
und automatisierten Validierungsverfahren erschwert die breite Einführung von
GenAI im Unternehmensumfeld.

Mit der fortschreitenden Integration von KI in die Softwareentwicklung
entstehen nicht nur neue technische Herausforderungen, sondern auch veränderte
Anforderungen an Sicherheit, Arbeitsorganisation und langfristige
Kollaborationsmodelle \cite{hazra_ai_2025}

Auf organisatorischer Ebene betonen sowohl Nguyen-Duc et
al.~\cite{nguyen-duc_generative_2023} als auch Schmitt et
al.~\cite{schmitt_generative_2024}, dass fehlende Akzeptanz und Unsicherheit im
Team, unklare Verantwortlichkeiten sowie mangelnde Schulung zu erheblichen
Implementierungsbarrieren führen können. Die Umstellung auf KI-gestützte
Prozesse erfordert häufig ein umfassendes Change Management, die Anpassung
bestehender Arbeitsweisen und neue Formen der Zusammenarbeit. Schmitt et
al.~\cite{schmitt_generative_2024} weisen darauf hin, dass die erfolgreiche
Einführung von GenAI-Tools nicht nur technisches Know-how, sondern auch
kulturelle Offenheit und kontinuierliche Weiterbildung im Team voraussetzt.
Richards und Wessel~\cite{richards_bridging_2025} argumentieren, dass für den
Erfolg conversationaler KI-Assistenzsysteme eine enge Zusammenarbeit von HCI-
und KI-Forschung erforderlich ist, um praxisnahe Evaluationsmethoden zu
etablieren.


\chapter{Wirtschaftliche und gesellschaftliche Auswirkungen}
\label{chap:wirtschaftliche_und_gesellschaftliche_auswirkungen}
Die Integration generativer KI in die Softwareentwicklung wirkt sich nicht nur auf technischer Ebene, sondern zunehmend auch auf wirtschaftliche Strukturen und gesellschaftliche Prozesse aus. Die folgenden Abschnitte analysieren, wie sich Unternehmen, Arbeitsmärkte und die Rolle von Softwareentwickler:innen im Zuge der KI-Einführung verändern. Im Zentrum stehen Fragen nach Produktivität, Wertschöpfung, Arbeitsplatzstruktur und den Chancen und Risiken für Unternehmen und Gesellschaft.

\section{Veränderungen in Softwareunternehmen}
Mit der Verbreitung generativer KI-Technologien stehen Softwareunternehmen vor
einem tiefgreifenden Wandel. Marguerit~\cite{marguerit_augmenting_2025}
beschreibt, dass Unternehmen zunehmend KI-basierte Automatisierungslösungen
einsetzen, um Entwicklungsprozesse zu beschleunigen und Ressourcen effizienter
zu nutzen. Diese Entwicklung führt dazu, dass klassische Rollenmodelle und
Teamstrukturen neu bewertet werden müssen.

Farach et al.~\cite{farach_evolving_2025} argumentieren, dass digitale Arbeit
durch KI-Tools einen eigenständigen Produktionsfaktor darstellt, der
traditionelle Vorstellungen von Arbeitsteilung und Wertschöpfung grundlegend
verändert. In vielen Unternehmen werden Aufgaben wie das Schreiben von
Standardcode, Testing oder das Generieren von Dokumentation zunehmend
automatisiert, während der Fokus auf kreative, überwachende und strategische
Tätigkeiten wächst.

Rothschild et~al.~\cite{rothschild_agentic_2025} diskutieren das Konzept der
„agentischen Ökonomie“, in der generative KI-Systeme zunehmend eigenständig
Entscheidungen treffen und so ganze Wertschöpfungsketten transformieren.
Unternehmen sind laut den Autoren gefordert, ihre Prozesse kontinuierlich an
diese neue Form digitaler Arbeit und Delegation anzupassen.

Storey et al.~\cite{storey_generative_2025} betonen, dass die Einführung
generativer KI-Tools nicht nur ökonomische, sondern auch organisatorische
Anpassungen erfordert. Unternehmen müssen neue Kompetenzen fördern,
Veränderungsbereitschaft unterstützen und klare ethische sowie regulatorische
Leitplanken setzen, um Risiken zu minimieren und die Akzeptanz im Team zu
erhöhen.

Die Literatur macht deutlich, dass der Erfolg von KI-Integration maßgeblich
davon abhängt, inwieweit Unternehmen nicht nur in Technologie, sondern auch in
Weiterbildung, Organisationsentwicklung und eine offene Innovationskultur
investieren.

% \begin{itemize}
%     \item Auswirkungen auf Geschäftsmodelle und Prozesse
%     \item Veränderungen in der Softwareentwicklung und im Projektmanagement
%     \item Rolle von KI bei der Automatisierung von Softwareentwicklungsaufgaben
% \end{itemize}

\section{Auswirkungen auf den Arbeitsmarkt und Entwickler:innen-Rollen}

Die zunehmende Verbreitung generativer KI wirkt sich bereits heute spürbar auf
den Arbeitsmarkt für Softwareentwickler:innen aus. Studien zeigen, dass nicht
nur klassische Routinetätigkeiten zunehmend automatisiert werden, sondern sich
auch die Qualifikationsprofile und Tätigkeitsfelder nachhaltig
verändern~\cite{siebert_generative_2024,braun_ki_2024,s_future_2024}. Während
repetitive Aufgaben wie das Schreiben von Boilerplate-Code, Testing oder
Dokumentation verstärkt durch KI-Tools übernommen werden, wächst der Bedarf an
Kompetenzen in Bereichen wie Prompt-Engineering, KI-Management und der
kritischen Bewertung KI-generierter Ergebnisse.

Analysen aktueller Jobprofile und -anzeigen belegen, dass die Nachfrage nach
Fähigkeiten im Umgang mit generativen KI-Anwendungen deutlich steigt. Neben
klassischen Programmierkenntnissen werden zunehmend auch Kompetenzen im Bereich
der Steuerung, Überwachung und dem Feintuning von KI-basierten Systemen
nachgefragt~\cite{ahmadi_generative_2024}. Insbesondere der sichere und
verantwortungsvolle Einsatz von Tools wie GitHub Copilot, ChatGPT oder
branchenspezifischen KI-Assistenzsystemen entwickelt sich zu einer
Schlüsselqualifikation moderner Entwickler:innen.

Mit der Einführung generativer KI-Tools verändert sich zudem die Rolle von
Entwickler:innen im Team und im Unternehmen. Der Fokus verschiebt sich von der
manuellen Implementierung hin zur strategischen Nutzung und Integration von
KI-Lösungen. Dies umfasst sowohl die Gestaltung von Entwicklungsprozessen als
auch die Bewertung und kontinuierliche Verbesserung von KI-basierten
Ergebnissen~\cite{storey_generative_2025}. Gleichzeitig entstehen neue
Rollenprofile, die Fähigkeiten in den Bereichen Data Science, Human-in-the-Loop
und ethische Bewertung von KI-Systemen erfordern.

Die Veränderungen auf dem Arbeitsmarkt sind jedoch ambivalent: Während
einerseits neue Aufgabenfelder und Qualifikationen entstehen, besteht zugleich
die Gefahr von Verunsicherung und möglichen Substitutionseffekten bei stark
standardisierten Tätigkeiten~\cite{farach_evolving_2025}. Die Literatur weist
darauf hin, dass Beschäftigungsstrukturen und Lohngefüge sich im Zuge der
Automatisierung durch KI nachhaltig wandeln
können~\cite{marguerit_augmenting_2025}.

Nicht zuletzt gewinnt die Fähigkeit zur Kollaboration mit KI-Systemen sowie die
Bereitschaft zu kontinuierlicher Weiterbildung an Bedeutung. Die Arbeitswelt
von Entwickler:innen wird damit vielseitiger, flexibler und erfordert neben
technischem Know-how zunehmend auch soziale und ethische
Kompetenzen~\cite{storey_generative_2025,siebert_generative_2024}.

% \begin{itemize} 
%     \item Verschiebung der gefragten Kompetenzen und Qualifikationen
%     \item Neue Berufsbilder und veränderte Karrierewege
%     \item Auswirkungen auf die Arbeitsplatzsicherheit und die Notwendigkeit der Weiterbildung
% \end{itemize}

\section{Zukunftsperspektiven und strategische Empfehlungen}

Die wissenschaftliche Literatur und aktuelle Branchenanalysen machen deutlich,
dass der Einsatz generativer KI in der Softwareentwicklung erst am Anfang einer
langfristigen Transformationsphase steht. Studien von Deloitte, IBM und
Fraunhofer IESE\cite{siebert_generative_2024, a_ki_2024, s_future_2024}
betonen, dass Unternehmen, die frühzeitig in KI-Kompetenzen, datenbasierte
Prozesse und ethische Leitlinien investieren, sich entscheidende
Wettbewerbsvorteile sichern können.

Eine zentrale Empfehlung der Literatur ist die **kontinuierliche
Weiterbildung** und Entwicklung neuer Rollenprofile. Fraunhofer IESE und
IBM\cite{siebert_generative_2024, a_ki_2024} unterstreichen, dass Unternehmen
gezielt in die Qualifikation ihrer Mitarbeitenden investieren müssen, um den
Anforderungen neuer Technologien gerecht zu werden.
Deloitte\cite{s_future_2024} hebt hervor, dass ein aktives Change Management
und eine offene Innovationskultur entscheidend sind, um Widerstände im Team und
Kompetenzlücken zu überwinden.

Die Literatur betont außerdem die Bedeutung von **Open-Source-Ansätzen** und
kollaborativen Entwicklungsmodellen, um Innovation und Transparenz im
KI-Ökosystem zu fördern\cite{siebert_generative_2024, a_ki_2024}. Zudem spielt
die Entwicklung klarer **ethischer und regulatorischer Leitplanken** eine
strategische Rolle. Insbesondere vor dem Hintergrund unterschiedlicher
rechtlicher Anforderungen und globaler Märkte ist die Ausgestaltung geeigneter
Governance-Strukturen eine zentrale Aufgabe für
Unternehmen\cite{siebert_generative_2024, s_future_2024}.

Die kontinuierliche Weiterentwicklung von Informationssystemen wird als Chance
gesehen, gesellschaftliche Teilhabe und Innovation zu fördern. Fraunhofer
IESE\cite{siebert_generative_2024} argumentiert, dass die gezielte Nutzung
generativer KI nicht nur ökonomische, sondern auch soziale und kreative
Potenziale freisetzen kann, wenn die Integration verantwortungsvoll erfolgt.

Zusammenfassend sind strategische Investitionen in Kompetenzen, Change
Management und Innovationskultur sowie die Entwicklung klarer ethischer
Leitlinien entscheidende Erfolgsfaktoren für die nachhaltige Integration
generativer KI in Unternehmen. Die langfristigen Potenziale können nur dann
ausgeschöpft werden, wenn technologische und organisatorische Transformation
Hand in Hand gehen.

% \begin{itemize}
%     \item Notwendige Anpassungen für Unternehmen und Entwickler
%     \item Möglichkeiten der Integration von KI in bestehende Entwicklungsprozesse
%     \item Regulatorische und ethische Implikationen für eine nachhaltige KI-Nutzung
% \end{itemize}

\section{Kosten-Nutzen-Analyse von KI-gestützter Softwareentwicklung}

Die Integration generativer KI in die Softwareentwicklung bringt sowohl
erhebliche Vorteile als auch neue Kosten- und Risikofaktoren mit sich. Aktuelle
Studien belegen, dass durch den Einsatz von KI-Tools die Produktivität in
vielen Bereichen signifikant steigt – insbesondere bei Routinetätigkeiten, der
Code-Generierung und der Testautomatisierung~\cite{marguerit_augmenting_2025,
    farach_evolving_2025, habibi_open_2025}. Dies führt zu messbaren
Effizienzgewinnen und kann langfristig die Entwicklungskosten pro Feature oder
Release deutlich senken.

Gleichzeitig entstehen neue Investitionen: Die Einführung und Wartung von
KI-Systemen erfordert gezielte Weiterbildung, Anpassungen der Infrastruktur und
oft eine Neuausrichtung bestehender Prozesse. Während proprietäre Lösungen
Lizenz- und Betriebskosten verursachen, sind Open-Source-Modelle zwar
kostengünstiger, erfordern aber häufig einen höheren Initialaufwand für
Anpassung und Integration~\cite{habibi_open_2025}. Song
et~al.~\cite{song_impact_2024} zeigen, dass KI-basierte Assistenzsysteme in
Open-Source-Projekten nicht nur die Kollaboration und Codequalität verbessern,
sondern auch zu einer Demokratisierung des Entwicklungsprozesses beitragen
können.

Zudem hängt der konkrete Nutzen von KI-gestützter Entwicklung maßgeblich davon
ab, inwieweit Unternehmen strategische Ziele, Kostenstruktur und
Wertschöpfungspotenziale aufeinander
abstimmen~\cite{mcnamara_exponential_2025}. Neben direkten Effizienzgewinnen
zählen auch Flexibilität, Innovationspotenzial und die Gewinnung von
Wettbewerbsvorteilen zu den zentralen
Nutzenaspekten~\cite{storey_generative_2025}. Dem stehen potenzielle
Folgekosten durch Fehlinvestitionen, Qualitätsprobleme bei KI-generiertem Code
und ethische Risiken gegenüber, die zu erheblichen finanziellen Belastungen
führen können, wenn sie nicht frühzeitig adressiert werden.

Nicht zuletzt verändern KI-Technologien auch kreative Branchen und ermöglichen
neue Formen gesellschaftlicher Teilhabe~\cite{anantrasirichai_artificial_2025}.
Eine sorgfältige Kosten-Nutzen-Abwägung sowie die kontinuierliche Anpassung der
Strategie bleiben daher zentrale Voraussetzungen für einen erfolgreichen und
nachhaltigen Einsatz generativer KI in der Softwareentwicklung.

% \begin{itemize}
%     \item Analyse der wirtschaftlichen Effizienz und Kostenersparnis
%     \item Vergleich der Investitionskosten und erwarteten Produktivitätsgewinne
%     \item Langfristige wirtschaftliche Auswirkungen für Unternehmen und die Softwarebranche
% \end{itemize}

\section{Fazit}

Die Analyse der wirtschaftlichen und gesellschaftlichen Auswirkungen
generativer KI in der Softwareentwicklung zeigt: Während Unternehmen und
Entwickler:innen deutliche Effizienzgewinne, neue Wertschöpfungsmodelle und
innovative Arbeitsformen erschließen können, entstehen zugleich neue
Herausforderungen für Arbeitsmärkte, Organisationsstrukturen und
Qualifikationsprofile. Die nachhaltige Integration von KI-Tools erfordert daher
nicht nur technologische Investitionen, sondern vor allem eine strategische
Anpassung von Unternehmensstrukturen, kontinuierliche Weiterbildung und die
Etablierung ethischer Leitlinien. Nur so lassen sich die Potenziale generativer
KI langfristig nutzen und soziale Risiken wirksam begrenzen.

% \section{Fazit}

% Die Analyse zeigt, dass der Einsatz generativer KI in der Softwareentwicklung
% weitreichende wirtschaftliche und gesellschaftliche Implikationen hat.
% Unternehmen profitieren von erheblichen Effizienzgewinnen und neuen
% Wertschöpfungspotenzialen, stehen jedoch zugleich vor der Herausforderung, ihre
% Organisationsstrukturen, Rollenmodelle und Qualifikationsprofile an die neue
% technologische Realität anzupassen.

% Die empirischen Befunde aus der Literatur machen deutlich, dass der nachhaltige
% Erfolg von KI-gestützter Entwicklung vor allem von strategischen Investitionen
% in Weiterbildung, flexibles Change Management und eine offene Innovationskultur
% abhängt. Zugleich sind Risiken wie Kostenfallen, ethische Konflikte und
% potenzielle Verwerfungen am Arbeitsmarkt frühzeitig zu adressieren, um die
% positiven Effekte langfristig zu sichern.

% Insgesamt eröffnet die Integration von GenAI-Tools neue Chancen für Unternehmen
% und Gesellschaft, verlangt aber ebenso verantwortungsvolle Gestaltung und eine
% kontinuierliche Anpassung an den rasanten technologischen Wandel.

\chapter{Fazit und Ausblick}

\section{Erwartete Erkenntnisse}
Es wird erwartet, dass KI-gestützte Softwareentwicklung nicht nur Effizienzsteigerungen ermöglicht, sondern auch die Arbeitsweise von Entwicklern nachhaltig verändert. Besonders relevant ist die Frage, inwieweit generative KI langfristig klassische Programmieraufgaben übernimmt oder ergänzt. Darüber hinaus soll analysiert werden, welche Herausforderungen in Bezug auf Sicherheit, Ethik und wirtschaftliche Auswirkungen entstehen.

\section{Zusammenfassung der Erkentnisse}
\input{content/Fazit & Ausblick/07_02_Zusammefassung}

\section{Handlungsempfehlungen und Zukunftsperpektiven}
\input{content/Fazit & Ausblick/07_03_Handlungsempfehlungen}

% Anhang (Bibliographie darf im deutschen nicht in den Anhang!)
\nocite{*} % Include all entries from the BibTeX database
\bibliography{bib/BibtexDatabase} % Reference your BibTeX database
\clearpage
% Symbolverzeichnis (Begriffserklärung)
% Damit die Begriffe hinzugefügt werden muss in Texmaker unter Optionen - Konfiguren 
% Der Makeindex Befehl folgendermaßen geändert werden: "/usr/texbin/makeindex" %.nlo -s nomencl.ist -o %.nls
\IfDefined{printindex}{\printindex}
\IfDefined{printnomenclature}{\printnomenclature}
% 'Symbolverzeichnis' ins Inhaltsverzeichnis
\addcontentsline{toc}{chapter}{Abkürzungsverzeichnis}
% Abbildungs- und Tabellenverzeichnis
\listoffigures
\listoftables
\lstlistoflistings.
% Anhang
\appendix
% 'Anhang' ins Inhaltsverzeichnis
%\phantomsection
%\addcontentsline{toc}{part}{Anhang}

\input{content/Z-Anhang}

%% Dokument ENDE %%%%%%%%%%%%%%%%%%%%%%%%%%%%%%%%%%%%%%%%%%%%%%%%%%%%%%%%%%
\end{document}