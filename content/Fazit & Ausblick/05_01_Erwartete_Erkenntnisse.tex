Es wird erwartet, dass KI-gestützte Softwareentwicklung nicht nur Effizienzsteigerungen ermöglicht, sondern auch zu einer veränderten Arbeitsweise in der Softwarebranche führt. Insbesondere soll untersucht werden, ob und in welchen Bereichen generative KI-Modelle langfristig klassische Programmieraufgaben ergänzen oder gar ersetzen könnten. Darüber hinaus wird erwartet, dass KI-gestützte Tools in bestimmten Entwicklungsphasen, wie etwa der Fehleranalyse oder Code-Optimierung, einen messbaren Mehrwert bieten. Gleichzeitig sollen potenzielle Risiken und Herausforderungen analysiert werden, die sich aus dem verstärkten Einsatz von KI in der Softwareentwicklung ergeben.