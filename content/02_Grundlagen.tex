\chapter{Theoretische Grundlagen}

Dieses Kapitel legt die zentralen theoretischen und technologischen Grundlagen
der Arbeit. Im Mittelpunkt stehen die Definition generativer KI, die
zugrundeliegenden Algorithmen und Modelle sowie ausgewählte Praxisbeispiele für
den Einsatz KI-gestützter Tools in der Softwareentwicklung. Diese Grundlagen
sind essenziell, um die Potenziale und Herausforderungen generativer KI im
weiteren Verlauf der Arbeit differenziert bewerten zu können.

\section{Einordnung generativer KI in die Softwareentwicklung}
\input{content/Grundlagen/02_01_Künstliche_Intelligenz.tex}

\subsection{Beispiele für generative KI-Tools in der Praxis}
Generative KI-Tools sind heute in der Softwareentwicklung weit verbreitet und
decken ein breites Spektrum an Aufgaben ab -- von der Code-Vervollständigung
über automatische Testgenerierung bis hin zur Unterstützung ganzer
Entwicklungsprojekte. Im Folgenden werden vier prominente Tools vorgestellt,
die in der Praxis besonders relevant sind: \textbf{GitHub Copilot},
\textbf{TabNine}, \textbf{Cursor AI} und \textbf{Devin AI}.

\paragraph{GitHub Copilot}
GitHub Copilot ist ein KI-basierter Codeassistent, der Entwicklern direkt im
IDE-Kontext Vorschläge für Code-Snippets, ganze Funktionen und sogar Tests
macht. Die Integration erfolgt beispielsweise in Visual Studio Code, IntelliJ
oder Eclipse. Copilot basiert auf dem OpenAI Codex-Modell und kann mit
Kommentaren oder natürlicher Sprache gesteuert werden. Typische Einsatzfelder
sind das schnelle Prototyping, die Generierung von Boilerplate-Code, aber auch
die Hilfestellung für Einsteiger und Onboarding-Prozesse.

\begin{quote}
    \enquote{GitHub Copilot can assist in quick prototyping of code by generating foundational code structure based on natural language description of the feature. It can assist in boilerplate code generation by providing the class and interface definition generation, API and Database Schema creation. Both of these features combined improve the developer efficiency and enhanced code quality.}
    \cite[S.~8]{donvir_role_2024}
\end{quote}

Praxiserfahrung zeigt, dass GitHub Copilot die Entwicklung etwa einer
React-Anwendung deutlich beschleunigen kann, indem es Codevorschläge für
Authentifizierung, Routing und Formularvalidierung generiert und Fehlerbehebung
unterstützt. Entwickler betonen, dass der Review und die Überprüfung der
generierten Vorschläge dennoch unerlässlich bleiben \cite{kerr_github_nodate}.

\paragraph{TabNine}
TabNine ist ein weiteres KI-gestütztes Tool zur Code-Vervollständigung, das
ursprünglich auf GPT-2 basierte und heute ein eigenes Modell nutzt. Es kann
Codevorschläge in Echtzeit für verschiedene Sprachen machen und passt sich über
die Zeit an den Coding-Stil des jeweiligen Entwicklers an. TabNine unterstützt
alle gängigen IDEs und bietet eine Chat-Funktion, um gezielt Code-Fragen zu
stellen. Besonders geschätzt wird die Flexibilität durch lokale und
cloudbasierte Modelle sowie die Anpassung an unterschiedliche
Datenschutzanforderungen \cite[S.~9]{donvir_role_2024}.

\paragraph{Cursor AI}
Cursor AI repräsentiert die nächste Generation von KI-Entwicklungstools. Es ist
in der Lage, auf Basis natürlicher Sprache ganze Applikationen zu generieren,
nutzt Techniken wie Retrieval-Augmented Generation (RAG) und Agentic AI und
kann selbstständig Code verfeinern und Projekte strukturieren. Der Fokus liegt
auf End-to-End-Entwicklung und vollständiger Projektgenerierung, was
insbesondere für Prototyping oder den schnellen Aufbau komplexer
Softwarelösungen geeignet ist.

\begin{quote}
    \enquote{Cursor AI can generate the entire codebase of the application from the feature description of the project provided in natural language. It uses advanced AI concepts such as Retrieval Augmented Generation (RAG), Agentic AI, and prompt chaining to achieve its objectives and provides a high degree of automation in software development.}
    \cite[S.~10]{donvir_role_2024}
\end{quote}

\paragraph{Devin AI}
Devin AI geht noch einen Schritt weiter und bezeichnet sich selbst als
\enquote{AI Software Engineer}. Devin kann komplette Softwareprojekte auf Basis
von Anforderungen in natürlicher Sprache umsetzen, das Projekt in einzelne
Aufgaben herunterbrechen, automatisiert testen und sogar Deployment-Skripte
erstellen. Ein wesentliches Merkmal ist die Fähigkeit zur langfristigen Planung
und zur kontinuierlichen Anpassung an neue Anforderungen
\cite[S.~11]{donvir_role_2024}.

\vspace{1em}
\noindent
\textbf{Typische Einsatzszenarien}

Die beschriebenen Tools werden in der Praxis in unterschiedlichen Bereichen
eingesetzt:
\begin{itemize}
    \item \textbf{Code-Generierung und Vervollständigung:} Automatisiertes Schreiben von Code, Vorschläge für Funktionen, Klassen, API-Integration etc.
    \item \textbf{Test- und Debugging-Unterstützung:} Generierung von Unit- und Integrationstests, Identifikation von Fehlern und Vorschläge für Bugfixes.
    \item \textbf{Projekt-Scaffolding und Boilerplate:} Automatisches Erstellen von Grundstrukturen für neue Projekte.
    \item \textbf{End-to-End-Entwicklung:} Vollständige Umsetzung von Projektanforderungen inklusive Deployment-Skripten und CI/CD-Konfiguration \cite[S.~11]{donvir_role_2024}.
\end{itemize}

\vspace{1em}
\noindent
\textbf{Vorteile und Grenzen in der Praxis}

Die Integration generativer KI-Tools führt zu Zeitersparnis, konsistenterem
Code und einer schnelleren Einarbeitung neuer Entwickler. Gleichzeitig sind
Review-Prozesse und ein kritischer Umgang mit den generierten Vorschlägen
unerlässlich, um Qualitäts- und Sicherheitsrisiken zu minimieren.
Fortgeschrittene Tools wie Cursor AI oder Devin AI bieten ein hohes Maß an
Automatisierung, sind aber oft kostenintensiv und noch nicht für alle
Anwendungsfälle ausgereift \cite[S.~13]{donvir_role_2024}.

% \vspace{1em}
% \noindent
% \textbf{Quellen:}
% \begin{itemize}
%     \item Donvir, A. et al. (2024): \textit{The Role of Generative AI Tools in
%               Application Development: A Comprehensive Review of Current Technologies and
%               Practices} \cite{donvir_role_2024}
%     \item Kerr, K. (2025): \textit{GitHub for Beginners: Building a React App with GitHub
%               Copilot - The GitHub Blog} \cite{kerr_github_nodate}
% \end{itemize}


\subsection{Wichtige Algorithmen und Modelle in der Softwareentwicklung}
Die Integration generativer Künstlicher Intelligenz (KI) in der
Softwareentwicklung basiert maßgeblich auf fortschrittlichen Modellen und
Algorithmen. Im Zentrum stehen insbesondere Large Language Models (LLMs) und
Transformer-Architekturen, die als Grundlage moderner Coding-Tools wie GitHub
Copilot, Cursor oder v0 dienen. Im Folgenden werden die wichtigsten Modelle,
deren Funktionsweise und deren Bedeutung für typische
Softwareentwicklungsaufgaben erläutert.

\paragraph{Grundprinzipien und Funktionsweise moderner KI-Modelle}

Large Language Models (LLMs), wie beispielsweise GPT-4 oder Code Llama, sind
tiefenlernende neuronale Netze, die auf sehr großen Mengen von Quellcode und
natürlicher Sprache trainiert wurden. Sie nutzen Transformer-Architekturen, die
durch sogenannte Self-Attention-Mechanismen Kontextinformationen über lange
Sequenzen hinweg erfassen können. Dies ermöglicht es ihnen, sowohl syntaktisch
als auch semantisch komplexe Strukturen – wie etwa Code-Logik oder Designmuster
– zu erkennen und zu generieren~\cite{nguyen-duc_generative_2023,
    esposito_generative_2025}.

Diffusionsmodelle spielen hingegen vor allem in der Bild- und Grafikgenerierung
eine Rolle und sind für den textbasierten Softwareentwicklungsprozess bislang
von untergeordneter Bedeutung. Für Aufgaben wie Codegenerierung und
Architektur-Design sind LLMs und Transformer-Modelle
maßgeblich~\cite{weisz_design_2024}.

\paragraph{Beispiele für KI-Modelle in Coding-Tools}

Moderne Coding-Assistenzsysteme wie \textit{GitHub Copilot}, \textit{Cursor}
und \textit{v0} basieren auf Varianten großer Sprachmodelle, meist speziell auf
Programmcode vortrainiert (z.\,B. OpenAI Codex, Code Llama, StarCoder). Diese
Modelle können anhand des jeweiligen Kontexts (z.\,B. bestehender Code,
Kommentare, Projekthistorie) automatisiert neue Code-Abschnitte generieren,
Vorschläge zur Fehlerbehebung machen oder Testfälle
entwerfen~\cite{coutinho_role_2024, esposito_generative_2025}.

Typische Anwendungsbereiche sind:
\begin{itemize}
    \item \textbf{Code-Generierung}: Erstellung neuer Funktionen, Methoden oder ganzer Module auf Basis von Kurzbeschreibungen oder natürlicher Sprache.
    \item \textbf{Testing und Qualitätssicherung}: Automatisierte Generierung von Unit-Tests und Testdaten, Unterstützung beim Review durch Erkennung von Anomalien oder Schwachstellen.
    \item \textbf{Architekturvorschläge}: Unterstützung bei der Auswahl geeigneter Softwarearchitekturen oder Design Patterns, teilweise mit Bezug auf bestehende Anforderungen oder Projektdaten~\cite{esposito_generative_2025}.
\end{itemize}

\paragraph{Rolle dieser Modelle für spezifische Aufgaben}

\begin{itemize}
    \item \textbf{Codegenerierung:} LLMs können auf Grundlage von Prompts oder bestehenden Code-Fragmente eigenständig funktionsfähigen Quellcode erzeugen. Dies umfasst Routineaufgaben (z.\,B. Boilerplate-Code) ebenso wie komplexere Algorithmen.
    \item \textbf{Testing:} Modelle wie GPT-4 sind in der Lage, automatisiert Tests zu erzeugen, Testdaten zu variieren und gängige Fehlerbilder zu erkennen.
    \item \textbf{Architekturvorschläge:} Moderne LLMs unterstützen zunehmend beim Entwurf und bei der Dokumentation von Softwarearchitekturen, indem sie z.\,B. Requirements in Design-Vorschläge oder Diagramme übersetzen~\cite{esposito_generative_2025, nguyen-duc_generative_2023}.
\end{itemize}

\paragraph{Herausforderungen und Grenzen}

Trotz des enormen Potenzials bestehen aktuelle Herausforderungen:
\begin{itemize}
    \item \textbf{Halluzinationen und Fehleranfälligkeit}: Generative Modelle können syntaktisch korrekten, aber fachlich unpassenden oder sogar gefährlichen Code erzeugen.
    \item \textbf{Erklärbarkeit und Transparenz}: Die Nachvollziehbarkeit, wie ein Modell zu bestimmten Ergebnissen kommt, ist oft eingeschränkt~\cite{esposito_generative_2025, nguyen-duc_generative_2023}.
    \item \textbf{Domänenspezifisches Wissen}: Ohne Anpassung (Fine-Tuning) auf projektspezifische Daten sind die Modelle oft auf allgemeines Wissen beschränkt und berücksichtigen spezifische Anforderungen nur begrenzt.
\end{itemize}

\paragraph{Bezug zu den im Praxisteil genutzten Tools}

Die im Praxisteil eingesetzten Tools (\textit{GitHub Copilot, Cursor, v0})
nutzen genau diese Algorithmen, um Entwickler:innen bei alltäglichen
Entwicklungsaufgaben zu unterstützen. Sie liefern damit nicht nur klassische
Code-Vervollständigungen, sondern wirken zunehmend als kollaborative Partner im
gesamten Entwicklungsprozess – von Architektur über Implementierung bis zum
Test~\cite{esposito_generative_2025, nguyen-duc_generative_2023}.



\section{Generative KI-Tools: Funktion und Anwendung}
\label{sec:generative-ki-tools}

Generative KI-Tools wie GitHub Copilot, Cursor oder v0 prägen den modernen
Softwareentwicklungsprozess entscheidend. Ihre Hauptfunktion besteht darin,
natürliche Sprache (Prompts) in ausführbaren Code, Testfälle oder
Dokumentationen umzusetzen. Damit verändern sie nicht nur technische Workflows,
sondern auch die Zusammenarbeit in Entwicklungsteams und die Anforderungen an
Kompetenzen der Beteiligten \cite{weisz_design_2024}.

\subsubsection{Grundfunktion generativer KI-Tools}

Die zugrundeliegenden Large Language Models (LLMs) ermöglichen das sogenannte
Prompt-basiertes Entwicklungsparadigma: Entwickler:innen beschreiben eine
Aufgabe in natürlicher Sprache – das Tool generiert dazu passende Vorschläge.
Diese erscheinen als Code Completion direkt beim Tippen oder als komplette
Funktionsblöcke \cite{kerr_github_nodate, weisz_design_2024}. Neuere Ansätze
wie SENAI integrieren generative KI von Anfang an in den
Software-Engineering-Prozess und erlauben eine hochautomatisierte,
KI-zentrierte Entwicklung \cite{saad_senai_2025}.

Systematische Literaturübersichten zeigen, dass die Integration generativer KI
in Entwicklungsumgebungen das Nutzererlebnis, die Akzeptanz und den praktischen
Nutzen maßgeblich beeinflusst. Insbesondere Aspekte wie Transparenz, Usability
und Feedbackmechanismen entscheiden über den Erfolg im Alltag
\cite{sergeyuk_human-ai_2025}.

\subsubsection{Schnittstellen und Integration in Entwicklungsumgebungen}

Die praktische Nutzung generativer KI erfolgt meist über Plugins (z. B. für
Visual Studio Code oder JetBrains IDEs) oder via API. Typischerweise
interagieren Entwickler:innen mit dem KI-Tool im Kontext der gewohnten
Umgebung, wodurch Aufgaben wie Refactoring, Testing oder Dokumentation direkt
integriert werden können \cite{kerr_github_nodate, shi_ai-assisted_2023,
    weisz_design_2024}.

Besonderes Augenmerk liegt inzwischen auch auf Barrierefreiheit: KI-gestützte
Assistenzsysteme eröffnen etwa Entwickelnden mit Sehbeeinträchtigung neue
Teilhabemöglichkeiten, vorausgesetzt die Tools sind entsprechend gestaltet
\cite{flores-saviaga_impact_2025}.

\subsubsection{Beispielhafte Workflows: Pair Programming mit Copilot}

Beim Pair Programming mit Copilot werden Aufgaben in Form von Prompts gestellt,
Copilot generiert daraufhin Codevorschläge, die geprüft, angepasst oder
verworfen werden können. So unterstützt Copilot u. a. die Testautomatisierung,
das Refactoring und die Dokumentation in allen Phasen des
Entwicklungsprozesses. Studien belegen signifikante Beschleunigungseffekte bei
Routineaufgaben \cite{kerr_github_nodate, weisz_design_2024,
    shi_ai-assisted_2023}. In agilen Teams kann generative KI zudem zur
Qualitätsbewertung und Dokumentation von Anforderungen beitragen
\cite{geyer_case_2025}.

\subsubsection{Vorteile und Optimierungspotenziale}

Der Einsatz generativer KI bietet zahlreiche Vorteile:
\begin{itemize}
    \item \textbf{Zeitersparnis:} Automatisierung von Routineaufgaben, Verkürzung von Entwicklungszyklen.
    \item \textbf{Verbesserte Codequalität:} Erkennung häufiger Fehler, Empfehlungen für Best Practices.
    \item \textbf{Niedrigere Einstiegshürden:} Unterstützung auch für weniger erfahrene Entwickler:innen durch kontextbasierte Vorschläge.
\end{itemize}
KI-gestützte Code Reviews fördern die Kollaboration zwischen Mensch und Maschine und erhöhen die Transparenz im Entwicklungsprozess \cite{alami_human_2025}. Kontinuierliche Modellverbesserung und flexible Integration in unterschiedliche Projekte steigern das Optimierungspotenzial weiter \cite{kerr_github_nodate, weisz_design_2024}.

\subsubsection{Grenzen und typische Fehlerquellen}

Zu den zentralen Herausforderungen zählen:
\begin{itemize}
    \item \textbf{Halluzinationen:} Syntaktisch korrekter, aber inhaltlich fehlerhafter oder unsicherer Code, vor allem bei vagen Prompts \cite{shi_ai-assisted_2023}.
    \item \textbf{Bias und Kontextdefizite:} Übernahme von Vorurteilen oder Nichtbeachtung spezifischer Projektregeln.
    \item \textbf{Sicherheitsrisiken:} Vorschläge für unsicheren Code (z. B. Hardcoded Credentials), die explizit überprüft werden müssen \cite{shi_ai-assisted_2023}.
    \item \textbf{Übermäßiges Vertrauen:} Unkritische Übernahme von KI-Vorschlägen ohne Review durch erfahrene Entwickler:innen \cite{weisz_design_2024}.
\end{itemize}

\subsubsection{Designprinzipien für den produktiven und sicheren Einsatz}

Für die sichere Nutzung generativer KI-Tools werden in der Literatur folgende
Designprinzipien empfohlen \cite{weisz_design_2024}:
\begin{itemize}
    \item \textbf{Design for Mental Models:} Nutzer:innen sollen die Funktionsweise und Grenzen nachvollziehen können.
    \item \textbf{Design for Appropriate Trust \& Reliance:} Feedbackmechanismen müssen sowohl Vertrauen fördern als auch zur kritischen Prüfung anregen.
    \item \textbf{Design for Imperfection:} Aktive Hinweise auf mögliche Fehler und die Befähigung zur Korrektur sind essenziell.
\end{itemize}

Die Gestaltung generativer GUIs und Entwicklungsprozesse muss diese Prinzipien
berücksichtigen, um eine nachhaltige und verantwortungsvolle Integration zu
ermöglichen \cite{lee_towards_2025, chen_genui_2025, gill_agile_2025}.
Flexibilität, Feedback und kontinuierliche Anpassung sind Schlüssel zum Erfolg.

% Optional: Kurzbezug auf Kapitel 3 (Praxisbeispiel), falls erwünscht
% Wie in Kapitel 3 praktisch demonstriert, lassen sich diese Prinzipien in der Entwicklung von React Native-Anwendungen mit KI-Tools exemplarisch umsetzen.



