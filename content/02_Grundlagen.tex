\chapter{Theoretische Grundlagen}

Dieses Kapitel legt die zentralen theoretischen und technologischen Grundlagen
der Arbeit. Im Mittelpunkt stehen die Definition generativer KI, die
zugrundeliegenden Algorithmen und Modelle sowie ausgewählte Praxisbeispiele für
den Einsatz KI-gestützter Tools in der Softwareentwicklung. Diese Grundlagen
sind essenziell, um die Potenziale und Herausforderungen generativer KI im
weiteren Verlauf der Arbeit differenziert bewerten zu können.

\section{Einordnung generativer KI in die Softwareentwicklung}

Künstliche Intelligenz (KI) leitet einen grundlegenden Wandel in der
Softwareentwicklung ein. Die aktuelle, rechtlich verbindliche Definition der
Europäischen Union beschreibt KI als maschinengestützte Systeme, die mit
unterschiedlichem Grad an Autonomie Vorhersagen, Empfehlungen oder
Entscheidungen generieren, welche physische oder virtuelle Umgebungen
beeinflussen können \cite{noauthor_verordnung_nodate}. Diese Definition
etabliert sich zunehmend als Referenzrahmen in Forschung und Praxis.

Moderne KI-Systeme unterscheiden sich deutlich von traditionellen
softwarebasierten Ansätzen. Während frühe KI-Tools vor allem Aufgaben wie
Syntaxprüfung unterstützten, übernehmen generative KI-Systeme (GenAI) heute
komplexe Aufgaben im Design, der Entwicklung und Wartung von Software, etwa
durch die automatische Generierung von Code-Snippets, Unterstützung bei
Unit-Tests oder die Automatisierung von Deployment-Prozessen
\cite{donvir_role_2024}.

Zu den wichtigsten Architekturen zählen Transformer-Modelle (insbesondere Large
Language Models wie GPT), Generative Adversarial Networks (GANs), Variational
Autoencoders (VAEs) und Diffusion Models. Während LLMs primär für Text- und
Codegenerierung eingesetzt werden, kommen GANs und Diffusion Models vor allem
in der Bild- und Medienerzeugung zum Einsatz \cite{donvir_role_2024}. Viele
aktuelle Tools wie ChatGPT, GitHub Copilot oder Stable Diffusion basieren auf
diesen Architekturen und treiben Innovationen in der Softwareentwicklung
maßgeblich voran.

Der Einsatz generativer KI verändert Methoden und Arbeitsweisen grundlegend.
Besonders Large Language Models prägen sämtliche Phasen des
Softwareentwicklungszyklus, von der Anforderungsanalyse bis zur Umsetzung und
Wartung. Entscheidungsunterstützung, die Rekonstruktion von
Softwarearchitekturen sowie Methoden wie Few-Shot-Prompting und
Retrieval-Augmented Generation (RAG) sind dabei zentrale Elemente
\cite{esposito_generative_2025}.

Trotz fortschreitender Automatisierung bleibt die sorgfältige Validierung der
von KI generierten Ergebnisse durch Entwickler:innen unerlässlich. Die größten
Herausforderungen betreffen Präzision und Verlässlichkeit der Modelle, den
Umgang mit sogenannten „Halluzinationen“, ethische Aspekte sowie das Fehlen
domänenspezifischer Benchmarks und Standards \cite{esposito_generative_2025}.

Die Entwicklung der Softwaretechnik unter dem Einfluss von KI lässt sich grob
in drei Phasen gliedern. Während Software Engineering~1.0 klassische,
code-zentrierte Entwicklungspraktiken beschreibt, kennzeichnet SE~2.0 die
Integration KI-gestützter Assistenzsysteme (z.\,B. Copilots) in etablierte
Prozesse. Die Vision von SE~3.0 sieht einen Paradigmenwechsel zu einer
intent-basierten, dialogorientierten Entwicklung vor, in der KI-Systeme als
intelligente Partner agieren~\cite{hassan_towards_2024}.

\begin{table}[H]
    \centering
    \vspace{1em}
    \footnotesize
    \begin{tabular}{|p{3cm}|p{3.3cm}|p{3.3cm}|p{3.3cm}|}
        \hline
                                                 & \textbf{SE 1.0 (Vergangenheit)} & \textbf{SE 2.0 (Gegenwart)} & \textbf{SE 3.0 (Zukunft)} \\
        \hline
        Leitmotiv                                &
        Code-first                               &
        Code-first + KI-Unterstützung            &
        Intent-first, KI als Partner                                                                                                         \\
        \hline
        Technologie                              &
        Klassische Tools, Programmanalyse        &
        KI-gestützte Copilots, Foundation Models &
        Konversationsorientiert, Reasoning-Modelle                                                                                           \\
        \hline
        Rolle von Mensch/KI                      &
        Mensch zentral, alles manuell            &
        Mensch + KI, KI assistiert               &
        Symbiose: Mensch \& KI kollaborieren                                                                                                 \\
        \hline
        Besonderheiten                           &
        Fokus auf Code, klare Abläufe            &
        Effizienzsteigerung, hohe kognitive Last &
        Dialog, Automatisierung, wissensgetrieben                                                                                            \\
        \hline
    \end{tabular}
    \caption{Entwicklung der Softwaretechnik unter KI-Einfluss (SE~1.0 bis SE~3.0). Quelle: Eigene Darstellung in Anlehnung an Hassan et al.~\cite{hassan_towards_2024}.}
    \label{tab:se-evolution}
\end{table}

% \vspace{1em}
% \noindent
% \textbf{Quellen:}
% \begin{itemize}
%     \item Verordnung (EU) 2024/1689 des Europäischen Parlaments und des Rates
%           \cite{noauthor_verordnung_nodate}
%     \item Donvir, A. et al. (2024): \textit{The Role of Generative AI Tools in
%               Application Development: A Comprehensive Review of Current Technologies and
%               Practices} \cite{donvir_role_2024}
%     \item Esposito, M. et al. (2025): \textit{Generative AI for Software Architecture.
%               Applications, Trends, Challenges, and Future Directions}
%           \cite{esposito_generative_2025}
% \end{itemize}


\subsection{Beispiele für generative KI-Tools in der Praxis}

Generative KI-Tools sind mittlerweile aus der Softwareentwicklung nicht mehr
wegzudenken und decken ein breites Spektrum an Aufgaben ab, von der
Code-Vervollständigung über automatische Testgenerierung bis hin zur
Unterstützung ganzer Entwicklungsprojekte \cite{donvir_role_2024}. Zu den
praxisrelevanten Vertretern zählen unter anderem GitHub Copilot, TabNine,
Cursor AI und Devin AI.

\textit{GitHub Copilot}
ist ein KI-basierter Codeassistent, der Entwickelnden direkt im Kontext der
Entwicklungsumgebung Vorschläge für Code-Snippets, komplette Funktionen und
sogar Tests unterbreitet. Die Integration erfolgt nahtlos in gängige IDEs wie
Visual Studio Code, IntelliJ oder Eclipse. Das zugrunde liegende Modell –
OpenAI Codex – wird mit Kommentaren oder natürlicher Sprache angesteuert.
Typische Anwendungsfelder sind schnelles Prototyping, die Generierung von
Boilerplate-Code und die Unterstützung beim Onboarding neuer Teammitglieder
\cite{donvir_role_2024}. Praxiserfahrungen zeigen, dass Copilot die Entwicklung
– etwa von React-Anwendungen – erheblich beschleunigen kann, indem es gezielte
Codevorschläge für Authentifizierung, Routing oder Formularvalidierung liefert
und bei der Fehlerbehebung unterstützt. Dennoch bleibt eine kritische
Überprüfung der KI-Vorschläge unerlässlich \cite{kerr_github_nodate}.

\textit{TabNine}
ist ein weiteres KI-gestütztes Tool zur Code-Vervollständigung, das
ursprünglich auf GPT-2 basierte und mittlerweile ein eigenes Modell verwendet.
Es generiert Codevorschläge in Echtzeit für zahlreiche Programmiersprachen und
passt sich sukzessive dem Stil der jeweiligen Entwicklerperson an. TabNine
unterstützt alle gängigen Entwicklungsumgebungen und bietet zusätzlich eine
Chat-Funktion für gezielte Code-Fragen. Die Flexibilität durch wahlweise lokale
oder cloudbasierte Modelle wird besonders im Hinblick auf unterschiedliche
Datenschutzanforderungen geschätzt \cite{donvir_role_2024}.

\textit{Cursor AI}
steht für die nächste Generation KI-basierter Entwicklungstools. Es kann auf
Grundlage natürlicher Sprache vollständige Applikationen generieren und nutzt
dabei fortschrittliche Ansätze wie Retrieval-Augmented Generation (RAG) und
Agentic AI. Die Stärke von Cursor AI liegt in der End-to-End-Generierung
kompletter Projekte, was insbesondere für schnelles Prototyping oder den Aufbau
komplexer Softwarelösungen von Vorteil ist \cite{donvir_role_2024}.

\textit{Devin AI}
geht noch einen Schritt weiter und versteht sich als \enquote{AI Software
    Engineer}. Das Tool setzt komplette Softwareprojekte auf Basis natürlicher
Sprache um, bricht Anforderungen in Aufgaben herunter, automatisiert
Testprozesse und erstellt Deployment-Skripte. Besonders hervorzuheben ist die
Fähigkeit von Devin, langfristige Planungen umzusetzen und kontinuierliche
Anpassungen an neue Anforderungen vorzunehmen \cite{donvir_role_2024}.

\vspace{1em}
\noindent
\textbf{Typische Einsatzszenarien}

In der Praxis kommen nach \cite{donvir_role_2024} diese Tools in verschiedenen
Bereichen zum Einsatz:
\begin{itemize}
    \item \textbf{Code-Generierung und Vervollständigung:} Automatisiertes Schreiben von Code, Vorschläge für Funktionen, Klassen oder API-Integrationen.
    \item \textbf{Test- und Debugging-Unterstützung:} Generierung von Unit- und Integrationstests, Erkennung von Fehlern und Vorschläge für Bugfixes.
    \item \textbf{Projekt-Scaffolding und Boilerplate:} Automatisches Erstellen von Grundstrukturen für neue Projekte.
    \item \textbf{End-to-End-Entwicklung:} Vollständige Umsetzung von Projektanforderungen inklusive Deployment-Skripten und CI/CD-Konfiguration.
\end{itemize}

\vspace{1em}
\noindent
\textbf{Vorteile und Grenzen in der Praxis}

Die Integration generativer KI-Tools führt nachweislich zu erheblichen
Zeitersparnissen, konsistenterem Code und einer schnelleren Einarbeitung neuer
Teammitglieder. Gleichzeitig bleiben strukturierte Review-Prozesse und ein
kritischer Umgang mit KI-generierten Vorschlägen unverzichtbar, um Qualitäts-
und Sicherheitsrisiken zu minimieren. Fortgeschrittene Werkzeuge wie Cursor AI
oder Devin AI bieten ein hohes Maß an Automatisierung, sind jedoch häufig
kostenintensiv und nicht in jedem Anwendungsfall ausgereift
\cite{donvir_role_2024}.

% \vspace{1em}
% \noindent
% \textbf{Quellen:}
% \begin{itemize}
%     \item Donvir, A. et al. (2024): \textit{The Role of Generative AI Tools in
%               Application Development: A Comprehensive Review of Current Technologies and
%               Practices} \cite{donvir_role_2024}
%     \item Kerr, K. (2025): \textit{GitHub for Beginners: Building a React App with GitHub
%               Copilot - The GitHub Blog} \cite{kerr_github_nodate}
% \end{itemize}


\subsection{Wichtige Algorithmen und Modelle in der Softwareentwicklung}
% TODO : Quellenangaben für absatz?
% Die Integration generativer Künstlicher Intelligenz (KI) in die
% Softwareentwicklung beruht maßgeblich auf dem Einsatz fortschrittlicher Modelle
% und Algorithmen. Insbesondere Large Language Models (LLMs) und
% Transformer-Architekturen bilden die technologische Grundlage moderner
% Coding-Tools wie GitHub Copilot, Cursor oder v0. Im Folgenden werden zentrale
% Modelle, deren Funktionsweise und Bedeutung für typische Entwicklungsaufgaben
% dargestellt.

\label{sec:wichtige-algorithmen-modelle}

Aktuelle Forschung betont, dass ein zukunftsorientiertes Software-Ökosystem
nicht nur technologische Innovation, sondern auch eine optimierte
Zusammenarbeit von Mensch und KI erfordert. Ein systematischer Umgang mit
technischer Verschuldung sowie die gezielte Nutzung externer Wissensquellen
gelten als Schlüsselfaktoren moderner Entwicklungsframeworks
\cite{matsumoto_conceptual_2021}. Darüber hinaus zeigen empirische Studien,
dass KI-Assistenzsysteme die Codequalität und Wartbarkeit nachweislich
verbessern können \cite{martinovic_impact_2024}. Der Ansatz des sogenannten
\glqq AI-Native Software Engineering\grqq{} (SE 3.0) fokussiert eine enge
Verzahnung von KI, Entwicklerkompetenz und Geschäftsprozessen und steht für
einen Wandel hin zu einer kooperativen, dialogorientierten Softwareentwicklung
\cite{hassan_towards_2024}.

Im Kern basieren moderne KI-Tools wie GitHub Copilot, Cursor oder Bolt auf
sogenannten Large Language Models (LLMs) und verwandten Architekturen wie
Transformers. Diese tiefenlernenden neuronalen Netze werden auf großen Mengen
an Quellcode und natürlicher Sprache trainiert und nutzen Mechanismen wie
Self-Attention, um den Kontext über längere Sequenzen hinweg zu erfassen. So
gelingt es den Modellen, sowohl syntaktisch als auch semantisch komplexe
Strukturen zu erkennen und zu generieren, etwa im Bereich der Code-Logik oder
bei der Anwendung typischer Designmuster \cite{nguyen-duc_generative_2023,
    esposito_generative_2025}. Während Diffusionsmodelle in der Bild- und
Mediengenerierung dominieren, bleiben für text- und codebasierte
Softwareentwicklung weiterhin LLMs und Transformer-Architekturen zentral
\cite{weisz_design_2024}.

In der praktischen Anwendung nutzen moderne Assistenzsysteme typischerweise
spezialisierte Sprachmodelle wie OpenAI Codex, Code Llama oder StarCoder, die
gezielt auf Programmcode vortrainiert wurden. Diese Modelle sind in der Lage,
ausgehend vom Kontext, etwa bestehender Code, Kommentare oder Projekthistorie,
automatisiert neue Code-Abschnitte zu generieren, Fehlerkorrekturen
vorzuschlagen oder passende Testfälle zu erstellen \cite{coutinho_role_2024,
    esposito_generative_2025}. Ihr Funktionsspektrum reicht von der automatischen
Erstellung kompletter Funktionen und Module auf Basis kurzer Beschreibungen in
natürlicher Sprache über die Generierung von Unit-Tests und die Unterstützung
beim Review bis hin zur Entwicklung und Auswahl geeigneter
Softwarearchitekturen oder Design Patterns, insbesondere im projektspezifischen
Kontext \cite{coutinho_role_2024, esposito_generative_2025, donvir_role_2024}.

LLMs übernehmen damit zunehmend Aufgaben, die von der reinen Code-Generierung
bis hin zu komplexeren Anforderungen reichen. Sie erstellen automatisiert
Tests, variieren Testdaten, erkennen typische Fehlerbilder und unterstützen
Entwickler:innen durch Vorschläge zur Architektur oder durch das Übersetzen von
Requirements in konkrete Design-Entwürfe oder Diagramme
\cite{esposito_generative_2025, nguyen-duc_generative_2023}.

Trotz des enormen Potenzials solcher Modelle sind wesentliche Herausforderungen
zu beachten. So können LLMs zwar häufig syntaktisch korrekten Code generieren,
dieser ist jedoch nicht immer inhaltlich passend oder sicher, insbesondere bei
vagen Prompts oder fehlendem Kontext (\textit{Halluzinationen}). Die
Erklärbarkeit und Transparenz der generierten Vorschläge bleibt oftmals
eingeschränkt, da die Entscheidungswege der Modelle schwer nachvollziehbar
sind. Ohne gezieltes Fine-Tuning auf spezifische Projekte oder Domänen bleibt
das Wissen zudem meist allgemein und kann individuellen Anforderungen nicht
immer gerecht werden \cite{esposito_generative_2025,
    nguyen-duc_generative_2023, donvir_role_2024}.

Im Praxisteil dieser Arbeit werden die beschriebenen Modelle durch Tools wie
GitHub Copilot, Cursor und Bolt eingesetzt, um Entwickler:innen in sämtlichen
Phasen der Softwareentwicklung, von Architekturentwurf bis Testing, aktiv zu
unterstützen. Diese Werkzeuge etablieren sich damit zunehmend als kollaborative
Partner und verändern klassische Entwicklungsprozesse nachhaltig
\cite{esposito_generative_2025, nguyen-duc_generative_2023}.



\section{Generative KI-Tools: Funktion und Anwendung}
\label{sec:generative-ki-tools}

Generative KI-Tools wie GitHub Copilot, Cursor oder v0 prägen den modernen
Softwareentwicklungsprozess entscheidend. Ihre Hauptfunktion besteht darin,
natürliche Sprache, sogenannte Prompts, in ausführbaren Code, Testfälle oder
Dokumentationen umzusetzen. Damit verändern sie sowohl technische Workflows als
auch die Zusammenarbeit in Entwicklungsteams und stellen neue Anforderungen an
die Kompetenzen der Beteiligten \cite{weisz_design_2024}.

Die zugrundeliegenden Large Language Models (LLMs) ermöglichen ein
Prompt-basiertes Entwicklungsparadigma: Entwickler:innen beschreiben Aufgaben
in natürlicher Sprache und erhalten daraufhin passende Vorschläge. Diese
erscheinen entweder als \textit{Code Completion} direkt beim Tippen oder als
vollständige Funktionsblöcke \cite{kerr_github_nodate, weisz_design_2024}.
Neuere Ansätze wie SENAI integrieren generative KI von Beginn an in den
Software-Engineering-Prozess und erlauben so eine hochautomatisierte,
KI-zentrierte Entwicklung \cite{saad_senai_2025}.

Systematische Literaturübersichten zeigen, dass die Integration generativer KI
in Entwicklungsumgebungen das Nutzererlebnis, die Akzeptanz und den praktischen
Nutzen maßgeblich beeinflusst. Besonders Faktoren wie Transparenz, Usability
und intelligente Feedbackmechanismen sind entscheidend für den Erfolg im Alltag
\cite{sergeyuk_human-ai_2025}. Die praktische Nutzung generativer KI erfolgt
heute meist direkt über Plugins für etablierte IDEs wie Visual Studio Code oder
JetBrains, oftmals auch über API-Schnittstellen. Dadurch lassen sich Aufgaben
wie Refactoring, Testing oder Dokumentation unmittelbar und nahtlos in die
gewohnte Arbeitsumgebung integrieren \cite{kerr_github_nodate,
    shi_ai-assisted_2023, weisz_design_2024}. Besonders hervorzuheben ist zudem,
dass KI-gestützte Assistenzsysteme neue Teilhabemöglichkeiten für
sehbeeinträchtigte Entwickler:innen eröffnen können, vorausgesetzt, die Tools
sind barrierefrei gestaltet \cite{flores-saviaga_impact_2025}.

Ein typischer Workflow zeigt sich etwa beim Pair Programming mit Copilot:
Aufgaben werden in Form von Prompts gestellt, das Tool generiert passende
Codevorschläge, die geprüft, angepasst oder verworfen werden können. Studien
belegen, dass diese Arbeitsweise Routineaufgaben wie Testautomatisierung,
Refactoring und Dokumentation signifikant beschleunigt
\cite{kerr_github_nodate, weisz_design_2024, shi_ai-assisted_2023}. In agilen
Teams kann generative KI zudem einen Beitrag zur Qualitätsbewertung und zur
Dokumentation von Anforderungen leisten \cite{geyer_case_2025}.

Der Einsatz generativer KI bietet zahlreiche Vorteile, die in aktuellen Studien
hervorgehoben werden: So wird insbesondere die Automatisierung von
Routineaufgaben und die damit verbundene Zeitersparnis hervorgehoben, genauso
wie die Verbesserung der Codequalität durch die Erkennung häufiger Fehler und
gezielte Empfehlungen für Best Practices. Außerdem werden niedrigere
Einstiegshürden für weniger erfahrene Entwickler:innen genannt, denen durch
kontextbasierte Vorschläge der Zugang zur Entwicklung erleichtert wird
\cite{donvir_role_2024, sergeyuk_human-ai_2025}. Darüber hinaus fördern
KI-gestützte Code Reviews die Kollaboration zwischen Mensch und Maschine und
erhöhen die Transparenz im Entwicklungsprozess \cite{alami_human_2025}. Durch
kontinuierliche Modellverbesserung und flexible Integration in unterschiedliche
Projekte kann das Optimierungspotenzial weiter gesteigert werden
\cite{kerr_github_nodate, weisz_design_2024}.

Trotz dieser Vorteile bestehen zentrale Herausforderungen. Ein häufig
diskutiertes Problem sind sogenannte \textit{Halluzinationen}: Modelle
generieren zwar syntaktisch korrekten, inhaltlich aber fehlerhaften oder
unsicheren Code, besonders bei unscharfen oder vagen Prompts
\cite{shi_ai-assisted_2023, weisz_design_2024}. Hinzu kommen Bias und
Kontextdefizite, also die Übernahme von Vorurteilen aus den Trainingsdaten oder
das Nichtbeachten projektspezifischer Regeln. Ein weiteres Risiko besteht in
der Generierung unsicheren Codes, etwa durch Vorschläge mit hardcodierten
Zugangsdaten, die explizit überprüft werden müssen. Zudem beobachten Studien
ein übermäßiges Vertrauen vieler Entwickler:innen in die Vorschläge der KI,
eine unkritische Übernahme ohne manuelle Überprüfung kann zu schwerwiegenden
Fehlern führen \cite{shi_ai-assisted_2023, weisz_design_2024}.

Um einen produktiven und sicheren Einsatz zu gewährleisten, empfiehlt die
Literatur spezifische Designprinzipien für generative KI-Tools
\cite{weisz_design_2024}: Erstens sollte die Gestaltung der Systeme so
erfolgen, dass Nutzer:innen die Funktionsweise und Grenzen nachvollziehen
können (\textit{Design for Mental Models}). Zweitens müssen Feedbackmechanismen
sowohl Vertrauen fördern als auch zur kritischen Prüfung anregen
(\textit{Design for Appropriate Trust \& Reliance}). Drittens ist die
Berücksichtigung von Fehlern essenziell: Tools sollten aktiv auf mögliche
Fehler hinweisen und die Nutzer:innen zur Korrektur befähigen (\textit{Design
    for Imperfection}). Die Gestaltung generativer GUIs und Entwicklungsprozesse
muss diese Prinzipien berücksichtigen, um eine nachhaltige und
verantwortungsvolle Integration zu ermöglichen \cite{lee_towards_2025,
    chen_genui_2025, gill_agile_2025}. Flexibilität, Feedback und kontinuierliche
Anpassung sind hier Schlüsselfaktoren.

Insgesamt zeigt sich, dass der Mehrwert generativer KI-Tools vor allem dann zum
Tragen kommt, wenn sie sinnvoll in bestehende Entwicklungsumgebungen
integriert, kritisch überprüft und auf die jeweiligen Team- und
Projektanforderungen angepasst werden.

% Optional: Kurzbezug auf Kapitel 3 (Praxisbeispiel), falls erwünscht
% Wie in Kapitel 3 praktisch demonstriert, lassen sich diese Prinzipien in der Entwicklung von React Native-Anwendungen mit KI-Tools exemplarisch umsetzen.



