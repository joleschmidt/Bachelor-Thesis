\chapter{Chancen}
Die Integration generativer KI-Technologien bietet vielfältige Chancen für die
moderne Softwareentwicklung. Neben der Automatisierung repetitiver Aufgaben
ermöglichen KI-Tools eine signifikante Steigerung der Entwicklungseffizienz und
eröffnen innovative Arbeitsweisen. Die praktischen Erfahrungen aus Kapitel~3
zeigen, dass der produktive Einsatz von KI nicht nur zu Zeitersparnis führt,
sondern auch die Qualität und Wartbarkeit des Codes verbessern kann. Im
Folgenden werden die zentralen Potenziale generativer KI im Entwicklungsprozess
systematisch analysiert und anhand von Literatur und Praxiserfahrungen
bewertet.

\section{Effizienzsteigerung und Automatisierung}
Zahlreiche Studien und Fallanalysen bescheinigen generativen KI-Tools das
Potenzial, die Effizienz im Entwicklungsprozess maßgeblich zu
steigern~\cite{donvir_role_2024,coutinho_role_2024,s_future_2024,esposito_generative_2025,braun_ki_2024,siebert_generative_2024}.
Auch in der eigenen praktischen Demonstration (vgl. Kapitel~3) zeigte sich,
dass Werkzeuge wie GitHub Copilot oder Cursor repetitive Aufgaben wie das
Erstellen von Boilerplate-Code, Standardkomponenten oder einfachen UI-Logiken
erheblich beschleunigen können. So konnte das Grundgerüst des Map-Screens in
der Locals-App mit Unterstützung von Copilot innerhalb weniger Minuten
generiert werden, während vergleichbare Aufgaben ohne KI deutlich
zeitaufwändiger wären.

Aktuelle Literatur und Praxisberichte belegen, dass der gezielte Einsatz
generativer KI-Tools zu signifikanten Effizienzsteigerungen in der
Softwareentwicklung führt. Donvir et~al.~\cite{donvir_role_2024} betonen, dass
moderne Coding-Assistenzsysteme wie Copilot oder Cursor insbesondere bei
repetitiven Aufgaben für eine starke Beschleunigung sorgen. Auch die Fallstudie
von Coutinho et~al.~\cite{coutinho_role_2024} weist nach, dass sich die
Entwicklungszeit bei Routineaufgaben durch KI-gestützte Werkzeuge deutlich
verringert. Sulabh~\cite{s_future_2024} und das Fraunhofer
IESE~\cite{siebert_generative_2024} berichten von Effizienzgewinnen von bis zu
50\,\%. Esposito et~al.~\cite{esposito_generative_2025} unterstreichen zudem,
dass der Einsatz von Large Language Models neue Automatisierungs- und
Optimierungsmöglichkeiten eröffnet.

\begin{quote}
    \enquote{GitHub Copilot can assist in quick prototyping of code by generating foundational code structure based on natural language description of the feature. It can assist in boilerplate code generation by providing the class and interface definition generation, API and Database Schema creation. Both of these features combined improve the developer efficiency and enhanced code quality.}
    \cite[S.~4]{donvir_role_2024}
\end{quote}

Generative KI-Tools wirken sich auf sämtliche Phasen des
Softwareentwicklungsprozesses aus – von der Planung über die Implementierung
bis hin zu Test und Deployment – und eröffnen dadurch neue Potenziale für die
Effizienzsteigerung~\cite{minikiewicz_impact_nodate}. Feldexperimente mit
Softwareentwickler:innen bestätigen, dass sich der Einsatz solcher Werkzeuge
unmittelbar positiv auf Produktivität und Arbeitsweise
auswirkt~\cite{cui_effects_2024}.

Auch komplexere Aufgaben wie Debugging oder die automatische Anpassung von
Datenstrukturen profitieren von KI-Unterstützung, wie insbesondere der
Vergleich zwischen Copilot und Cursor verdeutlicht. Die Literatur verweist
dabei auf Effizienzsteigerungen von bis zu 50\,\% bei
Routinetätigkeiten~\cite{s_future_2024}, was sich mit den im Praxisteil
beobachteten Zeitersparnissen und Produktivitätsgewinnen deckt.

Die Qualität der Automatisierung bleibt jedoch stark abhängig von der Präzision
der Prompts und der Kontextintegration der eingesetzten Tools. Wie die Arbeit
mit Cursor gezeigt hat, ist gerade bei komplexeren Aufgaben ein dialogischer
Ansatz mit Feedback-Loops und manueller Kontrolle weiterhin unverzichtbar.
Dennoch legen sowohl Forschung als auch Praxis nahe, dass generative KI einen
spürbaren Effizienzgewinn im Entwicklungsalltag ermöglicht.
Wangoo~\cite{wangoo_artificial_2018} hebt hervor, dass KI-Technologien nicht
nur den Entwicklungsprozess beschleunigen, sondern auch die Wiederverwendung
bestehender Komponenten und das Design von Software nachhaltig vereinfachen
können.


\section{Neue Werkzeuge und Methoden}
Der verstärkte Einsatz generativer KI hat in den letzten Jahren eine Vielzahl
neuer Werkzeuge und Methoden in der Softwareentwicklung etabliert. Besonders
die Integration von Large Language Models (LLMs) in Entwicklungsumgebungen hat
die Art, wie Entwickler*innen arbeiten, maßgeblich verändert.

Zu den wichtigsten Werkzeugen zählen unter anderem \textbf{GitHub Copilot},
\textbf{Cursor AI}, \textbf{Amazon CodeWhisperer} und \textbf{Devin AI}. Diese
Tools werden in der Literatur umfassend dargestellt und spielen laut Esposito
et al.~\cite{esposito_generative_2025} sowie Nguyen-Duc et
al.~\cite{duc_generative_2023} eine zentrale Rolle in der aktuellen
Entwicklungspraxis.

GitHub Copilot wird besonders häufig eingesetzt und unterstützt
Entwickler*innen bei der automatischen Codegenerierung und Vervollständigung
direkt in der IDE. Esposito et al.~\cite[S.~2]{esposito_generative_2025}
beschreiben, dass solche Werkzeuge zunehmend in frühen Phasen des
Entwicklungsprozesses verwendet werden, etwa beim Übergang von Anforderungen zu
Architektur oder bei der Erstellung von Code aus natürlichsprachigen
Beschreibungen.

Cursor AI und ähnliche Tools ermöglichen einen dialogorientierten Workflow, bei
dem nicht nur einzelne Codezeilen, sondern ganze Features, Module oder sogar
Projekte automatisch erstellt und verfeinert werden können. Dabei kommen
Methoden wie Prompt Engineering, Retrieval-Augmented Generation (RAG) und
agentenbasierte Ansätze zum Einsatz (vgl. Esposito et
al.,~\cite[S.~3--4]{esposito_generative_2025}).

Im praktischen Teil dieser Arbeit (vgl. Kapitel~3) zeigte sich, dass die
Kombination dieser Werkzeuge erhebliche Produktivitätsgewinne ermöglicht, vor
allem beim schnellen Prototyping, bei Standardaufgaben (Boilerplate) und bei
der automatischen Generierung von Tests. Cursor AI konnte darüber hinaus durch
die Möglichkeit, Kontext wie Screenshots oder Fehlermeldungen einzubinden, bei
der Fehlersuche und dem Debugging zusätzliche Mehrwerte bieten.

Neben den Werkzeugen haben sich auch neue Methoden etabliert:
\begin{itemize}
    \item \textbf{Prompt Engineering:} Entwickler*innen formulieren Anforderungen in natürlicher Sprache, die direkt von der KI interpretiert werden (vgl. Esposito et al.,~\cite[S.~2--3]{esposito_generative_2025}).
    \item \textbf{Retrieval-Augmented Generation (RAG):} KI-Tools kombinieren projektspezifische Kontextdaten (z.\,B. Dokumentation, vorhandener Code) mit aktuellen Benutzeranfragen, um passgenaue Lösungen zu generieren (vgl. Esposito et al.,~\cite[S.~4]{esposito_generative_2025}).
    \item \textbf{Human-in-the-Loop und Pair Programming:} Laut Nguyen-Duc et al.~\cite[S.~8]{nguyen-duc_generative_2023} und Fraunhofer IESE~\cite{siebert_generative_2024} wird die Zusammenarbeit von Mensch und KI (z.\,B. durch Feedback-Loops) immer wichtiger, um Qualität und Anpassungsfähigkeit der Entwicklung zu sichern.
\end{itemize}

Im Blog von Fraunhofer IESE~\cite{siebert_generative_2024} wird betont, dass
diese neuen Tools nicht nur als Autovervollständigung dienen, sondern immer
mehr Aufgaben im gesamten Entwicklungsprozess übernehmen – bis hin zur
automatischen Erstellung von Tests und zum Refactoring.

% \begin{itemize}
%     \item Innovative Ansätze für die Softwareentwicklung
% \end{itemize}
% ... Hier kommt der Text für die Subsektion Optimierung der Kollaboration durch KI ... 

% \chapter{Chancen}
% Der Einsatz von KI in der Softwareentwicklung bietet eine Vielzahl an Vorteilen. Besonders hervorzuheben sind Effizienzsteigerungen durch Automatisierung, die Entwickler:innen von repetitiven Aufgaben entlasten und ihnen mehr Zeit für kreative und konzeptionelle Arbeit geben. Dieses Kapitel untersucht die wichtigsten Potenziale, die KI-Technologien für den Softwareentwicklungsprozess mit sich bringen.

% Im praktischen Teil dieser Arbeit wird zudem untersucht, wie sich durch KI-Tools Entwicklungsaufgaben für eine interaktive Kartenansicht automatisieren lassen und welche konkreten Effizienzsteigerungen hier auftreten können.
% \section{Effizienzsteigerung und Automatisierung}
% Zahlreiche Studien und Fallanalysen bescheinigen generativen KI-Tools das
Potenzial, die Effizienz im Entwicklungsprozess maßgeblich zu
steigern~\cite{donvir_role_2024,coutinho_role_2024,s_future_2024,esposito_generative_2025,braun_ki_2024,siebert_generative_2024}.
Auch in der eigenen praktischen Demonstration (vgl. Kapitel~3) zeigte sich,
dass Werkzeuge wie GitHub Copilot oder Cursor repetitive Aufgaben wie das
Erstellen von Boilerplate-Code, Standardkomponenten oder einfachen UI-Logiken
erheblich beschleunigen können. So konnte das Grundgerüst des Map-Screens in
der Locals-App mit Unterstützung von Copilot innerhalb weniger Minuten
generiert werden, während vergleichbare Aufgaben ohne KI deutlich
zeitaufwändiger wären.

Aktuelle Literatur und Praxisberichte belegen, dass der gezielte Einsatz
generativer KI-Tools zu signifikanten Effizienzsteigerungen in der
Softwareentwicklung führt. Donvir et~al.~\cite{donvir_role_2024} betonen, dass
moderne Coding-Assistenzsysteme wie Copilot oder Cursor insbesondere bei
repetitiven Aufgaben für eine starke Beschleunigung sorgen. Auch die Fallstudie
von Coutinho et~al.~\cite{coutinho_role_2024} weist nach, dass sich die
Entwicklungszeit bei Routineaufgaben durch KI-gestützte Werkzeuge deutlich
verringert. Sulabh~\cite{s_future_2024} und das Fraunhofer
IESE~\cite{siebert_generative_2024} berichten von Effizienzgewinnen von bis zu
50\,\%. Esposito et~al.~\cite{esposito_generative_2025} unterstreichen zudem,
dass der Einsatz von Large Language Models neue Automatisierungs- und
Optimierungsmöglichkeiten eröffnet.

\begin{quote}
    \enquote{GitHub Copilot can assist in quick prototyping of code by generating foundational code structure based on natural language description of the feature. It can assist in boilerplate code generation by providing the class and interface definition generation, API and Database Schema creation. Both of these features combined improve the developer efficiency and enhanced code quality.}
    \cite[S.~4]{donvir_role_2024}
\end{quote}

Generative KI-Tools wirken sich auf sämtliche Phasen des
Softwareentwicklungsprozesses aus – von der Planung über die Implementierung
bis hin zu Test und Deployment – und eröffnen dadurch neue Potenziale für die
Effizienzsteigerung~\cite{minikiewicz_impact_nodate}. Feldexperimente mit
Softwareentwickler:innen bestätigen, dass sich der Einsatz solcher Werkzeuge
unmittelbar positiv auf Produktivität und Arbeitsweise
auswirkt~\cite{cui_effects_2024}.

Auch komplexere Aufgaben wie Debugging oder die automatische Anpassung von
Datenstrukturen profitieren von KI-Unterstützung, wie insbesondere der
Vergleich zwischen Copilot und Cursor verdeutlicht. Die Literatur verweist
dabei auf Effizienzsteigerungen von bis zu 50\,\% bei
Routinetätigkeiten~\cite{s_future_2024}, was sich mit den im Praxisteil
beobachteten Zeitersparnissen und Produktivitätsgewinnen deckt.

Die Qualität der Automatisierung bleibt jedoch stark abhängig von der Präzision
der Prompts und der Kontextintegration der eingesetzten Tools. Wie die Arbeit
mit Cursor gezeigt hat, ist gerade bei komplexeren Aufgaben ein dialogischer
Ansatz mit Feedback-Loops und manueller Kontrolle weiterhin unverzichtbar.
Dennoch legen sowohl Forschung als auch Praxis nahe, dass generative KI einen
spürbaren Effizienzgewinn im Entwicklungsalltag ermöglicht.
Wangoo~\cite{wangoo_artificial_2018} hebt hervor, dass KI-Technologien nicht
nur den Entwicklungsprozess beschleunigen, sondern auch die Wiederverwendung
bestehender Komponenten und das Design von Software nachhaltig vereinfachen
können.


% \section{Neue Werkzeuge und Methoden}
% Der verstärkte Einsatz generativer KI hat in den letzten Jahren eine Vielzahl
neuer Werkzeuge und Methoden in der Softwareentwicklung etabliert. Besonders
die Integration von Large Language Models (LLMs) in Entwicklungsumgebungen hat
die Art, wie Entwickler*innen arbeiten, maßgeblich verändert.

Zu den wichtigsten Werkzeugen zählen unter anderem \textbf{GitHub Copilot},
\textbf{Cursor AI}, \textbf{Amazon CodeWhisperer} und \textbf{Devin AI}. Diese
Tools werden in der Literatur umfassend dargestellt und spielen laut Esposito
et al.~\cite{esposito_generative_2025} sowie Nguyen-Duc et
al.~\cite{duc_generative_2023} eine zentrale Rolle in der aktuellen
Entwicklungspraxis.

GitHub Copilot wird besonders häufig eingesetzt und unterstützt
Entwickler*innen bei der automatischen Codegenerierung und Vervollständigung
direkt in der IDE. Esposito et al.~\cite[S.~2]{esposito_generative_2025}
beschreiben, dass solche Werkzeuge zunehmend in frühen Phasen des
Entwicklungsprozesses verwendet werden, etwa beim Übergang von Anforderungen zu
Architektur oder bei der Erstellung von Code aus natürlichsprachigen
Beschreibungen.

Cursor AI und ähnliche Tools ermöglichen einen dialogorientierten Workflow, bei
dem nicht nur einzelne Codezeilen, sondern ganze Features, Module oder sogar
Projekte automatisch erstellt und verfeinert werden können. Dabei kommen
Methoden wie Prompt Engineering, Retrieval-Augmented Generation (RAG) und
agentenbasierte Ansätze zum Einsatz (vgl. Esposito et
al.,~\cite[S.~3--4]{esposito_generative_2025}).

Im praktischen Teil dieser Arbeit (vgl. Kapitel~3) zeigte sich, dass die
Kombination dieser Werkzeuge erhebliche Produktivitätsgewinne ermöglicht, vor
allem beim schnellen Prototyping, bei Standardaufgaben (Boilerplate) und bei
der automatischen Generierung von Tests. Cursor AI konnte darüber hinaus durch
die Möglichkeit, Kontext wie Screenshots oder Fehlermeldungen einzubinden, bei
der Fehlersuche und dem Debugging zusätzliche Mehrwerte bieten.

Neben den Werkzeugen haben sich auch neue Methoden etabliert:
\begin{itemize}
    \item \textbf{Prompt Engineering:} Entwickler*innen formulieren Anforderungen in natürlicher Sprache, die direkt von der KI interpretiert werden (vgl. Esposito et al.,~\cite[S.~2--3]{esposito_generative_2025}).
    \item \textbf{Retrieval-Augmented Generation (RAG):} KI-Tools kombinieren projektspezifische Kontextdaten (z.\,B. Dokumentation, vorhandener Code) mit aktuellen Benutzeranfragen, um passgenaue Lösungen zu generieren (vgl. Esposito et al.,~\cite[S.~4]{esposito_generative_2025}).
    \item \textbf{Human-in-the-Loop und Pair Programming:} Laut Nguyen-Duc et al.~\cite[S.~8]{nguyen-duc_generative_2023} und Fraunhofer IESE~\cite{siebert_generative_2024} wird die Zusammenarbeit von Mensch und KI (z.\,B. durch Feedback-Loops) immer wichtiger, um Qualität und Anpassungsfähigkeit der Entwicklung zu sichern.
\end{itemize}

Im Blog von Fraunhofer IESE~\cite{siebert_generative_2024} wird betont, dass
diese neuen Tools nicht nur als Autovervollständigung dienen, sondern immer
mehr Aufgaben im gesamten Entwicklungsprozess übernehmen – bis hin zur
automatischen Erstellung von Tests und zum Refactoring.

% \begin{itemize}
%     \item Innovative Ansätze für die Softwareentwicklung
% \end{itemize}
% ... Hier kommt der Text für die Subsektion Optimierung der Kollaboration durch KI ... 

