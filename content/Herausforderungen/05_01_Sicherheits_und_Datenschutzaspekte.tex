
Die Integration generativer KI-Tools in die Softwareentwicklung eröffnet nicht
nur neue Chancen, sondern bringt auch Risiken für IT-Sicherheit und Datenschutz
mit sich. Aktuelle Studien zeigen, dass generative KI einerseits dazu beitragen
kann, Sicherheitslücken zu erkennen und Best Practices wie Security-by-Design
umzusetzen, andererseits aber auch neue Angriffsflächen schafft, wenn
KI-generierte Vorschläge ungeprüft übernommen werden
\cite{shi_ai-assisted_2023, alwageed_role_nodate}.

Insbesondere wird darauf hingewiesen, dass der Einsatz generativer KI zu
spezifischen Risiken wie \enquote{Prompt Injection} und \enquote{adversarial
    attacks} führen kann. Dabei werden durch gezielte oder manipulierte Eingaben
der KI unsichere oder schadhafte Codefragmente entlockt
\cite{shi_ai-assisted_2023}. Ein weiteres Risiko ist das sogenannte
\enquote{Model Poisoning}, bei dem durch fehlerhafte oder bösartige
Trainingsdaten gezielt Schwachstellen in das Modell eingeschleust werden
\cite{alwageed_role_nodate}.

Regelmäßige Security-Reviews, Audit-Trails und eine konsequente Einbindung
menschlicher Expertise (\enquote{Human-in-the-Loop}) werden in der Literatur
als zentrale Maßnahmen zur Absicherung empfohlen. Die Gefahr besteht
insbesondere darin, dass KI-Tools potenziell sicherheitskritische Muster zwar
erkennen und kennzeichnen können, ihre Vorschläge jedoch stets von
Entwickler:innen geprüft und angepasst werden müssen
\cite{shi_ai-assisted_2023, alwageed_role_nodate, siebert_generative_2024}.

Auch Datenschutzfragen treten verstärkt in den Vordergrund, insbesondere beim
Einsatz externer KI-Modelle in Cloud-Umgebungen. Hier besteht die Gefahr, dass
sensible Daten unbeabsichtigt an Dritte weitergegeben werden oder aus den
generierten Vorschlägen rekonstruiert werden können
\cite{siebert_generative_2024}.

