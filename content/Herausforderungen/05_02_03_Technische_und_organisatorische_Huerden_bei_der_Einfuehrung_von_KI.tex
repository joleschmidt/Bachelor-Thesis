Die Einführung generativer KI in der Softwareentwicklung ist mit zahlreichen
technischen und organisatorischen Hürden verbunden. Nguyen-Duc et
al.~\cite{nguyen-duc_generative_2023} machen deutlich, dass viele Unternehmen
Schwierigkeiten bei der Integration von GenAI-Tools in bestehende
Entwicklungsprozesse haben. Ein zentrales Problem ist die fehlende
Standardisierung von Schnittstellen, Datenformaten und Entwicklungsumgebungen,
was die nahtlose Integration verschiedener Tools erschwert.

Weitere technische Herausforderungen bestehen in der Qualitätssicherung der
generierten Ergebnisse. Laut Nguyen-Duc et
al.~\cite{nguyen-duc_generative_2023} ist es bislang schwierig, die
Zuverlässigkeit, Sicherheit und Wartbarkeit von KI-generiertem Code
systematisch zu überprüfen. Auch der Mangel an geeigneten Benchmarks, Testdaten
und automatisierten Validierungsverfahren erschwert die breite Einführung von
GenAI im Unternehmensumfeld.

Auf organisatorischer Ebene betonen sowohl Nguyen-Duc et
al.~\cite{nguyen-duc_generative_2023} als auch Schmitt et
al.~\cite{schmitt_generative_2024}, dass fehlende Akzeptanz und Unsicherheit im
Team, unklare Verantwortlichkeiten sowie mangelnde Schulung zu erheblichen
Implementierungsbarrieren führen können. Die Umstellung auf KI-gestützte
Prozesse erfordert häufig ein umfassendes Change Management, die Anpassung
bestehender Arbeitsweisen und neue Formen der Zusammenarbeit. Schmitt et
al.~\cite{schmitt_generative_2024} weisen darauf hin, dass die erfolgreiche
Einführung von GenAI-Tools nicht nur technisches Know-how, sondern auch
kulturelle Offenheit und kontinuierliche Weiterbildung im Team voraussetzt.
