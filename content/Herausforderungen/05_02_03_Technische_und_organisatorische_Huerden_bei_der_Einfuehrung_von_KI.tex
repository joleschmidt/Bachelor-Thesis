
Die Einführung generativer KI ist mit zahlreichen technischen und
organisatorischen Herausforderungen verbunden. Zu den größten Hürden zählen der
Mangel an Standards, geeigneten Benchmarks und validen Testdaten
\cite{nguyen-duc_generative_2023}. Agile Entwicklungsprozesse müssen speziell
für KI-Projekte weiterentwickelt werden, um den Anforderungen neuer
Technologien gerecht zu werden \cite{gill_agile_2025}. Unsicherheiten im Team
und eine fehlende Akzeptanz der neuen Werkzeuge stellen weitere zentrale
Barrieren dar, wie auch das Fraunhofer IESE hervorhebt
\cite{siebert_generative_2024}.

Usability und UX-Design spielen bei der Integration von KI-Tools eine immer
größere Rolle. Häufig wird die subjektive Nutzererfahrung unterschätzt, obwohl
die Akzeptanz im Team maßgeblich von transparenter Kommunikation,
kontinuierlichem Feedback und einer benutzerzentrierten Gestaltung der Tools
abhängt \cite{sergeyuk_human-ai_2025, sifi_how_2025}.

Weitere technische Herausforderungen bestehen in der Qualitätssicherung der
generierten Ergebnisse. Es ist bislang schwierig, die Zuverlässigkeit,
Sicherheit und Wartbarkeit von KI-generiertem Code systematisch zu überprüfen.
Auch der Mangel an geeigneten Benchmarks, Testdaten und automatisierten
Validierungsverfahren erschwert die breite Einführung von GenAI im
Unternehmensumfeld \cite{nguyen-duc_generative_2023}. Mit der fortschreitenden
Integration von KI in die Softwareentwicklung entstehen zudem neue
Anforderungen an Sicherheit, Arbeitsorganisation und langfristige
Kollaborationsmodelle \cite{hazra_ai_2025}.

Auf organisatorischer Ebene wirken sich fehlende Akzeptanz und Unsicherheiten
im Team, unklare Verantwortlichkeiten sowie mangelnde Schulung als erhebliche
Implementierungsbarrieren aus \cite{nguyen-duc_generative_2023,
    schmitt_generative_2024}. Die Umstellung auf KI-gestützte Prozesse erfordert
häufig umfassendes Change Management, die Anpassung bestehender Arbeitsweisen
und neue Formen der Zusammenarbeit. Eine erfolgreiche Einführung von
GenAI-Tools setzt nicht nur technisches Know-how, sondern auch kulturelle
Offenheit und kontinuierliche Weiterbildung voraus
\cite{schmitt_generative_2024}.

Für den Erfolg conversationaler KI-Assistenzsysteme ist zudem eine enge
Zusammenarbeit von Human-Computer-Interaction- und KI-Forschung notwendig, um
praxisnahe Evaluationsmethoden zu etablieren \cite{richards_bridging_2025}.
