Die Einführung generativer KI ist mit zahlreichen technischen und
organisatorischen Hürden verbunden. Nguyen-Duc
et~al.~\cite{nguyen-duc_generative_2023} betonen den Mangel an Standards,
geeigneten Benchmarks und validen Testdaten als zentrale Herausforderungen.
Gill~\cite{Gil} verweist auf die Notwendigkeit, agile Prozesse speziell für
KI-Projekte weiterzuentwickeln, während das Fraunhofer IESE~\cite{Sie} auf
Unsicherheiten im Team und fehlende Akzeptanz hinweist. Sergeyuk
et~al.~\cite{Ser} unterstreichen in ihrer Übersicht, dass auch Usability und
UX-Design bei der Integration von KI-Tools stärker berücksichtigt werden
müssen. Sifi~\cite{sifi_how_2025} hebt hervor, dass die subjektive
Nutzererfahrung bei der Einführung neuer KI-Tools oft unterschätzt wird und die
Akzeptanz entscheidend von transparenter Kommunikation und kontinuierlichem
Feedback abhängt.

Weitere technische Herausforderungen bestehen in der Qualitätssicherung der
generierten Ergebnisse. Laut Nguyen-Duc et
al.~\cite{nguyen-duc_generative_2023} ist es bislang schwierig, die
Zuverlässigkeit, Sicherheit und Wartbarkeit von KI-generiertem Code
systematisch zu überprüfen. Auch der Mangel an geeigneten Benchmarks, Testdaten
und automatisierten Validierungsverfahren erschwert die breite Einführung von
GenAI im Unternehmensumfeld.

Mit der fortschreitenden Integration von KI in die Softwareentwicklung
entstehen nicht nur neue technische Herausforderungen, sondern auch veränderte
Anforderungen an Sicherheit, Arbeitsorganisation und langfristige
Kollaborationsmodelle \cite{hazra_ai_2025}

Auf organisatorischer Ebene betonen sowohl Nguyen-Duc et
al.~\cite{nguyen-duc_generative_2023} als auch Schmitt et
al.~\cite{schmitt_generative_2024}, dass fehlende Akzeptanz und Unsicherheit im
Team, unklare Verantwortlichkeiten sowie mangelnde Schulung zu erheblichen
Implementierungsbarrieren führen können. Die Umstellung auf KI-gestützte
Prozesse erfordert häufig ein umfassendes Change Management, die Anpassung
bestehender Arbeitsweisen und neue Formen der Zusammenarbeit. Schmitt et
al.~\cite{schmitt_generative_2024} weisen darauf hin, dass die erfolgreiche
Einführung von GenAI-Tools nicht nur technisches Know-how, sondern auch
kulturelle Offenheit und kontinuierliche Weiterbildung im Team voraussetzt.
Richards und Wessel~\cite{richards_bridging_2025} argumentieren, dass für den
Erfolg conversationaler KI-Assistenzsysteme eine enge Zusammenarbeit von HCI-
und KI-Forschung erforderlich ist, um praxisnahe Evaluationsmethoden zu
etablieren.
