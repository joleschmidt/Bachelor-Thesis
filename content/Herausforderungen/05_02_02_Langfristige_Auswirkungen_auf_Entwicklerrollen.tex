
Der verstärkte Einsatz generativer KI-Tools in der Softwareentwicklung
verändert die Rolle von Entwickler:innen grundlegend. Während
Routinetätigkeiten und repetitive Aufgaben zunehmend automatisiert werden,
gewinnen Kompetenzen wie Prompt-Engineering, Systembewertung und die kritische
Reflexion von KI-Ergebnissen deutlich an Bedeutung
\cite{schmitt_generative_2024}.

Viele Entwickler:innen sehen in der Integration von GenAI neue Möglichkeiten
zur Kompetenzentwicklung, etwa in der Mensch-KI-Kollaboration oder im Aufbau
von Schnittstellenwissen zwischen Entwicklung, Domänenkenntnis und KI-Nutzung.
Gleichzeitig entstehen durch die Neuverteilung von Aufgaben und die wachsende
Abhängigkeit von KI-Systemen Unsicherheiten und Identitätskonflikte
\cite{schmitt_generative_2024}.

Auf organisatorischer Ebene sind Anpassungen der Rollenprofile, neue
Schulungskonzepte und die Überarbeitung von Verantwortlichkeiten notwendig, um
den Wandel aktiv zu gestalten und kontinuierliche Weiterbildung zu ermöglichen
\cite{nguyen-duc_generative_2023}. Entscheidend ist, dass Unternehmen und
Entwickler:innen sich auf die Veränderungen einlassen und die Transformation
aktiv begleiten.

Auch die Erfahrungen aus dem Praxisteil dieser Arbeit zeigen, dass der Fokus
bei der Entwicklung zunehmend auf der Formulierung präziser Prompts, der
kritischen Prüfung von KI-Vorschlägen und der aktiven Gestaltung der
Mensch-KI-Zusammenarbeit liegt. Die Rolle von Entwickler:innen wandelt sich
damit immer stärker hin zu einer Schnittstellenfunktion zwischen Mensch und
Maschine.
