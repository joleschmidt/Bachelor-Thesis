Der verstärkte Einsatz generativer KI verändert nicht nur technische Prozesse,
sondern wirkt sich auch langfristig auf die Rolle von Entwickler:innen aus.
Schmitt et al.~\cite{schmitt_generative_2024} zeigen, dass der zunehmende
Einsatz von KI-Tools wie Copilot oder ChatGPT zu einem Wandel der beruflichen
Identität und des Selbstverständnisses von Softwareentwickler:innen führen
kann. Während einerseits Routinetätigkeiten und repetitive Aufgaben zunehmend
automatisiert werden, rücken Kompetenzen wie Prompt-Engineering,
Systembewertung und die kritische Reflexion von KI-Ergebnissen stärker in den
Vordergrund.

Die Studie von Schmitt et al.~\cite{schmitt_generative_2024} verdeutlicht
zudem, dass viele Entwickler:innen durch die Integration von GenAI neue
Möglichkeiten zur Kompetenzentwicklung sehen -- etwa im Bereich
Mensch-KI-Kollaboration oder im Aufbau von Schnittstellenkompetenzen zwischen
Entwicklung, Domänenwissen und KI-Nutzung. Gleichzeitig berichten die
Autor:innen aber auch von Verunsicherung und Identitätskonflikten, die durch
die Neuverteilung von Aufgaben und die wachsende Abhängigkeit von KI-Systemen
entstehen können.

Auch auf organisatorischer Ebene ergeben sich laut Nguyen-Duc et
al.~\cite{nguyen-duc_generative_2023} langfristige Veränderungen: Die
Einführung von GenAI-Tools erfordert nicht nur technisches, sondern auch
soziales Change Management, etwa durch die Entwicklung neuer Rollenprofile,
angepasster Schulungskonzepte und überarbeiteter Verantwortlichkeiten im Team.
Entscheidend ist laut beiden Quellen, dass Unternehmen und Entwickler:innen die
Veränderungen aktiv gestalten und sich auf eine kontinuierliche
Weiterentwicklung der beruflichen Rollen einlassen.

Im praktischen Teil dieser Arbeit zeigte sich dieser Wandel exemplarisch:
Während der Implementierung des Map-Screens in der Locals-App verlagerte sich
der Fokus zunehmend von der reinen Code-Implementierung hin zur Fähigkeit, die
richtigen Prompts zu formulieren, generierte Vorschläge kritisch zu prüfen und
die Zusammenarbeit mit KI-Tools aktiv zu gestalten. Anstelle von klassischen
Routinetätigkeiten dominierten Tätigkeiten wie das Feintuning von Prompts, die
Auswahl zwischen unterschiedlichen KI-generierten Lösungswegen und die
Integration von Feedback in den Entwicklungsprozess.

Diese Entwicklung bestätigt die Beobachtung von Schmitt et
al.~\cite{schmitt_generative_2024}, dass Entwickler:innen künftig vermehrt als
Schnittstelle zwischen Mensch und KI agieren und Kompetenzen wie
Systembewertung, Prompt-Engineering und kritische Reflexion an Bedeutung
gewinnen.
