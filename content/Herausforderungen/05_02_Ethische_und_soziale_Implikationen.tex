Die zunehmende Integration generativer KI in den Entwicklungsalltag wirft
weitreichende ethische und soziale Fragen auf. Weisz et
al.~\cite{weisz_design_2024} betonen, dass mit der Automatisierung von
Entwicklungsaufgaben die Notwendigkeit steigt, Verantwortung und
Entscheidungsprozesse klar zu definieren. Insbesondere die Nachvollziehbarkeit
und Transparenz von KI-generierten Lösungen gelten als zentrale
Herausforderungen im Hinblick auf ethische Standards und die Überprüfbarkeit
von Ergebnissen.

Ein zentrales ethisches Problem besteht im sogenannten Bias: Generative
KI-Modelle können bestehende Vorurteile aus Trainingsdaten übernehmen und
reproduzieren. Wie Weisz et al.~\cite{weisz_design_2024} herausstellen, müssen
Entwickler:innen und Unternehmen darauf achten, geeignete Kontrollmechanismen
(\enquote{Guardrails}) zu etablieren, um Diskriminierung, Fehlinformationen und
unfaire Vorschläge zu vermeiden.

Schmitt et al.~\cite{schmitt_generative_2024} untersuchen darüber hinaus die
Auswirkungen generativer KI auf die berufliche Identität von Entwickler:innen.
Ihre Studie zeigt, dass der Einsatz von KI-Tools zu Verunsicherung hinsichtlich
der eigenen Rolle und Wertschätzung führen kann. Gleichzeitig ergeben sich neue
Möglichkeiten zur Kompetenzentwicklung und Zusammenarbeit, sofern der Fokus auf
Mensch-KI-Kollaboration liegt.

Auch im organisatorischen Kontext ergeben sich Herausforderungen: Laut
Nguyen-Duc et al.~\cite{nguyen-duc_generative_2023} sind Unternehmen gefordert,
klare Leitlinien für die Nutzung von GenAI-Tools zu formulieren, um
Verantwortlichkeiten, Qualitätsstandards und ethische Prinzipien verbindlich zu
verankern.
