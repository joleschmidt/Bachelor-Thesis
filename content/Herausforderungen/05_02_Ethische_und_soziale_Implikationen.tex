
Die zunehmende Integration generativer KI in die Softwareentwicklung wirft
weitreichende ethische und soziale Fragen auf. Die Automatisierung von
Entwicklungsaufgaben macht es notwendig, Verantwortung und
Entscheidungsprozesse klar zu definieren. Insbesondere Nachvollziehbarkeit und
Transparenz von KI-generierten Lösungen sind zentrale Herausforderungen für
ethische Standards und die Überprüfbarkeit von Ergebnissen
\cite{weisz_design_2024}.

Ein wesentliches ethisches Problem besteht im sogenannten Bias. Generative
KI-Modelle übernehmen häufig bestehende Vorurteile oder Stereotypen aus den
Trainingsdaten und können diese unreflektiert reproduzieren. Um
Diskriminierung, Fehlinformationen und unfaire Vorschläge zu vermeiden, sind
technische und organisatorische Kontrollmechanismen erforderlich, wie etwa
Guardrails, kontrollierte Testdatensätze und Diversity-Checks
\cite{weisz_design_2024, schmitt_generative_2024}.

Ethische und soziale Fragestellungen gewinnen immer mehr an Bedeutung, da
Barrierefreiheit und Teilhabe bei der Entwicklung von KI-Systemen noch oft
unzureichend berücksichtigt werden, obwohl KI-Technologien neue Möglichkeiten
der Inklusion bieten könnten \cite{flores-saviaga_impact_2025}.

Gleichzeitig zeigt die Literatur, dass der Einsatz generativer KI die
berufliche Identität von Entwickler:innen beeinflusst. Es entstehen
Unsicherheiten hinsichtlich der eigenen Rolle und Wertschätzung, aber auch neue
Möglichkeiten zur Kompetenzentwicklung und Zusammenarbeit, wenn der Fokus auf
Mensch-KI-Kollaboration gelegt wird \cite{schmitt_generative_2024}.

Auch auf organisatorischer Ebene sind Unternehmen gefordert, klare Leitlinien
für die Nutzung von GenAI-Tools zu formulieren und Verantwortlichkeiten,
Qualitätsstandards sowie ethische Prinzipien verbindlich zu verankern
\cite{nguyen-duc_generative_2023}.
