\subsubsection{Sicherheitsrisiken durch generative Modelle}

Generative KI-Modelle bringen neben allgemeinen IT-Sicherheitsfragen
spezifische neue Bedrohungen mit sich. Shi et al.~\cite{shi_ai-assisted_2023}
beschreiben, dass Angriffe wie \enquote{Prompt Injection}, bei denen gezielt
manipulierte Eingaben dazu führen, dass die KI unsicheren oder schädlichen Code
generiert, eine reale Gefahr darstellen. Ebenso können sogenannte
\enquote{adversarial attacks} dazu führen, dass durch minimale Veränderungen an
den Eingabedaten unerwartete und potenziell sicherheitskritische
Verhaltensweisen im generativen Modell ausgelöst werden.

Aktuelle Forschung zeigt, dass der Einsatz generativer KI neue Angriffsflächen
und Risiken für die IT-Sicherheit schafft. Shi
et~al.~\cite{shi_ai-assisted_2023} weisen darauf hin, dass insbesondere
sogenannte Prompt Injections und adversarielle Angriffe gezielt genutzt werden
können, um unsicheren oder schädlichen Code zu erzeugen. Alwageed und
Khan~\cite{alwageed_role_nodate} betonen zudem, dass eine Übernahme von
KI-generierten Vorschlägen ohne sorgfältige Überprüfung zu subtilen
Sicherheitslücken führen kann. Das Fraunhofer
IESE~\cite{siebert_generative_2024} hebt hervor, dass externe KI-Modelle
entlang der Software-Supply-Chain zusätzliche Datenschutzprobleme verursachen
können.

Ein weiteres Risiko besteht im sogenannten \enquote{Model Poisoning}: Hierbei
werden während des Trainings gezielt fehlerhafte oder bösartige Daten
eingespeist, um die KI zu manipulieren und Schwachstellen zu platzieren.
Alwageed und Khan~\cite{alwageed_role_nodate} weisen darauf hin, dass große
generative Modelle wie LLMs besonders anfällig für solche Angriffe sind, da sie
auf umfangreiche und oftmals öffentlich verfügbare Datenquellen zurückgreifen.
In diesem Zusammenhang betonen sie die Bedeutung von regelmäßigen
Security-Reviews und der Entwicklung von Abwehrmechanismen gegen adversarial
attacks.

Der Blog von Fraunhofer IESE~\cite{siebert_generative_2024} hebt hervor, dass
gerade bei der Nutzung externer KI-Modelle entlang der Software-Supply-Chain
neue Risiken entstehen können. Unzureichend geprüfter oder von Dritten
generierter Code kann unbemerkt Schwachstellen in den Entwicklungsprozess
einschleusen.
