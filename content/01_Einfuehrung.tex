\chapter{Einführung}
Künstliche Intelligenz (KI) hat in den vergangenen Jahren einen rasanten Aufschwung erlebt und beeinflusst bereits vielfältige Branchen von der Medizin bis zur Finanzwelt. Auch die Softwareentwicklung bleibt nicht verschont: Dort eröffnen KI-gestützte Verfahren ein breites Spektrum neuer Einsatzfelder. So kann KI nicht nur das Schreiben von Code und das Durchführen automatisierter Tests erleichtern, sondern auch innovative Methoden für Fehlersuche und Qualitätssicherung bereitstellen.

Diese Potenziale gehen jedoch mit weitreichenden Fragestellungen einher. Neben technischen Aspekten wie Sicherheit und Code-Qualität spielt auch die gesellschaftliche Dimension eine Rolle, etwa die Frage nach ethischen Standards oder Veränderungen im Berufsbild „Softwareentwickler“. Insbesondere generative KI, zum Beispiel in Form von Large Language Models (LLMs), wirft Fragen zu Datenschutz, Verantwortung und methodischer Einbindung in agile Prozesse auf.

Vor diesem Hintergrund setzt die vorliegende Arbeit an: Sie soll beleuchten, wie KI-gestützte Technologien den Softwareentwicklungsprozess langfristig prägen und welche Herausforderungen sich dabei ergeben. Dies betrifft sowohl die konkrete Arbeitssituation von Entwickler:innen als auch die strategischen Überlegungen von Unternehmen.

\section{Motivation}
% \subsection{Relevanz des Themas}
% % Diese Arbeit adressiert diese Problematik, indem sie die Chancen und Herausforderungen von KI in der Softwareentwicklung analysiert. Ziel ist es, sowohl wissenschaftliche als auch praktische Erkenntnisse zu gewinnen, die Unternehmen und Entwicklern helfen, informierte Entscheidungen über den Einsatz von KI-gestützten Technologien zu treffen.

% % Künstliche Intelligenz (KI) hat sich in den letzten Jahren als transformative Technologie in der Softwareentwicklung etabliert. Insbesondere generative KI-Modelle wie Large Language Models (LLMs) haben das Potenzial, Entwicklungsprozesse signifikant zu verändern. Die Automatisierung von Codegenerierung, Fehleranalyse und Softwarewartung führt zu einer gesteigerten Effizienz und ermöglicht es Entwicklern, sich auf konzeptionell anspruchsvollere Aufgaben zu konzentrieren.

% % Die zunehmende Integration von KI in den Softwareentwicklungsprozess eröffnet neue Möglichkeiten, birgt jedoch auch Herausforderungen. Während einige Studien eine gesteigerte Produktivität und Codequalität durch KI-gestützte Tools belegen, gibt es gleichzeitig Bedenken hinsichtlich Sicherheitsrisiken, algorithmischer Verzerrung und langfristigen Veränderungen in der Arbeitsweise von Entwicklern. Diese Gegensätze verdeutlichen die Notwendigkeit einer fundierten wissenschaftlichen Auseinandersetzung mit den Auswirkungen von KI auf die Softwareentwicklung.

% \subsection{Motivation}
% % Die zunehmende Verbreitung künstlicher Intelligenz (KI) in der Softwareentwicklung stellt sowohl Wissenschaft als auch Praxis vor bedeutende Herausforderungen und Chancen. Unternehmen integrieren KI-gestützte Tools in ihre Entwicklungsprozesse, um Produktivität und Codequalität zu steigern, doch der langfristige Einfluss dieser Technologie auf die Arbeitsweise von Softwareentwicklern ist noch nicht vollständig erforscht.

% % Besonders relevant ist die Frage, wie sich KI-gestützte Entwicklungsumgebungen auf traditionelle Softwareentwicklungspraktiken auswirken. Während einige Forschungen darauf hindeuten, dass KI-Tools repetitive Aufgaben reduzieren und Entwicklern mehr Raum für kreative Problemlösungen geben, bestehen weiterhin Unsicherheiten hinsichtlich der Verlässlichkeit der generierten Codevorschläge und möglicher ethischer Bedenken.

Die hohe Relevanz des Themas ergibt sich aus aktuellen Entwicklungen in
Forschung und Praxis: Immer mehr Unternehmen setzen auf KI-Technologien, um
Effizienzpotenziale und Innovationsschübe zu realisieren. Während generative
KI, etwa durch automatisierte Code-Generierung oder intelligente
Projektsteuerung enorme Fortschritte und Produktivitätsgewinne verspricht,
zeigen empirische Studien zugleich ein Spannungsverhältnis zwischen den
erhofften Vorteilen und realen Risiken wie Intransparenz, Sicherheitslücken und
ethischen Verzerrungen.

Nach aktuellen Schätzungen (Stand: 2025) erreicht der Softwaremarkt in
Deutschland ein Volumen von rund 31 Milliarden US-Dollar. Entwickler:innen
verbringen im Durchschnitt bis zu 17 Stunden pro Woche mit Wartungs- und
Routineaufgaben, ein Indikator dafür, wie groß der Bedarf an Automatisierung
und Qualitätsverbesserung in der Praxis ist. Genau hier setzt die generative KI
an: Tools wie GitHub Copilot können durch automatisierte
Boilerplate-Code-Generierung oder intelligente Code-Vervollständigung nicht nur
die Produktivität erhöhen, sondern auch das Fachkräftethema teilweise
entschärfen. Laut einer von Deloitte zitierten Studie lässt sich durch
KI-basierte Coding-Tools die für Routineaufgaben benötigte Entwicklerzeit um
bis zu 50\,\% reduzieren (vgl. \cite{s_future_2024}
\cite{siebert_generative_2024}).

Ein Beleg für die zunehmende wirtschaftliche Bedeutung ist der stetige Anstieg
der Investitionen in KI-Technologien für die Softwareentwicklung in den letzten
Jahren. Abbildung~\ref{fig:ki-investitionen} veranschaulicht diese Entwicklung
und unterstreicht, wie stark Unternehmen auf innovative KI-Lösungen setzen, um
sich Wettbewerbsvorteile zu sichern.

\begin{figure}[H]
    \centering
    \vspace{1em}
    \includegraphics[width=0.7\textwidth]{images/abbildungen/statistic_id1481131_jaehrliche-investitionen-von-unternehmen-in-ki-fuer-softwareentwicklung-bis-2025.png}
    \caption{Jährliche Investitionen von Unternehmen in KI für Softwareentwicklung 2023–2025. Quelle: Capgemini~\cite{statista_ki_investitionen_2025}.}
    \label{fig:ki-investitionen}
\end{figure}

Allerdings werfen diese Entwicklungen auch kontroverse Fragen auf: In welchem
Maße verändert die zunehmende Abhängigkeit von KI-Tools die Rollen und
Kompetenzen von Softwareentwickler:innen? Wie kann sichergestellt werden, dass
durch KI-gestützte Automatisierung weiterhin qualitativ hochwertige, wartbare
und sichere Software entsteht~\cite{siebert_generative_2024}? Hinzu kommt, dass
die Softwareentwicklung durch agile Methoden wie Scrum oder Kanban bereits
stark dynamisiert ist. Die zusätzliche Integration von KI als Tool oder
„Co-Entwickler“ erhöht die Anforderungen an Prozessgestaltung, Rollenverteilung
und Qualitätsmanagement weiter.



\section{Zielsetzung und Fragestellungen}

Ziel dieser Arbeit ist es, die Auswirkungen generativer KI auf die
Softwareentwicklung umfassend zu analysieren und daraus praxisnahe
Handlungsempfehlungen für Unternehmen und Entwickler:innen abzuleiten. Im Fokus
steht insbesondere, wie KI-gestützte Tools bestehende Entwicklungspraktiken
verändern, welche Herausforderungen damit verbunden sind und wie sich Chancen
und Risiken in der Praxis ausbalancieren lassen.

Daraus ergeben sich folgende zentrale Forschungsfragen:
\begin{itemize}
      \item Wie verändert generative KI traditionelle Entwicklungspraktiken in der
            Softwareentwicklung?
      \item Welche spezifischen Herausforderungen entstehen durch KI-gestützte
            Softwareentwicklung hinsichtlich Sicherheit, Ethik und Code-Qualität?
      \item Wie kann generative KI Softwareentwickler:innen in einem agilen
            Entwicklungsprozess unterstützen?
      \item Wie lassen sich bestehende generative KI-Tools, wie Cursor, GitHub Copilot,
            bolt.new (Bolt), in den Entwicklungsprozess einer React-Native-App integrieren
            und welchen Einfluss hat dies auf Entwicklungszeit und Code-Qualität?
\end{itemize}

Um diese Fragen nicht nur theoretisch, sondern auch praxisnah zu beleuchten,
wird im praktischen Teil der Arbeit eine React-Native-App als Fallstudie
herangezogen. Ziel ist es, exemplarisch zu untersuchen, wie generative KI-Tools
in reale Entwicklungsprozesse integriert werden können und welchen konkreten
Mehrwert sie im Entwicklungsalltag bieten.


\section{Methodik}

Die Arbeit folgt einem literaturbasierten, qualitativen Ansatz, um ein
umfassendes Bild der Potenziale und Herausforderungen generativer KI in der
Softwareentwicklung zu zeichnen. Ziel ist es, die aktuelle Forschungslage
systematisch zu erfassen, zu bewerten und praxisrelevante Schlüsse zu ziehen.
Die Methodik gliedert sich in folgende Schritte:

\begin{enumerate}
    % 1. punkt überprüfen: springer link & weiter quellen?
    \item \textbf{Literaturrecherche:} Umfassende Analyse wissenschaftlicher Publikationen aus IEEE Xplore, arXiv, SpringerLink sowie den in der Literaturdatenbank dieser Arbeit aufgeführten Fachquellen mit Fokus auf aktuelle Entwicklungen in der KI-gestützten Softwareentwicklung.
    \item \textbf{Kategorisierung der Forschungsthemen:} Strukturierte Erfassung und Gruppierung zentraler Themenfelder wie Automatisierung, Produktivität, Sicherheit und ethische Fragestellungen.
    \item \textbf{Vergleichende Analyse:} Systematischer Vergleich und Gegenüberstellung der Chancen und Herausforderungen auf Basis ausgewählter Studien und Fachbeiträge.
    \item \textbf{Synthese und Ableitung von Schlussfolgerungen:} Entwicklung praxisorientierter Empfehlungen für Unternehmen und Entwickler:innen zur Integration von KI in der Softwareentwicklung.
    \item \textbf{Praktische Demonstration:} Ergänzend zur Literaturauswertung wird ein Map-Screen (interaktive Kartenansicht) in der React Native-App „Locals“ exemplarisch implementiert. Dabei kommen verschiedene generative KI-Tools (u.a. Cursor, v0, GitHub Copilot) zum Einsatz, um die Unterstützung bei Code-Generierung, Testing und Qualitätsverbesserung zu evaluieren. Die gewonnenen Erkenntnisse aus diesem praktischen Teil werden systematisch mit den theoretischen Ergebnissen abgeglichen.
\end{enumerate}

Dieses methodische Vorgehen ermöglicht es, den aktuellen Forschungsstand
kritisch einzuordnen, praxisnahe Einblicke zu gewinnen und fundierte
Empfehlungen für die erfolgreiche Nutzung generativer KI in der
Softwareentwicklung abzuleiten.


\section{Aufbau der Arbeit}
Die Arbeit gliedert sich in folgende Kapitel:

\begin{itemize}
    \item \textbf{Kapitel 1: Einleitung} 
        \begin{itemize}
            \item Darstellung von Hintergrund, Motivation, Zielsetzung, Forschungsfragen, methodischer Vorgehensweise und Abgrenzung.
        \end{itemize}
    \item \textbf{Kapitel 2: Theoretische Grundlagen}   
        \begin{itemize}
            \item Definition und Funktionsweise von generativer KI in der Softwareentwicklung
            \item Übersicht relevanter KI-Modelle, Algorithmen und Beispiele für KI-gestützte Entwicklungswerkzeuge
        \end{itemize}
    \item \textbf{Kapitel 3: Praktische Demonstration}
        \begin{itemize}
            \item Vorstellung des Projekts "Locals" und dessen Architektur
            \item Implementierung einer interaktiven Kartenansicht in der React Native-Anwendung mithilfe generativer KI-Technologien
            \item Darstellung der Implementierungsschritte, Code-Beispiele und erste Evaluationsergebnisse 
        \end{itemize}
    \item \textbf{Kapitel 4: Chancen durch KI} 
        \begin{itemize}
            \item Effizienzsteigerung und Automatisierung
            \item Neue Werkzeuge und Methoden
            \item Verbesserte Code-Qualität und Fehlerminimierung
            \item Einfluss von KI auf agile Entwicklungsmethoden
            \item Bezugnahme auf die Erkenntnisse aus Kapitel 3
        \end{itemize}
    \item \textbf{Kapitel 5: Herausforderungen durch KI} 
        \begin{itemize}
            \item Sicherheits- und Datenschutzaspekte
            \item Ethische Implikationen und Bias in KI-Modellen
            \item Langfristige Auswirkungen auf Entwickler:innen-Rollen
            \item Organisatorische und technologische Hürden
            \item Risiken durch Abhängigkeit von KI-generiertem Code
            \item Analyse der in Kapitel 3 möglicherweise aufgetretenen Herausforderungen und Problematiken
        \end{itemize}
    \item \textbf{Kapitel 6: Wirtschaftliche und gesellschaftliche Auswirkungen} 
        \begin{itemize}
            \item Veränderungen in Softwareunternehmen
            \item Auswirkungen auf den Arbeitsmarkt und Entwickler:innen-Rollen
            \item Zukunftsperspektiven und strategische Empfehlungen
            \item Kosten-Nutzen-Analyse von KI-gestützter Softwareentwicklung
        \end{itemize}
    \item \textbf{Kapitel 7: Fazit und Ausblick} 
        \begin{itemize}
            \item Zusammenfassung der theoretischen und praktischen Erkenntnisse
            \item Diskussion offener Forschungsfragen
            \item Ableitung von Handlungsempfehlungen und Ausblick auf zukünftige Entwicklungen
        \end{itemize}
\end{itemize}

\section{Abgrenzung}

Die Arbeit konzentriert sich auf die theoretische Analyse der Chancen und
Herausforderungen von KI in der Softwareentwicklung. Folgende Aspekte werden
bewusst ausgeklammert:

\begin{itemize}
    \item \textbf{Technische Implementierungen:} Es werden keine eigenen KI-Modelle oder neuen Algorithmen entwickelt.
    \item \textbf{Empirische Studien:} Die Arbeit basiert auf einer literaturgestützten Analyse; eigene Befragungen, Experimente oder quantitative Erhebungen werden nicht durchgeführt.
    \item \textbf{Rechtliche Rahmenbedingungen:} Eine detaillierte Untersuchung rechtlicher oder regulatorischer Aspekte findet nicht statt.
\end{itemize}

Obwohl ein begrenzter praktischer Teil in Form einer exemplarischen
Funktionsimplementierung integriert wird, dient dieser ausschließlich als Proof
of Concept zur Veranschaulichung des KI-Einsatzes. Eine umfassende empirische
oder technische Evaluation erfolgt nicht.

Diese Abgrenzungen und methodischen Einschränkungen sind bei der Interpretation
der Ergebnisse sowie im abschließenden Fazit und Ausblick dieser Arbeit stets
zu berücksichtigen.


% Die Softwareentwicklung erlebt durch den Einsatz Künstlicher Intelligenz (KI) einen tiefgreifenden Wandel. Insbesondere generative KI-Modelle wie Large Language Models (LLMs) verändern etablierte Entwicklungspraktiken, indem sie Code generieren, automatisierte Tests durchführen und Entwicklungsprozesse effizienter gestalten. Diese Fortschritte werfen jedoch auch zentrale Fragen zur Sicherheit, zu ethischen Implikationen und zur langfristigen Rolle von Entwicklern auf.
%
% Erste Studien zeigen, dass KI-gestützte Tools signifikante Produktivitätssteigerungen ermöglichen, aber auch potenzielle Risiken wie Sicherheitslücken oder algorithmische Verzerrungen mit sich bringen. Daher ist es essenziell, eine fundierte wissenschaftliche Analyse durchzuführen, um Chancen und Herausforderungen umfassend zu bewerten.
%
% Diese Arbeit untersucht, wie KI-gestützte Softwareentwicklung die Produktivität und Qualität von Software beeinflusst, welche Herausforderungen sich daraus ergeben und welche praxisorientierten Anwendungen sich bereits abzeichnen. Dabei werden sowohl technologische als auch wirtschaftliche und gesellschaftliche Aspekte beleuchtet.
