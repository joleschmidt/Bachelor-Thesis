\chapter{Einführung}
Künstliche Intelligenz (KI) verändert zunehmend die Softwareentwicklung, indem sie Automatisierungsmöglichkeiten bietet und Entwicklungsprozesse effizienter gestaltet. In den letzten Jahren haben generative KI-Modelle erhebliche Fortschritte erzielt, insbesondere in der automatisierten Codegenerierung und Qualitätssicherung. Erste Studien zeigen, dass KI-gestützte Tools Produktivitätssteigerungen für Entwickler ermöglichen, indem sie automatisierte Codevorschläge liefern und repetitive Aufgaben reduzieren.

Neben den Vorteilen dieser Technologien gibt es jedoch auch Herausforderungen. Sicherheitsaspekte, ethische Fragen und die langfristige Veränderung der Arbeitsweise von Softwareentwicklern sind zentrale Diskussionspunkte in der aktuellen Forschung. Die Auswirkungen von KI auf die Softwareentwicklung sind nicht nur technischer Natur, sondern haben auch weitreichende Konsequenzen für Unternehmensstrukturen und die Ausbildung zukünftiger Entwickler.

Diese Arbeit setzt sich mit den Chancen, Herausforderungen und praxisorientierten Anwendungen von KI in der Softwareentwicklung auseinander. Dabei wird untersucht, wie KI-gestützte Softwareentwicklung sowohl die Produktivität als auch die Qualität von Code beeinflusst und welche ethischen und sicherheitstechnischen Fragen sich daraus ergeben.
\newpage
\section{Kontext und Motivation}
\subsection{Relevanz des Themas}
Diese Arbeit adressiert diese Problematik, indem sie die Chancen und Herausforderungen von KI in der Softwareentwicklung analysiert. Ziel ist es, sowohl wissenschaftliche als auch praktische Erkenntnisse zu gewinnen, die Unternehmen und Entwicklern helfen, informierte Entscheidungen über den Einsatz von KI-gestützten Technologien zu treffen.

Künstliche Intelligenz (KI) hat sich in den letzten Jahren als transformative Technologie in der Softwareentwicklung etabliert. Insbesondere generative KI-Modelle wie Large Language Models (LLMs) haben das Potenzial, Entwicklungsprozesse signifikant zu verändern. Die Automatisierung von Codegenerierung, Fehleranalyse und Softwarewartung führt zu einer gesteigerten Effizienz und ermöglicht es Entwicklern, sich auf konzeptionell anspruchsvollere Aufgaben zu konzentrieren.

Die zunehmende Integration von KI in den Softwareentwicklungsprozess eröffnet neue Möglichkeiten, birgt jedoch auch Herausforderungen. Während einige Studien eine gesteigerte Produktivität und Codequalität durch KI-gestützte Tools belegen, gibt es gleichzeitig Bedenken hinsichtlich Sicherheitsrisiken, algorithmischer Verzerrung und langfristigen Veränderungen in der Arbeitsweise von Entwicklern. Diese Gegensätze verdeutlichen die Notwendigkeit einer fundierten wissenschaftlichen Auseinandersetzung mit den Auswirkungen von KI auf die Softwareentwicklung.

\subsection{Motivation}
Die zunehmende Verbreitung künstlicher Intelligenz (KI) in der Softwareentwicklung stellt sowohl Wissenschaft als auch Praxis vor bedeutende Herausforderungen und Chancen. Unternehmen integrieren KI-gestützte Tools in ihre Entwicklungsprozesse, um Produktivität und Codequalität zu steigern, doch der langfristige Einfluss dieser Technologie auf die Arbeitsweise von Softwareentwicklern ist noch nicht vollständig erforscht.

Besonders relevant ist die Frage, wie sich KI-gestützte Entwicklungsumgebungen auf traditionelle Softwareentwicklungspraktiken auswirken. Während einige Forschungen darauf hindeuten, dass KI-Tools repetitive Aufgaben reduzieren und Entwicklern mehr Raum für kreative Problemlösungen geben, bestehen weiterhin Unsicherheiten hinsichtlich der Verlässlichkeit der generierten Codevorschläge und möglicher ethischer Bedenken.

\section{Zielsetzung und Fragestellungen}

Ziel dieser Arbeit ist es, die Auswirkungen generativer KI auf die
Softwareentwicklung umfassend zu analysieren und daraus praxisnahe
Handlungsempfehlungen für Unternehmen und Entwickler:innen abzuleiten. Im Fokus
steht insbesondere, wie KI-gestützte Tools bestehende Entwicklungspraktiken
verändern, welche Herausforderungen damit verbunden sind und wie sich Chancen
und Risiken in der Praxis ausbalancieren lassen.

Daraus ergeben sich folgende zentrale Forschungsfragen:
\begin{itemize}
      \item Wie verändert generative KI traditionelle Entwicklungspraktiken in der
            Softwareentwicklung?
      \item Welche spezifischen Herausforderungen entstehen durch KI-gestützte
            Softwareentwicklung hinsichtlich Sicherheit, Ethik und Code-Qualität?
      \item Wie kann generative KI Softwareentwickler:innen in einem agilen
            Entwicklungsprozess unterstützen?
      \item Wie lassen sich bestehende generative KI-Tools, wie Cursor, GitHub Copilot,
            bolt.new (Bolt), in den Entwicklungsprozess einer React-Native-App integrieren
            und welchen Einfluss hat dies auf Entwicklungszeit und Code-Qualität?
\end{itemize}

Um diese Fragen nicht nur theoretisch, sondern auch praxisnah zu beleuchten,
wird im praktischen Teil der Arbeit eine React-Native-App als Fallstudie
herangezogen. Ziel ist es, exemplarisch zu untersuchen, wie generative KI-Tools
in reale Entwicklungsprozesse integriert werden können und welchen konkreten
Mehrwert sie im Entwicklungsalltag bieten.


\section{Methodik und Vorgehensweise}
Diese Arbeit verfolgt eine theoretische und literaturbasierte Herangehensweise. Es wird eine systematische Analyse bestehender wissenschaftlicher Literatur durchgeführt, um ein umfassendes Verständnis der aktuellen Forschungslage zu generativer KI in der Softwareentwicklung zu erhalten. Die Methodik umfasst folgende Schritte:
\begin{enumerate}
    \item \textbf{Literaturrecherche:} Auswahl relevanter wissenschaftlicher Publikationen aus anerkannten Datenbanken wie IEEE Xplore, arXiv und SpringerLink. Dabei wird ein Fokus auf aktuelle Studien gelegt, die die Auswirkungen von KI auf die Softwareentwicklung untersuchen.
    \item \textbf{Kategorisierung der Forschungsthemen:} Identifikation und Gruppierung zentraler Themenfelder wie Produktivitätssteigerung, Automatisierung, Sicherheitsrisiken und ethische Fragestellungen.
    \item \textbf{Vergleichende Analyse:} Gegenüberstellung der identifizierten Chancen und Herausforderungen durch KI in der Softwareentwicklung.
    \item \textbf{Synthese und Ableitung von Schlussfolgerungen:} Basierend auf der Literaturauswertung werden praxisorientierte Empfehlungen für den Einsatz von KI in der Softwareentwicklung erarbeitet.
\end{enumerate}

\section{Aufbau der Arbeit}
Die Arbeit gliedert sich in folgende Kapitel:

\begin{itemize}
    \item \textbf{Kapitel 1: Einleitung} 
        \begin{itemize}
            \item Darstellung von Hintergrund, Motivation, Zielsetzung, Forschungsfragen, methodischer Vorgehensweise und Abgrenzung.
        \end{itemize}
    \item \textbf{Kapitel 2: Theoretische Grundlagen}   
        \begin{itemize}
            \item Definition und Funktionsweise von generativer KI in der Softwareentwicklung
            \item Übersicht relevanter KI-Modelle, Algorithmen und Beispiele für KI-gestützte Entwicklungswerkzeuge
        \end{itemize}
    \item \textbf{Kapitel 3: Praktische Demonstration}
        \begin{itemize}
            \item Vorstellung des Projekts "Locals" und dessen Architektur
            \item Implementierung einer interaktiven Kartenansicht in der React Native-Anwendung mithilfe generativer KI-Technologien
            \item Darstellung der Implementierungsschritte, Code-Beispiele und erste Evaluationsergebnisse 
        \end{itemize}
    \item \textbf{Kapitel 4: Chancen durch KI} 
        \begin{itemize}
            \item Effizienzsteigerung und Automatisierung
            \item Neue Werkzeuge und Methoden
            \item Verbesserte Code-Qualität und Fehlerminimierung
            \item Einfluss von KI auf agile Entwicklungsmethoden
            \item Bezugnahme auf die Erkenntnisse aus Kapitel 3
        \end{itemize}
    \item \textbf{Kapitel 5: Herausforderungen durch KI} 
        \begin{itemize}
            \item Sicherheits- und Datenschutzaspekte
            \item Ethische Implikationen und Bias in KI-Modellen
            \item Langfristige Auswirkungen auf Entwickler:innen-Rollen
            \item Organisatorische und technologische Hürden
            \item Risiken durch Abhängigkeit von KI-generiertem Code
            \item Analyse der in Kapitel 3 möglicherweise aufgetretenen Herausforderungen und Problematiken
        \end{itemize}
    \item \textbf{Kapitel 6: Wirtschaftliche und gesellschaftliche Auswirkungen} 
        \begin{itemize}
            \item Veränderungen in Softwareunternehmen
            \item Auswirkungen auf den Arbeitsmarkt und Entwickler:innen-Rollen
            \item Zukunftsperspektiven und strategische Empfehlungen
            \item Kosten-Nutzen-Analyse von KI-gestützter Softwareentwicklung
        \end{itemize}
    \item \textbf{Kapitel 7: Fazit und Ausblick} 
        \begin{itemize}
            \item Zusammenfassung der theoretischen und praktischen Erkenntnisse
            \item Diskussion offener Forschungsfragen
            \item Ableitung von Handlungsempfehlungen und Ausblick auf zukünftige Entwicklungen
        \end{itemize}
\end{itemize}

\section{Abgrenzung}

Die Arbeit konzentriert sich auf die theoretische Analyse der Chancen und
Herausforderungen von KI in der Softwareentwicklung. Folgende Aspekte werden
bewusst ausgeklammert:

\begin{itemize}
    \item \textbf{Technische Implementierungen:} Es werden keine eigenen KI-Modelle oder neuen Algorithmen entwickelt.
    \item \textbf{Empirische Studien:} Die Arbeit basiert auf einer literaturgestützten Analyse; eigene Befragungen, Experimente oder quantitative Erhebungen werden nicht durchgeführt.
    \item \textbf{Rechtliche Rahmenbedingungen:} Eine detaillierte Untersuchung rechtlicher oder regulatorischer Aspekte findet nicht statt.
\end{itemize}

Obwohl ein begrenzter praktischer Teil in Form einer exemplarischen
Funktionsimplementierung integriert wird, dient dieser ausschließlich als Proof
of Concept zur Veranschaulichung des KI-Einsatzes. Eine umfassende empirische
oder technische Evaluation erfolgt nicht.

Diese Abgrenzungen und methodischen Einschränkungen sind bei der Interpretation
der Ergebnisse sowie im abschließenden Fazit und Ausblick dieser Arbeit stets
zu berücksichtigen.


