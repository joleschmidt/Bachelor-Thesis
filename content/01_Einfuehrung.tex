\chapter{Einführung}
Künstliche Intelligenz (KI) verändert zunehmend die Softwareentwicklung, indem sie Automatisierungsmöglichkeiten bietet und Entwicklungsprozesse effizienter gestaltet. In den letzten Jahren haben generative KI-Modelle erhebliche Fortschritte erzielt, insbesondere in der automatisierten Codegenerierung und Qualitätssicherung. Erste Studien zeigen, dass KI-gestützte Tools Produktivitätssteigerungen für Entwickler ermöglichen, indem sie automatisierte Codevorschläge liefern und repetitive Aufgaben reduzieren.

Neben den Vorteilen dieser Technologien gibt es jedoch auch Herausforderungen. Sicherheitsaspekte, ethische Fragen und die langfristige Veränderung der Arbeitsweise von Softwareentwicklern sind zentrale Diskussionspunkte in der aktuellen Forschung. Die Auswirkungen von KI auf die Softwareentwicklung sind nicht nur technischer Natur, sondern haben auch weitreichende Konsequenzen für Unternehmensstrukturen und die Ausbildung zukünftiger Entwickler.

Diese Arbeit setzt sich mit den Chancen, Herausforderungen und praxisorientierten Anwendungen von KI in der Softwareentwicklung auseinander. Dabei wird untersucht, wie KI-gestützte Softwareentwicklung sowohl die Produktivität als auch die Qualität von Code beeinflusst und welche ethischen und sicherheitstechnischen Fragen sich daraus ergeben.
\newpage
\section{Kontext und Motivation}
\subsection{Relevanz des Themas}
Diese Arbeit adressiert diese Problematik, indem sie die Chancen und Herausforderungen von KI in der Softwareentwicklung analysiert. Ziel ist es, sowohl wissenschaftliche als auch praktische Erkenntnisse zu gewinnen, die Unternehmen und Entwicklern helfen, informierte Entscheidungen über den Einsatz von KI-gestützten Technologien zu treffen.

Künstliche Intelligenz (KI) hat sich in den letzten Jahren als transformative Technologie in der Softwareentwicklung etabliert. Insbesondere generative KI-Modelle wie Large Language Models (LLMs) haben das Potenzial, Entwicklungsprozesse signifikant zu verändern. Die Automatisierung von Codegenerierung, Fehleranalyse und Softwarewartung führt zu einer gesteigerten Effizienz und ermöglicht es Entwicklern, sich auf konzeptionell anspruchsvollere Aufgaben zu konzentrieren.

Die zunehmende Integration von KI in den Softwareentwicklungsprozess eröffnet neue Möglichkeiten, birgt jedoch auch Herausforderungen. Während einige Studien eine gesteigerte Produktivität und Codequalität durch KI-gestützte Tools belegen, gibt es gleichzeitig Bedenken hinsichtlich Sicherheitsrisiken, algorithmischer Verzerrung und langfristigen Veränderungen in der Arbeitsweise von Entwicklern. Diese Gegensätze verdeutlichen die Notwendigkeit einer fundierten wissenschaftlichen Auseinandersetzung mit den Auswirkungen von KI auf die Softwareentwicklung.

\subsection{Motivation}
Die zunehmende Verbreitung künstlicher Intelligenz (KI) in der Softwareentwicklung stellt sowohl Wissenschaft als auch Praxis vor bedeutende Herausforderungen und Chancen. Unternehmen integrieren KI-gestützte Tools in ihre Entwicklungsprozesse, um Produktivität und Codequalität zu steigern, doch der langfristige Einfluss dieser Technologie auf die Arbeitsweise von Softwareentwicklern ist noch nicht vollständig erforscht.

Besonders relevant ist die Frage, wie sich KI-gestützte Entwicklungsumgebungen auf traditionelle Softwareentwicklungspraktiken auswirken. Während einige Forschungen darauf hindeuten, dass KI-Tools repetitive Aufgaben reduzieren und Entwicklern mehr Raum für kreative Problemlösungen geben, bestehen weiterhin Unsicherheiten hinsichtlich der Verlässlichkeit der generierten Codevorschläge und möglicher ethischer Bedenken.

\section{Zielsetzung und Fragestellungen}
Das Ziel dieser Arbeit ist es, die Auswirkungen von KI auf die Softwareentwicklung zu analysieren und praxisnahe Handlungsempfehlungen für Unternehmen und Entwickler abzuleiten. Dabei werden insbesondere folgende Forschungsfragen untersucht:

\begin{itemize}
    \item[FF-1] Wie verändert generative KI traditionelle Entwicklungspraktiken in der Softwareentwicklung?
    \item[FF-2] Welche spezifischen Herausforderungen entstehen durch KI-gestützte Softwareentwicklung hinsichtlich Sicherheit, Ethik und Code-Qualität?
    \item[FF-3] Wie kann Generative KI Softwareentwickler in einem agilen Entwicklungsprozess unterstützen? 
    \item[FF-4] Wie lassen sich bestehende generative KI-Tools (Cursor, GitHub Copilot, v0 etc.) in den Entwicklungsprozess einer React-Native-App integrieren, und welchen Einfluss hat das auf Entwicklungszeit und Code-Qualität?
    % \item[FF-4] Welche langfristigen Implikationen hat die Nutzung von KI auf die Rolle und Qualifikation von Softwareentwicklern?
    % \item[FF-5] Welche wirtschaftlichen und organisatorischen Auswirkungen hat die Integration von KI auf Softwareunternehmen und den Arbeitsmarkt?
\end{itemize}

Darüber hinaus wird ein praktisches Beispiel in Form einer React Native-App (Locals) angeführt, um zu untersuchen, wie bestehende generative KI-Tools (etwa Cursor, GitHub Copilot etc.) in einen realen Entwicklungsprozess integriert werden können.

\section{Methodik und Vorgehensweise}
Diese Arbeit verfolgt eine theoretische und literaturbasierte Herangehensweise. Es wird eine systematische Analyse bestehender wissenschaftlicher Literatur durchgeführt, um ein umfassendes Verständnis der aktuellen Forschungslage zu generativer KI in der Softwareentwicklung zu erhalten. Die Methodik umfasst folgende Schritte:
\begin{enumerate}
    \item \textbf{Literaturrecherche:} Auswahl relevanter wissenschaftlicher Publikationen aus anerkannten Datenbanken wie IEEE Xplore, arXiv und SpringerLink. Dabei wird ein Fokus auf aktuelle Studien gelegt, die die Auswirkungen von KI auf die Softwareentwicklung untersuchen.
    \item \textbf{Kategorisierung der Forschungsthemen:} Identifikation und Gruppierung zentraler Themenfelder wie Produktivitätssteigerung, Automatisierung, Sicherheitsrisiken und ethische Fragestellungen.
    \item \textbf{Vergleichende Analyse:} Gegenüberstellung der identifizierten Chancen und Herausforderungen durch KI in der Softwareentwicklung.
    \item \textbf{Synthese und Ableitung von Schlussfolgerungen:} Basierend auf der Literaturauswertung werden praxisorientierte Empfehlungen für den Einsatz von KI in der Softwareentwicklung erarbeitet.
\end{enumerate}

\section{Aufbau der Arbeit}
Die Arbeit gliedert sich in folgende Kapitel:


\begin{itemize}
    \item \textbf{Kapitel 2:} Theoretische Grundlagen, in denen relevante Definitionen und Technologien der Künstlichen Intelligenz sowie Beispiele für generative KI-Tools vorgestellt werden.
    \item \textbf{Kapitel 3:} Chancen durch KI, mit einem Fokus auf Effizienzsteigerung und neuen Werkzeugen.
    \item \textbf{Kapitel 4:} Herausforderungen, einschließlich Sicherheits- und Datenschutzaspekten sowie ethischen Fragen.
    \item \textbf{Kapitel 5:} Fazit und Ausblick, mit einer Zusammenfassung der Ergebnisse und Handlungsempfehlungen.
\end{itemize}

\section{Abgrenzung}
Die Arbeit konzentriert sich auf die theoretische Analyse der Chancen und Herausforderungen von KI in der Softwareentwicklung. Folgende Aspekte werden bewusst ausgeklammert:

\begin{itemize}
    \item \textbf{Technische Implementierungen:} Es werden keine neuen KI-Modelle oder Algorithmen entwickelt.
    \item  \textbf{Empirische Studien:} Die Arbeit basiert auf einer literaturgestützten Analyse und führt keine Befragungen oder Experimente durch.
    \item \textbf{Rechtliche Rahmenbedingungen:} Eine detaillierte Untersuchung rechtlicher oder regulatorischer Aspekte wird nicht vorgenommen.
\end{itemize}

Obwohl ein begrenzter praktischer Teil in Form einer Funktionsimplementierung gezeigt wird, dient dieser in erster Linie als Proof of Concept. Eine umfassende empirische Evaluierung oder die Entwicklung eigener KI-Modelle findet nicht statt.

