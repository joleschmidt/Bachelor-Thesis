\chapter{Herausforderungen durch KI in der Softwareentwicklung}
Trotz der vielversprechenden Möglichkeiten von KI-gestützten Entwicklungsmethoden existieren Herausforderungen, die nicht vernachlässigt werden dürfen. Besonders Datenschutz- und Sicherheitsaspekte spielen eine zentrale Rolle, ebenso wie ethische Fragen zur Fairness und Transparenz von KI-Modellen. In diesem Kapitel werden die wesentlichen Problembereiche diskutiert, die mit der zunehmenden Integration von KI in die Softwareentwicklung verbunden sind. 
\section{Sicherheits- und Datenschutzaspekte}
\input{content/Herausforderungen/04_01_Sicherheits_und_Datenschutzaspekte.tex}

\subsection{Sicherheitsrisiken durch generative Modelle}
\input{content/Herausforderungen/04_01_01_Sicherheitsrisiken_durch_generative_Modelle.tex}

\section{Ethische und soziale Implikationen}
\input{content/Herausforderungen/04_02_Ethische_und_soziale_Implikationen.tex}

\subsection{Ethische Konflikte und Bias in KI-Systemen}
% ... Hier kommt der Text für die Subsektion Ethische Konflikte und Bias in KI-Systemen ... 

\subsection{Langfristige Auswirkungen auf Entwicklerrollen}
\input{content/Herausforderungen/04_02_02_Langfristige_Auswirkungen_auf_Entwicklerrollen.tex}

\subsection{Technische und organisatorische Hürden bei der Einführung von KI}
\input{content/Herausforderungen/04_02_03_Technische_und_organisatorische_Huerden_bei_der_Einfuehrung_von_KI.tex} 
