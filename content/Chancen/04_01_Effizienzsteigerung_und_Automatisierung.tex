Zahlreiche Studien und Fallanalysen bescheinigen generativen KI-Tools das
Potenzial, die Effizienz im Entwicklungsprozess maßgeblich zu
steigern~\cite{donvir_role_2024,coutinho_role_2024,s_future_2024,esposito_generative_2025,braun_ki_2024,siebert_generative_2024}.
Auch in der eigenen praktischen Demonstration (vgl. Kapitel~3) zeigte sich,
dass Werkzeuge wie GitHub Copilot oder Cursor repetitive Aufgaben wie das
Erstellen von Boilerplate-Code, Standardkomponenten oder einfachen UI-Logiken
erheblich beschleunigen können. So konnte das Grundgerüst des Map-Screens in
der Locals-App mit Unterstützung von Copilot innerhalb weniger Minuten
generiert werden, während vergleichbare Aufgaben ohne KI deutlich
zeitaufwändiger wären.

Aktuelle Literatur und Praxisberichte belegen, dass der gezielte Einsatz
generativer KI-Tools zu signifikanten Effizienzsteigerungen in der
Softwareentwicklung führt. Donvir et~al.~\cite{donvir_role_2024} betonen, dass
moderne Coding-Assistenzsysteme wie Copilot oder Cursor insbesondere bei
repetitiven Aufgaben für eine starke Beschleunigung sorgen. Auch die Fallstudie
von Coutinho et~al.~\cite{coutinho_role_2024} weist nach, dass sich die
Entwicklungszeit bei Routineaufgaben durch KI-gestützte Werkzeuge deutlich
verringert. Sulabh~\cite{s_future_2024} und das Fraunhofer
IESE~\cite{siebert_generative_2024} berichten von Effizienzgewinnen von bis zu
50\,\%. Esposito et~al.~\cite{esposito_generative_2025} unterstreichen zudem,
dass der Einsatz von Large Language Models neue Automatisierungs- und
Optimierungsmöglichkeiten eröffnet.

\begin{quote}
    \enquote{GitHub Copilot can assist in quick prototyping of code by generating foundational code structure based on natural language description of the feature. It can assist in boilerplate code generation by providing the class and interface definition generation, API and Database Schema creation. Both of these features combined improve the developer efficiency and enhanced code quality.}
    \cite[S.~4]{donvir_role_2024}
\end{quote}

Generative KI-Tools wirken sich auf sämtliche Phasen des
Softwareentwicklungsprozesses aus – von der Planung über die Implementierung
bis hin zu Test und Deployment – und eröffnen dadurch neue Potenziale für die
Effizienzsteigerung~\cite{minikiewicz_impact_nodate}. Feldexperimente mit
Softwareentwickler:innen bestätigen, dass sich der Einsatz solcher Werkzeuge
unmittelbar positiv auf Produktivität und Arbeitsweise
auswirkt~\cite{cui_effects_2024}.

Auch komplexere Aufgaben wie Debugging oder die automatische Anpassung von
Datenstrukturen profitieren von KI-Unterstützung, wie insbesondere der
Vergleich zwischen Copilot und Cursor verdeutlicht. Die Literatur verweist
dabei auf Effizienzsteigerungen von bis zu 50\,\% bei
Routinetätigkeiten~\cite{s_future_2024}, was sich mit den im Praxisteil
beobachteten Zeitersparnissen und Produktivitätsgewinnen deckt.

Die Qualität der Automatisierung bleibt jedoch stark abhängig von der Präzision
der Prompts und der Kontextintegration der eingesetzten Tools. Wie die Arbeit
mit Cursor gezeigt hat, ist gerade bei komplexeren Aufgaben ein dialogischer
Ansatz mit Feedback-Loops und manueller Kontrolle weiterhin unverzichtbar.
Dennoch legen sowohl Forschung als auch Praxis nahe, dass generative KI einen
spürbaren Effizienzgewinn im Entwicklungsalltag ermöglicht.
Wangoo~\cite{wangoo_artificial_2018} hebt hervor, dass KI-Technologien nicht
nur den Entwicklungsprozess beschleunigen, sondern auch die Wiederverwendung
bestehender Komponenten und das Design von Software nachhaltig vereinfachen
können.
