Zahlreiche Studien und Fallanalysen bescheinigen generativen KI-Tools das
Potenzial, die Effizienz im Entwicklungsprozess maßgeblich zu
steigern~\cite{donvir_role_2024,coutinho_role_2024,s_future_2024,esposito_generative_2025,braun_ki_2024,siebert_generative_2024}.
In der eigenen praktischen Demonstration (vgl. Kapitel~3) zeigte sich
beispielsweise, dass Tools wie GitHub Copilot oder Cursor repetitive Aufgaben
wie das Erstellen von Boilerplate-Code, Standardkomponenten oder einfachen
UI-Logiken erheblich beschleunigen können. So wurde das Grundgerüst des
Map-Screens in der Locals-App mit Unterstützung von Copilot innerhalb weniger
Minuten generiert, wohingegen für vergleichbare Aufgaben ohne KI deutlich mehr
Zeit einzuplanen wäre.

Zahlreiche aktuelle Studien und Praxisberichte belegen die
Effizienzsteigerungen, die durch den gezielten Einsatz generativer KI-Tools in
der Softwareentwicklung erzielt werden können. Donvir
et~al.~\cite{donvir_role_2024} heben hervor, dass moderne
Coding-Assistenzsysteme wie Copilot oder Cursor repetitive Aufgaben stark
beschleunigen. Coutinho et~al.~\cite{coutinho_role_2024} zeigen in ihrer
Fallstudie, dass sich die Entwicklungszeit bei Routineaufgaben durch den
KI-Einsatz deutlich verringert. Auch Sulabh~\cite{s_future_2024} und das
Fraunhofer IESE~\cite{siebert_generative_2024} bestätigen diese Beobachtungen
und verweisen auf Effizienzgewinne von bis zu 50 Prozent. Esposito
et~al.~\cite{esposito_generative_2025} betonen zudem, dass insbesondere der
Einsatz von Large Language Models neue Möglichkeiten zur Automatisierung und
Optimierung bietet.

\begin{quote}
    \enquote{GitHub Copilot can assist in quick prototyping of code by generating foundational code structure based on natural language description of the feature. It can assist in boilerplate code generation by providing the class and interface definition generation, API and Database Schema creation. Both of these features combined improve the developer efficiency and enhanced code quality.}
    \cite[S.~8]{donvir_role_2024}
\end{quote}

Generative KI-Tools beeinflussen sämtliche Phasen des
Softwareentwicklungsprozesses – von der Planung über die Implementierung bis
hin zu Test und Deployment – und eröffnen dadurch neue Potenziale für die
Effizienzsteigerung \cite{minikiewicz_impact_nodate}. Feldexperimente mit
Softwareentwickler:innen belegen, dass sich der Einsatz generativer KI-Tools
unmittelbar positiv auf Produktivität und Arbeitsweise auswirken kann
\cite{cui_effects_2024}.

Auch komplexere Aufgaben wie das Debugging oder die automatische Anpassung von
Datenstrukturen wurden durch KI-gestützte Tools unterstützt, wie insbesondere
im Vergleich zwischen Copilot und Cursor deutlich wurde. Die Literatur verweist
auf Effizienzsteigerungen von bis zu 50\,\% bei
Routinetätigkeiten~\cite{s_future_2024}. Dies deckt sich mit den im Praxisteil
beobachteten Zeitersparnissen und der damit verbundenen Steigerung der
Produktivität.

Trotz dieser Potenziale bleibt die Qualität der Automatisierung stark abhängig
von der Präzision der Prompts und der Kontextintegration der eingesetzten
Tools. Wie die Arbeit mit Cursor zeigte, ist insbesondere bei komplexeren
Aufgaben ein dialogischer Ansatz mit Feedback-Loops und manueller Kontrolle
weiterhin unverzichtbar. Dennoch zeigen sowohl Forschung als auch Praxis, dass
generative KI einen spürbaren Effizienzgewinn im Entwicklungsalltag ermöglicht.
Wangoo~\cite{wangoo_artificial_2018} stellt heraus, dass KI-Technologien nicht
nur den Entwicklungsprozess beschleunigen, sondern auch die Wiederverwendung
bestehender Komponenten und das Design von Software nachhaltig vereinfachen
können.

