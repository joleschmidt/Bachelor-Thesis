Zahlreiche Studien und Fallanalysen bescheinigen generativen KI-Tools das
Potenzial, die Effizienz im Entwicklungsprozess maßgeblich zu
steigern~\cite{10741797,coutinho2024rolegenerativeaisoftware,deloitte2024,computerwoche2023,fraunhofer2024}.
In der eigenen praktischen Demonstration (vgl. Kapitel~3) zeigte sich
beispielsweise, dass Tools wie GitHub Copilot oder Cursor repetitive Aufgaben
wie das Erstellen von Boilerplate-Code, Standardkomponenten oder einfachen
UI-Logiken erheblich beschleunigen können. So wurde das Grundgerüst des
Map-Screens in der Locals-App mit Unterstützung von Copilot innerhalb weniger
Minuten generiert, wohingegen für vergleichbare Aufgaben ohne KI deutlich mehr
Zeit einzuplanen wäre.

\begin{quote}
    \enquote{GitHub Copilot can assist in quick prototyping of code by generating foundational code structure based on natural language description of the feature. It can assist in boilerplate code generation by providing the class and interface definition generation, API and Database Schema creation. Both of these features combined improve the developer efficiency and enhanced code quality.}
    \cite[S.~8]{10741797}
\end{quote}

Auch komplexere Aufgaben wie das Debugging oder die automatische Anpassung von
Datenstrukturen wurden durch KI-gestützte Tools unterstützt, wie insbesondere
im Vergleich zwischen Copilot und Cursor deutlich wurde. Die Literatur verweist
auf Effizienzsteigerungen von bis zu 50\,\% bei
Routinetätigkeiten~\cite{deloitte2024}. Dies deckt sich mit den im Praxisteil
beobachteten Zeitersparnissen und der damit verbundenen Steigerung der
Produktivität.

Trotz dieser Potenziale bleibt die Qualität der Automatisierung stark abhängig
von der Präzision der Prompts und der Kontextintegration der eingesetzten
Tools. Wie die Arbeit mit Cursor zeigte, ist insbesondere bei komplexeren
Aufgaben ein dialogischer Ansatz mit Feedback-Loops und manueller Kontrolle
weiterhin unverzichtbar. Dennoch zeigen sowohl Forschung als auch Praxis, dass
generative KI einen spürbaren Effizienzgewinn im Entwicklungsalltag ermöglicht.

% \begin{itemize}
%     \item Automatisierung von Entwicklungsprozessen
%     \item Optimierung der Kollaboration durch KI
%     \item Verbesserung der Codequalität
% \end{itemize}

