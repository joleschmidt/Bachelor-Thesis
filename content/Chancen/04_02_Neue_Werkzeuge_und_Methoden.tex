Der verstärkte Einsatz generativer KI hat in den letzten Jahren eine Vielzahl
neuer Werkzeuge und Methoden in der Softwareentwicklung hervorgebracht.
Insbesondere die Integration von Large Language Models (LLMs) in moderne
Entwicklungsumgebungen verändert die Arbeitsweise von Entwickler:innen
grundlegend~\cite{esposito_generative_2025,nguyen-duc_generative_2023}.

Zu den wichtigsten Werkzeugen zählen \textbf{GitHub Copilot}, \textbf{Cursor
    AI}, \textbf{Amazon CodeWhisperer} und \textbf{Devin AI}. Diese Tools werden in
der Literatur umfassend dargestellt und spielen laut Esposito et
al.~\cite{esposito_generative_2025} sowie Nguyen-Duc et
al.~\cite{nguyen-duc_generative_2023} eine zentrale Rolle in der aktuellen
Entwicklungspraxis. Besonders GitHub Copilot wird häufig eingesetzt, um
Entwickler:innen bei der automatischen Codegenerierung und Vervollständigung in
der IDE zu unterstützen~\cite{esposito_generative_2025}. Solche Werkzeuge
finden zunehmend in frühen Phasen des Entwicklungsprozesses Anwendung, etwa
beim Übergang von Anforderungen zu Architektur oder bei der Erstellung von Code
aus natürlichsprachigen Beschreibungen.

Cursor AI und ähnliche Tools ermöglichen darüber hinaus einen
dialogorientierten Workflow, bei dem nicht nur einzelne Codezeilen, sondern
ganze Features oder sogar komplette Projekte automatisch erstellt und iterativ
verfeinert werden können. Hierbei kommen Methoden wie Prompt Engineering,
Retrieval-Augmented Generation (RAG) und agentenbasierte Ansätze zum
Einsatz~\cite{esposito_generative_2025}. Die aktuelle Literatur weist jedoch
auch auf Herausforderungen hin: Karhu et al.~\cite{karhu_expectations_2025}
verweisen auf Diskrepanzen zwischen hohen Erwartungen an KI-gestützte
Testautomatisierung und den realen Schwierigkeiten in der Praxis, während Islam
und Sandborn~\cite{islam_multimodal_2025} zeigen, dass multimodale generative
KI zunehmend auch für Aufgaben wie die automatische Aufwandsschätzung (z.\,B.
Story Points) genutzt wird.

Die Einführung dieser neuen Werkzeuge wird in zahlreichen aktuellen
Veröffentlichungen beschrieben. So erläutert Kerr~\cite{kerr_github_nodate} an
Praxisbeispielen die Integration von Copilot in Entwicklungsprozesse.
Nguyen-Duc et al.~\cite{nguyen-duc_generative_2023} betonen die wachsende
Bedeutung von Prompt Engineering und dialogbasierten Workflows. Weisz et
al.~\cite{weisz_design_2024} formulieren Designprinzipien für den produktiven
Einsatz generativer KI-Tools, während Geyer et al.~\cite{geyer_case_2025} den
Einfluss auf agile Qualitätssicherung untersuchen. Gill~\cite{gill_agile_2025}
hebt die Notwendigkeit speziell angepasster, agiler Entwicklungsprozesse für
KI-basierte Projekte hervor. Erfahrungen aus der Praxis, wie sie von
Rogachev~\cite{rogachev_my_nodate} beschrieben werden, bestätigen zudem, dass
die Kombination dieser Ansätze sowohl Produktivität als auch Codequalität
nachhaltig verbessern kann.

Generative KI-Systeme werden heute nicht mehr nur als reine
Automatisierungstools betrachtet, sondern zunehmend als kreative Partner
verstanden, die neue Formen der kollaborativen Ideenfindung und Gestaltung
ermöglichen~\cite{khan_beyond_2025}.

Im Praxisteil dieser Arbeit (vgl. Kapitel~3) zeigte sich, dass insbesondere die
Kombination dieser Werkzeuge deutliche Produktivitätsgewinne ermöglicht – etwa
beim schnellen Prototyping, bei Standardaufgaben (Boilerplate) und der
automatischen Testgenerierung. Cursor AI ermöglichte durch Kontextintegration
wie Screenshots oder Fehlermeldungen einen zusätzlichen Mehrwert bei der
Fehlersuche und beim Debugging.

Neben den Werkzeugen haben sich in der Praxis auch neue Methoden etabliert:
\begin{itemize}
    \item \textbf{Prompt Engineering:} Anforderungen werden in natürlicher Sprache formuliert und direkt von der KI interpretiert~\cite{esposito_generative_2025}.
    \item \textbf{Retrieval-Augmented Generation (RAG):} KI-Tools kombinieren projektspezifische Kontextdaten mit aktuellen Benutzeranfragen, um passgenaue Lösungen zu generieren~\cite{esposito_generative_2025}.
    \item \textbf{Human-in-the-Loop und Pair Programming:} Die Zusammenarbeit von Mensch und KI (z.\,B. über Feedback-Loops) wird immer wichtiger, um Qualität und Anpassungsfähigkeit der Entwicklung zu sichern~\cite{nguyen-duc_generative_2023,siebert_generative_2024}. J et al.~\cite{j_integration_2023} zeigen, dass insbesondere bei der Automatisierung von Routineaufgaben KI-Tools signifikante Vorteile bieten und zunehmend als Standardwerkzeuge in agilen Teams akzeptiert werden.
\end{itemize}

Der Blog des Fraunhofer IESE~\cite{siebert_generative_2024} betont, dass diese
neuen Tools längst über die reine Autovervollständigung hinausgehen und immer
mehr Aufgaben im gesamten Entwicklungsprozess übernehmen – von der
automatischen Erstellung von Tests bis zum Refactoring.
