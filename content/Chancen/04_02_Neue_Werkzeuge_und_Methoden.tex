Der Siegeszug generativer KI hat in den letzten Jahren eine Vielzahl
innovativer Werkzeuge und Methoden in die Softwareentwicklung eingeführt.
Insbesondere die Entstehung von KI-gestützten Entwicklungsumgebungen und
Coding-Assistenten wie GitHub Copilot, Cursor oder Bolt.new hat dazu geführt,
dass Entwickler:innen in ihrem Workflow neue Arbeitsweisen adaptieren
müssen~\cite{10741797,Weisz_2024,fraunhofer2024}.

Im praktischen Teil dieser Arbeit (vgl. Kapitel~3) zeigte sich deutlich, wie
unterschiedlich diese Tools in der Entwicklungspraxis agieren und welche neuen
Möglichkeiten sie eröffnen. Während Copilot insbesondere für die schnelle
Generierung von Standard-Code, Boilerplate und UI-Elementen eingesetzt werden
konnte, überzeugte Cursor durch seine Fähigkeit, Kontextinformationen wie
Screenshots oder Fehlermeldungen direkt in die Codegenerierung und
Problemlösung zu integrieren. Bolt.new wiederum ermöglichte eine weitgehend
automatisierte Initialisierung und das cloudbasierte Setup von Projekten,
inklusive direkter Web-Deployments.

Diese neuen Werkzeuge führen zu einer Verschiebung der klassischen
Rollenverteilung und eröffnen neue Methoden im Entwicklungsprozess. Die
Literatur beschreibt, dass insbesondere die Integration von Large Language
Models (LLMs) und Retrieval-Augmented Generation (RAG) zu einer höheren
Anpassungsfähigkeit und Kontextsensitivität
führt~\cite{esposito2025,donvir2024}. Methoden wie Prompt Chaining, dialogische
Interaktion mit der KI oder die Nutzung von Kontextdaten in Echtzeit gehören
zunehmend zum Alltag in modernen Entwicklerteams~\cite{weisz2024}.

\begin{quote}
    \enquote{Tools like Copilot and Cursor AI have introduced dialog-driven, context-aware development workflows, in which code suggestions, debugging, and refactoring are conducted interactively, often utilizing screenshots or project artifacts as additional context.}
    \cite[S.~10]{donvir2024}
\end{quote}

Ein weiteres zentrales Ergebnis der praktischen Demonstration ist die
Erkenntnis, dass die Qualität der Ergebnisse maßgeblich von der Fähigkeit zur
Integration und Steuerung der KI im Entwicklungsprozess abhängt. Besonders
Cursor konnte durch die Nutzung von Prompt Chaining und der aktiven Einbindung
von User-Feedback die Entwicklungsgeschwindigkeit und Problemlösungskompetenz
deutlich steigern (vgl. Kapitel~3). Die Erfahrung zeigt zudem, dass die Wahl
des passenden Tools und der dazugehörigen Methode stark vom Projekttyp, den
Teamstrukturen und den angestrebten Zielen abhängt.

Insgesamt lässt sich festhalten, dass generative KI nicht nur neue Werkzeuge,
sondern auch grundlegend neue Methoden und Arbeitsweisen in die
Softwareentwicklung gebracht hat, die zukünftig weiter an Bedeutung gewinnen
werden.

% \begin{itemize}
%     \item Innovative Ansätze für die Softwareentwicklung
% \end{itemize}
% ... Hier kommt der Text für die Subsektion Optimierung der Kollaboration durch KI ... 