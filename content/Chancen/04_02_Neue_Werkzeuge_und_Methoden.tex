Der verstärkte Einsatz generativer KI hat in den letzten Jahren eine Vielzahl
neuer Werkzeuge und Methoden in der Softwareentwicklung etabliert. Besonders
die Integration von Large Language Models (LLMs) in Entwicklungsumgebungen hat
die Art, wie Entwickler*innen arbeiten, maßgeblich verändert.

Zu den wichtigsten Werkzeugen zählen unter anderem \textbf{GitHub Copilot},
\textbf{Cursor AI}, \textbf{Amazon CodeWhisperer} und \textbf{Devin AI}. Diese
Tools werden in der Literatur umfassend dargestellt und spielen laut Esposito
et al.~\cite{esposito_generative_2025} sowie Nguyen-Duc et
al.~\cite{duc_generative_2023} eine zentrale Rolle in der aktuellen
Entwicklungspraxis.

GitHub Copilot wird besonders häufig eingesetzt und unterstützt
Entwickler*innen bei der automatischen Codegenerierung und Vervollständigung
direkt in der IDE. Esposito et al.~\cite[S.~2]{esposito_generative_2025}
beschreiben, dass solche Werkzeuge zunehmend in frühen Phasen des
Entwicklungsprozesses verwendet werden, etwa beim Übergang von Anforderungen zu
Architektur oder bei der Erstellung von Code aus natürlichsprachigen
Beschreibungen.

Cursor AI und ähnliche Tools ermöglichen einen dialogorientierten Workflow, bei
dem nicht nur einzelne Codezeilen, sondern ganze Features, Module oder sogar
Projekte automatisch erstellt und verfeinert werden können. Dabei kommen
Methoden wie Prompt Engineering, Retrieval-Augmented Generation (RAG) und
agentenbasierte Ansätze zum Einsatz (vgl. Esposito et
al.,~\cite[S.~3--4]{esposito_generative_2025}). Karhu
et~al.~\cite{karhu_expectations_2025} verweisen in ihrer Übersicht auf die
Diskrepanz zwischen den hohen Erwartungen an KI-gestützte Testautomatisierung
und den realen Herausforderungen bei deren Einführung in der industriellen
Praxis. Aktuelle Ansätze wie von Islam und
Sandborn~\cite{islam_multimodal_2025} demonstrieren, dass multimodale
generative KI zunehmend auch in der agilen Aufwandsschätzung, beispielsweise
bei der automatischen Ermittlung von Story Points, eingesetzt wird.

Die Entwicklung neuer Werkzeuge und Methoden wird in der aktuellen Literatur
ausführlich beschrieben. So erläutert Kerr~\cite{kerr_github_nodate} anhand von
Praxisbeispielen die Integration von Copilot in moderne Entwicklungsprozesse.
Nguyen-Duc et~al.~\cite{nguyen-duc_generative_2023} betonen die wachsende
Bedeutung von Prompt Engineering und dialogbasierten Workflows. Weisz
et~al.~\cite{weisz_design_2024} formulieren Designprinzipien für die
erfolgreiche Nutzung generativer KI-Tools, während Geyer
et~al.~\cite{geyer_case_2025} den Einfluss auf die agile Qualitätssicherung
untersuchen. Gill~\cite{gill_agile_2025} unterstreicht die Bedeutung von
speziell angepassten, agilen Entwicklungsprozessen für KI-basierte Projekte.
Die Erfahrungen aus der Praxis - wie sie etwa von
Rogachev~\cite{rogachev_my_nodate} beschrieben werden - bestätigen, dass sich
durch die Kombination dieser Ansätze sowohl Produktivität als auch Codequalität
nachhaltig verbessern lassen.

KI-Systeme werden dabei nicht mehr nur als Automatisierungstools verstanden,
sondern zunehmend als kreative Partner, die neue Formen der kollaborativen
Ideenfindung und Gestaltung ermöglichen \cite{khan_beyond_2025}.

Im praktischen Teil dieser Arbeit (vgl. Kapitel~3) zeigte sich, dass die
Kombination dieser Werkzeuge erhebliche Produktivitätsgewinne ermöglicht, vor
allem beim schnellen Prototyping, bei Standardaufgaben (Boilerplate) und bei
der automatischen Generierung von Tests. Cursor AI konnte darüber hinaus durch
die Möglichkeit, Kontext wie Screenshots oder Fehlermeldungen einzubinden, bei
der Fehlersuche und dem Debugging zusätzliche Mehrwerte bieten.

Neben den Werkzeugen haben sich auch neue Methoden etabliert:
\begin{itemize}
    \item \textbf{Prompt Engineering:} Entwickler*innen formulieren Anforderungen in natürlicher Sprache, die direkt von der KI interpretiert werden (vgl. Esposito et al.,~\cite[S.~2--3]{esposito_generative_2025}).
    \item \textbf{Retrieval-Augmented Generation (RAG):} KI-Tools kombinieren projektspezifische Kontextdaten (z.\,B. Dokumentation, vorhandener Code) mit aktuellen Benutzeranfragen, um passgenaue Lösungen zu generieren (vgl. Esposito et al.,~\cite[S.~4]{esposito_generative_2025}).
    \item \textbf{Human-in-the-Loop und Pair Programming:} Laut Nguyen-Duc et al.~\cite[S.~8]{nguyen-duc_generative_2023} und Fraunhofer IESE~\cite{siebert_generative_2024} wird die Zusammenarbeit von Mensch und KI (z.\,B. durch Feedback-Loops) immer wichtiger, um Qualität und Anpassungsfähigkeit der Entwicklung zu sichern. J et al. \cite{j_integration_2023} zeigen, dass insbesondere bei der Automatisierung von Routineaufgaben KI-Tools signifikante Vorteile bieten und zunehmend als Standardwerkzeuge in agilen Teams akzeptiert werden.
\end{itemize}

Im Blog von Fraunhofer IESE~\cite{siebert_generative_2024} wird betont, dass
diese neuen Tools nicht nur als Autovervollständigung dienen, sondern immer
mehr Aufgaben im gesamten Entwicklungsprozess übernehmen – bis hin zur
automatischen Erstellung von Tests und zum Refactoring.

% \begin{itemize}
%     \item Innovative Ansätze für die Softwareentwicklung
% \end{itemize}
% ... Hier kommt der Text für die Subsektion Optimierung der Kollaboration durch KI ... 