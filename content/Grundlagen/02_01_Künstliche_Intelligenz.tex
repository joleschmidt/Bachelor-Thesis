
Künstliche Intelligenz (KI) leitet einen grundlegenden Wandel in der
Softwareentwicklung ein. Die aktuelle, rechtlich verbindliche Definition der
Europäischen Union beschreibt KI als maschinengestützte Systeme, die mit
unterschiedlichem Grad an Autonomie Vorhersagen, Empfehlungen oder
Entscheidungen generieren, welche physische oder virtuelle Umgebungen
beeinflussen können \cite{noauthor_verordnung_nodate}. Diese Definition
etabliert sich zunehmend als Referenzrahmen in Forschung und Praxis.

Moderne KI-Systeme unterscheiden sich deutlich von traditionellen
softwarebasierten Ansätzen. Während frühe KI-Tools vor allem Aufgaben wie
Syntaxprüfung unterstützten, übernehmen generative KI-Systeme (GenAI) heute
komplexe Aufgaben im Design, der Entwicklung und Wartung von Software, etwa
durch die automatische Generierung von Code-Snippets, Unterstützung bei
Unit-Tests oder die Automatisierung von Deployment-Prozessen
\cite{donvir_role_2024}.

Zu den wichtigsten Architekturen zählen Transformer-Modelle (insbesondere Large
Language Models wie GPT), Generative Adversarial Networks (GANs), Variational
Autoencoders (VAEs) und Diffusion Models. Während LLMs primär für Text- und
Codegenerierung eingesetzt werden, kommen GANs und Diffusion Models vor allem
in der Bild- und Medienerzeugung zum Einsatz \cite{donvir_role_2024}. Viele
aktuelle Tools wie ChatGPT, GitHub Copilot oder Stable Diffusion basieren auf
diesen Architekturen und treiben Innovationen in der Softwareentwicklung
maßgeblich voran.

Der Einsatz generativer KI verändert Methoden und Arbeitsweisen grundlegend.
Besonders Large Language Models prägen sämtliche Phasen des
Softwareentwicklungszyklus, von der Anforderungsanalyse bis zur Umsetzung und
Wartung. Entscheidungsunterstützung, die Rekonstruktion von
Softwarearchitekturen sowie Methoden wie Few-Shot-Prompting und
Retrieval-Augmented Generation (RAG) sind dabei zentrale Elemente
\cite{esposito_generative_2025}.

Trotz fortschreitender Automatisierung bleibt die sorgfältige Validierung der
von KI generierten Ergebnisse durch Entwickler:innen unerlässlich. Die größten
Herausforderungen betreffen Präzision und Verlässlichkeit der Modelle, den
Umgang mit sogenannten „Halluzinationen“, ethische Aspekte sowie das Fehlen
domänenspezifischer Benchmarks und Standards \cite{esposito_generative_2025}.

Die Entwicklung der Softwaretechnik unter dem Einfluss von KI lässt sich grob
in drei Phasen gliedern. Während Software Engineering~1.0 klassische,
code-zentrierte Entwicklungspraktiken beschreibt, kennzeichnet SE~2.0 die
Integration KI-gestützter Assistenzsysteme (z.\,B. Copilots) in etablierte
Prozesse. Die Vision von SE~3.0 sieht einen Paradigmenwechsel zu einer
intent-basierten, dialogorientierten Entwicklung vor, in der KI-Systeme als
intelligente Partner agieren~\cite{hassan_towards_2024}.

\begin{table}[H]
    \centering
    \vspace{1em}
    \footnotesize
    \begin{tabular}{|p{3cm}|p{3.3cm}|p{3.3cm}|p{3.3cm}|}
        \hline
                                                 & \textbf{SE 1.0 (Vergangenheit)} & \textbf{SE 2.0 (Gegenwart)} & \textbf{SE 3.0 (Zukunft)} \\
        \hline
        Leitmotiv                                &
        Code-first                               &
        Code-first + KI-Unterstützung            &
        Intent-first, KI als Partner                                                                                                         \\
        \hline
        Technologie                              &
        Klassische Tools, Programmanalyse        &
        KI-gestützte Copilots, Foundation Models &
        Konversationsorientiert, Reasoning-Modelle                                                                                           \\
        \hline
        Rolle von Mensch/KI                      &
        Mensch zentral, alles manuell            &
        Mensch + KI, KI assistiert               &
        Symbiose: Mensch \& KI kollaborieren                                                                                                 \\
        \hline
        Besonderheiten                           &
        Fokus auf Code, klare Abläufe            &
        Effizienzsteigerung, hohe kognitive Last &
        Dialog, Automatisierung, wissensgetrieben                                                                                            \\
        \hline
    \end{tabular}
    \caption{Entwicklung der Softwaretechnik unter KI-Einfluss (SE~1.0 bis SE~3.0). Quelle: Eigene Darstellung in Anlehnung an Hassan et al.~\cite{hassan_towards_2024}.}
    \label{tab:se-evolution}
\end{table}

% \vspace{1em}
% \noindent
% \textbf{Quellen:}
% \begin{itemize}
%     \item Verordnung (EU) 2024/1689 des Europäischen Parlaments und des Rates
%           \cite{noauthor_verordnung_nodate}
%     \item Donvir, A. et al. (2024): \textit{The Role of Generative AI Tools in
%               Application Development: A Comprehensive Review of Current Technologies and
%               Practices} \cite{donvir_role_2024}
%     \item Esposito, M. et al. (2025): \textit{Generative AI for Software Architecture.
%               Applications, Trends, Challenges, and Future Directions}
%           \cite{esposito_generative_2025}
% \end{itemize}
