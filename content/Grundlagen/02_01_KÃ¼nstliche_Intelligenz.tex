
Künstliche Intelligenz (KI) leitet einen grundlegenden Wandel in der
Softwareentwicklung ein und verändert die Art und Weise, wie Software
entwickelt, getestet und betrieben wird \cite{esposito_generative_2025}. Eine
aktuelle, rechtlich verbindliche Definition der Europäischen Union beschreibt
KI als maschinengestützte Systeme, die mit unterschiedlichem Grad an Autonomie
Vorhersagen, Empfehlungen oder Entscheidungen generieren, welche physische oder
virtuelle Umgebungen beeinflussen können \cite{noauthor_verordnung_nodate}.
Diese Definition etabliert sich zunehmend als Referenzrahmen in Forschung und
Praxis.

Die Entwicklung moderner KI-Systeme unterscheidet sich deutlich von
traditionellen softwarebasierten Ansätzen. Während frühe KI-Tools vor allem
grundlegende Aufgaben wie Syntaxprüfung oder Codeformatierung unterstützten,
übernehmen generative KI-Systeme (GenAI) heute komplexe Aufgaben im Design, der
Entwicklung und Wartung von Software. Dazu gehören etwa die automatische
Generierung von Code-Snippets, Funktionen oder Modulen, die Unterstützung bei
Unit-Tests sowie die Automatisierung von Deployment-Prozessen
\cite{donvir_role_2024}.

Grundlage dieser Anwendungen sind verschiedene fortschrittliche
Modellarchitekturen, darunter Transformer-Modelle (insbesondere Large Language
Models wie GPT), Generative Adversarial Networks (GANs), Variational
Autoencoders (VAEs) und Diffusion Models. Während LLMs primär für Text- und
Codegenerierung eingesetzt werden, kommen GANs und Diffusion Models vor allem
in der Bild- und Medienerzeugung zum Einsatz. VAEs spielen insbesondere bei der
Generierung und Verarbeitung komplexer Datensätze eine Rolle
\cite{donvir_role_2024}.

Diese Vielfalt an Modelltypen ermöglicht den breiten Einsatz generativer
KI-Systeme und fördert kontinuierlich die Entwicklung innovativer
Softwarewerkzeuge. Zahlreiche aktuelle Tools wie ChatGPT, GitHub Copilot oder
Stable Diffusion basieren auf diesen Architekturen und bieten vielfältige
Funktionen von der Text- bis zur Mediengenerierung. Damit bilden sie zugleich
die Grundlage für eine stete Weiterentwicklung und Innovation in der
Softwareentwicklung \cite{donvir_role_2024}.

Neben den technischen Fortschritten betonen Studien die grundlegenden
Veränderungen, die generative KI für Methoden und Arbeitsweisen in der
Entwicklung mit sich bringt. Besonders der Einsatz von Large Language Models
prägt sämtliche Phasen des Softwareentwicklungszyklus, angefangen bei der
Anforderungsanalyse bis hin zur Umsetzung von Softwarearchitektur und
Quellcode. Entscheidungsunterstützung und die Rekonstruktion von
Softwarearchitekturen zählen dabei zu den wichtigsten Anwendungsfeldern;
Methoden wie Few-Shot-Prompting und Retrieval-Augmented Generation (RAG) sind
heute fest etabliert \cite{esposito_generative_2025}.

Trotz fortschreitender Automatisierung bleibt der Mensch ein zentraler Faktor
für den Erfolg generativer KI-Systeme. Eine sorgfältige Validierung der von KI
generierten Ergebnisse durch Entwicklerinnen und Entwickler ist weiterhin
notwendig, um die Qualität und Sicherheit der Lösungen sicherzustellen. Die
wichtigsten Herausforderungen betreffen insbesondere die Präzision und
Verlässlichkeit der Modelle, den Umgang mit sogenannten „Halluzinationen“,
ethische Aspekte sowie das Fehlen domänenspezifischer Benchmarks und Standards
zur Evaluation \cite{esposito_generative_2025}.

% \vspace{1em}
% \noindent
% \textbf{Quellen:}
% \begin{itemize}
%     \item Verordnung (EU) 2024/1689 des Europäischen Parlaments und des Rates
%           \cite{noauthor_verordnung_nodate}
%     \item Donvir, A. et al. (2024): \textit{The Role of Generative AI Tools in
%               Application Development: A Comprehensive Review of Current Technologies and
%               Practices} \cite{donvir_role_2024}
%     \item Esposito, M. et al. (2025): \textit{Generative AI for Software Architecture.
%               Applications, Trends, Challenges, and Future Directions}
%           \cite{esposito_generative_2025}
% \end{itemize}
