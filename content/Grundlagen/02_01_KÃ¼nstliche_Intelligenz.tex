Künstliche Intelligenz (KI) hat sich in den letzten Jahren zu einer
Schlüsseltechnologie in der Softwareentwicklung entwickelt. Eine anerkannte und
aktuelle Definition liefert die Europäische Union. Demnach bezeichnet

\begin{quote}
    „Künstliche Intelligenz (KI) [...] ein maschinengestütztes System, das mit unterschiedlichem Grad an Autonomie für ausdrücklich oder implizit gesetzte Ziele Vorhersagen, Empfehlungen oder Entscheidungen generiert, die physische oder virtuelle Umgebungen beeinflussen können.“
\end{quote}
\cite{noauthor_verordnung_nodate}

Diese Definition orientiert sich an der aktuellen europäischen Gesetzgebung und
wird in der wissenschaftlichen Diskussion zunehmend als Standardbegriff
verwendet.

Die Entwicklung von KI-Systemen unterscheidet sich deutlich von klassischen
softwarebasierten Ansätzen. Während erste KI-gestützte Werkzeuge lediglich
grundlegende Aufgaben wie Syntaxprüfung oder Code-Formatierung unterstützten,
übernehmen moderne Generative-AI-Tools Aufgaben, die maßgeblichen Einfluss auf
das Design, die Entwicklung und Wartung von Software haben. Dazu zählen die
Generierung von Code-Snippets, ganzen Funktionen oder Modulen, die
Unterstützung bei der Erstellung von Unit-Tests sowie die Automatisierung von
Deployment-Prozessen \cite{donvir_role_2024}.

In modernen generativen KI-Systemen kommen unterschiedliche Modellarchitekturen
zum Einsatz, die als Grundlage für viele aktuelle Anwendungen dienen. Donvir et
al. \cite{donvir_role_2024} nennen in ihrer Übersicht insbesondere
Transformer-Modelle (wie Large Language Models, z.B. GPT), Generative
Adversarial Networks (GANs), Variational Autoencoders (VAEs) und Diffusion
Models als zentrale Modelltypen:

\begin{quote}
    „They also present a taxonomy of generative AI models, such as Generative Adversarial Networks (GANs), variational autoencoders (VAEs), and transformers, which are foundational to many modern GenAI applications.“
\end{quote}
\cite[S.~2]{donvir_role_2024}

Transformer-Modelle, insbesondere Large Language Models wie GPT, bilden die
Grundlage für viele KI-gestützte Text- und Codegenerierungswerkzeuge (z.B.
ChatGPT, GitHub Copilot oder Bard). GANs werden häufig für die Generierung von
Bildern, Videos oder anderen Medieninhalten verwendet und stellen einen
wichtigen Baustein generativer Anwendungen dar. Variational Autoencoders (VAEs)
sind vor allem in der Datengenerierung und bei der effizienten Verarbeitung
komplexer Eingabedaten relevant. Diffusion Models wiederum kommen insbesondere
in der Bildgenerierung zum Einsatz und sind Grundlage moderner Tools wie Stable
Diffusion. Donvir et al. \cite{donvir_role_2024} heben hervor, dass diese
Modelle die Vielfalt und Leistungsfähigkeit generativer KI-Systeme wesentlich
bestimmen und die Entwicklung neuer Softwarewerkzeuge und -anwendungen
maßgeblich vorantreiben:

\begin{quote}
    „...the paper Advancements in Generative AI: A Comprehensive Review of GANs, GPT, Autoencoders, Diffusion Model, and Transformers [5] explores state-of-the-art GenAI models that power tools like ChatGPT, Bard, and Stable Diffusion. This review highlights the wide range of capabilities these models enable, from text generation to image creation, and emphasizes their applications in software development.“
\end{quote}
\cite[S.~2]{donvir_role_2024}

Die kontinuierliche Weiterentwicklung dieser Modellarchitekturen bedingt auch
die rasante Entwicklung neuer Werkzeuge und Methoden, die aktuelle Trends im
Bereich der KI-gestützten Softwareentwicklung bestimmen. Insbesondere
Generative KI (GenAI) prägt laut Esposito et al.
\cite{esposito_generative_2025} „die Art und Weise, wie Entwickler Code
entwerfen, schreiben und warten, grundlegend“. Besonders hervorzuheben ist der
breite Einsatz von Large Language Models (LLMs) wie den OpenAI GPT-Modellen,
die vorrangig in frühen Phasen des Softwareentwicklungszyklus, etwa im Übergang
von Requirements zu Architektur und von Architektur zu Code, Anwendung finden.

Entscheidungsunterstützung und die Rekonstruktion von Softwarearchitekturen
gehören zu den dominanten Anwendungsfeldern, wobei Methoden wie
Few-Shot-Prompting und Retrieval-Augmented Generation (RAG) etabliert sind. Ein
zentrales Merkmal aktueller Ansätze ist der weiterhin hohe Grad an menschlicher
Interaktion und Validierung, wodurch vollständig autonome KI-gestützte
Architekturentscheidungen bislang selten sind
\cite[S.~2,~10]{esposito_generative_2025}. Zu den Herausforderungen gehören
insbesondere die Sicherstellung von Präzision und Verlässlichkeit der Modelle,
der Umgang mit Halluzinationen und ethischen Fragestellungen sowie der Mangel
an domänenspezifischen Benchmarks und Evaluationsstandards
\cite[S.~2,~16]{esposito_generative_2025}.

\vspace{1em}
\noindent
\textbf{Quellen:}
\begin{itemize}
    \item Verordnung (EU) 2024/1689 des Europäischen Parlaments und des Rates
          \cite{noauthor_verordnung_nodate}
    \item Donvir, A. et al. (2024): \textit{The Role of Generative AI Tools in
              Application Development: A Comprehensive Review of Current Technologies and
              Practices} \cite{donvir_role_2024}
    \item Esposito, M. et al. (2025): \textit{Generative AI for Software Architecture.
              Applications, Trends, Challenges, and Future Directions}
          \cite{esposito_generative_2025}
\end{itemize}
