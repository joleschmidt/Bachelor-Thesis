Generative KI-Tools sind heute in der Softwareentwicklung weit verbreitet und
decken ein breites Spektrum an Aufgaben ab -- von der Code-Vervollständigung
über automatische Testgenerierung bis hin zur Unterstützung ganzer
Entwicklungsprojekte. Im Folgenden werden vier prominente Tools vorgestellt,
die in der Praxis besonders relevant sind: \textbf{GitHub Copilot},
\textbf{TabNine}, \textbf{Cursor AI} und \textbf{Devin AI}.

\paragraph{GitHub Copilot}
GitHub Copilot ist ein KI-basierter Codeassistent, der Entwicklern direkt im
IDE-Kontext Vorschläge für Code-Snippets, ganze Funktionen und sogar Tests
macht. Die Integration erfolgt beispielsweise in Visual Studio Code, IntelliJ
oder Eclipse. Copilot basiert auf dem OpenAI Codex-Modell und kann mit
Kommentaren oder natürlicher Sprache gesteuert werden. Typische Einsatzfelder
sind das schnelle Prototyping, die Generierung von Boilerplate-Code, aber auch
die Hilfestellung für Einsteiger und Onboarding-Prozesse.

\begin{quote}
    \enquote{GitHub Copilot can assist in quick prototyping of code by generating foundational code structure based on natural language description of the feature. It can assist in boilerplate code generation by providing the class and interface definition generation, API and Database Schema creation. Both of these features combined improve the developer efficiency and enhanced code quality.}
    \cite[S.~8]{donvir_role_2024}
\end{quote}

Praxiserfahrung zeigt, dass GitHub Copilot die Entwicklung etwa einer
React-Anwendung deutlich beschleunigen kann, indem es Codevorschläge für
Authentifizierung, Routing und Formularvalidierung generiert und Fehlerbehebung
unterstützt. Entwickler betonen, dass der Review und die Überprüfung der
generierten Vorschläge dennoch unerlässlich bleiben \cite{kerr_github_nodate}.

\paragraph{TabNine}
TabNine ist ein weiteres KI-gestütztes Tool zur Code-Vervollständigung, das
ursprünglich auf GPT-2 basierte und heute ein eigenes Modell nutzt. Es kann
Codevorschläge in Echtzeit für verschiedene Sprachen machen und passt sich über
die Zeit an den Coding-Stil des jeweiligen Entwicklers an. TabNine unterstützt
alle gängigen IDEs und bietet eine Chat-Funktion, um gezielt Code-Fragen zu
stellen. Besonders geschätzt wird die Flexibilität durch lokale und
cloudbasierte Modelle sowie die Anpassung an unterschiedliche
Datenschutzanforderungen \cite[S.~9]{donvir_role_2024}.

\paragraph{Cursor AI}
Cursor AI repräsentiert die nächste Generation von KI-Entwicklungstools. Es ist
in der Lage, auf Basis natürlicher Sprache ganze Applikationen zu generieren,
nutzt Techniken wie Retrieval-Augmented Generation (RAG) und Agentic AI und
kann selbstständig Code verfeinern und Projekte strukturieren. Der Fokus liegt
auf End-to-End-Entwicklung und vollständiger Projektgenerierung, was
insbesondere für Prototyping oder den schnellen Aufbau komplexer
Softwarelösungen geeignet ist.

\begin{quote}
    \enquote{Cursor AI can generate the entire codebase of the application from the feature description of the project provided in natural language. It uses advanced AI concepts such as Retrieval Augmented Generation (RAG), Agentic AI, and prompt chaining to achieve its objectives and provides a high degree of automation in software development.}
    \cite[S.~10]{donvir_role_2024}
\end{quote}

\paragraph{Devin AI}
Devin AI geht noch einen Schritt weiter und bezeichnet sich selbst als
\enquote{AI Software Engineer}. Devin kann komplette Softwareprojekte auf Basis
von Anforderungen in natürlicher Sprache umsetzen, das Projekt in einzelne
Aufgaben herunterbrechen, automatisiert testen und sogar Deployment-Skripte
erstellen. Ein wesentliches Merkmal ist die Fähigkeit zur langfristigen Planung
und zur kontinuierlichen Anpassung an neue Anforderungen
\cite[S.~11]{donvir_role_2024}.

\vspace{1em}
\noindent
\textbf{Typische Einsatzszenarien}

Die beschriebenen Tools werden in der Praxis in unterschiedlichen Bereichen
eingesetzt:
\begin{itemize}
    \item \textbf{Code-Generierung und Vervollständigung:} Automatisiertes Schreiben von Code, Vorschläge für Funktionen, Klassen, API-Integration etc.
    \item \textbf{Test- und Debugging-Unterstützung:} Generierung von Unit- und Integrationstests, Identifikation von Fehlern und Vorschläge für Bugfixes.
    \item \textbf{Projekt-Scaffolding und Boilerplate:} Automatisches Erstellen von Grundstrukturen für neue Projekte.
    \item \textbf{End-to-End-Entwicklung:} Vollständige Umsetzung von Projektanforderungen inklusive Deployment-Skripten und CI/CD-Konfiguration \cite[S.~11]{donvir_role_2024}.
\end{itemize}

\vspace{1em}
\noindent
\textbf{Vorteile und Grenzen in der Praxis}

Die Integration generativer KI-Tools führt zu Zeitersparnis, konsistenterem
Code und einer schnelleren Einarbeitung neuer Entwickler. Gleichzeitig sind
Review-Prozesse und ein kritischer Umgang mit den generierten Vorschlägen
unerlässlich, um Qualitäts- und Sicherheitsrisiken zu minimieren.
Fortgeschrittene Tools wie Cursor AI oder Devin AI bieten ein hohes Maß an
Automatisierung, sind aber oft kostenintensiv und noch nicht für alle
Anwendungsfälle ausgereift \cite[S.~13]{donvir_role_2024}.

\vspace{1em}
\noindent
\textbf{Quellen:}
\begin{itemize}
    \item Donvir, A. et al. (2024): \textit{The Role of Generative AI Tools in
              Application Development: A Comprehensive Review of Current Technologies and
              Practices} \cite{donvir_role_2024}
    \item Kerr, K. (2025): \textit{GitHub for Beginners: Building a React App with GitHub
              Copilot - The GitHub Blog} \cite{kerr_github_nodate}
\end{itemize}
