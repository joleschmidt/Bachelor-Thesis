\label{sec:generative-ki-tools}

Generative KI-Tools wie GitHub Copilot, Cursor oder v0 prägen den modernen
Softwareentwicklungsprozess entscheidend. Ihre Hauptfunktion besteht darin,
natürliche Sprache (Prompts) in ausführbaren Code, Testfälle oder
Dokumentationen umzusetzen. Damit verändern sie nicht nur technische Workflows,
sondern auch die Zusammenarbeit in Entwicklungsteams und die Anforderungen an
Kompetenzen der Beteiligten \cite{weisz_design_2024}.

\subsubsection{Grundfunktion generativer KI-Tools}

Die zugrundeliegenden Large Language Models (LLMs) ermöglichen das sogenannte
Prompt-basiertes Entwicklungsparadigma: Entwickler:innen beschreiben eine
Aufgabe in natürlicher Sprache – das Tool generiert dazu passende Vorschläge.
Diese erscheinen als Code Completion direkt beim Tippen oder als komplette
Funktionsblöcke \cite{kerr_github_nodate, weisz_design_2024}. Neuere Ansätze
wie SENAI integrieren generative KI von Anfang an in den
Software-Engineering-Prozess und erlauben eine hochautomatisierte,
KI-zentrierte Entwicklung \cite{saad_senai_2025}.

Systematische Literaturübersichten zeigen, dass die Integration generativer KI
in Entwicklungsumgebungen das Nutzererlebnis, die Akzeptanz und den praktischen
Nutzen maßgeblich beeinflusst. Insbesondere Aspekte wie Transparenz, Usability
und Feedbackmechanismen entscheiden über den Erfolg im Alltag
\cite{sergeyuk_human-ai_2025}.

\subsubsection{Schnittstellen und Integration in Entwicklungsumgebungen}

Die praktische Nutzung generativer KI erfolgt meist über Plugins (z. B. für
Visual Studio Code oder JetBrains IDEs) oder via API. Typischerweise
interagieren Entwickler:innen mit dem KI-Tool im Kontext der gewohnten
Umgebung, wodurch Aufgaben wie Refactoring, Testing oder Dokumentation direkt
integriert werden können \cite{kerr_github_nodate, shi_ai-assisted_2023,
    weisz_design_2024}.

Besonderes Augenmerk liegt inzwischen auch auf Barrierefreiheit: KI-gestützte
Assistenzsysteme eröffnen etwa Entwickelnden mit Sehbeeinträchtigung neue
Teilhabemöglichkeiten, vorausgesetzt die Tools sind entsprechend gestaltet
\cite{flores-saviaga_impact_2025}.

\subsubsection{Beispielhafte Workflows: Pair Programming mit Copilot}

Beim Pair Programming mit Copilot werden Aufgaben in Form von Prompts gestellt,
Copilot generiert daraufhin Codevorschläge, die geprüft, angepasst oder
verworfen werden können. So unterstützt Copilot u. a. die Testautomatisierung,
das Refactoring und die Dokumentation in allen Phasen des
Entwicklungsprozesses. Studien belegen signifikante Beschleunigungseffekte bei
Routineaufgaben \cite{kerr_github_nodate, weisz_design_2024,
    shi_ai-assisted_2023}. In agilen Teams kann generative KI zudem zur
Qualitätsbewertung und Dokumentation von Anforderungen beitragen
\cite{geyer_case_2025}.

\subsubsection{Vorteile und Optimierungspotenziale}

Der Einsatz generativer KI bietet zahlreiche Vorteile:
\begin{itemize}
    \item \textbf{Zeitersparnis:} Automatisierung von Routineaufgaben, Verkürzung von Entwicklungszyklen.
    \item \textbf{Verbesserte Codequalität:} Erkennung häufiger Fehler, Empfehlungen für Best Practices.
    \item \textbf{Niedrigere Einstiegshürden:} Unterstützung auch für weniger erfahrene Entwickler:innen durch kontextbasierte Vorschläge.
\end{itemize}
KI-gestützte Code Reviews fördern die Kollaboration zwischen Mensch und Maschine und erhöhen die Transparenz im Entwicklungsprozess \cite{alami_human_2025}. Kontinuierliche Modellverbesserung und flexible Integration in unterschiedliche Projekte steigern das Optimierungspotenzial weiter \cite{kerr_github_nodate, weisz_design_2024}.

\subsubsection{Grenzen und typische Fehlerquellen}

Zu den zentralen Herausforderungen zählen:
\begin{itemize}
    \item \textbf{Halluzinationen:} Syntaktisch korrekter, aber inhaltlich fehlerhafter oder unsicherer Code, vor allem bei vagen Prompts \cite{shi_ai-assisted_2023}.
    \item \textbf{Bias und Kontextdefizite:} Übernahme von Vorurteilen oder Nichtbeachtung spezifischer Projektregeln.
    \item \textbf{Sicherheitsrisiken:} Vorschläge für unsicheren Code (z. B. Hardcoded Credentials), die explizit überprüft werden müssen \cite{shi_ai-assisted_2023}.
    \item \textbf{Übermäßiges Vertrauen:} Unkritische Übernahme von KI-Vorschlägen ohne Review durch erfahrene Entwickler:innen \cite{weisz_design_2024}.
\end{itemize}

\subsubsection{Designprinzipien für den produktiven und sicheren Einsatz}

Für die sichere Nutzung generativer KI-Tools werden in der Literatur folgende
Designprinzipien empfohlen \cite{weisz_design_2024}:
\begin{itemize}
    \item \textbf{Design for Mental Models:} Nutzer:innen sollen die Funktionsweise und Grenzen nachvollziehen können.
    \item \textbf{Design for Appropriate Trust \& Reliance:} Feedbackmechanismen müssen sowohl Vertrauen fördern als auch zur kritischen Prüfung anregen.
    \item \textbf{Design for Imperfection:} Aktive Hinweise auf mögliche Fehler und die Befähigung zur Korrektur sind essenziell.
\end{itemize}

Die Gestaltung generativer GUIs und Entwicklungsprozesse muss diese Prinzipien
berücksichtigen, um eine nachhaltige und verantwortungsvolle Integration zu
ermöglichen \cite{lee_towards_2025, chen_genui_2025, gill_agile_2025}.
Flexibilität, Feedback und kontinuierliche Anpassung sind Schlüssel zum Erfolg.

% Optional: Kurzbezug auf Kapitel 3 (Praxisbeispiel), falls erwünscht
% Wie in Kapitel 3 praktisch demonstriert, lassen sich diese Prinzipien in der Entwicklung von React Native-Anwendungen mit KI-Tools exemplarisch umsetzen.

