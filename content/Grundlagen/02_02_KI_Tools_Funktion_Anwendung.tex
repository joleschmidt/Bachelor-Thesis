\label{sec:generative-ki-tools}

Generative KI-Tools wie GitHub Copilot, Cursor oder Bolt prägen den modernen
Softwareentwicklungsprozess entscheidend. Ihre Hauptfunktion besteht darin,
natürliche Sprache, sogenannte Prompts, in ausführbaren Code, Testfälle oder
Dokumentationen umzusetzen. Damit verändern sie sowohl technische Workflows als
auch die Zusammenarbeit in Entwicklungsteams und stellen neue Anforderungen an
die Kompetenzen der Beteiligten \cite{weisz_design_2024}.

Die zugrundeliegenden Large Language Models (LLMs) ermöglichen ein
Prompt-basiertes Entwicklungsparadigma. Entwickler:innen beschreiben Aufgaben
in natürlicher Sprache und erhalten daraufhin passende Vorschläge. Diese
erscheinen entweder als \textit{Code Completion} direkt beim Tippen oder als
vollständige Funktionsblöcke \cite{kerr_github_nodate, weisz_design_2024}.
Neuere Ansätze wie SENAI integrieren generative KI von Beginn an in den
Software-Engineering-Prozess und erlauben so eine hochautomatisierte,
KI-zentrierte Entwicklung \cite{saad_senai_2025}.

Systematische Literaturübersichten zeigen, dass die Integration generativer KI
in Entwicklungsumgebungen das Nutzererlebnis, die Akzeptanz und den praktischen
Nutzen maßgeblich beeinflusst. Besonders Faktoren wie Transparenz,
Benutzerfreundlichkeit und intelligente Feedbackmechanismen sind entscheidend
für den Erfolg im Alltag \cite{sergeyuk_human-ai_2025}. Die praktische Nutzung
generativer KI erfolgt heute meist direkt über Plugins für etablierte IDEs wie
Visual Studio Code oder JetBrains, oftmals auch über API-Schnittstellen.
Dadurch lassen sich Aufgaben wie Refaktorisierung, Testing oder Dokumentation
unmittelbar und nahtlos in die gewohnte Arbeitsumgebung integrieren
\cite{kerr_github_nodate, shi_ai-assisted_2023, weisz_design_2024}. Besonders
hervorzuheben ist zudem, dass KI-gestützte Assistenzsysteme neue
Teilhabemöglichkeiten für sehbeeinträchtigte Entwickler:innen eröffnen können,
vorausgesetzt, die Tools sind barrierefrei gestaltet
\cite{flores-saviaga_impact_2025}.

Ein typischer Workflow zeigt sich etwa beim Pair Programming mit Copilot:
Aufgaben werden in Form von Prompts gestellt, das Tool generiert passende
Codevorschläge, die geprüft, angepasst oder verworfen werden können. Studien
belegen, dass diese Arbeitsweise Routineaufgaben wie Testautomatisierung,
Refaktorisierung und Dokumentation signifikant beschleunigt
\cite{kerr_github_nodate, weisz_design_2024, shi_ai-assisted_2023}. In agilen
Teams kann generative KI zudem einen Beitrag zur Qualitätsbewertung und zur
Dokumentation von Anforderungen leisten \cite{geyer_case_2025}.

Der Einsatz generativer KI bietet zahlreiche Vorteile, die in aktuellen Studien
hervorgehoben werden. So wird insbesondere die Automatisierung von
Routineaufgaben und die damit verbundene Zeitersparnis hervorgehoben, genauso
wie die Verbesserung der Codequalität durch die Erkennung häufiger Fehler und
gezielte Empfehlungen für Best Practices. Außerdem werden niedrigere
Einstiegshürden für weniger erfahrene Entwickler:innen genannt, denen durch
kontextbasierte Vorschläge der Zugang zur Entwicklung erleichtert wird
\cite{donvir_role_2024, sergeyuk_human-ai_2025}. Darüber hinaus fördern
KI-gestützte Code Reviews die Kollaboration zwischen Mensch und Maschine und
erhöhen die Transparenz im Entwicklungsprozess \cite{alami_human_2025}. Durch
kontinuierliche Modellverbesserung und flexible Integration in unterschiedliche
Projekte kann das Optimierungspotenzial weiter gesteigert werden
\cite{kerr_github_nodate, weisz_design_2024}.

Trotz dieser Vorteile bestehen zentrale Herausforderungen. Ein häufig
diskutiertes Problem sind sogenannte \textit{Halluzinationen}. Modelle
generieren zwar syntaktisch korrekten, inhaltlich aber fehlerhaften oder
unsicheren Code, besonders bei unscharfen oder vagen Prompts
\cite{shi_ai-assisted_2023, weisz_design_2024}. Hinzu kommen Bias und
Kontextdefizite, also die Übernahme von Vorurteilen aus den Trainingsdaten oder
das Nichtbeachten projektspezifischer Regeln. Ein weiteres Risiko besteht in
der Generierung unsicheren Codes, etwa durch Vorschläge mit hardcodierten
Zugangsdaten, die explizit überprüft werden müssen. Zudem beobachten Studien
ein übermäßiges Vertrauen vieler Entwickler:innen in die Vorschläge der KI,
eine unkritische Übernahme ohne manuelle Überprüfung kann zu schwerwiegenden
Fehlern führen \cite{shi_ai-assisted_2023, weisz_design_2024}.

Um einen produktiven und sicheren Einsatz zu gewährleisten, empfiehlt die
Literatur spezifische Designprinzipien für generative KI-Tools
\cite{weisz_design_2024}. Erstens sollte die Gestaltung der Systeme so
erfolgen, dass Nutzer:innen die Funktionsweise und Grenzen nachvollziehen
können (\textit{Design for Mental Models}). Zweitens müssen Feedbackmechanismen
sowohl Vertrauen fördern als auch zur kritischen Prüfung anregen
(\textit{Design for Appropriate Trust \& Reliance}). Drittens ist die
Berücksichtigung von Fehlern essenziell. Tools sollten aktiv auf mögliche
Fehler hinweisen und die Nutzer:innen zur Korrektur befähigen (\textit{Design
    for Imperfection}). Die Gestaltung generativer GUIs und Entwicklungsprozesse
muss diese Prinzipien berücksichtigen, um eine nachhaltige und
verantwortungsvolle Integration zu ermöglichen \cite{lee_towards_2025,
    chen_genui_2025, gill_agile_2025}. Flexibilität, Feedback und kontinuierliche
Anpassung sind hier Schlüsselfaktoren.

Insgesamt zeigt sich, dass der Mehrwert generativer KI-Tools vor allem dann zum
Tragen kommt, wenn sie sinnvoll in bestehende Entwicklungsumgebungen
integriert, kritisch überprüft und auf die jeweiligen Team- und
Projektanforderungen angepasst werden.

% Optional: Kurzbezug auf Kapitel 3 (Praxisbeispiel), falls erwünscht
% Wie in Kapitel 3 praktisch demonstriert, lassen sich diese Prinzipien in der Entwicklung von React Native-Anwendungen mit KI-Tools exemplarisch umsetzen.

