\label{sec:generative-ki-tools}

\subsection{Grundfunktion generativer KI-Tools}

Generative KI-Tools wie GitHub Copilot, Cursor und v0 basieren auf
leistungsfähigen Large Language Models (LLMs), die natürliche Sprache
interpretieren und daraus konkrete Vorschläge für Code, Tests oder
Dokumentation generieren. Mit SENAI wurde ein KI-natives Framework vorgestellt,
das generative KI von Beginn an in den Software-Engineering-Prozess integriert
und so die Entwicklung hochautomatisierter, KI-zentrierter Anwendungen
ermöglicht \cite{saad_senai_2025}. Charakteristisch ist das sogenannte
\textit{Prompt-Output-Paradigma}: Entwickler:innen geben eine
Aufgabenbeschreibung als Prompt ein, worauf das KI-Tool passenden Code
vorschlägt. Besonders in modernen Entwicklungsumgebungen erscheinen diese
Vorschläge direkt beim Tippen (Code Completion) oder als vollständige
Funktionsblöcke \cite{kerr_github_nodate,weisz_design_2024}.

Eine systematische Literaturstudie belegt, dass die Integration von KI in
Entwicklungsumgebungen das Nutzererlebnis maßgeblich beeinflusst. Aspekte wie
Transparenz, Usability und Feedbackmechanismen entscheiden laut Sergeyuk et al.
darüber, ob generative Tools in der Praxis akzeptiert und effektiv genutzt
werden \cite{sergeyuk_human-ai_2025}.

\subsection{Schnittstellen und Integration in Entwicklungsumgebungen}

Die praktische Integration generativer KI erfolgt überwiegend über IDE-Plugins
(z.B. Visual Studio Code, JetBrains) oder APIs. GitHub Copilot etwa lässt sich
als Erweiterung in gängigen Editoren installieren und unterstützt
Entwickler:innen unmittelbar im Arbeitsprozess. Zusätzlich stehen Chat-basierte
Interaktionen zur Verfügung, um Aufgaben wie Refactoring, Debugging oder
Testautomatisierung effizient zu bearbeiten. Die nahtlose Einbindung in
bestehende Entwicklungsumgebungen erleichtert den Zugang und fördert die
Akzeptanz \cite{kerr_github_nodate,shi_ai-assisted_2023,weisz_design_2024}.

Ein zunehmend wichtiger Aspekt ist die Barrierefreiheit: Studien zeigen, dass
KI-gestützte Coding-Assistenzsysteme insbesondere für Entwickler:innen mit
Sehbeeinträchtigung neue Möglichkeiten zur Teilhabe schaffen – vorausgesetzt,
die Tools sind inklusiv gestaltet \cite{flores-saviaga_impact_2025}.

\subsection{Beispielhafte Workflows: Pair Programming mit Copilot}

Im Pair Programming mit GitHub Copilot werden Aufgaben als Prompts formuliert
(z.B. „Implementiere eine Authentifizierung in React“). Copilot erzeugt
daraufhin passenden Beispielcode, der übernommen oder angepasst werden kann.
Der Entwicklungsprozess bleibt interaktiv: Entwickler:innen prüfen die
Vorschläge, passen sie an oder verwerfen sie. Copilot kann in allen
Entwicklungsphasen eingesetzt werden – etwa für Testautomatisierung,
Refactoring oder Dokumentation. Studien zeigen, dass insbesondere
Routineaufgaben dadurch erheblich beschleunigt werden
\cite{kerr_github_nodate,weisz_design_2024,shi_ai-assisted_2023}. Darüber
hinaus belegt eine Fallstudie, dass generative KI-Tools bereits in agilen
Entwicklungsprojekten zur Qualitätsbewertung von Epics beitragen und so die
Teamarbeit unterstützen können \cite{geyer_case_2025}.

\subsection{Vorteile und Optimierungspotenziale}

Die Nutzung generativer KI-Tools bietet zahlreiche Vorteile:
\begin{itemize}
    \item \textbf{Zeitersparnis:} Routineaufgaben werden automatisiert, wodurch sich Entwicklungszeiten deutlich verkürzen.
    \item \textbf{Verbesserte Codequalität:} Tools wie Copilot erkennen häufige Fehlerquellen und schlagen bewährte Lösungen vor.
    \item \textbf{Niedrigere Einstiegshürden:} Auch weniger erfahrene Entwickler:innen profitieren von kontextabhängigen Vorschlägen und Beispielen.
\end{itemize}
KI-gestützte Code Reviews stärken zudem die Kollaboration zwischen menschlichen Entwickler:innen und automatisierten Tools, da Bewertungen und Verbesserungsvorschläge gezielter kommuniziert werden können \cite{alami_human_2025}. Weiteres Optimierungspotenzial ergibt sich durch die fortlaufende Verbesserung der zugrunde liegenden Modelle und deren flexible Integration in unterschiedlichste Projekte \cite{kerr_github_nodate,weisz_design_2024}.

\subsection{Grenzen und typische Fehlerquellen}

Trotz ihres Potenzials sind generative KI-Tools nicht fehlerfrei. Zu den
häufigsten Herausforderungen zählen:
\begin{itemize}
    \item \textbf{Halluzinationen:} KI kann syntaktisch korrekten, aber fachlich falschen oder unsicheren Code vorschlagen, insbesondere bei unpräzisen Prompts \cite{shi_ai-assisted_2023}.
    \item \textbf{Bias und Kontextdefizite:} Die KI reproduziert möglicherweise Vorurteile oder ignoriert projektspezifische Regeln.
    \item \textbf{Sicherheitsrisiken:} Copilot schlägt mitunter unsicheren Code (wie SQL-Injection oder Hardcoded Credentials) vor, sofern nicht explizit nach sicheren Lösungen gefragt wird. Moderne Versionen reagieren jedoch sensibler auf entsprechende Prompts \cite{shi_ai-assisted_2023}.
    \item \textbf{Übermäßiges Vertrauen:} Entwickler:innen übernehmen KI-Vorschläge mitunter ungeprüft, weshalb Mechanismen zur kritischen Prüfung essenziell sind \cite{weisz_design_2024}.
\end{itemize}

\subsection{Designprinzipien für den produktiven und sicheren Einsatz}

Für einen erfolgreichen und sicheren Einsatz generativer KI-Tools empfiehlt die
Literatur zentrale Designprinzipien \cite{weisz_design_2024}:
\begin{itemize}
    \item \textbf{Design for Mental Models:} Tools sollten so gestaltet sein, dass Nutzende die Funktionsweise nachvollziehen können.
    \item \textbf{Design for Appropriate Trust \& Reliance:} Es sollten Mechanismen existieren, die sowohl Vertrauen fördern als auch zur kritischen Prüfung anregen (z.B. Hinweise auf Unsicherheiten, Feedbackmechanismen).
    \item \textbf{Design for Imperfection:} Nutzer:innen müssen aktiv auf potenzielle Fehler hingewiesen und zur Korrektur befähigt werden.
\end{itemize}

Die Gestaltung generativer Benutzeroberflächen (GUIs) bringt eigene
Anforderungen mit sich, die sich von klassischen UI-Methoden unterscheiden und
neue Prinzipien für Usability und Interaktion verlangen
\cite{lee_towards_2025}. Ergänzend betonen aktuelle Studien, dass für
UX-Praktiker:innen die Einbindung generativer UI-Tools neue methodische Ansätze
und Werkzeugunterstützung erfordert \cite{chen_genui_2025}.

Nicht zuletzt sollte der Entwicklungsprozess selbst an die neuen Möglichkeiten
und Herausforderungen angepasst werden: Für KI-basierte Systeme sind flexible,
entscheidungsorientierte Entwicklungszyklen empfehlenswert, die eine
nachhaltige und erfolgreiche KI-Integration ermöglichen \cite{gill_agile_2025}.

% \textbf{Quellen:}

% \begin{itemize}
%     \item\cite{geyer_case_2025},\cite{kerr_github_nodate},\cite{weisz_design_2024},\cite{martinovic_impact_2024},\cite{shi_ai-assisted_2023}
% \end{itemize}