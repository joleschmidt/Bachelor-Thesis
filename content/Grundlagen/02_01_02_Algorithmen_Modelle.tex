Die Integration generativer Künstlicher Intelligenz (KI) in der
Softwareentwicklung basiert maßgeblich auf fortschrittlichen Modellen und
Algorithmen. Im Zentrum stehen insbesondere Large Language Models (LLMs) und
Transformer-Architekturen, die als Grundlage moderner Coding-Tools wie GitHub
Copilot, Cursor oder v0 dienen. Im Folgenden werden die wichtigsten Modelle,
deren Funktionsweise und deren Bedeutung für typische
Softwareentwicklungsaufgaben erläutert.

\paragraph{Grundprinzipien und Funktionsweise moderner KI-Modelle}

Large Language Models (LLMs), wie beispielsweise GPT-4 oder Code Llama, sind
tiefenlernende neuronale Netze, die auf sehr großen Mengen von Quellcode und
natürlicher Sprache trainiert wurden. Sie nutzen Transformer-Architekturen, die
durch sogenannte Self-Attention-Mechanismen Kontextinformationen über lange
Sequenzen hinweg erfassen können. Dies ermöglicht es ihnen, sowohl syntaktisch
als auch semantisch komplexe Strukturen – wie etwa Code-Logik oder Designmuster
– zu erkennen und zu generieren~\cite{nguyen-duc_generative_2023,
    esposito_generative_2025}.

Diffusionsmodelle spielen hingegen vor allem in der Bild- und Grafikgenerierung
eine Rolle und sind für den textbasierten Softwareentwicklungsprozess bislang
von untergeordneter Bedeutung. Für Aufgaben wie Codegenerierung und
Architektur-Design sind LLMs und Transformer-Modelle
maßgeblich~\cite{weisz_design_2024}.

\paragraph{Beispiele für KI-Modelle in Coding-Tools}

Moderne Coding-Assistenzsysteme wie \textit{GitHub Copilot}, \textit{Cursor}
und \textit{v0} basieren auf Varianten großer Sprachmodelle, meist speziell auf
Programmcode vortrainiert (z.\,B. OpenAI Codex, Code Llama, StarCoder). Diese
Modelle können anhand des jeweiligen Kontexts (z.\,B. bestehender Code,
Kommentare, Projekthistorie) automatisiert neue Code-Abschnitte generieren,
Vorschläge zur Fehlerbehebung machen oder Testfälle
entwerfen~\cite{coutinho_role_2024, esposito_generative_2025}.

Typische Anwendungsbereiche sind:
\begin{itemize}
    \item \textbf{Code-Generierung}: Erstellung neuer Funktionen, Methoden oder ganzer Module auf Basis von Kurzbeschreibungen oder natürlicher Sprache.
    \item \textbf{Testing und Qualitätssicherung}: Automatisierte Generierung von Unit-Tests und Testdaten, Unterstützung beim Review durch Erkennung von Anomalien oder Schwachstellen.
    \item \textbf{Architekturvorschläge}: Unterstützung bei der Auswahl geeigneter Softwarearchitekturen oder Design Patterns, teilweise mit Bezug auf bestehende Anforderungen oder Projektdaten~\cite{esposito_generative_2025}.
\end{itemize}

\paragraph{Rolle dieser Modelle für spezifische Aufgaben}

\begin{itemize}
    \item \textbf{Codegenerierung:} LLMs können auf Grundlage von Prompts oder bestehenden Code-Fragmente eigenständig funktionsfähigen Quellcode erzeugen. Dies umfasst Routineaufgaben (z.\,B. Boilerplate-Code) ebenso wie komplexere Algorithmen.
    \item \textbf{Testing:} Modelle wie GPT-4 sind in der Lage, automatisiert Tests zu erzeugen, Testdaten zu variieren und gängige Fehlerbilder zu erkennen.
    \item \textbf{Architekturvorschläge:} Moderne LLMs unterstützen zunehmend beim Entwurf und bei der Dokumentation von Softwarearchitekturen, indem sie z.\,B. Requirements in Design-Vorschläge oder Diagramme übersetzen~\cite{esposito_generative_2025, nguyen-duc_generative_2023}.
\end{itemize}

\paragraph{Herausforderungen und Grenzen}

Trotz des enormen Potenzials bestehen aktuelle Herausforderungen:
\begin{itemize}
    \item \textbf{Halluzinationen und Fehleranfälligkeit}: Generative Modelle können syntaktisch korrekten, aber fachlich unpassenden oder sogar gefährlichen Code erzeugen.
    \item \textbf{Erklärbarkeit und Transparenz}: Die Nachvollziehbarkeit, wie ein Modell zu bestimmten Ergebnissen kommt, ist oft eingeschränkt~\cite{esposito_generative_2025, nguyen-duc_generative_2023}.
    \item \textbf{Domänenspezifisches Wissen}: Ohne Anpassung (Fine-Tuning) auf projektspezifische Daten sind die Modelle oft auf allgemeines Wissen beschränkt und berücksichtigen spezifische Anforderungen nur begrenzt.
\end{itemize}

\paragraph{Bezug zu den im Praxisteil genutzten Tools}

Die im Praxisteil eingesetzten Tools (\textit{GitHub Copilot, Cursor, v0})
nutzen genau diese Algorithmen, um Entwickler:innen bei alltäglichen
Entwicklungsaufgaben zu unterstützen. Sie liefern damit nicht nur klassische
Code-Vervollständigungen, sondern wirken zunehmend als kollaborative Partner im
gesamten Entwicklungsprozess – von Architektur über Implementierung bis zum
Test~\cite{esposito_generative_2025, nguyen-duc_generative_2023}.

