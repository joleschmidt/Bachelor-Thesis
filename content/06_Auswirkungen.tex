\chapter{Wirtschaftliche und gesellschaftliche Auswirkungen}
\label{chap:wirtschaftliche_und_gesellschaftliche_auswirkungen}
Die Integration generativer KI in die Softwareentwicklung wirkt sich nicht nur auf technischer Ebene, sondern zunehmend auch auf wirtschaftliche Strukturen und gesellschaftliche Prozesse aus. Die folgenden Abschnitte analysieren, wie sich Unternehmen, Arbeitsmärkte und die Rolle von Softwareentwickler:innen im Zuge der KI-Einführung verändern. Im Zentrum stehen Fragen nach Produktivität, Wertschöpfung, Arbeitsplatzstruktur und den Chancen und Risiken für Unternehmen und Gesellschaft.

\section{Veränderungen in Softwareunternehmen}
\input{content/Auswirkungen/06_01_Veränderungen_in_Softwareunternehmen.tex}

\section{Auswirkungen auf den Arbeitsmarkt und Entwickler:innen-Rollen}
Die Verbreitung generativer KI wirkt sich bereits heute spürbar auf den
Arbeitsmarkt für Softwareentwickler:innen aus.
Marguerit~\cite{marguerit_augmenting_2025} argumentiert, dass KI nicht nur
Automatisierung, sondern auch eine qualitative Veränderung von Arbeit mit sich
bringt: Während repetitive Tätigkeiten und Routinetasks zunehmend automatisiert
werden, entstehen neue Aufgabenfelder rund um die Steuerung, Überwachung und
das Feintuning von KI-basierten Systemen.

Ahmadi et al.~\cite{ahmadi_generative_2024} zeigen anhand von Analysen
aktueller Jobanzeigen, dass Kompetenzen im Umgang mit Tools wie ChatGPT,
Copilot und anderen generativen KI-Anwendungen zunehmend nachgefragt werden.
Dies deutet darauf hin, dass die Nachfrage nach klassischen
Programmierkenntnissen zwar bestehen bleibt, aber zunehmend durch Fähigkeiten
im Prompt-Engineering, KI-Management und in der Bewertung KI-generierter
Ergebnisse ergänzt wird.

Farach et al.~\cite{farach_evolving_2025} sehen in der digitalen Arbeit mit
KI-Tools einen neuen Produktionsfaktor, der es Unternehmen ermöglicht,
Entwicklungsleistungen flexibler und globaler zu organisieren. Zugleich betonen
sie, dass die Einführung generativer KI auch zu Unsicherheiten auf dem
Arbeitsmarkt führen kann, etwa durch die Verlagerung von Aufgaben,
Veränderungen im Qualifikationsprofil oder mögliche Substitutionseffekte bei
stark standardisierten Tätigkeiten.

Storey et al.~\cite{storey_generative_2025} machen darauf aufmerksam, dass der
Wandel nicht nur technische, sondern auch soziale Kompetenzen erfordert. Die
Fähigkeit, mit KI-Systemen kollaborativ zu arbeiten, ethische Risiken zu
erkennen und verantwortungsvoll mit automatisierten Vorschlägen umzugehen, wird
zu einer Schlüsselkompetenz in modernen Entwicklerteams. Aktuelle Studien
zeigen, dass generative KI-Tools die Arbeitsmuster von Wissensarbeiter:innen
spürbar verändern und insbesondere Aufgaben mit hohem Automatisierungspotenzial
nachhaltig beeinflussen \cite{dillon_shifting_2025}.

Marguerit~\cite{marguerit_augmenting_2025} weist zudem darauf hin, dass sich
nicht nur die Art der Arbeit, sondern auch Beschäftigungsstrukturen und
Lohngefüge im Zuge der Automatisierung durch KI deutlich wandeln werden. Die
langfristigen Auswirkungen auf die Arbeitsplatzsicherheit und berufliche
Entwicklung sind dabei weiterhin Gegenstand intensiver Forschung.

% \begin{itemize} 
%     \item Verschiebung der gefragten Kompetenzen und Qualifikationen
%     \item Neue Berufsbilder und veränderte Karrierewege
%     \item Auswirkungen auf die Arbeitsplatzsicherheit und die Notwendigkeit der Weiterbildung
% \end{itemize}

\section{Zukunftsperspektiven und strategische Empfehlungen}
Die wissenschaftliche Literatur macht deutlich, dass der Einsatz generativer KI
in der Softwareentwicklung noch am Anfang einer langfristigen
Transformationsphase steht. Storey et al.~\cite{storey_generative_2025} sehen
großes Potenzial für Unternehmen, sich durch frühzeitige Investitionen in
KI-Kompetenzen, datenbasierte Prozesse und ethische Leitlinien
Wettbewerbsvorteile zu sichern. Marguerit~\cite{marguerit_augmenting_2025}
betont, dass kontinuierliche Weiterbildung und die Entwicklung neuer
Rollenprofile entscheidend sind, damit Beschäftigte mit dem technologischen
Wandel Schritt halten können.

Habibi~\cite{habibi_open_2025} unterstreicht die Bedeutung von
Open-Source-Ansätzen und kollaborativen Entwicklungsmodellen, um Innovation und
Transparenz im KI-Ökosystem zu fördern. McNamara und
Marpu~\cite{mcnamara_exponential_2025} weisen darauf hin, dass Unternehmen
verstärkt auf flexible, adaptive Strukturen setzen müssen, um auf die
exponentiell steigende Geschwindigkeit technologischer Veränderungen reagieren
zu können.

Die Autoren sind sich einig, dass die Integration von GenAI-Tools nur dann
nachhaltig gelingt, wenn Unternehmen strategisch in Kompetenzen, Change
Management und eine offene Innovationskultur investieren. Farach et
al.~\cite{farach_evolving_2025} sehen zudem in der Betrachtung digitaler Arbeit
als eigenständigen Produktionsfaktor einen wichtigen Schritt, um die
Wertschöpfungspotenziale von KI systematisch zu erschließen und zugleich
soziale Risiken abzufedern.

% \begin{itemize}
%     \item Notwendige Anpassungen für Unternehmen und Entwickler
%     \item Möglichkeiten der Integration von KI in bestehende Entwicklungsprozesse
%     \item Regulatorische und ethische Implikationen für eine nachhaltige KI-Nutzung
% \end{itemize}

\section{Kosten-Nutzen-Analyse von KI-gestützter Softwareentwicklung}
\input{content/Auswirkungen/06_04_Kosten-Nutzen-Analyse_von_KI-gestützter_Softwareentwicklung.tex}