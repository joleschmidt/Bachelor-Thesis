\chapter{Wirtschaftliche und gesellschaftliche Auswirkungen}
\label{chap:wirtschaftliche_und_gesellschaftliche_auswirkungen}
Die Integration generativer KI in die Softwareentwicklung hat nicht nur technische Implikationen, sondern prägt zunehmend auch wirtschaftliche Strukturen und gesellschaftliche Prozesse. Die folgenden Abschnitte untersuchen, wie sich Unternehmen, Arbeitsmärkte und die Rolle von Softwareentwickler:innen durch die Einführung von KI verändern. Im Mittelpunkt stehen Fragen zu Produktivität, Wertschöpfung, Arbeitsplatzstruktur sowie Chancen und Risiken für Unternehmen und Gesellschaft.

\section{Veränderungen in Softwareunternehmen}
Mit der Verbreitung generativer KI-Technologien stehen Softwareunternehmen vor
einem tiefgreifenden Wandel. Marguerit~\cite{marguerit_augmenting_2025}
beschreibt, dass Unternehmen zunehmend KI-basierte Automatisierungslösungen
einsetzen, um Entwicklungsprozesse zu beschleunigen und Ressourcen effizienter
zu nutzen. Diese Entwicklung führt dazu, dass klassische Rollenmodelle und
Teamstrukturen neu bewertet werden müssen.

Farach et al.~\cite{farach_evolving_2025} argumentieren, dass digitale Arbeit
durch KI-Tools einen eigenständigen Produktionsfaktor darstellt, der
traditionelle Vorstellungen von Arbeitsteilung und Wertschöpfung grundlegend
verändert. In vielen Unternehmen werden Aufgaben wie das Schreiben von
Standardcode, Testing oder das Generieren von Dokumentation zunehmend
automatisiert, während der Fokus auf kreative, überwachende und strategische
Tätigkeiten wächst.

Rothschild et~al.~\cite{rothschild_agentic_2025} diskutieren das Konzept der
„agentischen Ökonomie“, in der generative KI-Systeme zunehmend eigenständig
Entscheidungen treffen und so ganze Wertschöpfungsketten transformieren.
Unternehmen sind laut den Autoren gefordert, ihre Prozesse kontinuierlich an
diese neue Form digitaler Arbeit und Delegation anzupassen.

Storey et al.~\cite{storey_generative_2025} betonen, dass die Einführung
generativer KI-Tools nicht nur ökonomische, sondern auch organisatorische
Anpassungen erfordert. Unternehmen müssen neue Kompetenzen fördern,
Veränderungsbereitschaft unterstützen und klare ethische sowie regulatorische
Leitplanken setzen, um Risiken zu minimieren und die Akzeptanz im Team zu
erhöhen.

Die Literatur macht deutlich, dass der Erfolg von KI-Integration maßgeblich
davon abhängt, inwieweit Unternehmen nicht nur in Technologie, sondern auch in
Weiterbildung, Organisationsentwicklung und eine offene Innovationskultur
investieren.

% \begin{itemize}
%     \item Auswirkungen auf Geschäftsmodelle und Prozesse
%     \item Veränderungen in der Softwareentwicklung und im Projektmanagement
%     \item Rolle von KI bei der Automatisierung von Softwareentwicklungsaufgaben
% \end{itemize}

\section{Auswirkungen auf den Arbeitsmarkt und Entwickler:innen-Rollen}

Die zunehmende Verbreitung generativer KI wirkt sich bereits heute spürbar auf
den Arbeitsmarkt für Softwareentwickler:innen aus. Studien zeigen, dass nicht
nur klassische Routinetätigkeiten zunehmend automatisiert werden, sondern sich
auch die Qualifikationsprofile und Tätigkeitsfelder nachhaltig
verändern~\cite{siebert_generative_2024,braun_ki_2024,s_future_2024}. Während
repetitive Aufgaben wie das Schreiben von Boilerplate-Code, Testing oder
Dokumentation verstärkt durch KI-Tools übernommen werden, wächst der Bedarf an
Kompetenzen in Bereichen wie Prompt-Engineering, KI-Management und der
kritischen Bewertung KI-generierter Ergebnisse.

Analysen aktueller Jobprofile und -anzeigen belegen, dass die Nachfrage nach
Fähigkeiten im Umgang mit generativen KI-Anwendungen deutlich steigt. Neben
klassischen Programmierkenntnissen werden zunehmend auch Kompetenzen im Bereich
der Steuerung, Überwachung und dem Feintuning von KI-basierten Systemen
nachgefragt~\cite{ahmadi_generative_2024}. Insbesondere der sichere und
verantwortungsvolle Einsatz von Tools wie GitHub Copilot, ChatGPT oder
branchenspezifischen KI-Assistenzsystemen entwickelt sich zu einer
Schlüsselqualifikation moderner Entwickler:innen.

Mit der Einführung generativer KI-Tools verändert sich zudem die Rolle von
Entwickler:innen im Team und im Unternehmen. Der Fokus verschiebt sich von der
manuellen Implementierung hin zur strategischen Nutzung und Integration von
KI-Lösungen. Dies umfasst sowohl die Gestaltung von Entwicklungsprozessen als
auch die Bewertung und kontinuierliche Verbesserung von KI-basierten
Ergebnissen~\cite{storey_generative_2025}. Gleichzeitig entstehen neue
Rollenprofile, die Fähigkeiten in den Bereichen Data Science, Human-in-the-Loop
und ethische Bewertung von KI-Systemen erfordern.

Die Veränderungen auf dem Arbeitsmarkt sind jedoch ambivalent: Während
einerseits neue Aufgabenfelder und Qualifikationen entstehen, besteht zugleich
die Gefahr von Verunsicherung und möglichen Substitutionseffekten bei stark
standardisierten Tätigkeiten~\cite{farach_evolving_2025}. Die Literatur weist
darauf hin, dass Beschäftigungsstrukturen und Lohngefüge sich im Zuge der
Automatisierung durch KI nachhaltig wandeln
können~\cite{marguerit_augmenting_2025}.

Nicht zuletzt gewinnt die Fähigkeit zur Kollaboration mit KI-Systemen sowie die
Bereitschaft zu kontinuierlicher Weiterbildung an Bedeutung. Die Arbeitswelt
von Entwickler:innen wird damit vielseitiger, flexibler und erfordert neben
technischem Know-how zunehmend auch soziale und ethische
Kompetenzen~\cite{storey_generative_2025,siebert_generative_2024}.

% \begin{itemize} 
%     \item Verschiebung der gefragten Kompetenzen und Qualifikationen
%     \item Neue Berufsbilder und veränderte Karrierewege
%     \item Auswirkungen auf die Arbeitsplatzsicherheit und die Notwendigkeit der Weiterbildung
% \end{itemize}

\section{Zukunftsperspektiven und strategische Empfehlungen}

Die wissenschaftliche Literatur und aktuelle Branchenanalysen machen deutlich,
dass der Einsatz generativer KI in der Softwareentwicklung erst am Anfang einer
langfristigen Transformationsphase steht. Studien von Deloitte, IBM und
Fraunhofer IESE\cite{siebert_generative_2024, a_ki_2024, s_future_2024}
betonen, dass Unternehmen, die frühzeitig in KI-Kompetenzen, datenbasierte
Prozesse und ethische Leitlinien investieren, sich entscheidende
Wettbewerbsvorteile sichern können.

Eine zentrale Empfehlung der Literatur ist die **kontinuierliche
Weiterbildung** und Entwicklung neuer Rollenprofile. Fraunhofer IESE und
IBM\cite{siebert_generative_2024, a_ki_2024} unterstreichen, dass Unternehmen
gezielt in die Qualifikation ihrer Mitarbeitenden investieren müssen, um den
Anforderungen neuer Technologien gerecht zu werden.
Deloitte\cite{s_future_2024} hebt hervor, dass ein aktives Change Management
und eine offene Innovationskultur entscheidend sind, um Widerstände im Team und
Kompetenzlücken zu überwinden.

Die Literatur betont außerdem die Bedeutung von **Open-Source-Ansätzen** und
kollaborativen Entwicklungsmodellen, um Innovation und Transparenz im
KI-Ökosystem zu fördern\cite{siebert_generative_2024, a_ki_2024}. Zudem spielt
die Entwicklung klarer **ethischer und regulatorischer Leitplanken** eine
strategische Rolle. Insbesondere vor dem Hintergrund unterschiedlicher
rechtlicher Anforderungen und globaler Märkte ist die Ausgestaltung geeigneter
Governance-Strukturen eine zentrale Aufgabe für
Unternehmen\cite{siebert_generative_2024, s_future_2024}.

Die kontinuierliche Weiterentwicklung von Informationssystemen wird als Chance
gesehen, gesellschaftliche Teilhabe und Innovation zu fördern. Fraunhofer
IESE\cite{siebert_generative_2024} argumentiert, dass die gezielte Nutzung
generativer KI nicht nur ökonomische, sondern auch soziale und kreative
Potenziale freisetzen kann, wenn die Integration verantwortungsvoll erfolgt.

Zusammenfassend sind strategische Investitionen in Kompetenzen, Change
Management und Innovationskultur sowie die Entwicklung klarer ethischer
Leitlinien entscheidende Erfolgsfaktoren für die nachhaltige Integration
generativer KI in Unternehmen. Die langfristigen Potenziale können nur dann
ausgeschöpft werden, wenn technologische und organisatorische Transformation
Hand in Hand gehen.

% \begin{itemize}
%     \item Notwendige Anpassungen für Unternehmen und Entwickler
%     \item Möglichkeiten der Integration von KI in bestehende Entwicklungsprozesse
%     \item Regulatorische und ethische Implikationen für eine nachhaltige KI-Nutzung
% \end{itemize}

\section{Kosten-Nutzen-Analyse von KI-gestützter Softwareentwicklung}

Die Integration generativer KI in die Softwareentwicklung bringt sowohl
erhebliche Vorteile als auch neue Kosten- und Risikofaktoren mit sich. Aktuelle
Studien belegen, dass durch den Einsatz von KI-Tools die Produktivität in
vielen Bereichen signifikant steigt – insbesondere bei Routinetätigkeiten, der
Code-Generierung und der Testautomatisierung~\cite{marguerit_augmenting_2025,
    farach_evolving_2025, habibi_open_2025}. Dies führt zu messbaren
Effizienzgewinnen und kann langfristig die Entwicklungskosten pro Feature oder
Release deutlich senken.

Gleichzeitig entstehen neue Investitionen: Die Einführung und Wartung von
KI-Systemen erfordert gezielte Weiterbildung, Anpassungen der Infrastruktur und
oft eine Neuausrichtung bestehender Prozesse. Während proprietäre Lösungen
Lizenz- und Betriebskosten verursachen, sind Open-Source-Modelle zwar
kostengünstiger, erfordern aber häufig einen höheren Initialaufwand für
Anpassung und Integration~\cite{habibi_open_2025}. Song
et~al.~\cite{song_impact_2024} zeigen, dass KI-basierte Assistenzsysteme in
Open-Source-Projekten nicht nur die Kollaboration und Codequalität verbessern,
sondern auch zu einer Demokratisierung des Entwicklungsprozesses beitragen
können.

Zudem hängt der konkrete Nutzen von KI-gestützter Entwicklung maßgeblich davon
ab, inwieweit Unternehmen strategische Ziele, Kostenstruktur und
Wertschöpfungspotenziale aufeinander
abstimmen~\cite{mcnamara_exponential_2025}. Neben direkten Effizienzgewinnen
zählen auch Flexibilität, Innovationspotenzial und die Gewinnung von
Wettbewerbsvorteilen zu den zentralen
Nutzenaspekten~\cite{storey_generative_2025}. Dem stehen potenzielle
Folgekosten durch Fehlinvestitionen, Qualitätsprobleme bei KI-generiertem Code
und ethische Risiken gegenüber, die zu erheblichen finanziellen Belastungen
führen können, wenn sie nicht frühzeitig adressiert werden.

Nicht zuletzt verändern KI-Technologien auch kreative Branchen und ermöglichen
neue Formen gesellschaftlicher Teilhabe~\cite{anantrasirichai_artificial_2025}.
Eine sorgfältige Kosten-Nutzen-Abwägung sowie die kontinuierliche Anpassung der
Strategie bleiben daher zentrale Voraussetzungen für einen erfolgreichen und
nachhaltigen Einsatz generativer KI in der Softwareentwicklung.

% \begin{itemize}
%     \item Analyse der wirtschaftlichen Effizienz und Kostenersparnis
%     \item Vergleich der Investitionskosten und erwarteten Produktivitätsgewinne
%     \item Langfristige wirtschaftliche Auswirkungen für Unternehmen und die Softwarebranche
% \end{itemize}

% TODO: anderen titel für fazit?

% \section{Fazit}

% Die Analyse zeigt, dass der Einsatz generativer KI in der Softwareentwicklung
% weitreichende wirtschaftliche und gesellschaftliche Implikationen hat.
% Unternehmen profitieren von erheblichen Effizienzgewinnen und neuen
% Wertschöpfungspotenzialen, stehen jedoch zugleich vor der Herausforderung, ihre
% Organisationsstrukturen, Rollenmodelle und Qualifikationsprofile an die neue
% technologische Realität anzupassen.

% Die empirischen Befunde aus der Literatur machen deutlich, dass der nachhaltige
% Erfolg von KI-gestützter Entwicklung vor allem von strategischen Investitionen
% in Weiterbildung, flexibles Change Management und eine offene Innovationskultur
% abhängt. Zugleich sind Risiken wie Kostenfallen, ethische Konflikte und
% potenzielle Verwerfungen am Arbeitsmarkt frühzeitig zu adressieren, um die
% positiven Effekte langfristig zu sichern.

% Insgesamt eröffnet die Integration von GenAI-Tools neue Chancen für Unternehmen
% und Gesellschaft, verlangt aber ebenso verantwortungsvolle Gestaltung und eine
% kontinuierliche Anpassung an den rasanten technologischen Wandel.
