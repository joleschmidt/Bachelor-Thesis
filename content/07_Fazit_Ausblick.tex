\chapter{Fazit und Ausblick}
Die Auswertung der theoretischen und praktischen Analysen verdeutlicht, wie
tiefgreifend generative KI-Tools bereits heute die Softwareentwicklung
beeinflussen. Das abschließende Kapitel fasst die zentralen Erkenntnisse
zusammen, bewertet ihre Tragweite für Unternehmen und Gesellschaft und
formuliert Perspektiven für zukünftige Entwicklungen und offene
Forschungsfragen.

\section{Fazit}
Die Analysen und die praktische Demonstration machen deutlich, dass generative
KI-Tools wie GitHub Copilot, Cursor oder Bolt die Softwareentwicklung
grundlegend verändern. Zu den wichtigsten Befunden zählen signifikante
Effizienzsteigerungen, die Automatisierung vieler Routineaufgaben sowie die
Einführung neuer Werkzeuge und Methoden im Entwicklungsalltag. Im Praxisteil
zeigte sich, wie sich durch KI-basierte Assistenzsysteme sowohl die
Entwicklungsdauer als auch Einstiegshürden verringern lassen, etwa bei der
Initialisierung neuer Komponenten oder beim Refactoring von Code.

Gleichzeitig wurden auch klare Grenzen und Herausforderungen sichtbar. Die
Qualität der KI-generierten Vorschläge hängt maßgeblich von der Präzision der
Prompts, domänenspezifischem Kontext und der kritischen Überprüfung durch
Entwickler:innen ab. Besonders bei sicherheitskritischen Anwendungen können
fehleranfällige oder nicht nachvollziehbare Lösungen entstehen. Ethische Fragen
wie der Umgang mit Bias, Fairness sowie Nachvollziehbarkeit und Verantwortung
bleiben weiterhin zentrale Herausforderungen, die nicht allein durch technische
Lösungen adressiert werden können.

Wirtschaftlich eröffnen sich neue Wertschöpfungsmodelle, da Unternehmen mit
KI-gestützter Softwareentwicklung flexibler und innovativer agieren können.
Gleichzeitig steigt der Druck, Kompetenzen im Umgang mit KI systematisch
auszubauen und Arbeitsprozesse anzupassen. Gesellschaftlich stehen
Qualifikation, Arbeitsplatzsicherheit und Akzeptanz im Mittelpunkt.
Entwickler:innen werden künftig verstärkt als Schnittstelle zwischen Mensch und
Maschine agieren und benötigen Fähigkeiten im Prompt-Engineering, in der
Systembewertung und in der kritischen Reflexion automatisierter Prozesse.

\section{Beantwortung der Forschungsfragen}

Im Folgenden werden die zu Beginn der Arbeit formulierten Forschungsfragen auf
Basis der erzielten Ergebnisse systematisch beantwortet.

\subsection*{Wie verändert generative KI traditionelle Entwicklungspraktiken in der Softwareentwicklung?}
Die Analysen und praktischen Beispiele dieser Arbeit zeigen deutlich, dass
generative KI-Tools viele ehemals manuelle und repetitive Tätigkeiten, etwa die
Code-Generierung, das Testing und die Dokumentation, automatisieren. Dadurch
verschiebt sich der Schwerpunkt der Entwicklungsarbeit hin zu anspruchsvolleren
Aufgaben wie dem Prompt-Engineering, der Qualitätskontrolle und der Bewertung
von KI-generierten Vorschlägen. Entwicklungsprozesse werden iterativer,
dialogorientierter und durch eine intensivere Zusammenarbeit zwischen Mensch
und KI geprägt. Insgesamt führt dies zu einer höheren Flexibilität und
Geschwindigkeit in der Softwareentwicklung.

\subsection*{Welche spezifischen Herausforderungen entstehen durch KI hinsichtlich Sicherheit, Ethik und Code-Qualität?}
Im Rahmen der Untersuchung wurde deutlich, dass durch den Einsatz generativer
KI neue Risiken im Bereich der IT-Sicherheit entstehen, beispielsweise durch
potenziell fehlerhafte oder unsichere Codevorschläge. Zudem werfen Fragen wie
Bias, Diskriminierung sowie die Nachvollziehbarkeit und Verantwortung für
KI-generierte Ergebnisse weiterhin erhebliche ethische und technische
Herausforderungen auf. Die Arbeit macht deutlich, dass ein sicherer und
verantwortungsvoller Einsatz generativer KI nur möglich ist, wenn regelmäßige
Überprüfungen, Peer-Reviews und klare Leitlinien fester Bestandteil des
Entwicklungsprozesses sind.

\subsection*{Wie unterstützt generative KI Softwareentwickler:innen in agilen Entwicklungsprozessen?}
Die Ergebnisse zeigen, dass generative KI-Tools die Arbeit in agilen Teams
spürbar beschleunigen. Sie ermöglichen schnellere Iterationen, erleichtern das
Prototyping und verbessern die Integration von Testing und Deployment in den
Entwicklungszyklus. Besonders wertvoll ist die Unterstützung bei der Anpassung
von Anforderungen und der automatischen Generierung von Tests oder
Dokumentation. Dadurch profitieren agile Teams von effizienteren Workflows,
mehr Flexibilität und einer insgesamt höheren Produktivität.

\subsection*{Wie integrieren bestehende generative KI-Tools sich praktisch in die React-Native-Entwicklung und welchen Einfluss hat dies auf Entwicklungszeit und Codequalität?}
Die praktische Demonstration belegt, dass Tools wie Copilot, Cursor oder Bolt
die Entwicklungszeit für typische Aufgaben, etwa beim Erstellen von Komponenten
oder beim Einbinden von APIs, deutlich reduzieren können. Die Codequalität
profitiert besonders bei Routinetätigkeiten von automatisierten Vorschlägen,
während bei komplexeren oder sicherheitskritischen Aufgaben weiterhin eine
kritische Prüfung und Anpassung durch erfahrene Entwickler:innen notwendig
bleibt. Generative KI-Tools sind somit ein effektives Hilfsmittel, das die
Entwicklung beschleunigt, aber eine sorgfältige Kontrolle der Ergebnisse durch
den Menschen erfordert.

% \section{Beantwortung der Forschungsfragen}

% \begin{enumerate}
%       \item \textbf{Wie verändert generative KI traditionelle Entwicklungspraktiken in der Softwareentwicklung?} \\
%             Generative KI-Tools automatisieren zahlreiche bislang manuelle und repetitive Tätigkeiten wie Code-Generierung, Testing und Dokumentation. Dadurch verschieben sich die Schwerpunkte der Entwicklungsarbeit zunehmend hin zu Aufgaben wie Prompt-Engineering, Qualitätskontrolle und Systembewertung. Die Entwicklungspraktiken werden iterativer, dialogorientierter und sind stärker von der Zusammenarbeit zwischen Mensch und KI geprägt. Insgesamt trägt die generative KI dazu bei, dass Entwicklungsprozesse flexibler, schneller und kollaborativer gestaltet werden können.

%       \item \textbf{Welche spezifischen Herausforderungen entstehen durch KI hinsichtlich Sicherheit, Ethik und Code-Qualität?} \\
%             Die größten Herausforderungen bestehen in neuen Angriffsflächen – etwa durch fehlerhafte oder unsichere KI-generierte Vorschläge –, ethischen Risiken wie Bias und Diskriminierung sowie Fragen der Nachvollziehbarkeit, Verantwortung und Wartbarkeit von Code. Damit generative KI verantwortungsvoll eingesetzt werden kann, sind regelmäßige Überprüfungen, Peer-Reviews, die Etablierung von Leitlinien sowie die gezielte Sensibilisierung der Entwickler:innen unabdingbar.

%       \item \textbf{Wie unterstützt generative KI Softwareentwickler:innen in agilen Entwicklungsprozessen?} \\
%             Generative KI-Tools fördern schnellere Iterationen, erleichtern die flexible Anpassung von Anforderungen und bieten wertvolle Unterstützung bei Testing, Debugging und Dokumentation. Sie beschleunigen den gesamten Entwicklungszyklus, ermöglichen schnelleres Prototyping und unterstützen eine engere Verzahnung von Entwicklung, Test und Deployment – insbesondere in agilen Teams. Die Zusammenarbeit zwischen Mensch und KI führt dabei zu effizienteren Workflows und zu einer höheren Anpassungsfähigkeit im Projektalltag.

%       \item \textbf{Wie integrieren bestehende generative KI-Tools sich praktisch in die React-Native-Entwicklung und welchen Einfluss hat dies auf Entwicklungszeit und Codequalität?} \\
%             Die praktische Demonstration in dieser Arbeit belegt, dass Tools wie Copilot, Cursor oder Bolt die Entwicklungszeit für typische Aufgaben – wie das Erstellen von Komponenten, das Einbinden von APIs oder das Prototyping neuer Features – deutlich reduzieren können. Gleichzeitig bleibt die Qualität des Codes in hohem Maße von der kritischen Bewertung, Anpassung und Überprüfung der KI-Vorschläge durch erfahrene Entwickler:innen abhängig. Insgesamt zeigt sich, dass generative KI insbesondere Routineaufgaben und Standardmuster zuverlässig automatisiert, während bei komplexeren oder sicherheitsrelevanten Aufgaben die menschliche Expertise und Verantwortung weiterhin unerlässlich sind.
% \end{enumerate}

\section{Handlungsempfehlungen für Praxis und Unternehmen}

Die Ergebnisse dieser Arbeit zeigen, dass die erfolgreiche Einführung
generativer KI-Tools in der Softwareentwicklung einen strategischen und
ganzheitlichen Ansatz erfordert. Unternehmen sollten die Integration solcher
Technologien als kontinuierlichen Veränderungsprozess verstehen, der sowohl
technische als auch organisationale und kulturelle Dimensionen umfasst.

Neben Investitionen in moderne Infrastruktur ist insbesondere die Aus- und
Weiterbildung der Beschäftigten ein Schlüsselfaktor für nachhaltigen Erfolg.
Regelmäßige Schulungen zu Prompt-Engineering, ein grundlegendes Verständnis für
die Funktionsweise von KI-Modellen sowie Trainings zu Sicherheits- und
Ethikstandards sind essenziell, um die Kompetenzen im Umgang mit KI
kontinuierlich auszubauen.

Die Praxis hat gezeigt, dass der Einsatz generativer KI mit klaren
Qualitätskontrollen und Feedback-Loops begleitet werden sollte.
Peer-Review-Prozesse, Human-in-the-Loop-Mechanismen und eine offene
Fehlerkultur stärken die Zusammenarbeit zwischen Mensch und KI und fördern die
Akzeptanz im Team. Die Einführung transparenter Richtlinien sowie die
Definition klarer Verantwortlichkeiten für den Einsatz von KI-Lösungen sind
weitere zentrale Erfolgsfaktoren.

Darüber hinaus sollte der kontinuierliche Dialog zwischen Entwickler:innen,
Management und betroffenen Stakeholdern aktiv gefördert werden, um
Akzeptanzprobleme und Widerstände frühzeitig zu erkennen und anzugehen. Eine
innovationsfreundliche Unternehmenskultur, die Experimente mit neuen Werkzeugen
ermöglicht und Lernprozesse unterstützt, ist eine zentrale Voraussetzung für
die nachhaltige Integration generativer KI in die Entwicklungsarbeit.

Die konsequente Umsetzung dieser Empfehlungen kann dazu beitragen, die
Potenziale generativer KI voll auszuschöpfen und zugleich Risiken im
Entwicklungsalltag wirksam zu minimieren.

% \section{Einschränkungen der Arbeit}
% Die vorliegende Untersuchung basiert auf einer literaturgestützten Analyse und
% einer exemplarischen praktischen Demonstration. Sie erhebt keinen Anspruch auf
% empirische Generalisierbarkeit und kann daher keine abschließenden Aussagen zur
% Wirksamkeit aller KI-Tools in jeder Unternehmenskonstellation treffen. Die
% rechtlichen und gesellschaftspolitischen Dimensionen der KI-Nutzung konnten nur
% angeschnitten werden; ebenso wurden regionale Unterschiede und
% branchenspezifische Besonderheiten weitgehend ausgeklammert. Die
% Praxiserfahrung beschränkt sich zudem auf ausgewählte Tools und einen konkreten
% Anwendungskontext (Entwicklung einer Map-Komponente in React Native).

\section{Offene Fragen und Forschungsbedarf}

Trotz der vielfältigen Potenziale generativer KI in der Softwareentwicklung
bestehen zahlreiche offene Fragen, die Gegenstand zukünftiger Forschung sein
sollten. Besonders der langfristige Einfluss auf Arbeitsmärkte, Teamstrukturen
sowie auf die Entwicklung und Wartung komplexer Softwaresysteme ist bislang
noch nicht abschließend untersucht.

Wichtige Themen für weitere Forschung sind dabei insbesondere:
\begin{itemize}
      \item Wie verändert sich die Qualität, Wartbarkeit und Nachhaltigkeit von Software
            bei zunehmender KI-Unterstützung im Entwicklungsprozess?
      \item Welche ethischen, rechtlichen und sicherheitstechnischen Standards müssen
            entwickelt werden, um den flächendeckenden und verantwortungsvollen Einsatz
            generativer KI in der Softwareentwicklung zu gewährleisten?
      \item Wie kann die Zusammenarbeit zwischen Mensch und KI gestaltet werden, sodass
            Kreativität, Transparenz, Fairness und Verantwortlichkeit in Softwareprojekten
            erhalten bleiben?
      \item In welchem Maße wird KI in Zukunft nicht nur als Assistenzsystem, sondern als
            eigenständige Entscheidungsinstanz in Entwicklungsprozessen agieren, und welche
            Implikationen hat dies für Kontrolle, Haftung und Governance?
\end{itemize}

Wie Treude und Storey~\cite{treude_generative_2025} betonen, ist eine
Neuausrichtung der empirischen Softwaretechnik erforderlich, die die Chancen
und Grenzen generativer KI konsequent in den Mittelpunkt methodischer
Innovation rückt. Der Diskurs um die nachhaltige und verantwortungsvolle
Integration von KI in die Softwareentwicklung bleibt damit eine zentrale
Herausforderung für Wissenschaft und Praxis.

\section{Ausblick}

Die kommenden Jahre werden durch eine noch stärkere Durchdringung der
Softwareentwicklung mit generativer KI geprägt sein. Unternehmen, die
frühzeitig auf eine intelligente Verzahnung von Mensch und Maschine setzen,
haben die Chance, Innovationspotenziale optimal auszuschöpfen und ihre
Wettbewerbsfähigkeit nachhaltig zu stärken.

Gleichzeitig bleibt der gesellschaftliche und ethische Diskurs von zentraler
Bedeutung: Ein nachhaltiger und verantwortungsvoller Umgang mit KI erfordert
klare Leitplanken, kontinuierliche Qualifizierung der Beschäftigten und die
Bereitschaft, etablierte Rollen und Prozesse im Entwicklungsalltag immer wieder
zu hinterfragen und weiterzuentwickeln.

Die Softwareentwicklung steht damit exemplarisch für den umfassenden Wandel der
Arbeitswelt im Zeitalter der Künstlichen Intelligenz. Entscheidend wird sein,
ob es gelingt, technologische Innovation mit menschlicher Kreativität,
Verantwortungsbewusstsein und einer offenen Lernkultur zu verbinden.

Abschließend bleibt festzuhalten, dass die Zukunft der Softwareentwicklung kein
rein technisches, sondern ein zutiefst soziales und gestalterisches Projekt
ist. Ihr Erfolg hängt maßgeblich vom Zusammenspiel aller beteiligten Akteure
ab.

% \section{Erwartete Erkenntnisse}
% Es wird erwartet, dass KI-gestützte Softwareentwicklung nicht nur Effizienzsteigerungen ermöglicht, sondern auch die Arbeitsweise von Entwicklern nachhaltig verändert. Besonders relevant ist die Frage, inwieweit generative KI langfristig klassische Programmieraufgaben übernimmt oder ergänzt. Darüber hinaus soll analysiert werden, welche Herausforderungen in Bezug auf Sicherheit, Ethik und wirtschaftliche Auswirkungen entstehen.

% \section{Zusammenfassung der Erkenntnisse}
% 


% \section{Handlungsempfehlungen und Zukunftsperpektiven}
% \input{content/Fazit & Ausblick/07_03_Handlungsempfehlungen}