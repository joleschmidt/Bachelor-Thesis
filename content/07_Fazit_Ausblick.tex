\chapter{Fazit und Ausblick}

\section{Zusammenfassung der Kernergebnisse}
Die vorliegende Arbeit zeigt, dass generative KI-Tools wie GitHub Copilot,
Cursor oder Devin die Softwareentwicklung nachhaltig verändern. Zentrale
Erkenntnisse sind signifikante Effizienzsteigerungen, die Automatisierung
vieler Routineaufgaben und die Einführung neuer Werkzeuge und Methoden im
Entwicklungsalltag. Die praktische Demonstration verdeutlichte, wie sich durch
KI-basierte Assistenzsysteme sowohl Entwicklungsdauer als auch Einstiegshürden
verringern lassen – etwa bei der Initialisierung neuer Komponenten oder beim
Refactoring von Code.

Gleichzeitig offenbarten sich aber auch klare Grenzen und Herausforderungen:
Die Qualität von KI-generierten Vorschlägen ist abhängig von präzisen Prompts,
domänenspezifischem Kontext und der kritischen Überprüfung durch
Entwickler:innen. Unsicherheiten entstehen insbesondere bei
sicherheitskritischen Anwendungen, bei denen KI-Modelle fehleranfällige oder
nicht nachvollziehbare Lösungen produzieren können. Zudem wurde deutlich, dass
ethische Fragen wie der Umgang mit Bias, Fairness sowie Nachvollziehbarkeit und
Verantwortung nicht allein durch die Technik gelöst werden können.

Wirtschaftlich eröffnen sich neue Wertschöpfungsmodelle, da Unternehmen mit
KI-gestützter Softwareentwicklung flexibler und innovativer agieren können.
Gleichzeitig steigt der Druck, Kompetenzen im Umgang mit KI systematisch
auszubauen und Arbeitsprozesse anzupassen. Gesellschaftlich stehen
Qualifikation, Arbeitsplatzsicherheit und Akzeptanz im Mittelpunkt:
Entwickler:innen werden künftig verstärkt als Schnittstelle zwischen Mensch und
Maschine agieren und benötigen Fähigkeiten im Prompt-Engineering, in der
Systembewertung und in der kritischen Reflexion automatisierter Prozesse.

\section{Beantwortung der Forschungsfragen}
\textbf{1. Wie verändert generative KI traditionelle Entwicklungspraktiken in der Softwareentwicklung?}\\
Generative KI-Tools automatisieren viele bislang manuelle und repetitive Tätigkeiten wie Code-Generierung, Testing und Dokumentation. Dadurch verschieben sich die Schwerpunkte der Entwicklungsarbeit hin zu Prompt-Engineering, Qualitätskontrolle und Systembewertung. Entwicklungspraktiken werden iterativer, dialogorientierter und stärker von Mensch-KI-Kollaboration geprägt.

\textbf{2. Welche spezifischen Herausforderungen entstehen durch KI hinsichtlich Sicherheit, Ethik und Code‑Qualität?}\\
Zu den größten Herausforderungen zählen neue Angriffsflächen (z.\,B. durch fehlerhafte KI-generierte Vorschläge), ethische Risiken wie Bias und Diskriminierung sowie Fragen der Nachvollziehbarkeit, Verantwortung und Wartbarkeit von Code. Regelmäßige Überprüfung, Peer-Reviews und klare Leitlinien sind für den sicheren Einsatz unerlässlich.

\textbf{3. Wie unterstützt generative KI Softwareentwickler in agilen Entwicklungsprozessen?}\\
Generative KI-Tools fördern schnellere Iterationen, erleichtern die Anpassung von Anforderungen und bieten Hilfestellungen beim Testen und Debugging. Sie beschleunigen den Entwicklungszyklus, unterstützen das Prototyping und ermöglichen eine engere Verzahnung von Entwicklung, Test und Deployment -- insbesondere in agilen Teams.

\textbf{4. Wie integrieren bestehende generative KI‑Tools sich praktisch in die React‑Native‑Entwicklung und welchen Einfluss hat dies auf Entwicklungszeit und Codequalität?}\\
Die praktische Demonstration zeigte, dass Tools wie Copilot, Cursor oder Bolt die Entwicklungszeit für typische Aufgaben (z.\,B. Erstellung von Komponenten, Einbindung von APIs) deutlich reduzieren können. Gleichzeitig hängt die Codequalität stark von der kritischen Bewertung, Anpassung und Überprüfung der KI-Vorschläge durch die Entwickler:innen ab.

\section{Handlungsempfehlungen für Praxis und Unternehmen}
Unternehmen sollten die Einführung generativer KI-Tools strategisch planen und
als kontinuierlichen Veränderungsprozess verstehen. Neben Investitionen in
technische Infrastruktur ist insbesondere die Aus- und Weiterbildung der
Beschäftigten ein Schlüsselfaktor für den nachhaltigen Erfolg. Ein gezieltes
Training in Prompt-Engineering, das Verständnis von KI-Modellen sowie
regelmäßige Schulungen zu Sicherheits- und Ethikstandards sind essenziell.

Es empfiehlt sich, den Einsatz von KI stets mit klaren Qualitätskontrollen und
Feedback-Loops zu begleiten: Mensch-KI-Kollaboration sollte durch
Peer-Review-Prozesse, Human-in-the-Loop-Mechanismen und eine offene
Fehlerkultur gestärkt werden. Unternehmen profitieren zudem von der Einführung
transparenter Richtlinien und der Definition von Verantwortlichkeiten für den
Umgang mit KI-basierten Lösungen.

Zudem sollte der Dialog zwischen Entwickler:innen, Management und betroffenen
Stakeholdern aktiv gefördert werden, um Akzeptanzprobleme und Widerstände
frühzeitig zu adressieren. Die Schaffung einer innovationsfreundlichen
Unternehmenskultur, die Experimente mit neuen Werkzeugen erlaubt und
Lernprozesse unterstützt, ist eine zentrale Voraussetzung für die erfolgreiche
Integration von KI in die Entwicklungsarbeit.

\section{Einschränkungen der Arbeit}
Die vorliegende Untersuchung basiert auf einer literaturgestützten Analyse und
einer exemplarischen praktischen Demonstration. Sie erhebt keinen Anspruch auf
empirische Generalisierbarkeit und kann daher keine abschließenden Aussagen zur
Wirksamkeit aller KI-Tools in jeder Unternehmenskonstellation treffen. Die
rechtlichen und gesellschaftspolitischen Dimensionen der KI-Nutzung konnten nur
angeschnitten werden; ebenso wurden regionale Unterschiede und
branchenspezifische Besonderheiten weitgehend ausgeklammert. Die
Praxiserfahrung beschränkt sich zudem auf ausgewählte Tools und einen konkreten
Anwendungskontext (Entwicklung einer Map-Komponente in React Native).

\section{Offene Fragen und Forschungsbedarf}
Viele offene Fragen bleiben für die Zukunft bestehen. Insbesondere der
langfristige Einfluss generativer KI auf Arbeitsmarkt, Teamstrukturen und die
Entwicklung komplexer Softwaresysteme ist noch nicht abschließend erforscht.
Wichtige Forschungsthemen sind:
\begin{itemize}
    \item Wie verändert sich die Qualität und Wartbarkeit von Code bei zunehmender
          KI-Unterstützung?
    \item Welche ethischen, rechtlichen und sicherheitstechnischen Standards sind für den
          flächendeckenden Einsatz von KI in der Softwareentwicklung notwendig?
    \item Wie lässt sich die Zusammenarbeit zwischen Mensch und KI so gestalten, dass
          Kreativität, Transparenz und Verantwortlichkeit erhalten bleiben?
    \item In welchem Maße wird KI künftig als Entscheidungsinstanz und nicht nur als
          Assistenzsystem agieren?
\end{itemize}

\section{Ausblick}
Die nächsten Jahre werden von einer weiteren Durchdringung der
Softwareentwicklung durch generative KI geprägt sein. Unternehmen, die
frühzeitig auf eine intelligente Verzahnung von Mensch und KI setzen, können
Innovationspotenziale optimal ausschöpfen und ihre Wettbewerbsfähigkeit
ausbauen. Gleichzeitig bleibt der gesellschaftliche und ethische Diskurs von
zentraler Bedeutung: Der nachhaltige und verantwortungsvolle Umgang mit KI
erfordert klare Leitplanken, kontinuierliche Qualifizierung und die
Bereitschaft, gewachsene Rollen und Prozesse im Entwicklungsalltag immer wieder
neu zu hinterfragen.\\ Die Softwareentwicklung steht damit exemplarisch für den
umfassenden Wandel der Arbeitswelt im KI-Zeitalter.

% \section{Erwartete Erkenntnisse}
% Es wird erwartet, dass KI-gestützte Softwareentwicklung nicht nur Effizienzsteigerungen ermöglicht, sondern auch die Arbeitsweise von Entwicklern nachhaltig verändert. Besonders relevant ist die Frage, inwieweit generative KI langfristig klassische Programmieraufgaben übernimmt oder ergänzt. Darüber hinaus soll analysiert werden, welche Herausforderungen in Bezug auf Sicherheit, Ethik und wirtschaftliche Auswirkungen entstehen.

% \section{Zusammenfassung der Erkenntnisse}
% \input{content/Fazit & Ausblick/07_02_Zusammefassung}

% \section{Handlungsempfehlungen und Zukunftsperpektiven}
% \input{content/Fazit & Ausblick/07_03_Handlungsempfehlungen}