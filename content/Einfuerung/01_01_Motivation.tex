% \subsection{Relevanz des Themas}
% % Diese Arbeit adressiert diese Problematik, indem sie die Chancen und Herausforderungen von KI in der Softwareentwicklung analysiert. Ziel ist es, sowohl wissenschaftliche als auch praktische Erkenntnisse zu gewinnen, die Unternehmen und Entwicklern helfen, informierte Entscheidungen über den Einsatz von KI-gestützten Technologien zu treffen.

% % Künstliche Intelligenz (KI) hat sich in den letzten Jahren als transformative Technologie in der Softwareentwicklung etabliert. Insbesondere generative KI-Modelle wie Large Language Models (LLMs) haben das Potenzial, Entwicklungsprozesse signifikant zu verändern. Die Automatisierung von Codegenerierung, Fehleranalyse und Softwarewartung führt zu einer gesteigerten Effizienz und ermöglicht es Entwicklern, sich auf konzeptionell anspruchsvollere Aufgaben zu konzentrieren.

% % Die zunehmende Integration von KI in den Softwareentwicklungsprozess eröffnet neue Möglichkeiten, birgt jedoch auch Herausforderungen. Während einige Studien eine gesteigerte Produktivität und Codequalität durch KI-gestützte Tools belegen, gibt es gleichzeitig Bedenken hinsichtlich Sicherheitsrisiken, algorithmischer Verzerrung und langfristigen Veränderungen in der Arbeitsweise von Entwicklern. Diese Gegensätze verdeutlichen die Notwendigkeit einer fundierten wissenschaftlichen Auseinandersetzung mit den Auswirkungen von KI auf die Softwareentwicklung.

% \subsection{Motivation}
% % Die zunehmende Verbreitung künstlicher Intelligenz (KI) in der Softwareentwicklung stellt sowohl Wissenschaft als auch Praxis vor bedeutende Herausforderungen und Chancen. Unternehmen integrieren KI-gestützte Tools in ihre Entwicklungsprozesse, um Produktivität und Codequalität zu steigern, doch der langfristige Einfluss dieser Technologie auf die Arbeitsweise von Softwareentwicklern ist noch nicht vollständig erforscht.

% % Besonders relevant ist die Frage, wie sich KI-gestützte Entwicklungsumgebungen auf traditionelle Softwareentwicklungspraktiken auswirken. Während einige Forschungen darauf hindeuten, dass KI-Tools repetitive Aufgaben reduzieren und Entwicklern mehr Raum für kreative Problemlösungen geben, bestehen weiterhin Unsicherheiten hinsichtlich der Verlässlichkeit der generierten Codevorschläge und möglicher ethischer Bedenken.

Die Relevanz des Themas ergibt sich aus den gegenwärtigen Entwicklungen in Forschung und Praxis: Immer mehr Unternehmen erforschen aktiv den Einsatz von KI-Technologien, um sich Effizienzvorteile und Innovationsschübe zu sichern. Gleichzeitig zeigt sich in vielen Studien ein Spannungsverhältnis zwischen den Versprechen generativer KI – zum Beispiel automatisierte Code-Generierung und intelligente Projektsteuerung – und den Risiken, etwa unzureichender Transparenz, Sicherheitslücken oder ethischen Verzerrungen.

Hinzu kommt, dass Softwareentwicklung durch agile Methoden wie Scrum oder Kanban bereits stark dynamisiert ist: Teams agieren flexibel, stehen aber auch unter stetigem Veränderungsdruck. Wenn dann zusätzlich KI als Tool oder „Co-Entwickler“ eingebunden wird, steigen die Anforderungen an Prozessgestaltung, Rollenverteilung und Qualitätsmanagement weiter. Genau hier setzt diese Arbeit an: Sie möchte klären, wie Entwickler und Entscheider KI sinnvoll in den Softwarelebenszyklus integrieren können, wo praxisnahe Chancen liegen und welche neuen Stolpersteine zu beachten sind.

Daraus ergibt sich die Notwendigkeit, Chancen und Herausforderungen systematisch zu analysieren und klare Handlungsempfehlungen aufzustellen, damit die Integration von KI in Entwicklungsprozessen nicht nur technisch, sondern auch ethisch und organisatorisch gut gelingt.