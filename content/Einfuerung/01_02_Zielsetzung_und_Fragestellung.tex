Das Ziel dieser Arbeit ist es, die Auswirkungen von KI auf die Softwareentwicklung zu analysieren und praxisnahe Handlungsempfehlungen für Unternehmen und Entwickler abzuleiten. Dabei werden insbesondere folgende Forschungsfragen untersucht:

\begin{itemize}
    \item[FF-1] Wie verändert generative KI traditionelle Entwicklungspraktiken in der Softwareentwicklung?
    \item[FF-2] Welche spezifischen Herausforderungen entstehen durch KI-gestützte Softwareentwicklung hinsichtlich Sicherheit, Ethik und Code-Qualität?
    \item[FF-3] Wie kann Generative KI Softwareentwickler in einem agilen Entwicklungsprozess unterstützen? 
    \item[FF-4] Wie lassen sich bestehende generative KI-Tools (Cursor, GitHub Copilot, v0 etc.) in den Entwicklungsprozess einer React-Native-App integrieren, und welchen Einfluss hat das auf Entwicklungszeit und Code-Qualität?
    % \item[FF-4] Welche langfristigen Implikationen hat die Nutzung von KI auf die Rolle und Qualifikation von Softwareentwicklern?
    % \item[FF-5] Welche wirtschaftlichen und organisatorischen Auswirkungen hat die Integration von KI auf Softwareunternehmen und den Arbeitsmarkt?
\end{itemize}

Darüber hinaus wird ein praktisches Beispiel in Form einer React Native-App (Locals) angeführt, um zu untersuchen, wie bestehende generative KI-Tools (etwa Cursor, GitHub Copilot etc.) in einen realen Entwicklungsprozess integriert werden können.