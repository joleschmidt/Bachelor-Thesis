
Ziel dieser Arbeit ist es, die Auswirkungen generativer KI auf die
Softwareentwicklung umfassend zu analysieren und daraus praxisnahe
Handlungsempfehlungen für Unternehmen und Entwickler:innen abzuleiten. Im Fokus
steht insbesondere, wie KI-gestützte Tools bestehende Entwicklungspraktiken
verändern, welche Herausforderungen damit verbunden sind und wie sich Chancen
und Risiken in der Praxis ausbalancieren lassen.

Daraus ergeben sich folgende zentrale Forschungsfragen:
\begin{itemize}
      \item Wie verändert generative KI traditionelle Entwicklungspraktiken in der
            Softwareentwicklung?
      \item Welche spezifischen Herausforderungen entstehen durch KI-gestützte
            Softwareentwicklung hinsichtlich Sicherheit, Ethik und Code-Qualität?
      \item Wie kann generative KI Softwareentwickler:innen in einem agilen
            Entwicklungsprozess unterstützen?
      \item Wie lassen sich bestehende generative KI-Tools, wie Cursor, GitHub Copilot,
            bolt.new (Bolt), in den Entwicklungsprozess einer React-Native-App integrieren
            und welchen Einfluss hat dies auf Entwicklungszeit und Code-Qualität?
\end{itemize}

Um diese Fragen nicht nur theoretisch, sondern auch praxisnah zu beleuchten,
wird im praktischen Teil der Arbeit eine React-Native-App als Fallstudie
herangezogen. Ziel ist es, exemplarisch zu untersuchen, wie generative KI-Tools
in reale Entwicklungsprozesse integriert werden können und welchen konkreten
Mehrwert sie im Entwicklungsalltag bieten.
