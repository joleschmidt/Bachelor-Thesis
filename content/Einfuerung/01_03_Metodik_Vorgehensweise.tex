Diese Arbeit verfolgt eine theoretische und literaturbasierte Herangehensweise. Es wird eine systematische Analyse bestehender wissenschaftlicher Literatur durchgeführt, um ein umfassendes Verständnis der aktuellen Forschungslage zu generativer KI in der Softwareentwicklung zu erhalten. Die Methodik umfasst folgende Schritte:
\begin{enumerate}
    \item \textbf{Literaturrecherche:} Auswahl relevanter wissenschaftlicher Publikationen aus anerkannten Datenbanken wie IEEE Xplore, arXiv und SpringerLink. Dabei wird ein Fokus auf aktuelle Studien gelegt, die die Auswirkungen von KI auf die Softwareentwicklung untersuchen.
    \item \textbf{Kategorisierung der Forschungsthemen:} Identifikation und Gruppierung zentraler Themenfelder wie Produktivitätssteigerung, Automatisierung, Sicherheitsrisiken und ethische Fragestellungen.
    \item \textbf{Vergleichende Analyse:} Gegenüberstellung der identifizierten Chancen und Herausforderungen durch KI in der Softwareentwicklung.
    \item \textbf{Synthese und Ableitung von Schlussfolgerungen:} Basierend auf der Literaturauswertung werden praxisorientierte Empfehlungen für den Einsatz von KI in der Softwareentwicklung erarbeitet.
\end{enumerate}