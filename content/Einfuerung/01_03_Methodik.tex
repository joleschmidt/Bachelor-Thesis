Diese Arbeit verfolgt eine theoretische und literaturbasierte Herangehensweise, um ein umfassendes Verständnis der aktuellen Forschungslage zu generativer KI in der Softwareentwicklung zu erhalten. Die Methodik umfasst folgende Schritte:
\begin{enumerate}
    \item \textbf{Literaturrecherche:} Analyse wissenschaftlicher Publikationen aus IEEE Xplore, arXiv, SpringerLink und weiteren relevanten Fachquellen mit Fokus auf aktuelle Studien zur KI-gestützten Softwareentwicklung.
    \item \textbf{Kategorisierung der Forschungsthemen:} Identifikation und Gruppierung zentraler Themenfelder wie Automatisierung, Produktivität, Sicherheitsrisiken und ethische Fragestellungen.
    \item \textbf{Vergleichende Analyse:} Gegenüberstellung der identifizierten Chancen und Herausforderungen auf Basis aktueller Studien und Fachbeiträge.
    \item \textbf{Synthese und Ableitung von Schlussfolgerungen:} Entwicklung praxisorientierter Handlungsempfehlungen für den Einsatz von KI in der Softwareentwicklung.
    \item \textbf{Praktische Demonstration:} Im Rahmen der Arbeit wird exemplarisch ein Map-Screen (interaktive Kartenansicht) in der React Native-App „Locals“ entwickelt. Dabei werden ausgewählte generative KI-Tools (z.B. Cursor, v0 oder GitHub Copilot) eingesetzt, um Code-Generierung, Tests und Qualitätsverbesserungen zu demonstrieren. Die Erfahrungen aus diesem praktischen Teil werden dokumentiert und anschließend mit den theoretischen Erkenntnissen abgeglichen, um aufzuzeigen, inwieweit KI die Effizienz und Qualität im Entwicklungsprozess tatsächlich steigern kann.
\end{enumerate}

Diese Methodik erlaubt es, die bestehende Forschung systematisch zu strukturieren und relevante Erkenntnisse für die Praxis abzuleiten.