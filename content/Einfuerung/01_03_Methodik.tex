
Die Arbeit folgt einem literaturbasierten, qualitativen Ansatz, um ein
umfassendes Bild der Potenziale und Herausforderungen generativer KI in der
Softwareentwicklung zu zeichnen. Ziel ist es, die aktuelle Forschungslage
systematisch zu erfassen, zu bewerten und praxisrelevante Schlüsse zu ziehen.
Die Methodik gliedert sich in folgende Schritte:

\begin{enumerate}
    % 1. punkt überprüfen: springer link & weiter quellen?
    \item \textbf{Literaturrecherche:} Umfassende Analyse wissenschaftlicher Publikationen aus IEEE Xplore, arXiv, SpringerLink sowie den in der Literaturdatenbank dieser Arbeit aufgeführten Fachquellen mit Fokus auf aktuelle Entwicklungen in der KI-gestützten Softwareentwicklung.
    \item \textbf{Kategorisierung der Forschungsthemen:} Strukturierte Erfassung und Gruppierung zentraler Themenfelder wie Automatisierung, Produktivität, Sicherheit und ethische Fragestellungen.
    \item \textbf{Vergleichende Analyse:} Systematischer Vergleich und Gegenüberstellung der Chancen und Herausforderungen auf Basis ausgewählter Studien und Fachbeiträge.
    \item \textbf{Synthese und Ableitung von Schlussfolgerungen:} Entwicklung praxisorientierter Empfehlungen für Unternehmen und Entwickler:innen zur Integration von KI in der Softwareentwicklung.
    \item \textbf{Praktische Demonstration:} Ergänzend zur Literaturauswertung wird ein Map-Screen (interaktive Kartenansicht) in der React Native-App „Locals“ exemplarisch implementiert. Dabei kommen verschiedene generative KI-Tools (u.a. Cursor, v0, GitHub Copilot) zum Einsatz, um die Unterstützung bei Code-Generierung, Testing und Qualitätsverbesserung zu evaluieren. Die gewonnenen Erkenntnisse aus diesem praktischen Teil werden systematisch mit den theoretischen Ergebnissen abgeglichen.
\end{enumerate}

Dieses methodische Vorgehen ermöglicht es, den aktuellen Forschungsstand
kritisch einzuordnen, praxisnahe Einblicke zu gewinnen und fundierte
Empfehlungen für die erfolgreiche Nutzung generativer KI in der
Softwareentwicklung abzuleiten.
