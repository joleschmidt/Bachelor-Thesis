Mit dem verstärkten Einsatz generativer KI-Technologien stehen
Softwareunternehmen vor einem tiefgreifenden Wandel. Studien und Praxisberichte
zeigen, dass KI-basierte Automatisierungslösungen zunehmend eingesetzt werden,
um Entwicklungsprozesse zu beschleunigen und Ressourcen effizienter zu
nutzen~\cite{siebert_generative_2024,braun_ki_2024}. Dies erfordert eine
grundlegende Neubewertung klassischer Rollenmodelle, Teamstrukturen und
Wertschöpfungsketten.

Die Automatisierung wiederkehrender Aufgaben wie Standardcode, Testing oder
Dokumentation ermöglicht durch den gezielten Einsatz von KI-Tools wie GitHub
Copilot oder ChatGPT Effizienzsteigerungen von bis zu 50\,\% bei
Routinetätigkeiten~\cite{braun_ki_2024,siebert_generative_2024}. Gleichzeitig
verschiebt sich der Fokus der Entwickler:innen auf kreative, strategische und
überwachende Tätigkeiten, etwa das Prompt-Engineering, die Qualitätssicherung
und die kritische Bewertung KI-generierter Ergebnisse.

Deloitte~\cite{s_future_2024} betont, dass generative KI-Tools es Unternehmen
ermöglichen, Entwicklungsleistungen flexibler zu organisieren und globale Teams
besser zu vernetzen. Dadurch werden traditionelle Arbeitsteilung und
Wertschöpfung neu definiert: Die Rolle der Entwickler:innen verändert sich, und
Kompetenzen im Umgang mit KI, Data Science und Human-in-the-Loop-Konzepten
werden essenziell.

Um diesen Wandel erfolgreich zu gestalten, müssen Unternehmen in die
Weiterbildung ihrer Mitarbeitenden und die Anpassung organisatorischer
Strukturen investieren. Nur durch eine offene Innovationskultur und
kontinuierliches Change Management lassen sich Risiken wie Widerstände im Team,
Kompetenzlücken oder ethische Konflikte nachhaltig adressieren.

Nicht zuletzt rücken auch ethische und regulatorische Fragen in den
Mittelpunkt. Die Literatur betont, dass klare Leitlinien und eine
verantwortungsbewusste Nutzung von KI-Systemen notwendig sind, um Vertrauen zu
schaffen, Risiken zu minimieren und die Akzeptanz der neuen Technologien im
Unternehmen zu fördern~\cite{siebert_generative_2024}.

Insgesamt zeigt sich, dass der Erfolg der KI-Integration maßgeblich davon
abhängt, inwieweit Unternehmen nicht nur in Technologie, sondern auch in
Organisationsentwicklung und Kompetenzausbau investieren. Die Transformation
von Rollen, Prozessen und Wertschöpfungsketten ist eine der zentralen
Herausforderungen und Chancen im Zeitalter der generativen KI.

% \begin{itemize}
%     \item Auswirkungen auf Geschäftsmodelle und Prozesse
%     \item Veränderungen in der Softwareentwicklung und im Projektmanagement
%     \item Rolle von KI bei der Automatisierung von Softwareentwicklungsaufgaben
% \end{itemize}