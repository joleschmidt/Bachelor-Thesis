Mit der Verbreitung generativer KI-Technologien stehen Softwareunternehmen vor
einem tiefgreifenden Wandel. Marguerit~\cite{marguerit_augmenting_2025}
beschreibt, dass Unternehmen zunehmend KI-basierte Automatisierungslösungen
einsetzen, um Entwicklungsprozesse zu beschleunigen und Ressourcen effizienter
zu nutzen. Diese Entwicklung führt dazu, dass klassische Rollenmodelle und
Teamstrukturen neu bewertet werden müssen.

Farach et al.~\cite{farach_evolving_2025} argumentieren, dass digitale Arbeit
durch KI-Tools einen eigenständigen Produktionsfaktor darstellt, der
traditionelle Vorstellungen von Arbeitsteilung und Wertschöpfung grundlegend
verändert. In vielen Unternehmen werden Aufgaben wie das Schreiben von
Standardcode, Testing oder das Generieren von Dokumentation zunehmend
automatisiert, während der Fokus auf kreative, überwachende und strategische
Tätigkeiten wächst.

Storey et al.~\cite{storey_generative_2025} betonen, dass die Einführung
generativer KI-Tools nicht nur ökonomische, sondern auch organisatorische
Anpassungen erfordert. Unternehmen müssen neue Kompetenzen fördern,
Veränderungsbereitschaft unterstützen und klare ethische sowie regulatorische
Leitplanken setzen, um Risiken zu minimieren und die Akzeptanz im Team zu
erhöhen.

Die Literatur macht deutlich, dass der Erfolg von KI-Integration maßgeblich
davon abhängt, inwieweit Unternehmen nicht nur in Technologie, sondern auch in
Weiterbildung, Organisationsentwicklung und eine offene Innovationskultur
investieren.

% \begin{itemize}
%     \item Auswirkungen auf Geschäftsmodelle und Prozesse
%     \item Veränderungen in der Softwareentwicklung und im Projektmanagement
%     \item Rolle von KI bei der Automatisierung von Softwareentwicklungsaufgaben
% \end{itemize}