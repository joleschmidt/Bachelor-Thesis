
Die zunehmende Verbreitung generativer KI wirkt sich bereits heute spürbar auf
den Arbeitsmarkt für Softwareentwickler:innen aus. Studien zeigen, dass nicht
nur klassische Routinetätigkeiten zunehmend automatisiert werden, sondern sich
auch die Qualifikationsprofile und Tätigkeitsfelder nachhaltig
verändern~\cite{siebert_generative_2024,braun_ki_2024,s_future_2024}. Während
repetitive Aufgaben wie das Schreiben von Boilerplate-Code, Testing oder
Dokumentation verstärkt durch KI-Tools übernommen werden, wächst der Bedarf an
Kompetenzen in Bereichen wie Prompt-Engineering, KI-Management und der
kritischen Bewertung KI-generierter Ergebnisse.

Analysen aktueller Jobprofile und -anzeigen belegen, dass die Nachfrage nach
Fähigkeiten im Umgang mit generativen KI-Anwendungen deutlich steigt. Neben
klassischen Programmierkenntnissen werden zunehmend auch Kompetenzen im Bereich
der Steuerung, Überwachung und dem Feintuning von KI-basierten Systemen
nachgefragt~\cite{ahmadi_generative_2024}. Insbesondere der sichere und
verantwortungsvolle Einsatz von Tools wie GitHub Copilot, ChatGPT oder
branchenspezifischen KI-Assistenzsystemen entwickelt sich zu einer
Schlüsselqualifikation moderner Entwickler:innen.

Mit der Einführung generativer KI-Tools verändert sich zudem die Rolle von
Entwickler:innen im Team und im Unternehmen. Der Fokus verschiebt sich von der
manuellen Implementierung hin zur strategischen Nutzung und Integration von
KI-Lösungen. Dies umfasst sowohl die Gestaltung von Entwicklungsprozessen als
auch die Bewertung und kontinuierliche Verbesserung von KI-basierten
Ergebnissen~\cite{storey_generative_2025}. Gleichzeitig entstehen neue
Rollenprofile, die Fähigkeiten in den Bereichen Data Science, Human-in-the-Loop
und ethische Bewertung von KI-Systemen erfordern.

Die Veränderungen auf dem Arbeitsmarkt sind jedoch ambivalent. Während
einerseits neue Aufgabenfelder und Qualifikationen entstehen, besteht zugleich
die Gefahr von Verunsicherung und möglichen Substitutionseffekten bei stark
standardisierten Tätigkeiten~\cite{farach_evolving_2025}. Die Literatur weist
darauf hin, dass Beschäftigungsstrukturen und Lohngefüge sich im Zuge der
Automatisierung durch KI nachhaltig wandeln
können~\cite{marguerit_augmenting_2025}.

Nicht zuletzt gewinnt die Fähigkeit zur Kollaboration mit KI-Systemen sowie die
Bereitschaft zu kontinuierlicher Weiterbildung an Bedeutung. Die Arbeitswelt
von Entwickler:innen wird damit vielseitiger, flexibler und erfordert neben
technischem Know-how zunehmend auch soziale und ethische
Kompetenzen~\cite{storey_generative_2025,siebert_generative_2024}.

% \begin{itemize} 
%     \item Verschiebung der gefragten Kompetenzen und Qualifikationen
%     \item Neue Berufsbilder und veränderte Karrierewege
%     \item Auswirkungen auf die Arbeitsplatzsicherheit und die Notwendigkeit der Weiterbildung
% \end{itemize}