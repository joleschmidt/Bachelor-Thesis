Die Verbreitung generativer KI wirkt sich bereits heute spürbar auf den
Arbeitsmarkt für Softwareentwickler:innen aus.
Marguerit~\cite{marguerit_augmenting_2025} argumentiert, dass KI nicht nur
Automatisierung, sondern auch eine qualitative Veränderung von Arbeit mit sich
bringt: Während repetitive Tätigkeiten und Routinetasks zunehmend automatisiert
werden, entstehen neue Aufgabenfelder rund um die Steuerung, Überwachung und
das Feintuning von KI-basierten Systemen.

Ahmadi et al.~\cite{ahmadi_generative_2024} zeigen anhand von Analysen
aktueller Jobanzeigen, dass Kompetenzen im Umgang mit Tools wie ChatGPT,
Copilot und anderen generativen KI-Anwendungen zunehmend nachgefragt werden.
Dies deutet darauf hin, dass die Nachfrage nach klassischen
Programmierkenntnissen zwar bestehen bleibt, aber zunehmend durch Fähigkeiten
im Prompt-Engineering, KI-Management und in der Bewertung KI-generierter
Ergebnisse ergänzt wird.

Farach et al.~\cite{farach_evolving_2025} sehen in der digitalen Arbeit mit
KI-Tools einen neuen Produktionsfaktor, der es Unternehmen ermöglicht,
Entwicklungsleistungen flexibler und globaler zu organisieren. Zugleich betonen
sie, dass die Einführung generativer KI auch zu Unsicherheiten auf dem
Arbeitsmarkt führen kann, etwa durch die Verlagerung von Aufgaben,
Veränderungen im Qualifikationsprofil oder mögliche Substitutionseffekte bei
stark standardisierten Tätigkeiten.

Storey et al.~\cite{storey_generative_2025} machen darauf aufmerksam, dass der
Wandel nicht nur technische, sondern auch soziale Kompetenzen erfordert. Die
Fähigkeit, mit KI-Systemen kollaborativ zu arbeiten, ethische Risiken zu
erkennen und verantwortungsvoll mit automatisierten Vorschlägen umzugehen, wird
zu einer Schlüsselkompetenz in modernen Entwicklerteams.

% \begin{itemize} 
%     \item Verschiebung der gefragten Kompetenzen und Qualifikationen
%     \item Neue Berufsbilder und veränderte Karrierewege
%     \item Auswirkungen auf die Arbeitsplatzsicherheit und die Notwendigkeit der Weiterbildung
% \end{itemize}