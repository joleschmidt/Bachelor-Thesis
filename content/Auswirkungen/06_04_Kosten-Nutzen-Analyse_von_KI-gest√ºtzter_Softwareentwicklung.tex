
Die Integration generativer KI in die Softwareentwicklung bringt sowohl
erhebliche Vorteile als auch neue Kosten- und Risikofaktoren mit sich. Aktuelle
Studien belegen, dass durch den Einsatz von KI-Tools die Produktivität in
vielen Bereichen signifikant steigt, insbesondere bei Routinetätigkeiten, der
Code-Generierung und der Testautomatisierung~\cite{marguerit_augmenting_2025,
    farach_evolving_2025, habibi_open_2025}. Dies führt zu messbaren
Effizienzgewinnen und kann langfristig die Entwicklungskosten pro Feature oder
Release deutlich senken.

Gleichzeitig entstehen neue Investitionen. Die Einführung und Wartung von
KI-Systemen erfordert gezielte Weiterbildung, Anpassungen der Infrastruktur und
oft eine Neuausrichtung bestehender Prozesse. Während proprietäre Lösungen
Lizenz- und Betriebskosten verursachen, sind Open-Source-Modelle zwar
kostengünstiger, erfordern aber häufig einen höheren Initialaufwand für
Anpassung und Integration~\cite{habibi_open_2025}. Song
et~al.~\cite{song_impact_2024} zeigen, dass KI-basierte Assistenzsysteme in
Open-Source-Projekten nicht nur die Kollaboration und Codequalität verbessern,
sondern auch zu einer Demokratisierung des Entwicklungsprozesses beitragen
können.

Zudem hängt der konkrete Nutzen von KI-gestützter Entwicklung maßgeblich davon
ab, inwieweit Unternehmen strategische Ziele, Kostenstruktur und
Wertschöpfungspotenziale aufeinander
abstimmen~\cite{mcnamara_exponential_2025}. Neben direkten Effizienzgewinnen
zählen auch Flexibilität, Innovationspotenzial und die Gewinnung von
Wettbewerbsvorteilen zu den zentralen
Nutzenaspekten~\cite{storey_generative_2025}. Dem stehen potenzielle
Folgekosten durch Fehlinvestitionen, Qualitätsprobleme bei KI-generiertem Code
und ethische Risiken gegenüber, die zu erheblichen finanziellen Belastungen
führen können, wenn sie nicht frühzeitig adressiert werden.

Nicht zuletzt verändern KI-Technologien auch kreative Branchen und ermöglichen
neue Formen gesellschaftlicher Teilhabe~\cite{anantrasirichai_artificial_2025}.
Eine sorgfältige Kosten-Nutzen-Abwägung sowie die kontinuierliche Anpassung der
Strategie bleiben daher zentrale Voraussetzungen für einen erfolgreichen und
nachhaltigen Einsatz generativer KI in der Softwareentwicklung.

Die Analyse der wirtschaftlichen und gesellschaftlichen Auswirkungen
generativer KI in der Softwareentwicklung macht deutlich. Unternehmen und
Entwickler:innen können durch den Einsatz von KI-Tools deutliche
Effizienzgewinne, neue Wertschöpfungsmodelle und innovative Arbeitsformen
erschließen. Gleichzeitig entstehen neue Herausforderungen für Arbeitsmärkte,
Organisationsstrukturen und Qualifikationsprofile. Für eine nachhaltige
Integration generativer KI sind nicht nur technologische Investitionen, sondern
insbesondere strategische Anpassungen von Unternehmensstrukturen,
kontinuierliche Weiterbildung und die Entwicklung ethischer Leitlinien
erforderlich. Nur so können die Potenziale generativer KI langfristig genutzt
und gesellschaftliche Risiken wirksam begrenzt werden.

% \begin{itemize}
%     \item Analyse der wirtschaftlichen Effizienz und Kostenersparnis
%     \item Vergleich der Investitionskosten und erwarteten Produktivitätsgewinne
%     \item Langfristige wirtschaftliche Auswirkungen für Unternehmen und die Softwarebranche
% \end{itemize}