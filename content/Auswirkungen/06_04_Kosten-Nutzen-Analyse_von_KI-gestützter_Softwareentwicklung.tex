Die Integration generativer KI in die Softwareentwicklung bringt sowohl
beträchtliche Vorteile als auch neue Kosten- und Risikofaktoren mit sich.
Marguerit~\cite{marguerit_augmenting_2025} und Farach et
al.~\cite{farach_evolving_2025} weisen darauf hin, dass durch den Einsatz von
KI-Tools die Produktivität in vielen Bereichen signifikant steigt –
insbesondere bei Routinetätigkeiten, der Code-Generierung und der
Testautomatisierung. Dies führt zu messbaren Effizienzgewinnen und kann
langfristig die Entwicklungskosten pro Feature oder Release deutlich senken.

Allerdings entstehen auch neue Investitionen: Die Einführung und Wartung von
KI-Systemen erfordert gezielte Weiterbildung, Anpassung von Infrastrukturen und
häufig eine Neuausrichtung bestehender Entwicklungsprozesse.
Habibi~\cite{habibi_open_2025} betont, dass proprietäre KI-Lösungen mit
Lizenzgebühren und laufenden Betriebskosten verbunden sind, während
Open-Source-Modelle zwar günstiger sein können, dafür aber einen höheren
Initialaufwand für Customizing und Integration erfordern. Song
et~al.~\cite{song_impact_2024} zeigen anhand von Open-Source-Projekten, dass
KI-basierte Assistenzsysteme die Kollaboration und Codequalität nicht nur
steigern, sondern auch zu einer Demokratisierung des Entwicklungsprozesses
beitragen können.

McNamara und Marpu~\cite{mcnamara_exponential_2025} verweisen darauf, dass der
Nutzen von KI-gestützter Entwicklung maßgeblich von der Fähigkeit der
Unternehmen abhängt, strategische Ziele, Kostenstruktur und
Wertschöpfungspotenziale aufeinander abzustimmen. Storey et
al.~\cite{storey_generative_2025} ergänzen, dass zu den nicht-monetären
Nutzenaspekten vor allem Flexibilität, Innovationspotenzial und die Gewinnung
von Wettbewerbsvorteilen zählen. Umgekehrt ist zu beachten, dass
Fehlinvestitionen, Qualitätsprobleme bei KI-generiertem Code und ethische
Risiken zu erheblichen Folgekosten führen können, wenn diese nicht frühzeitig
adressiert werden.

Anantrasirichai et~al.~\cite{anantrasirichai_artificial_2025} betonen, dass die
fortschreitende Integration von KI-Technologien auch die kreativen Branchen
nachhaltig verändert und neue Formen gesellschaftlicher Teilhabe ermöglicht.
Die Rolle von KI als Impulsgeber für Innovation wird in Zukunft weiter an
Bedeutung gewinnen.

Eine sorgfältige Kosten-Nutzen-Abwägung und eine ständige Anpassung der
Strategie sind daher zentrale Voraussetzungen für den erfolgreichen und
nachhaltigen Einsatz von generativer KI in der Softwareentwicklung.

% \begin{itemize}
%     \item Analyse der wirtschaftlichen Effizienz und Kostenersparnis
%     \item Vergleich der Investitionskosten und erwarteten Produktivitätsgewinne
%     \item Langfristige wirtschaftliche Auswirkungen für Unternehmen und die Softwarebranche
% \end{itemize}