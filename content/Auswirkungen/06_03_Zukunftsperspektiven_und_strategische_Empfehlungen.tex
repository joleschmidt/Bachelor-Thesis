Die wissenschaftliche Literatur macht deutlich, dass der Einsatz generativer KI
in der Softwareentwicklung noch am Anfang einer langfristigen
Transformationsphase steht. Storey et al.~\cite{storey_generative_2025} sehen
großes Potenzial für Unternehmen, sich durch frühzeitige Investitionen in
KI-Kompetenzen, datenbasierte Prozesse und ethische Leitlinien
Wettbewerbsvorteile zu sichern. Marguerit~\cite{marguerit_augmenting_2025}
betont, dass kontinuierliche Weiterbildung und die Entwicklung neuer
Rollenprofile entscheidend sind, damit Beschäftigte mit dem technologischen
Wandel Schritt halten können.

Habibi~\cite{habibi_open_2025} unterstreicht die Bedeutung von
Open-Source-Ansätzen und kollaborativen Entwicklungsmodellen, um Innovation und
Transparenz im KI-Ökosystem zu fördern. McNamara und
Marpu~\cite{mcnamara_exponential_2025} weisen darauf hin, dass Unternehmen
verstärkt auf flexible, adaptive Strukturen setzen müssen, um auf die
exponentiell steigende Geschwindigkeit technologischer Veränderungen reagieren
zu können.

Die Autoren sind sich einig, dass die Integration von GenAI-Tools nur dann
nachhaltig gelingt, wenn Unternehmen strategisch in Kompetenzen, Change
Management und eine offene Innovationskultur investieren. Farach et
al.~\cite{farach_evolving_2025} sehen zudem in der Betrachtung digitaler Arbeit
als eigenständigen Produktionsfaktor einen wichtigen Schritt, um die
Wertschöpfungspotenziale von KI systematisch zu erschließen und zugleich
soziale Risiken abzufedern.

% \begin{itemize}
%     \item Notwendige Anpassungen für Unternehmen und Entwickler
%     \item Möglichkeiten der Integration von KI in bestehende Entwicklungsprozesse
%     \item Regulatorische und ethische Implikationen für eine nachhaltige KI-Nutzung
% \end{itemize}