% Wie schreibe ich das Abstract?
% Bei der Vorbereitung des Abstracts sollten Sie sich an folgenden grundlegenden Punkten orientieren:

%% Motivation des Textes:
%% worin liegt die Bedeutung der entsprechenden Forschung, warum sollte der längere Text gelesen werden?

%% Fragestellung:
%% welche Fragestellung(en) versucht der Text zu beantworten, was ist der Umfang der Forschung, was sind die zentralen Argumente und Behauptungen?

%% Methodologie:
%% welche Methoden/Zugänge nutzt der Autor/die Autorin, auf welche empirische Basis stützt sich der Text?

%% Ergebnisse:
%% zu welchen Ergebnissen kam die Forschung, was sind die zentralen Schlussfolgerungen des Textes?

%% Implikationen:
%% welche Schlussfolgerungen ergeben sich aus dem Text für die Forschung, was fügt der Text unserem Wissen über das Thema hinzu?
 
\begin{abstract}
    Künstliche Intelligenz (KI) hat sich in den letzten Jahren von einem theoretischen Konzept zu einer zentralen Technologie in vielen Bereichen der Softwareentwicklung entwickelt. Besonders generative KI-Modelle wie GitHub Copilot und Cursor AI revolutionieren den Entwicklungsprozess, indem sie automatisierte Code-Vorschläge, Fehlerkorrekturen und sogar vollständige Funktionen generieren. Diese Fortschritte werfen jedoch grundlegende Fragen auf: Welche langfristigen Auswirkungen hat der verstärkte Einsatz von KI auf Softwareentwickler? Inwieweit verändert KI etablierte Prinzipien und Methoden der Softwareentwicklung? Diese Arbeit untersucht die Chancen, Herausforderungen und praxisorientierten Anwendungen von KI in der Zukunft der Softwareentwicklung.
\end{abstract}
\cleardoublepage