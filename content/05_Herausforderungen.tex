\chapter{Herausforderungen durch KI in der Softwareentwicklung}
Trotz der erheblichen Chancen, die der Einsatz generativer KI in der
Softwareentwicklung bietet, sind mit ihrer Integration zahlreiche
Herausforderungen verbunden. Neben technischen Fragen stehen insbesondere
Aspekte der Sicherheit, des Datenschutzes, der ethischen Verantwortung sowie
soziale und organisatorische Veränderungen im Mittelpunkt der aktuellen
Diskussion. Die folgenden Abschnitte beleuchten diese Herausforderungen anhand
aktueller Literatur und reflektieren sie unter Einbezug der im praktischen Teil
gewonnenen Erfahrungen.

\section{Sicherheits- und Datenschutzaspekte}
Die Integration generativer KI-Tools in die Softwareentwicklung eröffnet nicht
nur neue Chancen, sondern schafft zugleich neue Risiken im Bereich der
IT-Sicherheit und des Datenschutzes. Shi et al.~\cite{shi_ai-assisted_2023}
zeigen, dass KI-gestützte Entwicklung sowohl zur Erkennung und Vermeidung von
Sicherheitslücken beitragen kann, aber auch neue Angriffsflächen entstehen,
insbesondere wenn unsichere oder fehlerhafte Codevorschläge ungeprüft
übernommen werden.

Alwageed und Khan~\cite{alwageed_role_nodate} betonen, dass generative
KI-Modelle wie LLMs dazu beitragen können, Best Practices im Bereich
Security-by-Design umzusetzen. Gleichzeitig weisen sie darauf hin, dass
\enquote{over-reliance on AI-generated code may lead to subtle vulnerabilities
    and propagate insecure coding patterns if not carefully validated by human
    experts}~\cite[S.~9]{alwageed_role_nodate}.

Auch Fragen des Datenschutzes stehen im Fokus: Bei der Nutzung externer
KI-Modelle (z.\,B. in Cloud-Umgebungen) können sensible Daten unbeabsichtigt an
Dritte weitergegeben werden. Shi et al.~\cite{shi_ai-assisted_2023} fordern
deshalb, dass Security-Reviews, Audit-Trails und eine konsequente Einbindung
menschlicher Expertise (\enquote{Human-in-the-Loop}) zentrale Bestandteile des
Entwicklungsprozesses bleiben müssen.

Weisz et~al.~\cite{weisz_design_2024} betonen, dass regelmäßige
Security-Reviews und der Einsatz von Human-in-the-Loop-Mechanismen zentrale
Bausteine für die sichere Nutzung generativer KI sind. Studien zeigen außerdem,
dass viele Angriffe darauf abzielen, gezielt das Verhalten generativer Modelle
zu manipulieren. Aus diesem Grund fordern Shi
et~al.~\cite{shi_ai-assisted_2023} und Alwageed
et~al.~\cite{alwageed_role_nodate} verstärkte Kontrollmechanismen und gezielte
Trainings, um die Robustheit der Systeme zu erhöhen.

Im Praxisteil dieser Arbeit zeigte sich, dass KI-Tools wie Copilot und Cursor
zwar potenziell sicherheitskritische Muster (z.\,B. hardcodierte Passwörter)
erkennen und warnen können, ihre Vorschläge jedoch stets von Entwickler*innen
geprüft und angepasst werden sollten.


\subsection{Sicherheitsrisiken durch generative Modelle}

Generative KI-Modelle bringen spezifische neue Bedrohungen mit sich. Zu den
wichtigsten Risiken zählen sogenannte \enquote{Prompt Injections}, bei denen
durch manipulierte Eingaben unsicherer oder schädlicher Code erzeugt werden
kann, sowie \enquote{adversarial attacks}, bei denen minimale Änderungen an den
Eingabedaten zu sicherheitskritischen Verhaltensweisen führen können
\cite{shi_ai-assisted_2023}.

Ein weiteres Risiko besteht im sogenannten \enquote{Model Poisoning}, bei dem
während des Trainings gezielt fehlerhafte oder bösartige Daten eingespeist
werden, um Schwachstellen in der KI zu platzieren. Besonders große generative
Modelle wie LLMs sind hierfür anfällig, da sie oft auf umfangreiche und
öffentlich verfügbare Datenquellen zurückgreifen \cite{alwageed_role_nodate}.

Auch im Bereich der Software-Supply-Chain entstehen neue Risiken: Unzureichend
geprüfter oder von Dritten generierter Code kann unbemerkt Schwachstellen in
den Entwicklungsprozess einschleusen. Entlang der gesamten Kette, von der
Entwicklung bis zur Bereitstellung, muss daher auf Sicherheit und regelmäßige
Überprüfung geachtet werden \cite{siebert_generative_2024}.

Die Literatur empfiehlt, Sicherheitsmechanismen wie regelmäßige
Security-Reviews, Human-in-the-Loop-Prozesse und gezielte Trainingsmaßnahmen
gegen adversarielle Angriffe und Model Poisoning zu etablieren, um die
Robustheit generativer KI-Systeme zu erhöhen \cite{shi_ai-assisted_2023,
    alwageed_role_nodate, siebert_generative_2024}.


\section{Ethische und soziale Implikationen}

Die zunehmende Integration generativer KI in die Softwareentwicklung wirft
weitreichende ethische und soziale Fragen auf. Die Automatisierung von
Entwicklungsaufgaben macht es notwendig, Verantwortung und
Entscheidungsprozesse klar zu definieren. Insbesondere Nachvollziehbarkeit und
Transparenz von KI-generierten Lösungen sind zentrale Herausforderungen für
ethische Standards und die Überprüfbarkeit von Ergebnissen
\cite{weisz_design_2024}.

Ein wesentliches ethisches Problem besteht im sogenannten Bias. Generative
KI-Modelle übernehmen häufig bestehende Vorurteile oder Stereotypen aus den
Trainingsdaten und können diese unreflektiert reproduzieren. Um
Diskriminierung, Fehlinformationen und unfaire Vorschläge zu vermeiden, sind
technische und organisatorische Kontrollmechanismen erforderlich, wie etwa
Guardrails, kontrollierte Testdatensätze und Diversity-Checks
\cite{weisz_design_2024, schmitt_generative_2024}.

Ethische und soziale Fragestellungen gewinnen immer mehr an Bedeutung, da
Barrierefreiheit und Teilhabe bei der Entwicklung von KI-Systemen noch oft
unzureichend berücksichtigt werden, obwohl KI-Technologien neue Möglichkeiten
der Inklusion bieten könnten \cite{flores-saviaga_impact_2025}.

Gleichzeitig zeigt die Literatur, dass der Einsatz generativer KI die
berufliche Identität von Entwickler:innen beeinflusst. Es entstehen
Unsicherheiten hinsichtlich der eigenen Rolle und Wertschätzung, aber auch neue
Möglichkeiten zur Kompetenzentwicklung und Zusammenarbeit, wenn der Fokus auf
Mensch-KI-Kollaboration gelegt wird \cite{schmitt_generative_2024}.

Auch auf organisatorischer Ebene sind Unternehmen gefordert, klare Leitlinien
für die Nutzung von GenAI-Tools zu formulieren und Verantwortlichkeiten,
Qualitätsstandards sowie ethische Prinzipien verbindlich zu verankern
\cite{nguyen-duc_generative_2023}.


\subsection{Ethische Konflikte und Bias in KI-Systemen}

Beim Einsatz generativer KI in der Softwareentwicklung besteht eine zentrale
Herausforderung darin, dass große Sprachmodelle und generative Systeme
bestehende Vorurteile oder Stereotype aus ihren Trainingsdaten übernehmen und
unbewusst reproduzieren können. Dies führt zu unfairen, potenziell
diskriminierenden Ergebnissen und stellt ein erhebliches ethisches Risiko dar
\cite{weisz_design_2024}.

Um diesen Risiken zu begegnen, sind technische und organisatorische Maßnahmen
erforderlich. Dazu zählen regelmäßige Prüfungen auf Fairness und Transparenz,
der Einsatz von Diversity-Checks, kontrollierten Testdatensätzen sowie
spezialisierte Überwachungsmechanismen, um Verzerrungen zu erkennen und
abzumildern. Auch die Entwicklung und Durchsetzung von sogenannten Guardrails
in KI-Systemen wird als essenziell erachtet, um Diskriminierung und
Fehlinformationen zu verhindern \cite{weisz_design_2024,
    schmitt_generative_2024}.

Neben den technischen Aspekten hat Bias auch soziale und organisationale
Auswirkungen. Eine unkritische Nutzung von KI-Systemen kann im Team zu
Unsicherheiten, Vertrauensverlust und Spannungen führen, insbesondere dann,
wenn Vorschläge der KI als objektiv wahrgenommen werden, obwohl sie verzerrt
oder unvollständig sind \cite{schmitt_generative_2024}.

Insgesamt ist der verantwortungsvolle Umgang mit Bias und ethischen Konflikten
eine Grundvoraussetzung für den erfolgreichen und nachhaltigen Einsatz
generativer KI in der Softwareentwicklung.


\subsection{Langfristige Auswirkungen auf Entwickler:innen-Rollen}
Der verstärkte Einsatz generativer KI verändert nicht nur technische Prozesse,
sondern wirkt sich auch langfristig auf die Rolle von Entwickler:innen aus.
Schmitt et al.~\cite{schmitt_generative_2024} zeigen, dass der zunehmende
Einsatz von KI-Tools wie Copilot oder ChatGPT zu einem Wandel der beruflichen
Identität und des Selbstverständnisses von Softwareentwickler:innen führen
kann. Während einerseits Routinetätigkeiten und repetitive Aufgaben zunehmend
automatisiert werden, rücken Kompetenzen wie Prompt-Engineering,
Systembewertung und die kritische Reflexion von KI-Ergebnissen stärker in den
Vordergrund.

Die Studie von Schmitt et al.~\cite{schmitt_generative_2024} verdeutlicht
zudem, dass viele Entwickler:innen durch die Integration von GenAI neue
Möglichkeiten zur Kompetenzentwicklung sehen -- etwa im Bereich
Mensch-KI-Kollaboration oder im Aufbau von Schnittstellenkompetenzen zwischen
Entwicklung, Domänenwissen und KI-Nutzung. Gleichzeitig berichten die
Autor:innen aber auch von Verunsicherung und Identitätskonflikten, die durch
die Neuverteilung von Aufgaben und die wachsende Abhängigkeit von KI-Systemen
entstehen können.

Auch auf organisatorischer Ebene ergeben sich laut Nguyen-Duc et
al.~\cite{nguyen-duc_generative_2023} langfristige Veränderungen: Die
Einführung von GenAI-Tools erfordert nicht nur technisches, sondern auch
soziales Change Management, etwa durch die Entwicklung neuer Rollenprofile,
angepasster Schulungskonzepte und überarbeiteter Verantwortlichkeiten im Team.
Entscheidend ist laut beiden Quellen, dass Unternehmen und Entwickler:innen die
Veränderungen aktiv gestalten und sich auf eine kontinuierliche
Weiterentwicklung der beruflichen Rollen einlassen.

Im praktischen Teil dieser Arbeit zeigte sich dieser Wandel exemplarisch:
Während der Implementierung des Map-Screens in der Locals-App verlagerte sich
der Fokus zunehmend von der reinen Code-Implementierung hin zur Fähigkeit, die
richtigen Prompts zu formulieren, generierte Vorschläge kritisch zu prüfen und
die Zusammenarbeit mit KI-Tools aktiv zu gestalten. Anstelle von klassischen
Routinetätigkeiten dominierten Tätigkeiten wie das Feintuning von Prompts, die
Auswahl zwischen unterschiedlichen KI-generierten Lösungswegen und die
Integration von Feedback in den Entwicklungsprozess.

Diese Entwicklung bestätigt die Beobachtung von Schmitt et
al.~\cite{schmitt_generative_2024}, dass Entwickler:innen künftig vermehrt als
Schnittstelle zwischen Mensch und KI agieren und Kompetenzen wie
Systembewertung, Prompt-Engineering und kritische Reflexion an Bedeutung
gewinnen.


\subsection{Technische und organisatorische Hürden bei der Einführung von KI}

Die Einführung generativer KI ist mit zahlreichen technischen und
organisatorischen Herausforderungen verbunden. Zu den größten Hürden zählen der
Mangel an Standards, geeigneten Benchmarks und validen Testdaten
\cite{nguyen-duc_generative_2023}. Agile Entwicklungsprozesse müssen speziell
für KI-Projekte weiterentwickelt werden, um den Anforderungen neuer
Technologien gerecht zu werden \cite{gill_agile_2025}. Unsicherheiten im Team
und eine fehlende Akzeptanz der neuen Werkzeuge stellen weitere zentrale
Barrieren dar, wie auch das Fraunhofer IESE hervorhebt
\cite{siebert_generative_2024}.

Benutzerfreundlichkeit und UX-Design spielen bei der Integration von KI-Tools
eine immer größere Rolle. Häufig wird die subjektive Nutzererfahrung
unterschätzt, obwohl die Akzeptanz im Team maßgeblich von transparenter
Kommunikation, kontinuierlichem Feedback und einer benutzerzentrierten
Gestaltung der Tools abhängt \cite{sergeyuk_human-ai_2025, sifi_how_2025}.

Weitere technische Herausforderungen bestehen in der Qualitätssicherung der
generierten Ergebnisse. Es ist bislang schwierig, die Zuverlässigkeit,
Sicherheit und Wartbarkeit von KI-generiertem Code systematisch zu überprüfen.
Auch der Mangel an geeigneten Benchmarks, Testdaten und automatisierten
Validierungsverfahren erschwert die breite Einführung von GenAI im
Unternehmensumfeld \cite{nguyen-duc_generative_2023}. Mit der fortschreitenden
Integration von KI in die Softwareentwicklung entstehen zudem neue
Anforderungen an Sicherheit, Arbeitsorganisation und langfristige
Kollaborationsmodelle \cite{hazra_ai_2025}.

Auf organisatorischer Ebene wirken sich fehlende Akzeptanz und Unsicherheiten
im Team, unklare Verantwortlichkeiten sowie mangelnde Schulung als erhebliche
Implementierungsbarrieren aus \cite{nguyen-duc_generative_2023,
    schmitt_generative_2024}. Die Umstellung auf KI-gestützte Prozesse erfordert
häufig umfassendes Change Management, die Anpassung bestehender Arbeitsweisen
und neue Formen der Zusammenarbeit. Eine erfolgreiche Einführung von
GenAI-Tools setzt nicht nur technisches Know-how, sondern auch kulturelle
Offenheit und kontinuierliche Weiterbildung voraus
\cite{schmitt_generative_2024}.

Für den Erfolg conversationaler KI-Assistenzsysteme ist zudem eine enge
Zusammenarbeit von Human-Computer-Interaction- und KI-Forschung notwendig, um
praxisnahe Evaluationsmethoden zu etablieren \cite{richards_bridging_2025}.

