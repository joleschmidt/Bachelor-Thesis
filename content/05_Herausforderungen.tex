\chapter{Herausforderungen durch KI in der Softwareentwicklung}
Trotz der erheblichen Chancen, die der Einsatz generativer KI in der
Softwareentwicklung bietet, sind mit ihrer Integration zahlreiche
Herausforderungen verbunden. Neben technischen Fragen stehen insbesondere
Aspekte der Sicherheit, des Datenschutzes, der ethischen Verantwortung sowie
soziale und organisatorische Veränderungen im Mittelpunkt der aktuellen
Diskussion. Die folgenden Abschnitte beleuchten diese Herausforderungen anhand
aktueller Literatur und reflektieren sie unter Einbezug der im praktischen Teil
gewonnenen Erfahrungen.

\section{Sicherheits- und Datenschutzaspekte}

Die Integration generativer KI-Tools in die Softwareentwicklung eröffnet nicht
nur neue Chancen, sondern bringt auch Risiken für IT-Sicherheit und Datenschutz
mit sich. Aktuelle Studien zeigen, dass generative KI einerseits dazu beitragen
kann, Sicherheitslücken zu erkennen und Best Practices wie Security-by-Design
umzusetzen, andererseits aber auch neue Angriffsflächen schafft, wenn
KI-generierte Vorschläge ungeprüft übernommen werden
\cite{shi_ai-assisted_2023, alwageed_role_nodate}.

Insbesondere wird darauf hingewiesen, dass der Einsatz generativer KI zu
spezifischen Risiken wie \enquote{Prompt Injection} und \enquote{adversarial
    attacks} führen kann. Dabei werden durch gezielte oder manipulierte Eingaben
der KI unsichere oder schadhafte Codefragmente entlockt
\cite{shi_ai-assisted_2023}. Ein weiteres Risiko ist das sogenannte
\enquote{Model Poisoning}, bei dem durch fehlerhafte oder bösartige
Trainingsdaten gezielt Schwachstellen in das Modell eingeschleust werden
\cite{alwageed_role_nodate}.

Regelmäßige Security-Reviews, Audit-Trails und eine konsequente Einbindung
menschlicher Expertise (\enquote{Human-in-the-Loop}) werden in der Literatur
als zentrale Maßnahmen zur Absicherung empfohlen. Die Gefahr besteht
insbesondere darin, dass KI-Tools potenziell sicherheitskritische Muster zwar
erkennen und kennzeichnen können, ihre Vorschläge jedoch stets von
Entwickler:innen geprüft und angepasst werden müssen
\cite{shi_ai-assisted_2023, alwageed_role_nodate, siebert_generative_2024}.

Auch Datenschutzfragen treten verstärkt in den Vordergrund, insbesondere beim
Einsatz externer KI-Modelle in Cloud-Umgebungen. Hier besteht die Gefahr, dass
sensible Daten unbeabsichtigt an Dritte weitergegeben werden oder aus den
generierten Vorschlägen rekonstruiert werden können
\cite{siebert_generative_2024}.

Im praktischen Teil dieser Arbeit zeigte sich, dass KI-Tools wie Copilot und
Cursor zwar potenziell sicherheitskritische Muster (z.\,B. hardcodierte
Passwörter) erkennen können, ihre Vorschläge aber stets kritisch geprüft und
gegebenenfalls angepasst werden müssen.


\subsection{Sicherheitsrisiken durch generative Modelle}

Generative KI-Modelle bringen spezifische neue Bedrohungen mit sich. Zu den
wichtigsten Risiken zählen sogenannte \enquote{Prompt Injections}, bei denen
durch manipulierte Eingaben unsicherer oder schädlicher Code erzeugt werden
kann, sowie \enquote{adversarial attacks}, bei denen minimale Änderungen an den
Eingabedaten zu sicherheitskritischen Verhaltensweisen führen können
\cite{shi_ai-assisted_2023}.

Ein weiteres Risiko besteht im sogenannten \enquote{Model Poisoning}, bei dem
während des Trainings gezielt fehlerhafte oder bösartige Daten eingespeist
werden, um Schwachstellen in der KI zu platzieren. Besonders große generative
Modelle wie LLMs sind hierfür anfällig, da sie oft auf umfangreiche und
öffentlich verfügbare Datenquellen zurückgreifen \cite{alwageed_role_nodate}.

Auch im Bereich der Software-Supply-Chain entstehen neue Risiken. Unzureichend
geprüfter oder von Dritten generierter Code kann unbemerkt Schwachstellen in
den Entwicklungsprozess einschleusen. Entlang der gesamten Kette, von der
Entwicklung bis zur Bereitstellung, muss daher auf Sicherheit und regelmäßige
Überprüfung geachtet werden \cite{siebert_generative_2024}.

Die Literatur empfiehlt, Sicherheitsmechanismen wie regelmäßige
Security-Reviews, Human-in-the-Loop-Prozesse und gezielte Trainingsmaßnahmen
gegen adversarielle Angriffe und Model Poisoning zu etablieren, um die
Robustheit generativer KI-Systeme zu erhöhen \cite{shi_ai-assisted_2023,
    alwageed_role_nodate, siebert_generative_2024}.


\section{Ethische und soziale Implikationen}

Die zunehmende Integration generativer KI in die Softwareentwicklung wirft
weitreichende ethische und soziale Fragen auf. Die Automatisierung von
Entwicklungsaufgaben macht es notwendig, Verantwortung und
Entscheidungsprozesse klar zu definieren. Insbesondere Nachvollziehbarkeit und
Transparenz von KI-generierten Lösungen sind zentrale Herausforderungen für
ethische Standards und die Überprüfbarkeit von Ergebnissen
\cite{weisz_design_2024}.

Ein wesentliches ethisches Problem besteht im sogenannten Bias. Generative
KI-Modelle übernehmen häufig bestehende Vorurteile oder Stereotypen aus den
Trainingsdaten und können diese unreflektiert reproduzieren. Um
Diskriminierung, Fehlinformationen und unfaire Vorschläge zu vermeiden, sind
technische und organisatorische Kontrollmechanismen erforderlich, wie etwa
Guardrails, kontrollierte Testdatensätze und Diversity-Checks
\cite{weisz_design_2024, schmitt_generative_2024}.

Ethische und soziale Fragestellungen gewinnen immer mehr an Bedeutung, da
Barrierefreiheit und Teilhabe bei der Entwicklung von KI-Systemen noch oft
unzureichend berücksichtigt werden, obwohl KI-Technologien neue Möglichkeiten
der Inklusion bieten könnten \cite{flores-saviaga_impact_2025}.

Gleichzeitig zeigt die Literatur, dass der Einsatz generativer KI die
berufliche Identität von Entwickler:innen beeinflusst. Es entstehen
Unsicherheiten hinsichtlich der eigenen Rolle und Wertschätzung, aber auch neue
Möglichkeiten zur Kompetenzentwicklung und Zusammenarbeit, wenn der Fokus auf
Mensch-KI-Kollaboration gelegt wird \cite{schmitt_generative_2024}.

Auch auf organisatorischer Ebene sind Unternehmen gefordert, klare Leitlinien
für die Nutzung von GenAI-Tools zu formulieren und Verantwortlichkeiten,
Qualitätsstandards sowie ethische Prinzipien verbindlich zu verankern
\cite{nguyen-duc_generative_2023}.


\subsection{Ethische Konflikte und Bias in KI-Systemen}
Eines der zentralen ethischen Probleme beim Einsatz generativer KI in der
Softwareentwicklung ist die Gefahr von Bias und Diskriminierung. Wie Weisz et
al.~\cite{weisz_design_2024} herausstellen, können große Sprachmodelle und
generative Systeme bestehende Vorurteile, Diskriminierungen oder Stereotypen
aus den Trainingsdaten übernehmen und diese im erzeugten Code oder in den
Vorschlägen reproduzieren. Dies kann zu unfairen, potenziell diskriminierenden
Ergebnissen führen und damit ethische Grundsätze sowie
Gleichbehandlungsprinzipien verletzen.

Um solchen Risiken zu begegnen, empfehlen Weisz et
al.~\cite{weisz_design_2024}, dass Entwickler:innen und Organisationen
technische und organisatorische Maßnahmen (\enquote{Guardrails})
implementieren, die sicherstellen, dass generative KI-Lösungen regelmäßig auf
Fairness, Transparenz und mögliche Verzerrungen geprüft werden. Dazu zählen
etwa kontrollierte Testdatensätze, Diversity-Checks oder der Einsatz
spezialisierter Überwachungsmechanismen.

Schmitt et al.~\cite{schmitt_generative_2024} weisen darauf hin, dass die
Gefahr von Bias nicht nur technischer Natur ist, sondern auch soziale und
organisationale Auswirkungen haben kann. Insbesondere im Kontext beruflicher
Identität und Teamdynamik kann eine unkritische Nutzung von KI-Systemen zu
Unsicherheiten, Vertrauensverlust und Spannungen führen – etwa wenn Vorschläge
der KI als neutral oder objektiv wahrgenommen werden, obwohl sie verzerrt oder
unvollständig sind.

Insgesamt machen beide Quellen deutlich, dass ethische Konflikte und der Umgang
mit Bias zentrale Herausforderungen für den erfolgreichen und
verantwortungsvollen Einsatz generativer KI in der Softwareentwicklung
darstellen.


\subsection{Langfristige Auswirkungen auf Entwickler:innen-Rollen}

Der verstärkte Einsatz generativer KI-Tools in der Softwareentwicklung
verändert die Rolle von Entwickler:innen grundlegend. Während
Routinetätigkeiten und repetitive Aufgaben zunehmend automatisiert werden,
gewinnen Kompetenzen wie Prompt-Engineering, Systembewertung und die kritische
Reflexion von KI-Ergebnissen deutlich an Bedeutung
\cite{schmitt_generative_2024}.

Viele Entwickler:innen sehen in der Integration von GenAI neue Möglichkeiten
zur Kompetenzentwicklung, etwa in der Mensch-KI-Kollaboration oder im Aufbau
von Schnittstellenwissen zwischen Entwicklung, Domänenkenntnis und KI-Nutzung.
Gleichzeitig entstehen durch die Neuverteilung von Aufgaben und die wachsende
Abhängigkeit von KI-Systemen Unsicherheiten und Identitätskonflikte
\cite{schmitt_generative_2024}.

Auf organisatorischer Ebene sind Anpassungen der Rollenprofile, neue
Schulungskonzepte und die Überarbeitung von Verantwortlichkeiten notwendig, um
den Wandel aktiv zu gestalten und kontinuierliche Weiterbildung zu ermöglichen.
Entscheidend ist, dass Unternehmen und Entwickler:innen sich auf die
Veränderungen einlassen und die Transformation aktiv begleiten
\cite{nguyen-duc_generative_2023}.

Auch die Erfahrungen aus dem Praxisteil dieser Arbeit zeigen, dass der Fokus
bei der Entwicklung zunehmend auf der Formulierung präziser Prompts, der
kritischen Prüfung von KI-Vorschlägen und der aktiven Gestaltung der
Mensch-KI-Zusammenarbeit liegt. Die Rolle von Entwickler:innen wandelt sich
damit immer stärker hin zu einer Schnittstellenfunktion zwischen Mensch und
Maschine.


\subsection{Technische und organisatorische Hürden bei der Einführung von KI}
Die Einführung generativer KI ist mit zahlreichen technischen und
organisatorischen Hürden verbunden. Nguyen-Duc
et~al.~\cite{nguyen-duc_generative_2023} betonen den Mangel an Standards,
geeigneten Benchmarks und validen Testdaten als zentrale Herausforderungen.
Gill~\cite{Gil} verweist auf die Notwendigkeit, agile Prozesse speziell für
KI-Projekte weiterzuentwickeln, während das Fraunhofer IESE~\cite{Sie} auf
Unsicherheiten im Team und fehlende Akzeptanz hinweist. Sergeyuk
et~al.~\cite{Ser} unterstreichen in ihrer Übersicht, dass auch Usability und
UX-Design bei der Integration von KI-Tools stärker berücksichtigt werden
müssen. Sifi~\cite{sifi_how_2025} hebt hervor, dass die subjektive
Nutzererfahrung bei der Einführung neuer KI-Tools oft unterschätzt wird und die
Akzeptanz entscheidend von transparenter Kommunikation und kontinuierlichem
Feedback abhängt.

Weitere technische Herausforderungen bestehen in der Qualitätssicherung der
generierten Ergebnisse. Laut Nguyen-Duc et
al.~\cite{nguyen-duc_generative_2023} ist es bislang schwierig, die
Zuverlässigkeit, Sicherheit und Wartbarkeit von KI-generiertem Code
systematisch zu überprüfen. Auch der Mangel an geeigneten Benchmarks, Testdaten
und automatisierten Validierungsverfahren erschwert die breite Einführung von
GenAI im Unternehmensumfeld.

Mit der fortschreitenden Integration von KI in die Softwareentwicklung
entstehen nicht nur neue technische Herausforderungen, sondern auch veränderte
Anforderungen an Sicherheit, Arbeitsorganisation und langfristige
Kollaborationsmodelle \cite{hazra_ai_2025}

Auf organisatorischer Ebene betonen sowohl Nguyen-Duc et
al.~\cite{nguyen-duc_generative_2023} als auch Schmitt et
al.~\cite{schmitt_generative_2024}, dass fehlende Akzeptanz und Unsicherheit im
Team, unklare Verantwortlichkeiten sowie mangelnde Schulung zu erheblichen
Implementierungsbarrieren führen können. Die Umstellung auf KI-gestützte
Prozesse erfordert häufig ein umfassendes Change Management, die Anpassung
bestehender Arbeitsweisen und neue Formen der Zusammenarbeit. Schmitt et
al.~\cite{schmitt_generative_2024} weisen darauf hin, dass die erfolgreiche
Einführung von GenAI-Tools nicht nur technisches Know-how, sondern auch
kulturelle Offenheit und kontinuierliche Weiterbildung im Team voraussetzt.
Richards und Wessel~\cite{richards_bridging_2025} argumentieren, dass für den
Erfolg conversationaler KI-Assistenzsysteme eine enge Zusammenarbeit von HCI-
und KI-Forschung erforderlich ist, um praxisnahe Evaluationsmethoden zu
etablieren.

