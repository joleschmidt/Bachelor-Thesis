Ziel dieses Kapitels ist es, die Potenziale und Herausforderungen generativer
KI-Tools in der Softwareentwicklung praxisnah zu evaluieren. Anhand der
exemplarischen Implementierung eines interaktiven Map-Screens in der mobilen
App „Locals“ wird untersucht, wie moderne KI-Tools den Entwicklungsprozess
unterstützen, wo ihre Stärken liegen und welche Grenzen deutlich werden.
Aufbauend auf den theoretischen Grundlagen bietet dieses Praxisbeispiel die
Möglichkeit, die zuvor beschriebenen Konzepte und Annahmen empirisch zu
überprüfen und kritisch zu reflektieren.

\subsubsection{Motivation für das praktische Beispiel}

Die Entscheidung für die Entwicklung einer interaktiven Kartenansicht beruht
auf folgenden Überlegungen:
\begin{itemize}
      \item Map-Views sind zentrale und technisch anspruchsvolle Features in Event- und
            Social-Apps; sie erfordern die Integration unterschiedlicher Technologien
            (Geolocation, Datenmanagement, UI/UX, Filterfunktionen).
      \item Die Aufgabe vereint klassische Frontend-Herausforderungen (State-Management,
            UI-Logik) mit der Integration externer Libraries (wie
            \texttt{react-native-maps}, Google Maps API).
      \item Aufgrund ihrer Vielschichtigkeit eignet sich die Map-View besonders gut, um die
            Leistungsfähigkeit generativer KI-Tools im realen Entwicklungsprozess kritisch
            zu beleuchten.
      \item Der unmittelbare Nutzen für Nutzer:innen unterstreicht die Praxisrelevanz des
            gewählten Beispiels.
\end{itemize}

\subsubsection{Eingesetzte KI-Tools}

Für die Umsetzung wurden drei führende KI-basierte Entwicklungstools
ausgewählt, die aktuelle Trends und Methoden abbilden:
\begin{itemize}
      \item \textbf{GitHub Copilot:} Kontextsensitiver Code-Assistent mit Echtzeit-Vervollständigung, automatischen Vorschlägen für Funktionen, Tests und Dokumentation; unterstützt Pull-Requests, Chatfunktionen, Wissensdatenbanken und arbeitet direkt in gängigen IDEs~\cite{github_copilot_2025}.
      \item \textbf{Cursor:} KI-basierter Editor mit Multi-Line-Edits, Prompt Chaining, Fehlerdiagnose und dialogorientiertem Interface für komplexe Aufgabenabwicklung; plattformübergreifend einsetzbar~\cite{cursor_welcome_2025}.
      \item \textbf{Bolt.new:} Cloudbasierte Entwicklungsumgebung ohne lokale Installation, ermöglicht die Entwicklung und das Deployment direkt aus Prompts heraus; Multi-Plattform-Support, Live-Bearbeitung, Teamfunktionen~\cite{bolt_support_2025}.
\end{itemize}

Die vergleichende Betrachtung dieser Tools ermöglicht eine differenzierte
Analyse von Mensch-KI-Interaktion im realen Entwicklungsalltag – von
klassischen Code-Vervollständigern bis zu agentenbasierten Komplettsystemen.