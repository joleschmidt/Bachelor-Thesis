Die Zielsetzung dieses Kapitels besteht darin, den Einsatz generativer KI-Tools
in der Softwareentwicklung nicht nur theoretisch zu betrachten, sondern anhand
eines konkreten, praxisnahen Beispiels zu evaluieren. Im Zentrum steht die
Implementierung eines interaktiven Map-Screens für die mobile App „Locals“. Das
Vorgehen umfasst die iterative Umsetzung dieses Features mithilfe mehrerer
moderner KI-gestützter Entwicklungstools, um deren Stärken, Schwächen und das
Entwicklererlebnis unter realen Bedingungen vergleichend zu analysieren.

\subsubsection{Motivation für das praktische Beispiel}

Die Entscheidung, als praktisches Beispiel die Entwicklung einer interaktiven
Kartenansicht zu wählen, beruht auf mehreren Überlegungen:
\begin{itemize}
      \item Die Map-View ist ein zentrales und anspruchsvolles Feature moderner Event- und
            Social-Apps und erfordert die Integration unterschiedlicher Technologien (u. a.
            Geolocation, Datenmanagement, UI/UX-Design, Filter- und Suchfunktionen).
      \item Die Aufgabe vereint sowohl klassische Herausforderungen der
            Frontend-Entwicklung (State-Management, UI-Logik) als auch typische
            Stolpersteine in der Zusammenarbeit mit externen Libraries (z. B.
            \texttt{react-native-maps}, Google Maps API).
      \item Durch die Komplexität und Vielschichtigkeit eignet sich das Beispiel besonders
            gut, um die Leistungsfähigkeit generativer KI-Tools im realen
            Entwicklungsprozess kritisch zu beleuchten.
      \item Das Feature ist in modernen Event-Apps allgegenwärtig und der Nutzen für
            Nutzer:innen unmittelbar erlebbar.
\end{itemize}
Die Auswahl des Map-Screens als Demonstrationsobjekt ermöglicht somit einen fundierten und praxisnahen Einblick in die Potenziale und Grenzen generativer KI im täglichen Entwickleralltag.

\subsubsection{Eingesetzte KI-Tools}

Für die Implementierung des Map-Screens wurden folgende KI-gestützte
Entwicklungstools eingesetzt:

\begin{itemize}
      \item \textbf{GitHub Copilot:} KI-basierter Code-Assistent zur automatischen Codegenerierung und Vervollständigung in Visual Studio Code.
      \item \textbf{Cursor:} KI-gestützte Entwicklungsumgebung mit besonderem Fokus auf Kontextintegration (Screenshots, Code, Fehlermeldungen) und dialogorientiertem Prompt Chaining.
      \item \textbf{Bolt.new:} Cloudbasierte Entwicklungsplattform mit direkter Anbindung an GitHub, automatisierter Fehlerbehebung und Multi-Plattform-Unterstützung (Mobile und Web).
\end{itemize}

Die Auswahl der Tools ermöglicht eine umfassende Betrachtung verschiedener
Ansätze generativer KI in der Softwareentwicklung – vom klassischen Pair
Programming bis zur cloudbasierten Komplettlösung.

