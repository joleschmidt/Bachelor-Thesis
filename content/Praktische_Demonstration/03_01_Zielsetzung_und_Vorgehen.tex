
Die Zielsetzung dieses Kapitels besteht darin, den Einsatz generativer KI-Tools
in der Softwareentwicklung nicht nur theoretisch zu betrachten, sondern anhand
eines konkreten, praxisnahen Beispiels zu evaluieren. Im Zentrum steht die
Implementierung eines interaktiven Map-Screens für die mobile App „Locals“. Das
Vorgehen umfasst die iterative Umsetzung dieses Features mithilfe mehrerer
moderner KI-gestützter Entwicklungstools, um deren Stärken, Schwächen und das
Entwicklererlebnis unter realen Bedingungen vergleichend zu analysieren.

Die Auswahl der eingesetzten Tools orientiert sich an aktuellen Empfehlungen
aus Wissenschaft und Praxis. Donvir et~al.~\cite{donvir_role_2024} zeigen in
ihrem Überblick, dass GitHub Copilot, Cursor und Bolt zu den
fortschrittlichsten und am weitesten verbreiteten generativen
KI-Entwicklungsumgebungen zählen und sich besonders für vergleichende Studien
in der Softwareentwicklung eignen. Auch Rogachev~\cite{rogachev_my_nodate}
beschreibt praxisnah die Vorteile und Grenzen von Copilot-Agenten bei der
Entwicklung von React Native-Anwendungen.

\subsubsection{Motivation für das praktische Beispiel}

Die Entscheidung, als praktisches Beispiel die Entwicklung einer interaktiven
Kartenansicht zu wählen, beruht auf mehreren Überlegungen:
\begin{itemize}
      \item Die Map-View ist ein zentrales und anspruchsvolles Feature moderner Event- und
            Social-Apps und erfordert die Integration unterschiedlicher Technologien (u. a.
            Geolocation, Datenmanagement, UI/UX-Design, Filter- und Suchfunktionen).
      \item Die Aufgabe vereint sowohl klassische Herausforderungen der
            Frontend-Entwicklung (State-Management, UI-Logik) als auch typische
            Stolpersteine in der Zusammenarbeit mit externen Libraries (z. B.
            \texttt{react-native-maps}, Google Maps API).
      \item Durch die Komplexität und Vielschichtigkeit eignet sich das Beispiel besonders
            gut, um die Leistungsfähigkeit generativer KI-Tools im realen
            Entwicklungsprozess kritisch zu beleuchten.
      \item Das Feature ist in modernen Event-Apps allgegenwärtig und der Nutzen für
            Nutzer:innen unmittelbar erlebbar.
\end{itemize}
Die Auswahl des Map-Screens als Demonstrationsobjekt ermöglicht somit einen fundierten und praxisnahen Einblick in die Potenziale und Grenzen generativer KI im täglichen Entwickleralltag.

\subsubsection{Eingesetzte KI-Tools}

Für die Implementierung des Map-Screens wurden folgende KI-gestützte
Entwicklungstools eingesetzt:

\begin{itemize}
      \item \textbf{GitHub Copilot:} KI-basierter Code-Assistent mit kontextsensitiver Echtzeit-Code-Vervollständigung, automatischen Codevorschlägen für Funktionen, Tests und Dokumentation. Bietet einen integrierten Chat für Refactoring, Hilfestellungen und Testgenerierung, Pull-Request-Zusammenfassungen sowie projektübergreifende Kontextfunktionen (z.B. Copilot Spaces und Wissensdatenbanken). Unterstützt die gängigen IDEs und arbeitet kontinuierlich im Hintergrund, um den gesamten Entwicklungsprozess zu begleiten~\cite{github_copilot_2025}.
      \item \textbf{Cursor:} Spezialisierter KI-Code-Editor mit Funktionen wie Multi-Line Edits, Smart Rewrites, Tab-Kommandos zur schnellen Navigation durch Änderungsvorschläge sowie vollständige Indexierung und kontextabhängige Analyse des gesamten Repos. Der Agent Mode erlaubt Prompt Chaining, Terminalbefehle, Fehlerdiagnose und komplette Aufgabenabwicklung, unterstützt durch ein dialogorientiertes Chat-Interface. Cursor ist plattformübergreifend (inkl. Web/Mobile) einsetzbar~\cite{cursor_welcome_2025}.
      \item \textbf{Bolt.new:} Cloudbasierte, browserbasierte Entwicklungsumgebung ohne lokalen Setup-Bedarf. Ermöglicht die Erstellung und das Deployment von Web- und Mobile-Anwendungen direkt aus Prompts heraus. Unterstützt Multi-Plattform-Development mit automatischer Paket-Installation, Live-Code-Bearbeitung und direkter Anbindung an GitHub. Bolt bietet ein tokenbasiertes Preismodell sowie Funktionen für Kollaboration und Teamarbeit~\cite{bolt_support_2025}.
\end{itemize}

Die Auswahl dieser Tools ermöglicht eine umfassende Betrachtung verschiedener
Ansätze generativer KI in der Softwareentwicklung – vom klassischen Pair
Programming bis zur cloudbasierten Komplettlösung.

