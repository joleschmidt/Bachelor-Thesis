\label{sec:zielsetzung-vorgehen}

Am Beispiel der exemplarischen Implementierung einer interaktiven Kartenansicht
(„Map-Screen“) in der mobilen App „Locals“ wird untersucht, wie moderne
KI-gestützte Tools den Entwicklungsprozess unterstützen, in welchen Bereichen
sie ihre Stärken ausspielen können und wo ihre Grenzen sichtbar werden.
Aufbauend auf den zuvor beschriebenen theoretischen Grundlagen bietet dieses
Praxisbeispiel die Gelegenheit, zentrale Konzepte und Annahmen empirisch zu
überprüfen und kritisch zu reflektieren.

Die Entscheidung, den Fokus auf die Entwicklung einer interaktiven
Kartenansicht zu legen, ist aus mehreren Gründen besonders sinnvoll. Erstens
handelt es sich bei Map-Views um zentrale und technisch anspruchsvolle Features
in Event- und Social-Apps, da sie die Integration verschiedener Technologien
wie Geolocation, Datenmanagement, UI/UX-Design und Filterfunktionen erfordern.
Zweitens vereint diese Aufgabe klassische Frontend-Herausforderungen,
beispielsweise im Bereich State-Management und UI-Logik, mit der Integration
externer Libraries wie \texttt{react-native-maps} oder der Google Maps API.
Gerade durch diese Vielschichtigkeit eignet sich die Map-View besonders gut, um
die Leistungsfähigkeit generativer KI-Tools im realen Entwicklungsprozess
kritisch zu beleuchten. Nicht zuletzt unterstreicht auch der unmittelbare
Nutzen für die Anwender:innen die hohe Praxisrelevanz des gewählten Beispiels.

Für die Umsetzung wurden drei führende KI-basierte Entwicklungstools
ausgewählt, die aktuelle Trends und Methoden im Bereich der KI-gestützten
Softwareentwicklung abbilden. Zum einen kommt \textit{GitHub Copilot} zum
Einsatz, ein kontextsensitiver Code-Assistent mit Echtzeit-Vervollständigung,
der automatische Vorschläge für Funktionen, Tests und Dokumentation generiert
und sich direkt in gängige IDEs integriert~\cite{github_copilot_2025}.
Ergänzend wird \textit{Cursor} verwendet, ein KI-basierter Editor, der durch
Multi-Line-Edits, Prompt Chaining, Fehlerdiagnose und eine dialogorientierte
Benutzeroberfläche die Umsetzung komplexer Aufgaben unterstützt und
plattformübergreifend einsetzbar ist~\cite{cursor_welcome_2025}. Als drittes
Tool kommt \textit{Bolt} hinzu, eine cloudbasierte Entwicklungsumgebung, die
ohne lokale Installation auskommt und die Entwicklung sowie das Deployment
direkt aus Prompts heraus ermöglicht. Bolt unterstützt mehrere Plattformen,
bietet Live-Bearbeitung und Teamfunktionen~\cite{bolt_support_2025}.

Die vergleichende Analyse dieser drei Tools ermöglicht es, unterschiedliche
Formen der Interaktion zwischen Mensch und KI im realen Entwicklungsalltag zu
bewerten, von klassischen Code-Vervollständigern wie Copilot über
dialogorientierte Systeme wie Cursor bis hin zu agentenbasierten
Komplettumgebungen wie Bolt. Dadurch lassen sich Stärken, Schwächen und
bestmögliche Einsatzszenarien generativer KI-Tools in der Praxis gezielt
herausarbeiten.
