\label{sec:vorstellung-app}

Die App \textit{Locals} ist als mobile Plattform konzipiert, die es
Nutzer:innen ermöglicht, lokale Events zu entdecken, zu erstellen und zu
verwalten. Zielgruppe sind vor allem junge, urbane Menschen, die soziale
Interaktionen rund um Veranstaltungen suchen. Die technische Umsetzung setzt
auf eine moderne, modulare und plattformübergreifende Architektur, die sowohl
schnelle Entwicklungszyklen als auch eine hohe Flexibilität bei der Erweiterung
von Funktionen erlaubt.

Im Frontend kommen React Native und TypeScript zum Einsatz, ergänzt durch das
Expo-Framework, das eine konsistente Entwicklung für iOS- und Android-Geräte
ermöglicht. Für die Backend-Funktionalitäten wie Authentifizierung,
Datenhaltung und Synchronisation wird Firebase genutzt. Die Navigation
innerhalb der App basiert auf den Bibliotheken
\texttt{@react-navigation/native} und \texttt{expo-router}. Das
State-Management wird über eigene Context-Provider wie \texttt{AuthProvider}
und \texttt{EventsProvider} realisiert. Für die Benutzeroberfläche werden
Icon-Bibliotheken wie \texttt{@expo/vector-icons} und
\texttt{lucide-react-native} verwendet. Karten- und Standortfunktionen werden
durch \texttt{react-native-maps} und \texttt{expo-location} integriert.

Die Anwendung gliedert sich in drei Hauptbereiche: Der Explore-Screen dient als
Event-Feed, der die Veranstaltungen nach Standort und Interessen filtert. Der
Map-Screen stellt eine interaktive Kartenansicht bereit, in der Events als
Marker angezeigt und mit Hilfe von Filtern durchsucht werden können. Im
Profil-Screen erhalten die Nutzer:innen eine Übersicht und Verwaltung ihrer
eigenen sowie besuchten Events und Profileinstellungen. Im Rahmen dieser Arbeit
wird insbesondere der Map-Screen als prototypisches, KI-gestütztes Feature
exemplarisch entwickelt und dient so als Grundlage für die weitere Analyse der
Potenziale generativer KI in der Softwareentwicklung.

Eine zentrale Rolle in der Architektur der App übernimmt das
\texttt{RootLayout}, das sowohl die benötigten Schriftarten als auch
Authentifizierungs- und Event-Kontexte lädt. Die Steuerung der Nutzerführung
ist strikt zustandsbasiert und erfolgt über die Hooks \texttt{useAuth},
\texttt{useSegments} und \texttt{useRouter}. Dieses Architekturprinzip
gewährleistet eine klare Trennung zwischen Zustandsverwaltung und Routing und
erleichtert so spätere Erweiterungen oder Integrationen, beispielsweise von
KI-gestützten Funktionen.

Bereits vor der Integration generativer KI-Tools wurden zentrale Funktionen
implementiert. Dazu zählen die Benutzerauthentifizierung mit Firebase
Authentication, eine umfassende Profilverwaltung für eigene und besuchte Events
sowie eine vollständige Eventverwaltung mit Funktionen zum Anlegen, Bearbeiten
und Löschen von Veranstaltungen. Die Navigation innerhalb der App erfolgt über
ein Tab-System, das einen schnellen Wechsel zwischen den Hauptbereichen
Explore, Map und Profil ermöglicht. Ein responsives Design sorgt zudem für eine
einheitliche Darstellung auf sämtlichen Endgeräten.

Die modulare Struktur und die klare Abgrenzung der einzelnen App-Komponenten
schaffen eine ideale Grundlage, um den gezielten Einsatz generativer KI-Tools
im Entwicklungsprozess zu evaluieren und deren Auswirkungen auf Wartbarkeit,
Erweiterbarkeit und Benutzerfreundlichkeit zu untersuchen.
