
\subsection{Architektur und technischer Stack}

Die App \textit{Locals} ist eine mobile Plattform, die Nutzer:innen ermöglicht,
lokale Events zu entdecken, zu erstellen und zu verwalten. Sie richtet sich an
ein junges, urbanes Publikum und unterstützt soziale Interaktionen rund um
Veranstaltungen.

Die modulare Architektur setzt auf moderne, plattformübergreifende
Technologien:

% TODO: einheitlich schreiben mit punkt oder ohne generell überall?
\begin{itemize}
    \item \textbf{Frontend:} React Native, TypeScript, Expo (für schnelle, konsistente Entwicklung auf iOS/Android).
    \item \textbf{Backend:} Firebase (Authentifizierung, Datenhaltung, Synchronisation).
    \item \textbf{Navigation:} \texttt{@react-navigation/native}, \texttt{expo-router}.
    \item \textbf{State-Management:} Eigene Context-Provider (\texttt{AuthProvider}, \texttt{EventsProvider}).
    \item \textbf{UI:} \texttt{@expo/vector-icons}, \texttt{lucide-react-native}.
    \item \textbf{Karten/Location:} \texttt{react-native-maps}, \texttt{expo-location}.
\end{itemize}

Die Anwendung ist in drei Hauptbereiche gegliedert:
\begin{itemize}
    \item \textbf{Explore-Screen:} Event-Feed nach Standort/Interesse.
    \item \textbf{Map-Screen:} Interaktive Kartenansicht mit Event-Markern und Filtern.
    \item \textbf{Profil-Screen:} Übersicht und Verwaltung eigener Events/Profile.
\end{itemize}

Der Map-Screen wird in dieser Arbeit als prototypisches, KI-gestütztes Feature
exemplarisch entwickelt und dient als Grundlage für die weitere Analyse.

\subsubsection{Einstiegskomponente und Navigation}

Das zentrale \texttt{RootLayout} lädt Fonts, Authentifizierungs- und
Event-Kontexte und steuert die Nutzerführung abhängig vom Login-Status. Die
Navigation erfolgt strikt zustandsbasiert (\texttt{useAuth},
\texttt{useSegments}, \texttt{useRouter}). Dieses Architekturprinzip
gewährleistet eine saubere Trennung von Zustandsverwaltung und Routing, was
spätere Erweiterungen und Integrationen (z.B. von KI-gestützten Funktionen)
erleichtert.

\subsection{Bestehende Funktionalitäten}

Bereits vor der KI-Integration implementiert sind:
\begin{itemize}
    \item \textbf{Benutzerauthentifizierung:} Firebase Authentication.
    \item \textbf{Profilverwaltung:} Eigene und besuchte Events.
    \item \textbf{Eventverwaltung:} Anlegen, Bearbeiten, Löschen.
    \item \textbf{Tab-Navigation:} Zwischen Explore, Map, Profil.
    \item \textbf{Responsives Design:} Einheitliche Darstellung auf allen Endgeräten.
\end{itemize}

Die modulare Struktur und klare Abgrenzung der App-Komponenten bieten die
ideale Basis, um generative KI-Tools gezielt einzusetzen und deren Auswirkungen
im Entwicklungsprozess zu evaluieren.